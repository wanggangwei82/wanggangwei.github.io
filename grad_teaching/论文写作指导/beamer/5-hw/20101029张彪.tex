% !Mode:: "TeX:UTF-8"
\documentclass{ctexart}
%\documentclass{ctexart}

\usepackage{enumitem}
%\usepackage{color}
\usepackage{xcolor}
\newcommand{\red}[1]{\textcolor{red}{#1}}
\newcommand{\blue}[1]{\textcolor{blue}{#1}}
\newcommand{\green}[1]{\textcolor{green}{#1}}

\usepackage{amsmath,amssymb,amsthm}             % AMS Math


\title{论文写作指导之{举例用法}}
\author{学号:20102019 }
\date{姓名:张彪}
\begin{document}
\maketitle


\begin{enumerate}


\item
句型: give several examples  \dots  ~ to illustrate \dots

例句:{We will \red{give several examples }throughout this section \emph{\red{to illustrate}} this contention.  P5
	
译文:我们将会在本节中给出一些例子加以说明。}

\item 
句型: give some  (clear-cut) examples  of \dots

例句:Now we give some \blue{ clear-cut} examples \emph{\red{of}} the distinction between combinatorical and \blue{non-combinatorial} proofs.

    clear-cut 明确的
    
译文:我们仅仅给出几个能够明显看到组合与非组合证明区别的例子。

\item 

句型:present various illustrations in the manipulation of  \dots

例句: We now \red{present various illustrations }\blue{in the manipulation of} generating functions...

译文:下面给出生成函数演算的若干例子。

\item 
%用在举例之后

句型: provides a simple illustration of




例句: Example 1.1.5 \red{provides a simple illustration of} a general principle that, informally speaking, states that⋯

译文:例1.1.5 为我们提供了一个一般原理的示例。不严格地说,\dots~\dots~


\item 
%如果还有一些例子没有举,可以参考下面的表达

句型: There are many examples \dots~Some of these will appear as \dots~

例句:{\red{There are many examples}} in the \blue{literature} of finite sets that are known to have the same number of elements but for which no combinatorial proof of this fact is known. \red{Some of these will appear as} exercises throughout this book.

literature文献

译文:在文献中有很多例子,其中的有限集合已知具有相同的元素个数,但还没有找到组合方法来证明。其中一些将在本书的习题中给出。

\end{enumerate}

%\end{CJK*}
\end{document} 