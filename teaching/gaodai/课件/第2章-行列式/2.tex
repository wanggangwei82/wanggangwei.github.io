% !Mode:: "TeX:UTF-8"
\documentclass[13pt,fontset=mac]{ctexbeamer}
\usepackage[utf8]{inputenc}


\usepackage{amsmath,amssymb,amsthm}             % AMS Math
\usepackage[T1]{fontenc}

\usepackage{graphicx}
\usepackage{epstopdf}
\usepackage{tikz}
\linespread{1.3}

\usepackage{mathrsfs}  %花写字母
 \usepackage{mathdots}

%%%=== theme ===%%%
% \usetheme{Warsaw}
%\usetheme{Copenhagen}
%\usetheme{Singapore}
\usetheme{Madrid}
%\usefonttheme{professionalfonts}
%\usefonttheme{serif}
% \usefonttheme{structureitalicserif}
%%\useinnertheme{rounded}
%%\useinnertheme{inmargin}
\useinnertheme{circles}
%\useoutertheme{miniframes}
\setbeamertemplate{navigation symbols}{}
\setbeamertemplate{footline}[page number]




\usepackage{minitoc}

\usepackage{array}
\newcommand{\PreserveBackslash}[1]{\let\temp=\\#1\let\\=\temp}
\newcolumntype{C}[1]{>{\PreserveBackslash\centering}p{#1}}
\newcolumntype{R}[1]{>{\PreserveBackslash\raggedleft}p{#1}}
\newcolumntype{L}[1]{>{\PreserveBackslash\raggedright}p{#1}}



\setbeamertemplate{theorems}[numbered]
\newtheorem{thm}{定理}
\newtheorem{lem}{引理}
\newtheorem{exa}{例}
\newtheorem*{theo}{定理}
\newtheorem*{defi}{定义}
\newtheorem*{coro}{推论}
\newtheorem*{ex}{练习}
\newtheorem*{rem}{注}
\newtheorem*{prop}{性质}
\newtheorem*{qst}{问题}

\def\qed{\nopagebreak\hfill{\rule{4pt}{7pt}}\medbreak}
\def\pf{{\bf 证明~~ }}
\def\sol{{\bf 解~~ }}



\def\R{\mathbb{R}}
\def\Rn{\mathbb{R}^n}
\def\A{\mathscr{A}}
\def\B{\mathscr{B}}
\def\D{\mathscr{D}}
\def\E{\mathscr{E}}
\def\O{\mathscr{O}}

\def\rank{\operatorname{rank}}
\def\dim{\operatorname{dim}}
\def\0{\mathbf{0}}
\def\a{\alpha}
\def\b{\beta}
\def\r{\gamma}

\usepackage{color}
\definecolor{linkcol}{rgb}{0,0,0.4}
\definecolor{citecol}{rgb}{0.5,0,0}

\definecolor{blue}{rgb}{0,0.08,1}
\newcommand{\blue}{\textcolor{blue}}

\definecolor{red}{rgb}{1,0.08,0}
\newcommand{\red}{\textcolor{red}}

  \usepackage{graphicx}
  \DeclareGraphicsExtensions{.eps}
%   \usepackage[a4paper,pagebackref,hyperindex=true,pdfnewwindow=true]{hyperref}


\begin{document}



\title[]{第二章 \quad 行列式}
\author[]{{\large 张彪}\\  }
\institute[]{{\normalsize
		天津师范大学\\[6pt]
		zhang@tjnu.edu.cn}}

\date{}


\AtBeginSection[]
{
\setcounter{exa}{0}
\setcounter{equation}{0}
}


\begin{frame}
\maketitle
\end{frame}

%\begin{frame}{Outline}
%	\tableofcontents
%\end{frame}

%\section{数域}
\begin{frame}{ 1.利用定义计算 }

\[
		\left|\begin{array}{ccccc}
			0 & 0 & \cdots & 0 & 1 \\
			0 & 0 & \cdots & 2 & 0 \\
			\vdots & \vdots & & \vdots & \vdots \\
			0 & n-1 & \cdots & 0 & 0 \\
			n & 0 & \cdots & 0 & 0
		\end{array}\right| 
%	=(-1)^{\tau(n \cdots 321)} n !=(-1)^{\frac{n(n-1)}{2}} n ! 
\]
\vspace{20pt}
\[
			\left|
	\begin{array}{ccccc}
			0 & 1 & 0 & \cdots & 0 \\
			0 & 0 & 2 & \cdots & 0 \\
			\vdots & \vdots & \vdots & & \vdots \\
			0 & 0 & 0 & \cdots & n-1 \\
			n & 0 & 0 & \cdots & 0
		\end{array}\right|
%	=(-1)^{\tau(23 \cdots n 1)} n !=(-1)^{n-1} n !
\]

\end{frame}



\begin{frame}{2. 利用行列式性质把行列式化为上、下三角形行列式.}
\[
	\left|\begin{array}{llll}
		3 & 1 & 1 & 1 \\
		1 & 3 & 1 & 1 \\
		1 & 1 & 3 & 1 \\
		1 & 1 & 1 & 3
	\end{array}\right|
\qquad
	\left|\begin{array}{cccc}
		1+x & 1 & 1 & 1 \\
		1 & 1-x & 1 & 1 \\
		1 & 1 & 1+y & 1 \\
		1 & 1 & 1 & 1-y
	\end{array}\right|
\]
\end{frame}



\begin{frame}{3. 行列式按一行(一列)展开,或按多行(多列)展开(Laplace 定理)}
		\text { 公式: }
\begin{align*}
\left|\begin{array}{cc}
		A & 0 \\
		C_{1} & B
	\end{array}\right| & =\left|\begin{array}{cc}
		A & C_{2} \\
		0 & B
	\end{array}\right|=|A||B|,\\
\left|\begin{array}{cc}
		0 & A \\
		B & D_{1}
	\end{array}\right| & =\left|\begin{array}{cc}
		D_{2} & A \\
		B & 0
	\end{array}\right|=(-1)^{m n}|A| B \mid, 
\end{align*}
 { 其中 } $A, B$  { 分别是 } $m,  n$  { 阶的方阵. }
\end{frame}


\begin{frame}{3. 行列式按一行(一列)展开,或按多行(多列)展开(Laplace 定理)}
$$
D_{2 n}=
\left|\begin{array}{cccccc}
	a_{n} & & & & & b_{n} \\
	& \ddots & & & \iddots & \\
	& & a_{1} & b_{1} & & \\
	& & c_{1} & d_{1} & & \\
	&  \iddots  & & & \ddots & \\
	c_{n} & & & &  & d_{n}
\end{array}\right|
=\prod_{i=1}^{n}\left(a_{i} d_{i}-b_{i} c_{i}\right) .
$$
\end{frame}


\begin{frame}{4. 箭头形行列式或者可以化为箭头形的行列式}
\begin{align*}
\left|\begin{array}{ccccc}
	a_{0} & 1 & 1 & \cdots & 1 \\
	1 & a_{1} & 0 & \cdots & 0 \\
	1 & 0 & a_{2} & \cdots & 0 \\
	\vdots & \vdots & \vdots & & \vdots \\
	1 & 0 & 0 & \cdots & a_{n}
\end{array}\right|
& =\left|\begin{array}{ccccc}
	a_{0}-\sum_{i=1}^{n} \frac{1}{a_{i}} & 1 & 1 & \cdots & 1 \\
	0 & a_{1} & 0 & \cdots & 0 \\
	0 & 0 & a_{2} & \cdots & 0 \\
	\vdots & \vdots & \vdots & & \vdots \\
	0 & 0 & 0 & \cdots & a_{n}
\end{array}\right|\\[6pt]
& =\left(a_{0}-\sum_{i=1}^{n} \frac{1}{a_{i}}\right) a_{1} a_{2} \cdots a_{n}
\end{align*}
\end{frame}


\begin{frame}
	\small{
\begin{align*}
& 	\left|\begin{array}{ccccc}
		x_{1}-m & x_{2} & x_{3} & \cdots & x_{n} \\
		x_{1} & x_{2}-m & x_{3} & \cdots & x_{n} \\
		x_{1} & x_{2} & x_{3}-m & \cdots & x_{n} \\
		\vdots & \vdots & \vdots & & \vdots \\
		x_{1} & x_{2} & x_{3} & \cdots & x_{n}-m
	\end{array}\right|\\[-3pt]
= & \left|\begin{array}{ccccc}
		x_{1}-m & x_{2} & x_{3} & \cdots & x_{n} \\
		m & -m & 0 & \cdots & 0 \\
		m & 0 & -m & \cdots & 0 \\
		\vdots & \vdots & \vdots & & \vdots \\
		m & 0 & 0 & \cdots & 0-m
	\end{array}\right|\\[-3pt]
 = & \left|\begin{array}{ccccc}
		\sum_{i=1}^{n} x_{i}-m & x_{2} & x_{3} & \cdots & x_{n} \\
		0 & -m & 0 & \cdots & 0 \\
		0 & 0 & -m & \cdots & 0 \\
		\vdots & \vdots & \vdots & & \vdots \\
		0 & 0 & 0 & \cdots & 0-m
	\end{array}\right|=\left(\sum_{i=1}^{n} x_{i}-m\right)(-m)^{n-1}
\end{align*}}
\end{frame}

\begin{frame}
\begin{align*}
& 	\left|\begin{array}{ccccc}
		1 & 1 & \cdots & 1 & a_{0} \\
		0 & 0 & \cdots & a_{1} & 1 \\
		\vdots & \vdots & & \vdots & \vdots \\
		0 & a_{n-1} & \cdots & 0 & 1 \\
		a_{n} & 0 & \cdots & 0 & 1
	\end{array}\right| \\[8pt]
= & \left|\begin{array}{ccccc}
		1 & 1 & \cdots & 1 & a_{0}-\sum_{i=1}^{n} \frac{1}{a_{i}} \\
		0 & 0 & \cdots & a_{1} & 0 \\
		\vdots & \vdots & & \vdots & \vdots \\
		0 & a_{n-1} & \cdots & 0 & 0 \\
		a_{n} & 0 & \cdots & 0 & 0
	\end{array}\right|  \\[8pt]
= & (-1)^{\frac{n(n+1)}{2}} a_{1} a_{2} \cdots a_{n}\left(a_{0}-\sum_{i=1}^{n} \frac{1}{a_{i}}\right)
\end{align*}
\end{frame}




\begin{frame}
\begin{align*}
	& \left|\begin{array}{ccccc}
		b & \cdots & b & b & a \\
		b & \cdots & b & a & b \\
		b & \cdots & a & b & b \\
		\vdots & & \vdots & \vdots & \vdots \\
		a & \cdots & b & b & b
	\end{array}\right|   
=	\left|\begin{array}{ccccc}
	b & b & \cdots & b & a \\
	0 & 0 & \cdots & a-b & b-a \\
	\vdots & \vdots & & \vdots & \vdots \\
	0 & a-b & \cdots & 0 & b -a\\
	a-b& 0 & \cdots & 0 & b-a
\end{array}\right|   \\[8pt]
=	& \left|\begin{array}{ccccc}
b & b & \cdots & b & a+(n-1)b \\
0 & 0 & \cdots & a-b & 0 \\
\vdots & \vdots & & \vdots & \vdots \\
0 & a-b & \cdots & 0 & 0 \\
a-b& 0 & \cdots & 0 & 0
\end{array}\right|   \\[8pt]
= & (-1)^{\frac{n(n-1)}{2}}(a+(n-1) b)(a-b)^{n-1}
\end{align*}
\end{frame}


\end{document} 