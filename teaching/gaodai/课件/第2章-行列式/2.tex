% !Mode:: "TeX:UTF-8"
\documentclass[13pt,fontset=mac]{ctexbeamer}
\usepackage[utf8]{inputenc}


\usepackage{amsmath,amssymb,amsthm}             % AMS Math
\usepackage[T1]{fontenc}

\usepackage{graphicx}
\usepackage{epstopdf}
\usepackage{tikz}
\linespread{1.3}

\usepackage{mathrsfs}  %花写字母
 \usepackage{mathdots}

%%%=== theme ===%%%
% \usetheme{Warsaw}
%\usetheme{Copenhagen}
%\usetheme{Singapore}
\usetheme{Madrid}
%\usefonttheme{professionalfonts}
%\usefonttheme{serif}
% \usefonttheme{structureitalicserif}
%%\useinnertheme{rounded}
%%\useinnertheme{inmargin}
\useinnertheme{circles}
%\useoutertheme{miniframes}
\setbeamertemplate{navigation symbols}{}
\setbeamertemplate{footline}[page number]




\usepackage{minitoc}

\usepackage{array}
\newcommand{\PreserveBackslash}[1]{\let\temp=\\#1\let\\=\temp}
\newcolumntype{C}[1]{>{\PreserveBackslash\centering}p{#1}}
\newcolumntype{R}[1]{>{\PreserveBackslash\raggedleft}p{#1}}
\newcolumntype{L}[1]{>{\PreserveBackslash\raggedright}p{#1}}



\setbeamertemplate{theorems}[numbered]
\newtheorem{thm}{定理}
\newtheorem{lem}{引理}
\newtheorem{exa}{例}
\newtheorem*{theo}{定理}
\newtheorem*{defi}{定义}
\newtheorem*{coro}{推论}
\newtheorem*{ex}{练习}
\newtheorem*{rem}{注}
\newtheorem*{prop}{性质}
\newtheorem*{qst}{问题}

\def\qed{\nopagebreak\hfill{\rule{4pt}{7pt}}\medbreak}
\def\pf{{\bf 证明~~ }}
\def\sol{{\bf 解~~ }}
\def\ex{{\bf 例~~ }}


\def\R{\mathbb{R}}
\def\Rn{\mathbb{R}^n}
\def\A{\mathscr{A}}
\def\B{\mathscr{B}}
\def\D{\mathscr{D}}
\def\E{\mathscr{E}}
\def\O{\mathscr{O}}

\def\rank{\operatorname{rank}}
\def\dim{\operatorname{dim}}
\def\0{\mathbf{0}}
\def\a{\alpha}
\def\b{\beta}
\def\r{\gamma}

\usepackage{color}
\definecolor{linkcol}{rgb}{0,0,0.4}
\definecolor{citecol}{rgb}{0.5,0,0}

\definecolor{blue}{rgb}{0,0.08,1}
\newcommand{\blue}{\textcolor{blue}}

\definecolor{red}{rgb}{1,0.08,0}
\newcommand{\red}{\textcolor{red}}

  \usepackage{graphicx}
  \DeclareGraphicsExtensions{.eps}
%   \usepackage[a4paper,pagebackref,hyperindex=true,pdfnewwindow=true]{hyperref}


\begin{document}



\title[]{行列式的计算方法总结}
\author[]{{\large 张彪}\\  }
\institute[]{{\normalsize
		天津师范大学\\[6pt]
		zhang@tjnu.edu.cn}}

\date{}


\AtBeginSection[]
{
\setcounter{exa}{0}
\setcounter{equation}{0}
}


\begin{frame}
\maketitle
\end{frame}

%\begin{frame}{Outline}
%	\tableofcontents
%\end{frame}

%\section{数域}
\begin{frame}{ 1.利用定义计算 }

\[
		\left|\begin{array}{ccccc}
			0 & 0 & \cdots & 0 & 1 \\
			0 & 0 & \cdots & 2 & 0 \\
			\vdots & \vdots & & \vdots & \vdots \\
			0 & n-1 & \cdots & 0 & 0 \\
			n & 0 & \cdots & 0 & 0
		\end{array}\right| 
%	=(-1)^{\tau(n \cdots 321)} n !=(-1)^{\frac{n(n-1)}{2}} n ! 
\]
\vspace{20pt}
\[
			\left|
	\begin{array}{ccccc}
			0 & 1 & 0 & \cdots & 0 \\
			0 & 0 & 2 & \cdots & 0 \\
			\vdots & \vdots & \vdots & & \vdots \\
			0 & 0 & 0 & \cdots & n-1 \\
			n & 0 & 0 & \cdots & 0
		\end{array}\right|
%	=(-1)^{\tau(23 \cdots n 1)} n !=(-1)^{n-1} n !
\]

\end{frame}



\begin{frame}{2. 利用行列式性质把行列式化为上、下三角形行列式.}
\[
	\left|\begin{array}{llll}
		3 & 1 & 1 & 1 \\
		1 & 3 & 1 & 1 \\
		1 & 1 & 3 & 1 \\
		1 & 1 & 1 & 3
	\end{array}\right|
\qquad
	\left|\begin{array}{cccc}
		1+x & 1 & 1 & 1 \\
		1 & 1-x & 1 & 1 \\
		1 & 1 & 1+y & 1 \\
		1 & 1 & 1 & 1-y
	\end{array}\right|
\]
\end{frame}

\begin{frame}
有时也可化成	斜上、斜下三角形行列式
\tiny{
\begin{align*}
 & \left|\begin{array}{ccccc}
	b & \cdots & b & b & a \\
	b & \cdots & b & a & b \\
	b & \cdots & a & b & b \\
	\vdots & & \vdots & \vdots & \vdots \\
	a & \cdots & b & b & b
\end{array}\right|
=   \left|\begin{array}{ccccc}
	b & \cdots & b & b & a+(n-1) b \\
	b & \cdots & b & a & a+(n-1) b \\
	b & \cdots & a & b & a+(n-1) b \\
	\vdots & & \vdots & \vdots & \vdots \\
	a & \cdots & b & b & a+(n-1) b
\end{array}\right|\\[8pt]
= & \,  (a+(n-1) b)\left|\begin{array}{ccccc}
	b & \cdots & b & b & 1 \\
	b & \cdots & b & a & 1 \\
	b & \cdots & a & b & 1 \\
	\vdots & & \vdots & \vdots & \vdots \\
	a & \cdots & b & b & 1
\end{array}\right|\\[8pt]
 = &  \,  (a+(n-1) b)\left|\begin{array}{ccccc}
	b & \cdots & b & b & 1 \\
	0 & \cdots & 0 & a-b & 0 \\
	0 & \cdots & a-b & 0 & 0 \\
	\vdots & & \vdots & \vdots & \vdots \\
	a-b & \cdots & 0 & 0 & 0
\end{array}\right|
= (-1)^{\frac{n(n-1)}{2}}(a+(n-1) b)(a-b)^{n-1}
\end{align*}}
\end{frame}

%\begin{frame}{3. 行列式按一行(一列)展开,或按多行(多列)展开(Laplace 定理)}
%		\text { 公式: }
%\begin{align*}
%\left|\begin{array}{cc}
%		A & 0 \\
%		C_{1} & B
%	\end{array}\right| & =\left|\begin{array}{cc}
%		A & C_{2} \\
%		0 & B
%	\end{array}\right|=|A||B|,\\
%\left|\begin{array}{cc}
%		0 & A \\
%		B & D_{1}
%	\end{array}\right| & =\left|\begin{array}{cc}
%		D_{2} & A \\
%		B & 0
%	\end{array}\right|=(-1)^{m n}|A| B \mid, 
%\end{align*}
% { 其中 } $A, B$  { 分别是 } $m,  n$  { 阶的方阵. }
%\end{frame}


\begin{frame}{3. 行列式按一行(一列)展开,或按多行(多列)展开(Laplace 定理)}
$$
D_{2 n}=
\left|\begin{array}{cccccc}
	a_{n} & & & & & b_{n} \\
	& \ddots & & & \iddots & \\
	& & a_{1} & b_{1} & & \\
	& & c_{1} & d_{1} & & \\
	&  \iddots  & & & \ddots & \\
	c_{n} & & & &  & d_{n}
\end{array}\right|
=\prod_{i=1}^{n}\left(a_{i} d_{i}-b_{i} c_{i}\right) .
$$
\end{frame}


\begin{frame}{4. 箭头形行列式或者可以化为箭头形的行列式}
\begin{align*}
\left|\begin{array}{ccccc}
	a_{0} & 1 & 1 & \cdots & 1 \\
	1 & a_{1} & 0 & \cdots & 0 \\
	1 & 0 & a_{2} & \cdots & 0 \\
	\vdots & \vdots & \vdots & & \vdots \\
	1 & 0 & 0 & \cdots & a_{n}
\end{array}\right|
& =\left|\begin{array}{ccccc}
	a_{0}-\sum_{i=1}^{n} \frac{1}{a_{i}} & 1 & 1 & \cdots & 1 \\
	0 & a_{1} & 0 & \cdots & 0 \\
	0 & 0 & a_{2} & \cdots & 0 \\
	\vdots & \vdots & \vdots & & \vdots \\
	0 & 0 & 0 & \cdots & a_{n}
\end{array}\right|\\[6pt]
& =\left(a_{0}-\sum_{i=1}^{n} \frac{1}{a_{i}}\right) a_{1} a_{2} \cdots a_{n}
\end{align*}
\end{frame}


\begin{frame}
	\small{
\begin{align*}
& 	\left|\begin{array}{ccccc}
		x_{1}-m & x_{2} & x_{3} & \cdots & x_{n} \\
		x_{1} & x_{2}-m & x_{3} & \cdots & x_{n} \\
		x_{1} & x_{2} & x_{3}-m & \cdots & x_{n} \\
		\vdots & \vdots & \vdots & & \vdots \\
		x_{1} & x_{2} & x_{3} & \cdots & x_{n}-m
	\end{array}\right|\\[-3pt]
= & \left|\begin{array}{ccccc}
		x_{1}-m & x_{2} & x_{3} & \cdots & x_{n} \\
		m & -m & 0 & \cdots & 0 \\
		m & 0 & -m & \cdots & 0 \\
		\vdots & \vdots & \vdots & & \vdots \\
		m & 0 & 0 & \cdots & 0-m
	\end{array}\right|\\[-3pt]
 = & \left|\begin{array}{ccccc}
		\sum_{i=1}^{n} x_{i}-m & x_{2} & x_{3} & \cdots & x_{n} \\
		0 & -m & 0 & \cdots & 0 \\
		0 & 0 & -m & \cdots & 0 \\
		\vdots & \vdots & \vdots & & \vdots \\
		0 & 0 & 0 & \cdots & 0-m
	\end{array}\right|=\left(\sum_{i=1}^{n} x_{i}-m\right)(-m)^{n-1}
\end{align*}}
\end{frame}

\begin{frame}
\begin{align*}
& 	\left|\begin{array}{ccccc}
		1 & 1 & \cdots & 1 & a_{0} \\
		0 & 0 & \cdots & a_{1} & 1 \\
		\vdots & \vdots & & \vdots & \vdots \\
		0 & a_{n-1} & \cdots & 0 & 1 \\
		a_{n} & 0 & \cdots & 0 & 1
	\end{array}\right| \\[8pt]
= & \left|\begin{array}{ccccc}
		1 & 1 & \cdots & 1 & a_{0}-\sum_{i=1}^{n} \frac{1}{a_{i}} \\
		0 & 0 & \cdots & a_{1} & 0 \\
		\vdots & \vdots & & \vdots & \vdots \\
		0 & a_{n-1} & \cdots & 0 & 0 \\
		a_{n} & 0 & \cdots & 0 & 0
	\end{array}\right|  \\[8pt]
= & (-1)^{\frac{n(n+1)}{2}} a_{1} a_{2} \cdots a_{n}\left(a_{0}-\sum_{i=1}^{n} \frac{1}{a_{i}}\right)
\end{align*}
\end{frame}




\begin{frame}
\begin{align*}
	& \left|\begin{array}{ccccc}
		b & \cdots & b & b & a \\
		b & \cdots & b & a & b \\
		b & \cdots & a & b & b \\
		\vdots & & \vdots & \vdots & \vdots \\
		a & \cdots & b & b & b
	\end{array}\right|   
=	\left|\begin{array}{ccccc}
	b & b & \cdots & b & a \\
	0 & 0 & \cdots & a-b & b-a \\
	\vdots & \vdots & & \vdots & \vdots \\
	0 & a-b & \cdots & 0 & b -a\\
	a-b& 0 & \cdots & 0 & b-a
\end{array}\right|   \\[8pt]
=	& \left|\begin{array}{ccccc}
b & b & \cdots & b & a+(n-1)b \\
0 & 0 & \cdots & a-b & 0 \\
\vdots & \vdots & & \vdots & \vdots \\
0 & a-b & \cdots & 0 & 0 \\
a-b& 0 & \cdots & 0 & 0
\end{array}\right|   \\[8pt]
= & (-1)^{\frac{n(n-1)}{2}}(a+(n-1) b)(a-b)^{n-1}
\end{align*}
\end{frame}


\begin{frame}{5. 逐行逐列相加减}
	行列式特点是每相邻两行(列)之间有许多元素相同.用逐行(列)相减可以化出零.
	
	$$
	\left|\begin{array}{ccccc}
		1 & 2 & 2 & \cdots & 2 \\
		2 & 2 & 2 & \cdots & 2 \\
		2 & 2 & 3 & \cdots & 2 \\
		\vdots & \vdots & \vdots & & \vdots \\
		2 & 2 & 2 & \cdots & n
	\end{array}\right|=\left|\begin{array}{ccccc}
		-1 & 0 & 0 & \cdots & 0 \\
		2 & 2 & 2 & \cdots & 2 \\
		0 & 0 & 1 & \cdots & 0 \\
		\vdots & \vdots & \vdots & & \vdots \\
		0 & 0 & 0 & \cdots & n-2
	\end{array}\right|=(-2)(n-2) !
	$$
\end{frame}


\begin{frame}
\begin{align*}
& \left|\begin{array}{cccccc}
	1 & 2 & 3 & \cdots & n-1 & n \\
	2 & 2 & 3 & \cdots & n-1 & n \\
	3 & 3 & 3 & \cdots & n-1 & n \\
	\vdots & \vdots & \vdots & & \vdots & \vdots \\
	n-1 & n-1 & n-1 & \cdots & n-1 & n \\
	n & n & n & \cdots & n & n
\end{array}\right|\\[8pt]
= & \left|\begin{array}{cccccc}
	-1 & 0 & 0 & \cdots & 0 & 0 \\
	-1 & -1 & 0 & \cdots & 0 & 0 \\
	-1 & -1 & -1 & \cdots & 0 & 0 \\
	\vdots & \vdots & \vdots & & \vdots & \vdots \\
	-1 & -1 & -1 & \cdots & -1 & 0 \\
	n & n & n & \cdots & n & n
\end{array}\right|=(-1)^{n-1} n .
\end{align*}
\end{frame}

\begin{frame}
	\tiny{
	$$
	\begin{aligned}
		&\left|\begin{array}{cccccc}
			1 & 2 & 3 & \cdots & n-1 & n \\
			2 & 3 & 4 & \cdots & n & 1 \\
			3 & 4 & 5 & \cdots & 1 & 2 \\
			\vdots & \vdots & \vdots & & \vdots & \vdots \\
			n-1 & n & 1 & \cdots & n-3 & n-2 \\
			n & 1 & 2 & \cdots & n-2 & n-1
		\end{array}\right|=\left|\begin{array}{cccccc}
			1 & 2 & 3 & \cdots & n-1 & n \\
			1 & 1 & 1 & \cdots & 1 & 1-n \\
			1 & 1 & 1 & \cdots & 1-n & 1 \\
			\vdots & \vdots & \vdots & & \vdots & \vdots \\
			1 & 1 & 1-n & \cdots & 1 & 1 \\
			1 & 1-n & 1 & \cdots & 1 & 1
		\end{array}\right|\\[8pt]
	= 	&\, \left|\begin{array}{cccccc}
			\frac{n(n+1)}{2} & 2 & 3 & \cdots & n-1 & n \\
			0 & 1 & 1 & \cdots & 1 & 1-n \\
			0 & 1 & 1 & \cdots & 1-n & 1 \\
			\vdots & \vdots & \vdots & & \vdots & \vdots \\
			0 & 1 & 1-n & \cdots & 1 & 1 \\
			0 & 1-n & 1 & \cdots & 1 & 1
		\end{array}\right|=\frac{n(n+1)}{2}\left|\begin{array}{ccccc}
			1 & 1 & \cdots & 1 & 1-n \\
			1 & 1 & \cdots & 1-n & 1 \\
			\vdots & \vdots & & \vdots & \vdots \\
			1 & 1-n & \cdots & 1 & 1 \\
			1-n & 1 & \cdots & 1 & 1
		\end{array}\right| \\[8pt]
	= & \,  \cdots 	\cdots 	
	\end{aligned}
	$$}
\end{frame}



%=\frac{n(n+1)}{2}\left|\begin{array}{ccccc}
%		1 & 1 & \cdots & 1 & -1 \\
%		0 & 0 & \cdots & -n & 0 \\
%		\vdots & \vdots & & \vdots & \vdots \\
%		0 & -n & \cdots & 0 & 0 \\
%		-n & 0 & \cdots & 0 & 0
%	\end{array}\right| \\
%	=\frac{n(n+1)}{2}(-1)^{\frac{(n-1)(n-2)}{2}}(-1)^{n-1} n^{n-2}=(-1)^{\frac{n(n-1)}{2}} \frac{n+1}{2} n^{n-1}


\begin{frame}
{6. 加边法 (添加一行一列,利于计算,但同时保持行列式不变)}
\small{
\begin{equation*}
	\begin{aligned}
		&\left|\begin{array}{ccccc}
			a+x_{1} & a & a & \cdots & a \\
			a & a+x_{2} & a & \cdots & a \\
			a & a & a+x_{3} & \cdots & a \\
			\vdots & \vdots & \vdots & & \vdots \\
			a & a & a & \cdots & a+x_{n}
		\end{array}\right|\\[8pt]
= 	& \left|\begin{array}{cccccc}
			1 & a & a & a & \cdots & a \\
			0 & a+x_{1} & a & a & \cdots & a \\
			0 & a & a+x_{2} & a & \cdots & a \\
			0 & a & a & a+x_{3} & \cdots & a \\
			\vdots & \vdots & \vdots & \vdots & & \vdots \\
			0 & a & a & a & \cdots & a+x_{n}
		\end{array}\right|
%		&=\left|\begin{array}{cccccc}
%			1 & a & a & a & \cdots & a \\
%			-1 & x_{1} & 0 & 0 & \cdots & 0 \\
%			-1 & 0 & x_{2} & 0 & \cdots & 0 \\
%			-1 & 0 & 0 & x_{3} & \cdots & 0 \\
%			\vdots & \vdots & \vdots & \vdots & & \vdots \\
%			-1 & 0 & 0 & 0 & \cdots & x_{n}
%		\end{array}\right|=\left|\begin{array}{cccccc}
%			1+\sum_{i=1}^{n} \frac{a}{x_{i}} & a & a & a & \cdots & a \\
%			0 & x_{1} & 0 & 0 & \cdots & 0 \\
%			0 & 0 & x_{2} & 0 & \cdots & 0 \\
%			0 & 0 & 0 & x_{3} & \cdots & 0 \\
%			\vdots & \vdots & \vdots & \vdots & & \vdots \\
%			0 & 0 & 0 & 0 & \cdots & x_{n}
%		\end{array}\right|=x_{1} x_{2} \cdots x_{n}\left(1+\sum_{i=1}^{n} \frac{a}{x_{i}}\right)
	\end{aligned}
\end{equation*}}
\end{frame}


\begin{frame}{7.利用数学归纳法证明行列式}
\begin{equation*}
	\left|\begin{array}{cccccc}
		\alpha+\beta & \alpha \beta & 0 & \cdots & 0 & 0 \\
		1 & \alpha+\beta & \alpha \beta & \cdots & 0 & 0 \\
		0 & 1 & \alpha+\beta & \cdots & 0 & 0 \\
		\vdots & \vdots & \vdots & & \vdots & \vdots \\
		0 & 0 & 0 & \cdots & \alpha+\beta & \alpha \beta \\
		0 & 0 & 0 & \cdots & 1 & \alpha+\beta
	\end{array}\right|
	=\frac{\alpha^{n+1}-\beta^{n+1}}{\alpha-\beta}
\end{equation*}

证明:利用第二数学归纳法:
若 $n=1,$ 则 $D_{1}=\alpha+\beta=\frac{\alpha^{2}-\beta^{2}}{\alpha-\beta},$ 成立.

假设结论对所有小于 $n$ 的都成立.

将
$D_{n}$ 按第一行(或第一列)展开得 $D_{n}=(\alpha+\beta) D_{n-1}-\alpha \beta D_{n-2}$,

利用归纳假设可得
$D_{n}=(\alpha+\beta) D_{n-1}-\alpha \beta D_{n-2}=(\alpha+\beta) \frac{\alpha^{n}-\beta^{n}}{\alpha-\beta}-\alpha \beta \frac{\alpha^{n-1}-\beta^{n-1}}{\alpha-\beta}=\frac{\alpha^{n+1}-\beta^{n+1}}{\alpha-\beta}$.
\end{frame}

\begin{frame}{8. 利用递推公式}


解: 按第一行展开得 $D_{n}=(\alpha+\beta) D_{n-1}-\alpha \beta D_{n-2}$, 将此式化为:
\begin{center}
 $D_{n}-\alpha D_{n-1}=\beta\left(D_{n-1}-\alpha D_{n-2}\right)$ (1)
 
 $D_{n}-\beta D_{n-1}=\alpha\left(D_{n-1}-\beta D_{n-2}\right)$ (2)
 \end{center}
利用公式(1)得 $D_{n}-\alpha D_{n-1}=\beta\left(D_{n-1}-\alpha D_{n-2}\right)=\beta^{2}\left(D_{n-2}-\alpha D_{n-3}\right)=\cdots=\beta^{n-2}\left(D_{2}-\alpha D_{1}\right)=\beta^{n},$ 

即
$D_{n}=\alpha D_{n-1}+\beta^{n}$
(3)


利用公式(2)得 $D_{n}-\beta D_{n-1}=\alpha\left(D_{n-1}-\beta D_{n-2}\right)=\alpha^{2}\left(D_{n-2}-\beta D_{n-3}\right)=\cdots=\alpha^{n-2}\left(D_{2}-\beta D_{1}\right)=\alpha^{n},$ 

即
$D_{n}=\beta D_{n-1}+\alpha^{n}$
(4)

$
	\text { 由(3)(4) 解得: } D_{n}=\left\{\begin{array}{ll}
		\frac{\alpha^{n+1}-\beta^{n+1}}{\alpha-\beta}, & \alpha \neq \beta \\
		(n+1) \alpha^{n}, & \alpha=\beta
	\end{array}\right.
$
\end{frame}

\begin{frame}
	\begin{equation*}
		D_{n}=\left|\begin{array}{ccccc}
			7 & 2 & & & \\
			5 & 7 & \ddots & & \\
			& \ddots & \ddots & \ddots & \\
			& & \ddots & 7 & 2 \\
			& & & 5 & 7 \\
			& & &
		\end{array}\right|=7 D_{n-1}-10 D_{n-2}
	\end{equation*}
\end{frame}


\begin{frame}{9. 拆项法: 将行列式的其中一行(列)拆成两个数的和}

	\small{	要点:分解成两个容易求的行列式的和.
	\begin{equation*}
		\begin{aligned}
			 D_{n}= & \left|\begin{array}{cccccc}
				a & b & b & \cdots & b & b \\
				c & a & b & \cdots & b & b \\
				c & c & a & \cdots & b & b \\
				\vdots & \vdots & \vdots & & \vdots & \vdots \\
				c & c & c & \cdots & a & b \\
				c & c & c & \cdots & c & a
			\end{array}\right|=\left|\begin{array}{cccccc}
				c+a-c & b & b & \cdots & b & b \\
				c & a & b & \cdots & b & b \\
				c & c & a & \cdots & b & b \\
				\vdots & \vdots & \vdots & & \vdots & \vdots \\
				c & c & c & \cdots & a & b \\
				c & c & c & \cdots & c & a
			\end{array}\right|\\
			= & \left|\begin{array}{cccccc}
				c & b & b & \cdots & b & b \\
				c & a & b & \cdots & b & b \\
				c & c & a & \cdots & b & b \\
				\vdots & \vdots & \vdots & & \vdots & \vdots \\
				c & c & c & \cdots & a & b \\
				c & c & c & \cdots & c & a
			\end{array}\right|+\left|\begin{array}{cccccc}
				a-c & b & b & \cdots & b & b \\
				0 & a & b & \cdots & b & b \\
				0 & c & a & \cdots & b & b \\
				\vdots & \vdots & \vdots & & \vdots & \vdots \\
				0 & c & c & \cdots & a & b \\
				0 & c & c & \cdots & c & a
			\end{array}\right|
		\end{aligned}
	\end{equation*}}
\end{frame}


\end{document} 