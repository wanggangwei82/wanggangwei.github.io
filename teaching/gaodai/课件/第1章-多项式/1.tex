% !Mode:: "TeX:UTF-8"
\documentclass[13pt,fontset=mac]{ctexbeamer}
\usepackage[utf8]{inputenc}


\usepackage{amsmath,amssymb,amsthm}             % AMS Math
\usepackage[T1]{fontenc}

\usepackage{graphicx}
\usepackage{epstopdf}
\usepackage{tikz}
\linespread{1.3}

\usepackage{mathrsfs}  %花写字母
 

%%%=== theme ===%%%
% \usetheme{Warsaw}
%\usetheme{Copenhagen}
%\usetheme{Singapore}
\usetheme{Madrid}
%\usefonttheme{professionalfonts}
%\usefonttheme{serif}
% \usefonttheme{structureitalicserif}
%%\useinnertheme{rounded}
%%\useinnertheme{inmargin}
\useinnertheme{circles}
%\useoutertheme{miniframes}
\setbeamertemplate{navigation symbols}{}
\setbeamertemplate{footline}[page number]




\usepackage{minitoc}

\usepackage{array}
\newcommand{\PreserveBackslash}[1]{\let\temp=\\#1\let\\=\temp}
\newcolumntype{C}[1]{>{\PreserveBackslash\centering}p{#1}}
\newcolumntype{R}[1]{>{\PreserveBackslash\raggedleft}p{#1}}
\newcolumntype{L}[1]{>{\PreserveBackslash\raggedright}p{#1}}



\setbeamertemplate{theorems}[numbered]
\newtheorem{thm}{定理}
\newtheorem{lem}{引理}
\newtheorem{exa}{例}
\newtheorem*{theo}{定理}
\newtheorem*{defi}{定义}
\newtheorem*{coro}{推论}
\newtheorem*{ex}{练习}
\newtheorem*{rem}{注}
\newtheorem*{prop}{性质}
\newtheorem*{qst}{问题}

\def\qed{\nopagebreak\hfill{\rule{4pt}{7pt}}\medbreak}
\def\pf{{\bf 证明~~ }}
\def\sol{{\bf 解~~ }}



\def\R{\mathbb{R}}
\def\Rn{\mathbb{R}^n}
\def\A{\mathscr{A}}
\def\B{\mathscr{B}}
\def\D{\mathscr{D}}
\def\E{\mathscr{E}}
\def\O{\mathscr{O}}

\def\rank{\operatorname{rank}}
\def\dim{\operatorname{dim}}
\def\0{\mathbf{0}}
\def\a{\alpha}
\def\b{\beta}
\def\r{\gamma}

\usepackage{color}
\definecolor{linkcol}{rgb}{0,0,0.4}
\definecolor{citecol}{rgb}{0.5,0,0}

\definecolor{blue}{rgb}{0,0.08,1}
\newcommand{\blue}{\textcolor{blue}}

\definecolor{red}{rgb}{1,0.08,0}
\newcommand{\red}{\textcolor{red}}

  \usepackage{graphicx}
  \DeclareGraphicsExtensions{.eps}
%   \usepackage[a4paper,pagebackref,hyperindex=true,pdfnewwindow=true]{hyperref}


\begin{document}



\title[]{第一章 \quad 多项式}
\author[]{{\large 张彪}\\  }
\institute[]{{\normalsize
		天津师范大学\\[6pt]
		zhang@tjnu.edu.cn}}

\date{}


\AtBeginSection[]
{
\setcounter{exa}{0}
\setcounter{equation}{0}
}


\begin{frame}
\maketitle
\end{frame}

\begin{frame}{Outline}
	\tableofcontents
\end{frame}

\section{数域}
\begin{frame}{\S 1    数域}

\begin{defi}
设$P$是由一些复数组成的集合,其中
	包括0与1.如果$P$中任意两个数的和、差、积、商
	(除数不为零)仍在$P$中,则称$P$为一个数域.
\end{defi}

   常用到的数域:有理数域$\mathbb{Q}$ 、实数域$\mathbb{R}$、复数域$\mathbb{C}$.
   
   \vspace{10pt}
数域定义的另一形式  
\begin{defi}
	设$P$是由一些复数组成的集合,其中
	包括0与1.如果对
	于加法、减法、乘法、除法(除数不为零)运算
	封闭,则称$P$为一个数域.
\end{defi}


\end{frame}


\begin{frame}
\begin{exa}
所有形如 $a+b \sqrt{2}(a ,b$ 是有理数) 的数 构成一个数域 $\mathbb{Q}(\sqrt{2}) .$
\end{exa} 

\pause
\pf (i) $0,1 \in \mathbb{Q}(\sqrt{2})$


(ii)对四则运算封闭.事实上 $\forall a, \beta \in \mathbb{Q}(\sqrt{2}),$ 设 $\a=a+b \sqrt{2}, \beta=c+d \sqrt{2},$ 有
\begin{align*}
a \pm \beta & =(a \pm c)+(b \pm d) \sqrt{2} \in \mathbb{Q}(\sqrt{2})\\[6pt]
a \beta & =(a c+2 b d)+(a d+b c) \sqrt{2} \in \mathbb{Q}(\sqrt{2})
\end{align*}

设 $\a=a+b \sqrt{2} \neq 0, \quad$ 则 $a-b \sqrt{2} \neq 0$ 且
\begin{align*}
\frac{\beta}{\a}= & \frac{c+d \sqrt{2}}{a+b \sqrt{2}}=\frac{(c+d \sqrt{2})(a-b \sqrt{2})}{(a+b \sqrt{2})(a-b \sqrt{2)}}\\[6pt]
= & \frac{a c-2 b d}{a^{2}-2 b^{2}}+\frac{a d-b c}{a^{2}-2 b^{2}} \sqrt{2} \in \mathbb{Q}(\sqrt{2})
\end{align*}

\end{frame}

\begin{frame}
\begin{rem}
有理数域是最小的数域。
%也就是说,任何数域都包含有理数域作为它的一部分。


\pf
设$P$为一个数域.
\begin{itemize}
    \item 由定义知$1\in P$,
    \item 又$P$对加法封闭知:1+1=2,
1+2=3, $\cdots$,  $P$包含所有\alert{自然数};
	\item 由$0\in P$及$P$对减法的封闭性知:$P$包含所有负
整数,因而$P$包含所有\alert{整数};
\item 
任何一个有理数都可以表为两个整数的商,
由P对除法的封闭性知:$P$包含所有\alert{有理数}.

\end{itemize}

即
    任何数域都包含有理数域作为它的一部分.

\end{rem}

\end{frame}


\begin{frame}

\begin{defi}
设$x$是一个符号(文字),$n$为非负整数. 
形式表达式  
\[
a_nx^n+a_{n-1}x^{n-1}+\cdots+a_1x+a_0,
\]                
其中$a_0, a_1, \ldots, a_n\in P$, 称为系数在数域$P$中的一元多项式,简称为\alert{数域$P$上的一元多项式}.
\end{defi}


\begin{rem}
\begin{itemize}
\item 符号$x$ 可以是为未知数,
也可以是其它待定事物.
\item 这里 $a_ix^i$ 称为 \alert{$i$次项}, $a_i$ 称为 \alert{$i$次项系数}.

若$a_n\neq 0$, 则称 $a_n x^n$ 为\alert{首项}.
\item 习惯上记为$f (x), g(x), \ldots$或 $f, g, \ldots$ 
上述形式表达式可写为         
\[
\sum_{i=0}a_i x^i.
\]       
\end{itemize}
\end{rem}
\end{frame}

\begin{frame}
\begin{itemize}
\item 零多项式  ——系数全为0的多项式
\item 多项式相等 ——$ f (x)=g(x)$当且仅当同次项的系
数全相等 (系数为零的项除外)
\item 多项式 $f (x)$的次数 —— $f (x)$的最高次项对应的幂
次,记作 $\deg (f (x))$.

如: $f (x) =3x^3+4x^2-5x+6$的次数为3,即
$\deg (f (x))=3$

\end{itemize}
\end{frame}


\begin{frame}
\begin{center}
	\setlength{\tabcolsep}{2mm}{
\begin{tabular}{R{2cm}|R{10pt} R{20pt}R{20pt}R{15pt}|R{1.3cm}}
\blue{$x^{2}-3 x +1$} & \red{$ 3 x^{3}$}& \red{$+4 x^{2}$} & \red{$-5 x$} & \red{$+6$}  & $3x +13$ \\
&  $3 x^{3}$ &$-9 x^{2}$ & $+3 x$  & \\
 \cline{2-5}
&& $13 x^{2}$& $-8 x$ & +6  & \\
 & &$13 x^{2}$ & $-39 x$ & +13 & \\
  \cline{2-5}
&&& $ 31 x$& -7 & 
\end{tabular}}
\end{center}
		\[
	\red{f(x)}=q(x) \blue{g(x)} +r(x)
	\]
\end{frame}


\begin{frame}
	\begin{exa}
$$
\begin{array}{l}
	f(x)=x^{4}+3 x^{3}-x^{2}-4 x-3 \\
	g(x)=3 x^{3}+10 x^{2}+2 x-3
\end{array}
$$
求 $(f(x), g(x))$, 并求 $u(x), v(x)$ 使
$$
(f(x), g(x))=u(x) f(x)+v(x) g(x)
$$
	\end{exa}
\end{frame}


\begin{frame}
辗转相除法可按下面的格式来作:
	\begin{center}
		\renewcommand\arraystretch{1.3}
		\begin{tabular}{R{3cm} |R{4cm}|L{1.5cm}}
			\blue{$3x^3+10x^2+2x-3$}&\red{$x^4+3x^3-x^2\, -4x-3$}&$\frac{1}{3}x-\frac{1}{9}$\\
			& $x^4+\frac{10}{3}x^3+\frac{2}{3}x^2-x\qquad $& $=q_1(x)$\\
			\cline{2-2}
			& $-\frac{1}{3}x^3-\frac{5}{3}x^2-3x-3\, $ & \\
			& $-\frac{1}{3}x^3-\frac{10}{9}x^2-\frac{2}{9}x+\frac{1}{3}$ & \\
			\cline{2-2}
			& $r_1(x)= -\frac{5}{9}x^2-\frac{25}{9}x-\frac{10}{3}$ & 
		\end{tabular}
	\end{center}
	
	
	\[
	\red{f(x)}=q_1(x) \blue{g(x)} +r_1(x)
	\]
\end{frame}




\begin{frame}
	\begin{center}
				\renewcommand\arraystretch{1.3}
		\begin{tabular}{R{1.5cm}|R{3cm} |R{4cm}|L{1.7cm}}
$-\frac{27}{5}x+9$ &\red{$3x^3+10x^2+2x-3$}&{$x^4+3x^3-x^2\, -4x-3$}&$\frac{1}{3}x-\frac{1}{9}$\\
$=q_2(x)$ & $3x^3+15x^2+18x\quad $ & $x^4+\frac{10}{3}x^3+\frac{2}{3}x^2-x\qquad $& $=q_1(x)$\\
 \cline{2-3}
 &$ -5x^2-16x-3$ & $-\frac{1}{3}x^3-\frac{5}{3}x^2-3x-3\, $ & \\
  &$ -5x^2-25x-30$ & $-\frac{1}{3}x^3-\frac{10}{9}x^2-\frac{2}{9}x+\frac{1}{3}$ & \\
   \cline{2-3}
  &$ r_2(x) = 9x+27$ & \blue{$r_1(x) =-\frac{5}{9}x^2-\frac{25}{9}x-\frac{10}{3}$} & 
		\end{tabular}
	\end{center}
	\pause
	
\begin{align*}
	{f(x)} & =q_1(x) {g(x)} +r_1(x)\\
\red{g(x)} & =q_2(x) \blue{r_1(x)} +r_2(x)
\end{align*}

\end{frame}




\begin{frame}
	\begin{center}
				\renewcommand\arraystretch{1.2}
		\begin{tabular}{R{1.5cm}|R{3cm} |R{4cm}|L{1.7cm}}
			$-\frac{27}{5}x+9$ &{$3x^3+10x^2+2x-3$}&{$x^4+3x^3-x^2\, -4x-3$}&$\frac{1}{3}x-\frac{1}{9}$\\
			$=q_2(x)$ & $3x^3+15x^2+18x\quad $ & $x^4+\frac{10}{3}x^3+\frac{2}{3}x^2-x\qquad $& $=q_1(x)$\\
			\cline{2-3}
			&$ -5x^2-16x-3$ & $-\frac{1}{3}x^3-\frac{5}{3}x^2-3x-3\, $ & \\
			&$ -5x^2-25x-30$ & $-\frac{1}{3}x^3-\frac{10}{9}x^2-\frac{2}{9}x+\frac{1}{3}$ & \\
			\cline{2-3}
			&\blue{$r_2(x) = 9x+27$} & \red{$r_1(x) =-\frac{5}{9}x^2-\frac{25}{9}x-\frac{10}{3}$} & $-\frac{5}{81}x-\frac{10}{81}$\\
			& & $-\frac{5}{9}x^2-\frac{5}{3}x$ & $=q_3(x)$\\
			\cline{3-3}
			& & $-\frac{10}{9}x-\frac{10}{3}$ & \\
			& & $-\frac{10}{9}x-\frac{10}{3}$ & \\
			\cline{3-3}
			& & 0 & \\
		\end{tabular}
	\end{center}
	
	\begin{align*}
		{f(x)} & =q_1(x) {g(x)} +r_1(x)\\
		{g(x)} & =q_2(x) {r_1(x)} +r_2(x)\\
		\red{r_1(x)} & = q_3(x) \blue{r_2(x)}
	\end{align*}
\end{frame}



\end{document} 