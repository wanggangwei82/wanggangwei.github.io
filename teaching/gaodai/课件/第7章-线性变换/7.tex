% !Mode:: "TeX:UTF-8"
\documentclass[13pt]{beamer}
\usepackage[utf8]{inputenc}


\usepackage{amsmath,amssymb,amsthm}             % AMS Math
\usepackage[T1]{fontenc}

\usepackage{graphicx}
\usepackage{epstopdf}
\usepackage{tikz}
\linespread{1.3}

\usepackage{mathrsfs}  %花写字母
 

%%%=== theme ===%%%
% \usetheme{Warsaw}
%\usetheme{Copenhagen}
%\usetheme{Singapore}
\usetheme{Madrid}
%\usefonttheme{professionalfonts}
%\usefonttheme{serif}
% \usefonttheme{structureitalicserif}
%%\useinnertheme{rounded}
%%\useinnertheme{inmargin}
\useinnertheme{circles}
%\useoutertheme{miniframes}
\setbeamertemplate{navigation symbols}{}
\setbeamertemplate{footline}[page number]




\usepackage{ctex}
% \usepackage{CJK,CJKnumb,CJKulem}
\usepackage{minitoc}


\setbeamertemplate{theorems}[numbered]
\newtheorem{thm}{定理}
\newtheorem{lem}{引理}
\newtheorem{exa}{例}
\newtheorem*{theo}{定理}
\newtheorem*{defi}{定义}
\newtheorem*{coro}{推论}
\newtheorem*{ex}{练习}
\newtheorem*{rem}{注}
\newtheorem*{prop}{性质}
\newtheorem*{qst}{问题}
\def\qed{\nopagebreak\hfill{\rule{4pt}{7pt}}\medbreak}
\def\pf{{\bf 证明~~ }}
\def\sol{{\bf 解~~ }}
\def\Rn{\mathbb{R}^n}
\def\A{\mathscr{A}}
\def\B{\mathscr{B}}
\def\D{\mathscr{D}}
\def\E{\mathscr{E}}
\def\O{\mathscr{O}}


\def\dim{\operatorname{dim}}
\def\0{\mathbf{0}}
\def\a{\alpha}
\def\b{\beta}
\def\r{\gamma}

\usepackage{color}
\definecolor{linkcol}{rgb}{0,0,0.4}
\definecolor{citecol}{rgb}{0.5,0,0}

\definecolor{blue}{rgb}{0,0.08,1}
\newcommand{\blue}{\textcolor{blue}}

  \usepackage{graphicx}
  \DeclareGraphicsExtensions{.eps}
%   \usepackage[a4paper,pagebackref,hyperindex=true,pdfnewwindow=true]{hyperref}


\begin{document}



\title[]{第七章 \quad  线性变换}
\author[]{{\large 张彪}\\  }
\institute[]{{\normalsize
		天津师范大学\\[6pt]
		zhang@tjnu.edu.cn}}

\date{}


\AtBeginSection[]
{
\setcounter{exa}{0}
\setcounter{equation}{0}
}


\begin{frame}
\maketitle
\end{frame}

\begin{frame}{Outline}
	\tableofcontents
\end{frame}

\section{线性变换的定义}
\begin{frame}{\S 1  线性变换的定义}

%\begin{itemize}
%\item 上一章我们看到,数域 $P$ 上任意一个 $n$ 维线性空间都与 $P^{n}$ 同构,因之, ,有限维线性空间的结构可以认为是完全清楚了.
%
%\item 线性 空间是某一类事物从量的方面的一个抽象.我们认识客观事物, 固然要弄清它们单个的和总体的性质,但是更重要的是研究它们之间的各种各样的联系.
%
%\item 在线性空间中,事物之间的联系就反映为线 性空间的映射.线性空间 V 到自身的映射通常称为 V 的一个变 换.这一章中要讨论的线性变换就是最简单的,同时也可以认为是 最基本的一种变换, 正如线性函数是最简单的和最基本的函数一 样.线性变换是线性代数的一个主要研究对象.
%\end{itemize}


下面如果不特别声明,所考虑的都是某一固定的数域 $P$ 上的
线性空间.

\begin{defi}
线性空间 $V$ 的一个变换 称为线性变换, 如果对于 $V$ 中任意的元素 ${\alpha}, {\beta}$ 和数域 $P$ 中任意数 $k$,都有
\[
\begin{array}{c}
\mathscr{A}({\alpha}+{\beta})=\mathscr{A}({\alpha})+\mathscr{A}({\beta}) \\
\mathscr{A}({k} {\alpha})=k \mathscr{A}({\alpha})
\end{array}
\]
\end{defi}

\begin{rem}
\begin{itemize}
\item 一般用花体拉丁字母 $\mathscr{A}, \mathscr{B}, \cdots$ 或 $\sigma,\tau, \cdots$代表 $V$ 的变换
\item $\mathscr{A}({\alpha})$或 $\A \a$ 代表元素 ${\alpha}$ 在变换 $\A$ 下的像
\item 线性变换保持向量的加法与数量乘法
\end{itemize}
\end{rem}
\end{frame}

\begin{frame}
下面我们来看几个简单的例子,它们表明线性变换这个概念 是有丰富的内容的.
\begin{exa}[{旋转变换}]
平面上的向量构成实数域上的二维线性空间.把平面 围绕坐标原点按反时针方向旋转 $\theta$ 角,就是一个线性变换,我们 用 $\mathscr{I}_{\theta}$ 表示.如果平面上一个向量 ${\alpha}$ 在直角坐标系下的坐标是 $\left( x_1,y_1 \right)$, 那么像 $\mathscr{I}_{\theta}({\alpha})$ 的坐标, 即 ${\alpha}$ 旋转 $\theta$ 角之后的坐标 $\left(x_2, y_2\right)$ 是按
照公式
\[
\left(\begin{array}{l}
x_2 \\
y_2
\end{array}\right)=\left(\begin{array}{ll}
\cos \theta & -\sin \theta \\
\sin \theta & \cos \theta
\end{array}\right)
\left(\begin{array}{l}
x_1 \\
y_1
\end{array}\right)
\]
来计算的.同样地,空间中绕轴的旋转也是一个线性变换. 
\end{exa}
\end{frame}


\begin{frame}
	\begin{exa}[投影变换]
设 $\a$是几何空间中一固定的非零向量,把每个向量 $\zeta$变到它在 ${\alpha}$ 上的内射影的变换也是一个线性变换,以 $\Pi_{\alpha}$ 表示它.
用公式表示就是
\[
\Pi_{\a}(\zeta)=\frac{({\alpha}, \zeta)}{({\alpha}, {\alpha})} {\alpha}
\]
这里 $({\alpha}, {\zeta}),({\alpha}, {\alpha})$ 表示内积.
	\end{exa}



\end{frame}

\begin{frame}
\begin{exa}
	线性空间 V 中的\alert{恒等变换}或称\alert{单位变换}\, $\mathscr{E}$,即
	\[
	\E({\alpha})={\alpha} \quad({\alpha} \in V)
	\]
	以及\alert{零变换} $\mathscr{O}$, 即
	\[
	\O({\alpha})=\mathbf{0} \quad({\alpha} \in V)
	\]
	都是线性变换. 
\end{exa}

\begin{exa}
	设 $V$ 是数域 $P$ 上的线性空间, $k$ 是 $P$ 中某个数,定义 $V$ 的变换如下:
$${\alpha} \rightarrow k {\alpha}, \quad {\alpha} \in V$$
可以证明,这是一个线性变换,称为由数 $k$ 决定的\alert{数乘变换}, 可用 $\mathscr{K}$表示. 
当 $k=1$ 时, 我们便得恒等变换,  当 $k=0$ 时,便得零变换.
\end{exa}
\end{frame}


\begin{frame}
\begin{exa}[导数]
在线性空间 $P[x]$或者 $P[x]_{n}$ 中,求导数是一个线性
变换.这个变换通常用 $\mathscr{D}$代表,即
\[
\mathscr{D}(f(x))=f^{\, \,\prime}(x)
\]
\end{exa}

\begin{exa}[积分]
	定义在闭区间$[a,b]$上的全体连续函数组成实数域上 一线性空间,以 $C(a,b)$代表.在这个空间中,变换
	\[
	\mathscr{S}(f(x))=\int_{a}^{x} f(t) \mathrm{d} t   \qquad (\mbox{这里}\mathscr{S} \mbox{是花体} S)
	\] 
	是一个线性变换.
\end{exa}

\end{frame}





\begin{frame}


从定义推出线性变换的以下简单性质:

\begin{itemize}
	\item[1] 设$\A$是线桂空间$V$的一个线性变换, 则
	$$\A(\0)=\0, \quad \A(-\alpha)=-\A(\alpha).$$
\end{itemize}

\pf\\[-30pt]
\begin{align*}
\A(\0)& =\A(0\, \a) =0\,  \A(\a) =\0, \\
\A(-\alpha) & =A\left( (-1) \alpha\right)=-A(\alpha)=-A(\alpha).
\end{align*}

\end{frame}


\begin{frame}
\begin{itemize}
	\item[2] 线性变换保持线性组合与线性关系式不变.
	
	换句话说, 如果 ${\beta}$ 是 ${\alpha}_{1}, {\alpha}_{2}, \cdots, {\alpha},$ 的线性组合:
	\[
	{\beta}=k_{1} {\alpha}_{1}+k_{2} {\alpha}_{2}+\cdots+k_{r} {\alpha}_{r}
	\]
	那么$\A(\beta)$ 是 $\A({\alpha}_{1}), \A({\alpha}_{2}), \cdots, \A({\alpha})$ 的线性组合:
	\[
	\mathscr{A}({\beta})=k_{1} \mathscr{A}\left({\alpha}_{1}\right)+k_{2} \mathscr{A}\left({\alpha}_{2}\right)+\cdots+k_{r}  \mathscr{A}\left({\alpha}_{r}\right)
	\]
	又如果 ${\alpha}_{1}, {\alpha}_{2}, \cdots, {\alpha},$ 之间有一线性关系式
\[
k_{1} {\alpha}_{1}+k_{2} {\alpha}_{2}+\cdots+k_{r} {\alpha}_{r}=\0
\]
那么它们的像之间也有同样的关系
\[
k_{1}  \mathscr{A}\left({\alpha}_{1}\right)+k_{2}  \mathscr{A}\left({\alpha}_{2}\right)+\cdots+k_{r}  \mathscr{A}\left({\alpha}_{r}\right)={\0}
\]

	
\end{itemize}
\end{frame}


\begin{frame}


\begin{itemize}
	\item[3] 线性变换把线性相关的向量组变成线性相关的向量组.  

\end{itemize}


	注意: \alert{它的逆是不对的.}
	
	线性变换可能把线性无关的向 量组也变成线性相关的向量组.
	
	例如零变换就是这样.


\end{frame}
\section{线性变换的运算}
\begin{frame}{\S 2 线性变换的运算}{一、线性变换的乘积}
首先,线性空间的线性变换作为映射的特殊情形当然可以定
义乘法.设 $\mathscr{A}, \mathscr{B}$ 是线性空间 $V$ 的两个线性变换,定义它们的\alert{乘积} $\mathscr{A B}$ 为
\[
(\mathscr{A} \mathscr{B})({\alpha})=\mathscr{A}(\mathscr{B}({\alpha})) \quad({\alpha} \in V)
\]
\\[-10pt]
\begin{itemize}
	\item {线性变换的乘积也是线性变换.} 
	\[
	\begin{aligned}
	(\mathscr{A} \mathscr{B})({\alpha}+{\beta}) &=\mathscr{A}(\mathscr{B}({\alpha}+{\beta}))=\mathscr{A}(\mathscr{B}({\alpha})+\mathscr{B}({\beta})) \\
	&=\mathscr{A}(\mathscr{B}({\alpha}))+\mathscr{A}(\mathscr{P}({\beta})) \\
	&=(\A \mathscr{B})({\alpha})+(\mathscr{A} \mathscr{B})({\beta}) \\
	(\mathscr{A} \mathscr{B})(k {\alpha}) &=\mathscr{A}(\mathscr{B}(k {\alpha}))=\mathscr{A}(k \mathscr{B}({\alpha})) \\
	&=k \mathscr{A}(\mathscr{B}({\alpha}))=k(\mathscr{A} \mathscr{B})({\alpha})
	\end{aligned}
	\]
	这说明 $\A \mathscr{B}$ 是线性的.  
\end{itemize}

\end{frame}


\begin{frame}
\begin{itemize}
	\item
既然一般映射的乘法适合结合律, 线性变换的乘法当然也适 合\alert{结合律},即
\[
(\mathscr{A} \mathscr{B}) \mathscr{C}=\mathscr{A}(\mathscr{B} \mathscr{C})
\]
	\item 但线性变换的乘法一般是\alert{不可交换}的.例如, 在实数域 R 上的线 性空间 $R[x]$ 中,线性变换
\[
\begin{array}{c}
\mathscr{D}(f(x))=f^{\,\, \prime}(x) \\
\mathscr{S}(f(x))=\int_{0}^{x} f(t) \mathrm{d} t
\end{array}
\]
的乘积 $\mathscr{D}  \mathscr{S} =\mathscr{E},$ 但一般 $\mathscr{S} \mathscr{D} \neq \mathscr{E}$.

	\item 对于乘法, 单位变换 $\E$有特殊的地位.对于任意线性变换. 都有
\[
\E \A=\A \E=\A
\]
\end{itemize}
\end{frame}

\begin{frame}{二、线性变换的加法}
其次,对于线性变换还可以定义加法.设 $\mathscr{A}, \mathscr{B}$ 是线性空间 $V$ 的两个线性变换,定义它们的和 为
$$
(\mathscr{A}+\mathscr{B})({\alpha})=\mathscr{A}({\alpha})+\mathscr{B}({\alpha}) \quad({\alpha} \in V)
$$
\\[-10pt]
\begin{itemize}
	\item 线性变换的和还是线性变换. 
\begin{align*}
(\alpha+\mathscr{B})({\alpha}+{\beta}) &=\mathscr{A}({\alpha}+{\beta})+\mathscr{B}({\alpha}+{\beta}) \\
&=(\mathscr{A}({\alpha})+\mathscr{A}({\beta}))+(\mathscr{B}({\alpha})+\mathscr{B}({\beta})) \\
&=(\mathscr{A}({\alpha})+\mathscr{B}({\alpha}))+(\mathscr{A}({\beta})+\mathscr{B}({\beta})) \\
&=(\mathscr{A}+\mathscr{B})({\alpha})+(\mathscr{A}+\mathscr{B})({\beta}) \\
(\mathscr{A}+\mathscr{B})(k {\alpha}) &=\mathscr{A}(k {\alpha})+\mathscr{B}(k {\alpha}) \\
&=k \mathscr{A}({\alpha})+k \mathscr{B}({\alpha})=k(\mathscr{A}({\alpha})+\mathscr{P}({\alpha})) \\
&=k(\mathscr{A}+\mathscr{B})({\alpha})
\end{align*}
%这就说明$\mathscr{A}+\mathscr{B}$是线性变换. 
\end{itemize}
\end{frame}

\begin{frame}
\begin{itemize}
	\item 线性变换的加法适合结合律与交换律,即
	\begin{align*}
	\mathscr{A}+(\mathscr{B}+\mathscr{C}) &=(\mathscr{A}+\mathscr{B})+\mathscr{C} \\
	\mathscr{A}+\mathscr{B} &=\mathscr{B}+\mathscr{A}
	\end{align*}
	\item 对于加法, 零变换 $\O$ 有着特殊的地位.它与所有线性变换 $\A$ 的和仍等于 $\A$:
	\[
	\A+\O=\mathscr{A}
	\]
\end{itemize}
\end{frame}

\begin{frame}
对于每个线性变换 $-\A$,我们可以定义它的负变换:
\[
(-\mathscr{A})({\alpha})=-\mathscr{A}({\alpha}) \quad({\alpha} \in V)
\]
\begin{itemize}
	\item 负变换 $-\A$ 也是线性的,且
	\[
	\A+(-\A)= \O
	\]
	\item 线性变换的乘法对加法有左右分配律,即
	\[
	\begin{array}{l}
	\mathscr{A}(\mathscr{B}+\mathscr{C})=\mathscr{A} \mathscr{B}+\mathscr{A} \mathscr{C} \\
	(\mathscr{B}+\mathscr{C}) \mathscr{A}=\mathscr{B} \mathscr{A}+\mathscr{C} \mathscr{A}
	\end{array}
	\]


\end{itemize}
\end{frame}

\begin{frame}{三、线性变换的数量乘法}
%在上一节例 4 中我们看到,数域 $P$ 中每个数 $k$ 都决定一个数 乘变换$\mathscr{K}$.
利用线性变换的乘法, 可以定义数域$P$ 中的数与线性 变换的数置乘法为
$$
(k \A)(\alpha)=(\mathscr{K} \A)(\alpha)=\mathscr{K}(\A(\alpha))
$$\\[-10pt]
\begin{itemize}
	\item $k \mathscr{A}$ 还是线性变换. 
	\item 线性变换的数量乘法适合以 下的规律:
\begin{itemize}
\item 数乘结合律 $(k \ell) \A=k(\ell \A)$
\item 数乘分配律 $({k}+\ell) \A={k} {\A}+\ell {\A}$
\item 数乘分配律 $k(\mathscr{A}+\mathscr{B})=k \mathscr{A}+k \mathscr{B}$
\item 1的数乘 \quad \, $1 \A=\A$
\end{itemize}
\item 
由加法与数量乘法的性质可知,线性空间 $V$ 上全体线性变换.  

对于如上定义的加法与数量乘法,也构成数域 $P$ 上一个线性空间.  
\end{itemize}


\end{frame}



\begin{frame}{四、线性变换的逆变换}
${V}$ 的变换$\A$ 称为可逆的, 如果存在 $V$的变换$\B$, 使
\[
\A \B=\B \A=\E
\]
这时, 变换$\B$称为$\A$的逆变换, 记为$\A^{-1}$
可逆线性变换$\A$的逆变换. 
\begin{itemize}
\item 可逆线性变换$\A$的逆变换$\A^{-1}$也是线性变换
\end{itemize}
\begin{align*} 
\A^{-1}(k \alpha+\ell \beta) &= \A^{-1}\left(k\left(\A \A^{-1}\right)(\alpha)+\ell\left(\A \A^{-1}\right)(\beta)\right) \\ &=\A^{-1}\left(\A\left(k \A^{-1}(\alpha)\right)+\A\left( \ell A^{-1}(\beta)\right)\right) \\ 
&=\A^{-1}\left(\A\left(k \A^{-1}(\alpha)+\ell \A^{-1}(\beta)\right)\right) \\
 &=\left(\A^{-1} \A\right)\left(k \A^{-1}(\alpha)+\ell \A^{-1}(\beta)\right) \\
  &=k \A^{-1}(\alpha)+\ell \A^{-1}(\beta) 
  \end{align*}

\end{frame}


\begin{frame}{五、线性变换的多项式}
\small{
当 $n$ 个($n$ 是正整数)线性变换相乘时,我们就可以用
\[
\overbrace{\A \A \cdots\A}^{n}
\]
来表示, 称为 $\A$的 $n$ 次幂,简单地记作 $\A^n$.
此外,作为定义, 令
$$
\mathscr{A}^{0}=\mathscr{E}.
$$

根据线性变换幂的定义, 可以推出指数法则:
\[
\mathscr{A}^{m+n}=\mathscr{A}^{m} \cdot \mathscr{A}^{n},\left(\mathscr{A}^{m}\right)^{n}=\mathscr{A}^{m n} \quad(m, n \geqslant 0)
\]
当线性变换$\A$可逆时,定义 $\A$的负整数幂为
\[
\mathscr{A}^{-n}=\left(\mathscr{A}^{-1}\right)^{n} \quad(n \text { 是正整数 })
\]
%这时, 指数法则可以推广到负整数幂的情形. 

值得注意的是,线性变换乘积的指数法则不成立, 即一般说来\[
(\mathscr{A} \mathscr{B})^{n} \neq \mathscr{A}^{n} \mathscr{B}^{n}
\] }
\end{frame}


\begin{frame}


设
\[
f(x)=a_{m} x^{m}+a_{m-1} x^{m-1}+\cdots+a_{0}
\]
是 $P[x]$ 中一多项式 $, \mathscr{A}$ 是 $V$ 的一线性变换,我们定义
\[
f(\A)=a_{m} \mathscr{A}^{m}+a_{m-1} \mathscr{A}^{m-1}+\cdots+a_{0} \mathscr{E}
\]
于是, $f(\A)$ 是一线性变换,它称为线性变换$\A$的多项式. 

如果在 $P[x]$中
\[
h(x)=f(x)+g(x),  \quad  p(x)=f(x) g(x)
\]
那么
\[
h(\mathscr{\A})=f(\mathscr{A})+g(\mathscr{A}), \quad p(\mathscr{A})=f(\mathscr{A}) g(\mathscr{A})
\]
特别地, 
\[
f(\A) g(\A)=g(\mathscr{A}) f(\mathscr{A})
\]
即同一个线性变换的多项式的乘法是可交换的.  

%下面的例子表明,线性变换之间的一些关系可以通过线性变 换的运算表示出来. 
\end{frame}

\begin{frame}
\begin{exa}
在线性空间 $P[\lambda]_{n}$ 中,求微商是一个线性变换,用$\mathscr{D}$表示.显然有
\[
\mathscr{D}^{n}=\O
\]
其次,变数的平移
\[
f(\lambda) \rightarrow f(\lambda+a) \quad(a \in P)
\]
\[
\begin{aligned}
f(\lambda+a)=f(\lambda)+a f^{\,\,\prime}(\lambda)+\frac{a^{2}}{2 !} f^{\, \, \prime \prime}(\lambda)+\cdots
+\frac{a^{n-1}}{(n-1) !} f^{\,\, (n-1)}(\lambda)
\end{aligned}
\]
因之 $\mathscr{S}_{a}$ 实质上是 $\mathscr{D}$ 的多项式:
\[
\mathscr{S}_{a}=\mathscr{E}+a \mathscr{D}+\frac{a^{2}}{2 !} \mathscr{D}^{2}+\cdots+\frac{a^{n-1}}{(n-1) !} {\mathscr{D}^{n-1}}
\]
\end{exa} 
\end{frame}


\section{线性变换的矩阵}
\begin{frame}{\S 3 线性变换的矩阵}
设 $V$ 是数域 $P$ 上 $n$ 维线性空间 $, {\varepsilon}_{1}, {\varepsilon}_{2}, \cdots, {\varepsilon}_{n}$ 是 $V$ 的一组基,现在我们来建立线性变换与矩阵的关系. 
\begin{itemize}
	\item 
空间 V 中任一向量 $\xi$可以由基${\varepsilon}_{1}, {\varepsilon}_{2}, \cdots, {\varepsilon}_{n}$线性表出, 即
\[
\xi=x_{1} \varepsilon_{1}+x_{2} \varepsilon_{2}+\cdots+x_{n} \varepsilon_{n}
\]
其中系数是唯一确定的, 它们就是 $\xi$ 在这组基下的坐标. 

	\item  由于线 性变换保持线 性关系不变, 因而在 $\xi$ 的像 $\A\xi$ 与基的像 $\A{\varepsilon}_{1}, \mathscr{A} {\varepsilon}_{2}, \cdots, \mathscr{A} {\varepsilon}_{n}$ 之间也必然有相同的关系 
\[
\begin{aligned}
\mathscr{A} {\xi} &=\mathscr{A}\left(x_{1} {\varepsilon}_{1}+x_{2} {\varepsilon}_{2}+\cdots+x_{n} {\varepsilon}_{n}\right) \\
&=x_{1} \mathscr{A}\left({\varepsilon}_{1}\right)+x_{2} \mathscr{A}\left({\varepsilon}_{2}\right)+\cdots+x_{n} \mathscr{A}\left({\varepsilon}_{n}\right)
\end{aligned}
\]

	\item 上式表明,如果我们知道了基 ${\varepsilon}_{1}, {\varepsilon}_{2}, \cdots, {\varepsilon}_{n}$ 的像,那么线性空间中
任意一个向量 $\xi$ 的像也就知道了
\end{itemize}
\end{frame}

\begin{frame}{一、线性变换作用在基上}
1. 设 ${\varepsilon}_{1}, {\varepsilon}_{2}, \cdots, {\varepsilon}_{n}$ 是线性空间 $V$ 的一组基.如果线性变换
$\A$与 $\B$在这组基上的作用相同,即
$\mathscr{A} {\varepsilon}_{i}=\mathscr{B} {\varepsilon}_{i}, \quad i=1,2, \cdots, n$
那么 $\mathscr{A}=\mathscr{B}$.

\pf 对$V$中任意向量
$\xi = x_{1} {\varepsilon}_{1}+x_{2} {\varepsilon}_{2}+\cdots+x_{n} {\varepsilon}_{n}$, 有
\begin{align*} 
\mathscr{A} {\xi} 
&=\mathscr{A}\left(x_{1} {\varepsilon}_{1}+x_{2} {\varepsilon}_{2}+\cdots+x_{n} {\varepsilon}_{n}\right) \\ 
&=x_{1} \mathscr{A}{\varepsilon}_{1}+x_{2} \mathscr{A}{\varepsilon}_{2}+\cdots+x_{n} \mathscr{A}{\varepsilon}_{n} \\
&=x_{1} \mathscr{B} {\varepsilon}_{1}+x_{2} \mathscr{B}{\varepsilon}_{2}+\cdots+x_{n} \mathscr{B}{\varepsilon}_{n} \\
&=\mathscr{B}\left(x_{1} {\varepsilon}_{1}+x_{2} {\varepsilon}_{2}+\cdots+x_{n} {\varepsilon}_{n}\right) \\ 
&= \mathscr{B} {\xi}.
\end{align*}
因此, $\mathscr{A}=\mathscr{B}$. \qed

结论 1 的意义就是, 一个线性变换完全被它在一组基上的作 用所决定.
\end{frame}

\begin{frame}
下面我们进一步指出,基向量的像却完全可以是任意 的, 也就是说, 

2. 设 ${\varepsilon}_{1}, {\varepsilon}_{2}, \cdots, {\varepsilon}_{n}$ 是线性空间 $V$ 的一组基.对任意一组
向量 ${\alpha}_{1}, {\alpha}_{2}, \cdots, {\alpha}_{n}$ 一定有一个线性变换 $\A$ 使
\[
\A \varepsilon_{i}={\a}_{i}, \quad i=1, 2, \cdots, n
\]

\pf 
我们来作出所要的线性变换.设
\[
\xi=\sum_{i=1}^{n} x_{i} \varepsilon_{i}
\]
是线性空间 V 的任意一个向量,我们定义 V 的变换 $\A$为
\[
\mathscr{A} \xi=\sum_{i=1}^{n} x_{i} {\alpha}_{i}
\]
\end{frame}

\begin{frame}
下面来证明变换 $\A$是线性的. 
在 $V$ 中任取两个向量, 
\[
{\beta}=\sum_{i=1}^{n} b_{i} \, {\varepsilon}_{i}, \quad \gamma=\sum_{i=1}^{n} c_{i}\, {\varepsilon}_{i}
\]
有
\begin{align*}
{\beta}+\gamma =\sum_{i=1}^{n}\left(b_{i}+c_{i}\right) {\varepsilon}_{i}, \qquad 
k {\beta}  =\sum_{i=1}^{n} k b_{i} \, {\varepsilon}_{i},  \quad k \in P
\end{align*}
于是, 
\begin{align*}
\mathscr{A}({\beta}+{\gamma})& =\sum_{i=1}^{n}\left(b_{i}+c_{i}\right) {\alpha}_{i}\\
&=\sum_{i=1}^{n} b_{i} {\alpha}_{i}+\sum_{i=1}^{n} c_{i} {\alpha}_{i}=\mathscr{A} {\beta}+\A {\gamma} \\
\mathscr{A}({k} {\beta}) &=\sum_{i=1}^{n} (k b_{i}) {\alpha}_{i}=k \sum_{i=1}^{n} b_{i} {\alpha}_{i}=k   {\beta}
\end{align*}
因此, $\mathscr{A}$是线性变换. 
\end{frame}

\begin{frame}
再来证$\A$满足$\A \varepsilon_{i}={\a}_{i}, \quad i=1, 2, \cdots, n$.
因为
\[
{\varepsilon}_{i}=0 {\varepsilon}_{1}+\cdots+0 {\varepsilon}_{i-1}+1 {\varepsilon}_{i}+0 {\varepsilon}_{i+1}+\cdots+0 {\varepsilon}_{n}, \quad i=1,2, \cdots, n
\]
所以
\[
\A \varepsilon_{i}=0 {\alpha}_{1}+\cdots+0 {\alpha}_{i-1}+1 {\alpha}_{i}+0 {\alpha}_{i+1}+\cdots+0 {\alpha}_{n}={\alpha}_{i} \quad i=1,2, \cdots, n 
\]
证毕.\qed
综合以上两点,得 
\begin{thm}
设 ${\varepsilon}_{1}, {\varepsilon}_{2}, \cdots, {\varepsilon}_{n}$ 是线性空间 $V$ 的一组基, ${\a}_{1},$
${\alpha}_{2}, \cdots, {\alpha}_{n}$ 是 $V$ 中任意 $n$ 个向量, 存在唯一的线性变换 $\A$ 使
\[
\mathcal{A} {\varepsilon}_{i}={\alpha}_{i}, \quad i=1,2, \cdots, n.
\]

\end{thm}
\end{frame}


\begin{frame}{二、线性变换在一组基下的矩阵}
有了以上的讨论, 我们就可以来建立线性变换与矩阵的联系.

\begin{itemize}
	\item  设 ${\varepsilon}_{1}, {\varepsilon}_{2}, \cdots, {\varepsilon}_{n}$ 是数域 $P$ 上 $n$ 维线性空间 $V$ 的一组基, $\A$ 是$V$ 中的一个线性变换, 基向量的像可以被基线性表出:
	$$\left\{\begin{array}{cc}
	\A \varepsilon_{1} & =  a_{11} \varepsilon_{1}+a_{21} \varepsilon_{2}+\cdots+a_{n 1} \varepsilon_{n} \\
	\A \varepsilon_{2} & =  a_{12} \varepsilon_{1}+a_{22} \varepsilon_{2}+\cdots+a_{n 2} \varepsilon_{n} \\ 
	 &   \cdots \cdots \\ 
	\A \varepsilon_{n} & =  a_{1 n} \varepsilon_{1}+a_{2 n} \varepsilon_{2}+\cdots+a_{n n} \varepsilon_{n}
	\end{array}\right.$$
	\item 写成矩阵的形式
	\begin{align*}
	\left(\mathscr{A} \varepsilon_{1}, \mathscr{A} \varepsilon_{2}, \ldots, \mathscr{A} \varepsilon_{n}\right)
	=\left(\varepsilon_{1}, \varepsilon_{2}, \ldots, \varepsilon_{n}\right)
	\left(
	\begin{array}{cccc} 
	a_{11} & a_{12} & \cdots & a_{1 n} \\
	a_{21} & a_{22} & \cdots & a_{2 n} \\
	\vdots & \vdots & & \vdots \\
	a_{n 1} & a_{n 2} & \cdots & a_{n n}
	\end{array}
	\right)
	\end{align*}
\end{itemize}



\end{frame}




\begin{frame}
\begin{defi}矩阵
	\[
	{A}=\left(\begin{array}{cccc}
	a_{11} & a_{12} & \cdots & a_{1 n} \\
	a_{21} & a_{22} & \cdots & a_{2 n} \\
	\vdots & \vdots & & \vdots \\
	a_{n 1} & a_{n 2} & \cdots & a_{n n}
	\end{array}\right)
	\]
	称为 线性变换$\A$在基 ${\varepsilon}_{1}, {\varepsilon}_{2}, \cdots, {\varepsilon}_{n}$ 下的矩阵.
\end{defi}
记$\A \left(\varepsilon_{1}, \varepsilon_{2}, \dots, \varepsilon_{n}\right):=\left(\A \varepsilon_{1}, \A \varepsilon_{2}, \dots, \A \varepsilon_{n}\right)$. 于是, 
\begin{align}\label{eq-A}
\A  \left(\varepsilon_{1}, \varepsilon_{2}, \dots, \varepsilon_{n}\right)  =\left( \varepsilon_{1}, \varepsilon_{2}, \dots, \varepsilon_{n}\right) A.
\end{align}

\end{frame}


\begin{frame}
\begin{exa}
设$V$是数域$P$上$n$维线性空间, 则
\begin{itemize}
\item 恒等变换在任一组基下的矩阵是\underline{\qquad \qquad}
\item 零变换在任一组基下的矩阵是\underline{\qquad \qquad}
\item 由$k$决定的数乘变换在任一组基下的矩阵是\underline{\qquad \qquad}
\end{itemize}
\pause
\end{exa}
\begin{itemize}
\item $n$阶单位矩 阵 $E$
\item $n$阶 零矩阵${O}$
\item $n$阶数量矩阵 $k E$
\end{itemize}
\end{frame}

\begin{frame}
\begin{exa}
设$P^{\,3}$的线性变换$\A$为 $\A\left(x_{1}, x_{2}, x_{3}\right)=\left(x_{1}, x_{2}, x_{1}+x_{2}\right)$
取一组基 $\varepsilon_{1}=(1,0,0), \varepsilon_{2}=(0,1,0), \varepsilon_{3}=(0,0,1)$
则 
$$
\begin{array}{llll}
\mathscr{A} \varepsilon_{1} = &  \varepsilon_{1} & \quad& +\varepsilon_{3}\\
\mathscr{A} \varepsilon_{2} = & \quad & \varepsilon_{2}& +\varepsilon_{3}\\
\mathscr{A} \varepsilon_{3} = &  0& &
\end{array}
$$
所以, $\A$在基$\varepsilon_{1}, \varepsilon_{2}, \varepsilon_{3}$ 的矩阵为
\pause  $$\left(\begin{array}{lll}1 & 0 & 0 \\ 0 & 1 & 0 \\ 1 & 1 & 0\end{array}\right)$$
\end{exa}
\end{frame}

\begin{frame}
\begin{exa}
	在空间 $P[x]_n$ 中,线性变换
	\[
	\mathscr{D} f(x)=f^{\,\prime}(x)
	\]
	在基 $1, x, \frac{x^{2}}{2 !}, \ldots, \frac{x^{n-1}}{(n-1) !}$ 下的矩阵是\pause
	\[
	{D}=\left(\begin{array}{ccccc}
	0 & 1 & 0 & \cdots & 0 \\
	0 & 0 & 1 & \cdots & 0 \\
	\vdots & \vdots & \vdots & & \vdots \\
	0 & 0 & 0 & & 1 \\
	0 & 0 & 0 & \cdots & 0
	\end{array}\right)
	\]
\end{exa}
\end{frame}


\begin{frame}
\begin{exa}
设 ${\varepsilon}_{1}, {\varepsilon}_{2}, \cdots, {\varepsilon}_{m}$ 是 $n(n>m)$ 维线性空间 $V$ 的子空间
W 的一组基, 把它扩充为 V 的一组基 $\varepsilon_{1}, \varepsilon_{2}, \cdots, \varepsilon_{n}$.  指定线性变换
$\A$ 如下
\[
\left\{\begin{array}{l}
\mathscr{A} {\varepsilon}_{i}={\varepsilon}_{i}, \text { 当 } i=1,2, \cdots, m \\
\mathscr{A} {\varepsilon}_{i}={0}, \text { 当 } i=m+1, \cdots, n
\end{array}\right.
\]
如此确定的线性变换 $\A$称为对子空间 $W$ 的一个投影. 于是
\[
\mathscr{A}^{2}=\mathscr{A}
\]
投影 $\A$在基 ${\varepsilon}_{1}, {\varepsilon}_{2}, \cdots, {\varepsilon}_{n}$ 下的矩阵是
$$\left(\begin{array}{cc}
E_m & 0\\
0& 0
\end{array}\right)$$
\end{exa}
\end{frame}


\begin{frame}
\begin{itemize}
\item 在取定一组基之后, 我们就建立了由数域 $P$ 上的 $n$ 维线性空间 $V$ 的线性变换到数域 $P$ 上的 $n \times n$ 矩阵的一个映射. 

\item 前 面的结论 1 说明这个映射是单射, 结论 2 说明这个映射是满射. 换 句话说,我们在这二者之间建立了一个双射. 
\item 这个对应的重要性表 现在它保持运算, 即有
\end{itemize}

\begin{thm}
设 ${\varepsilon}_{1}, {\varepsilon}_{2}, \cdots, {\varepsilon}_{n}$ 是数域 $P$ 上 $n$ 维线性空间 $V$ 的一
组基,在这组基下,每个线性变换按公式\eqref{eq-A}对应一个 $n \times n$ 矩阵. 这个对应具有以下的性质:
\begin{enumerate}
	\item 线性变换的和对应于矩阵的和,
	\item 线性变换的乘积对应于矩阵的乘积,
	\item 线性变换的数量乘积对应于矩阵的数量乘积,
	\item 可逆的线性变换与可逆矩阵对应,且逆变换对应于逆矩
	阵.
\end{enumerate}
\end{thm}
\end{frame}

\begin{frame}
	\pf 设 $\mathscr{A}, \mathscr{B}$ 是两个线性变换,它们在基 ${\varepsilon}_{1}, {\varepsilon}_{2}, \cdots, {\varepsilon}_{n}$ 下的
	矩阵分别是 ${A}, {B},$ 即
	\[
	\begin{array}{l}
	\mathscr{A}\left({\varepsilon}_{1}, {\varepsilon}_{2}, \cdots, {\varepsilon}_{n}\right)=\left({\varepsilon}_{1}, {\varepsilon}_{2}, \cdots, {\varepsilon}_{n}\right) {A} \\
	\mathscr{B}\left({\varepsilon}_{1}, {\varepsilon}_{2}, \cdots, {\varepsilon}_{n}\right)=\left({\varepsilon}_{1}, {\varepsilon}_{2}, \cdots, {\varepsilon}_{n}\right) {B}
	\end{array}
	\]
	1) 由
	\[
	\begin{aligned}
	&(\mathscr{A}+\mathscr{B})\left({\varepsilon}_{1}, {\varepsilon}_{2}, \cdots, {\varepsilon}_{n}\right) \\
	=& \mathscr{A}\left({\varepsilon}_{1}, {\varepsilon}_{2}, \cdots, {\varepsilon}_{n}\right)+\mathscr{P}\left({\varepsilon}_{1}, {\varepsilon}_{2}, \cdots, {\varepsilon}_{n}\right) \\
	=&\left({\varepsilon}_{1}, {\varepsilon}_{2}, \cdots, {\varepsilon}_{n}\right) {A}+\left({\varepsilon}_{1}, {\varepsilon}_{2}, \cdots, {\varepsilon}_{n}\right) {B} \\
	=&\left({\varepsilon}_{1}, {\varepsilon}_{2}, \cdots, {\varepsilon}_{n}\right)({A}+{B})
	\end{aligned}
	\]
	可知,在基 ${\varepsilon}_{1}, {\varepsilon}_{2}, \cdots, {\varepsilon}_{n}$ 下,线性变换 $\mathscr{A}+\mathscr{B}$ 的矩阵是 ${A}+{B}$
	
	2) 相仿地, 
	\[
	\begin{aligned}
	&(\mathscr{A} \mathscr{B})\left({\varepsilon}_{1}, {\varepsilon}_{2}, \cdots, {\varepsilon}_{n}\right)=\mathscr{A}\left(\mathscr{B}\left({\varepsilon}_{1}, {\varepsilon}_{2}, \cdots, {\varepsilon}_{n}\right)\right) \\
	=& \mathscr{A}\left(\left({\varepsilon}_{1}, {\varepsilon}_{2}, \cdots, {\varepsilon}_{n}\right) {B}\right)=\left(\mathscr{A}\left({\varepsilon}_{1}, {\varepsilon}_{2}, \cdots, {\varepsilon}_{n}\right)\right) {B} \\
	=&\left({\varepsilon}_{1}, {\varepsilon}_{2}, \cdots, {\varepsilon}_{n}\right) {A} {B}
	\end{aligned}
	\]
	因此, 在基 ${\varepsilon}_{1}, {\varepsilon}_{2}, \cdots, {\varepsilon}_{n}$ 下,线性变换 $\mathscr{A B}$ 的矩阵是 ${A} {B}$
\end{frame}

\begin{frame}
	3) 因为
	\[
	\left(k \varepsilon_{1}, k \varepsilon_{2}, \cdots, k \varepsilon_{n}\right)=\left(\varepsilon_{1}, \varepsilon_{2}, \cdots, \varepsilon_{n}\right) k E
	\]
	所以数乘变换 从在任何一组基下都对应于数量矩阵 $k {E}$. 由此可 知,数量乘积 $k \A$ 对应于矩阵的数量乘积 $k {A}$
	
	4) 单位变换 $\E$对应于单位矩阵,因之等式
	\[
	\mathscr{A} \mathscr{B}=\mathscr{B} \mathscr{A}=\mathscr{E}
	\]
	与等式
	\[
	{A B}={B A}={E}
	\]
	相对应, 从而可逆线性变换与可逆矩阵对应, 而且逆变换与逆矩阵 对应. 
	\qed
\end{frame}


\begin{frame}
用 $L(V)$ 或者$End(V)$	表示 数域 $P$ 上 $n$ 维线性空间 $V$ 的全部线性变换组成	的集合.
	

\begin{prop}
$L(V)$ 对于线性变换的加法与数量乘法构成 $P$ 上一个线性 空间,与数域 $P$ 上 $n$ 级方阵构成的线性空间 $P^{n \times n}$ 同构. 
\end{prop}	

\end{frame}


\begin{frame}{三、向量的象的坐标}

\begin{thm}
设线性变换$\A$在基 ${\varepsilon}_{1}, {\varepsilon}_{2}, \cdots, {\varepsilon}_{n},$ 下的矩阵是 ${A}$,  $\xi$在基
${\varepsilon}_{1}, {\varepsilon}_{2}, \cdots, {\varepsilon}_{n}$ 下的坐标 $\left(x_{1}, x_{2}, \cdots, x_{n}\right)$,
则
向量$\A \xi$在基
${\varepsilon}_{1}, {\varepsilon}_{2}, \cdots, {\varepsilon}_{n}$ 下的坐标 $\left(y_{1}, y_{2}, \cdots, y_{n}\right)$ 可以按公式
\[
\left(\begin{array}{c}
y_{1} \\
y_{2} \\
\vdots \\
y_{n}
\end{array}\right)=A\left(\begin{array}{c}
x_{1} \\
x_{2} \\
\vdots \\
x_{n}
\end{array}\right)
\]
计算. 
\end{thm}
\end{frame}


\begin{frame}
\pf 由假设
\[
\xi=\left(\varepsilon_{1}, \varepsilon_{2}, \cdots, \varepsilon_{n}\right)\left(\begin{array}{c}
x_{1} \\
x_{2} \\
\vdots \\
x_{n}
\end{array}\right)
\]
于是
\begin{align*}
\A \xi & =\left(\A \varepsilon_{1}, \A \varepsilon_{2}, \cdots, \A \varepsilon_{n}\right)\left(\begin{array}{c}
x_{1} \\
x_{2} \\
\vdots \\
x_{n}
\end{array}\right) \\
& =\left(\varepsilon_{1}, \varepsilon_{2}, \cdots, \varepsilon_{n}\right) A\left(\begin{array}{c}
x_{1} \\
x_{2} \\
\vdots \\
x_{n}
\end{array}\right)
\end{align*}

\end{frame}


\begin{frame}
另一方面,由假设
\[
d \xi=\left(\varepsilon_{1}, \varepsilon_{2}, \cdots, \varepsilon_{n}\right)\left(\begin{array}{c}
y_{1} \\
y_{2} \\
\vdots \\
y_{n}
\end{array}\right)
\]
由于 ${\varepsilon}_{1}, {\varepsilon}_{2}, \cdots, {\varepsilon}_{n}$ 线性无关,所以
\[
\left(\begin{array}{c}
y_{1} \\
y_{2} \\
\vdots \\
y_{n}
\end{array}\right)=A\left(\begin{array}{c}
x_{1} \\
x_{2} \\
\vdots \\
x_{n}
\end{array}\right) .
\]
\qed

\end{frame}

\begin{frame}{四、线性变换在不同基下的矩阵}
\begin{itemize}
	\item 线性变换的矩阵是与空间中一组基联系在一起的.
	\item 一般说来,  随着基的改变,同一个线性变换就有不同的矩阵.
	\item 为了利用矩阵来 研究线性变换,我们有必要弄清楚线性变换的矩阵是如何随着基的改变而改变的. 
\end{itemize}

\begin{thm}
设线性空间 $V$ 中线性变换 $\A$在两组基
\begin{align}
	{\varepsilon}_{1}, {\varepsilon}_{2}, \cdots, {\varepsilon}_{n}  \label{b-1} \tag{I}\\
	{\eta}_{1}, {\eta}_{2}, \cdots, {\eta}_{n} \label{b-2} \tag{II}
\end{align}
下的矩阵分别为$A$和$B$, 从基\eqref{b-1}到\eqref{b-2}的过渡矩阵是 $X$, 于是 ${B}={X}^{-1} {A X}$
\end{thm}
\end{frame}

\begin{frame}
\pf 已知
\[
\begin{array}{c}
\left(\mathscr{A} {\varepsilon}_{1}, \A {\varepsilon}_{2}, \cdots, \mathscr{A} {\varepsilon}_{n}\right)=\left({\varepsilon}_{1}, {\varepsilon}_{2}, \cdots, {\varepsilon}_{n}\right) {A} \\
\left(\A{\eta}_{1}, \A {\eta}_{2}, \cdots, \A {\eta}_{n}\right)=\left({\eta}_{1}, {\eta}_{2}, \cdots, {\eta}_{n}\right) {B} \\
\left({\eta}_{1}, {\eta}_{2}, \cdots, {\eta}_{n}\right)=\left({\varepsilon}_{1}, {\varepsilon}_{2}, \cdots, {\varepsilon}_{n}\right) {X}
\end{array}
\]
于是
$$
\begin{aligned} 
& \left( \mathscr{A} \eta_{1}, \mathscr{A} \eta_{2}, \ldots, \mathscr{A} \eta_{n}\right)
=   \mathscr{A} \left({\eta}_{1}, {\eta}_{2}, \cdots, {\eta}_{n}\right)  \\
= & \, \A\left(\left( \varepsilon_{1}, \varepsilon_{2}, \ldots, \varepsilon_{n}\right) X \right) 
=    \left( \A \left(\varepsilon_{1}, \varepsilon_{2}, \ldots, \varepsilon_{n}\right)\right) X \\
=  &  \left( \A \varepsilon_{1}, \A \varepsilon_{2}, \ldots, \A \varepsilon_{n}\right) X
=   \left(\varepsilon_{1}, \varepsilon_{2}, \ldots, \varepsilon_{n}\right) A X \\
= &     \left(\eta_{1}, \eta_{2}, \ldots, \eta_{n}\right) X^{-1} A X \end{aligned}
 $$
由此即得
\[
{B}={X}^{-1} {A} {X}.
\]
\qed
定理 4 告诉我们,同一个线性变换 \&在不同基下的矩阵之间 的关系.这个基本关系在以后的讨论中是重要的.


\end{frame}
 
 
\begin{frame}
现在,我们对于 矩阵引进相应的定义. 
\begin{defi}
 设 A,B 为数域 P上两个 n 级矩阵,如果可以找到 数域 $P$ 上的 $n$ 级可逆矩阵 X ,使得 ${B}={X}^{-1} {A X}$,就说 ${A}$ 相似手
${B},$ 记作 ${A} \sim {B}$
\end{defi}
相似是矩阵之间的一种关系, 这种关系具有下面三个性质:
\begin{enumerate}
	\item 反身性 : $A \sim A$ 这是因为 ${A}={E}^{-1} {A} {E}$
	\item 对称性:如果 ${A} \sim {B},$ 那么 ${B} \sim {A}$.
	
	如果 ${A} \sim {B}$,那么有 ${X}$ 使 ${B}={X}^{-1} {A} {X}$.
	
	令 ${Y}={X}^{-1}$,就有 ${A}=$
	${X B X}^{-1} = {Y}^{-1} {B Y},$ 所以 ${B} \sim {A}$
	\item 传递性:如果 ${A} \sim {B}, {B} \sim {C},$ 那么 ${A} \sim {C}$.
	
	已知有 ${X}, {Y}$ 使 ${B}={X}^{-1} {A} {X}, {C}={Y}^{-1} {B} {Y}$.
	
	令 ${Z}={X} {Y},$ 就有
	${C}={Y}^{-1} {X}^{-1} {A} {X} {Y}={Z}^{-1} {A} {Z}$. 
\end{enumerate}

\end{frame}

\begin{frame}
因之
有了矩阵相似的概念之后,定理 4 可以补充成: 
\begin{thm}
\begin{itemize}
	\item 线性变换在不同基下所对应的矩阵是相似的; 反过
	来,
	\item 如果两个矩阵相似,那么它们可以看作同一个线性变换在两组 基下所对应的矩阵. 
\end{itemize}
\end{thm}

\pf  前一部分已经为定理 4 证明.

现在证明后一部分.设 级矩阵 A 和 B 相似. A 可以看做是 $n$ 维线性空间 V 中一个线性 变换 $\A$ 在基 ${\varepsilon}_{1}, {\varepsilon}_{2}, \cdots, {\varepsilon}_{n}$ 下的矩阵.因为 ${B}={X}^{-1} {A} {X},$ 令
\[
\left({\eta}_{1}, {\eta}_{2}, \cdots, {\eta}_{n}\right)=\left({\varepsilon}_{1}, {\varepsilon}_{2}, \cdots, {\varepsilon}_{n}\right) {X}
\]
因为$X$可逆, 所以 ${\eta}_{1}, {\eta}_{2}, \cdots, {\eta}_{n}$ 也是一组基, $\A$ 在这组基下的矩阵就是 ${B}$. \qed
\end{frame}


\begin{frame}
矩阵的相似对于运算有下面的性质.  

\begin{itemize}
	\item 如果 ${B}_{1}={X}^{-1} {A}_{1} {X}, {B}_{2}={X}^{-1} {A}_{2} {X}$, 那么
\begin{align*}
	{B}_{1}+{B}_{2} & ={X}^{-1}\left({A}_{1}+{A}_{2}\right) {X}, \\
	{B}_{1} {B}_{2} & ={X}^{-1}\left({A}_{1} {A}_{2}\right) {X}
\end{align*}
	\item 如果 ${B}={X}^{-1} {A} {X},$ 且 $f(x)$ 是数域 $P$ 上一多项式,那么
	\[
	f({B})={X}^{-1} f({A}) {X}
	\]
	
\end{itemize}

\end{frame}

\begin{frame}
利用矩阵相似的这个性质可以简化矩
阵的计算. 
\begin{exa}
设 $V$ 是数域 $P$ 上一个二维线性空间, ${\varepsilon}_{1}, {\varepsilon}_{2}$ 是一组基,
线性变换 $\A$在 ${\varepsilon}_{1}, {\varepsilon}_{2}$ 下的矩阵是
\[
\left(\begin{array}{rr}
2 & 1 \\
-1 & 0
\end{array}\right)
\]
现在来计算 $\A$在 $V$ 的另一组基 ${\eta}_{1}, {\eta}_{2}$ 下的矩阵,这里
\[
\left({\eta}_{1}, {\eta}_{2}\right)=\left({\varepsilon}_{1}, {\varepsilon}_{2}\right)\left(\begin{array}{rr}
1 & -1 \\
-1 & 2
\end{array}\right)
\]
由定理 4, $\A$在 ${\eta}_{1}, {\eta}_{2}$ 下的矩阵为
\[
\begin{array}{l}
\left(\begin{array}{rr}
1 & -1 \\
-1 & 2
\end{array}\right)^{-1}\left(\begin{array}{rr}
2 & 1 \\
-1 & 0
\end{array}\right)\left(\begin{array}{rr}
1 & -1 \\
-1 & 2
\end{array}\right) =\left(\begin{array}{rr}
1 & 1 \\
0 & 1
\end{array}\right)
\end{array}
\]
\end{exa}
\end{frame}

\begin{frame}
归纳可知, 
\[
\left(\begin{array}{ll}
1 & 1 \\
0 & 1
\end{array}\right)^{k}=\left(\begin{array}{ll}
1 & k \\
0 & 1
\end{array}\right)
\]
再利用上面得到的关系
\[
\left(\begin{array}{rr}
1 & -1 \\
-1 & 2
\end{array}\right)^{-1}\left(\begin{array}{rr}
2 & 1 \\
-1 & 0
\end{array}\right)\left(\begin{array}{rr}
1 & -1 \\
-1 & 2
\end{array}\right)=\left(\begin{array}{ll}
1 & 1 \\
0 & 1
\end{array}\right)
\]
即
\[
\left(\begin{array}{rr}
2 & 1 \\
-1 & 0
\end{array}\right)=\left(\begin{array}{rr}
1 & -1 \\
-1 & 2
\end{array}\right)\left(\begin{array}{rr}
1 & 1 \\
0 & 1
\end{array}\right)\left(\begin{array}{rr}
1 & -1 \\
-1 & 2
\end{array}\right)^{-1}
\]
\end{frame}

\begin{frame}


我们可以得到
\[
\begin{aligned}
\left(\begin{array}{rr}
2 & 1 \\
-1 & 0
\end{array}\right)^{k} &=\left(\begin{array}{rr}
1 & -1 \\
-1 & 2
\end{array}\right)\left(\begin{array}{rr}
1 & 1 \\
0 & 1
\end{array}\right)^{k}\left(\begin{array}{rr}
1 & -1 \\
-1 & 2
\end{array}\right)^{-1} \\
&=\left(\begin{array}{rr}
1 & -1 \\
-1 & 2
\end{array}\right)\left(\begin{array}{rr}
1 & k \\
0 & 1
\end{array}\right)\left(\begin{array}{rr}
2 & 1 \\
1 & 1
\end{array}\right) \\
&=\left( \begin{array}{cc}
k+1 & k \\
-k & -k+1
\end{array}\right).
\end{aligned}
\]

\end{frame}

\begin{frame}
\begin{exa}
	取 $P^{3}$ 的线性变换 $\sigma(a, b, c)=(2 a-b, b+c, a)$
\begin{enumerate}

\item 求 $\sigma$ 在基 $\varepsilon_{1}=(1,0,0), \varepsilon_{2}=(0,1,0), \varepsilon_{3}=(0,0,1)$ 下的矩阵;
\item 求 $\sigma$ 在基 $\eta_{1}=(1,0,0), \eta_{2}=(1,1,0), \eta_{3}=(1,1,1)$ 下的矩阵;
\item 求向量 $\alpha=(1,2,3)$ 的像 $\sigma \alpha$ 分别在基 $\varepsilon_{1}, \varepsilon_{2}, \varepsilon_{3}$ 和 $\eta_{1}, \eta_{2}, \eta_{3}$ 下的坐标.	
\end{enumerate}
\end{exa}
\end{frame}

\begin{frame}

\sol~ 

\small{(1) 
$\sigma \varepsilon_{1}
=\sigma(1,0,0)=(2,0,1)
=\left(\varepsilon_{1}, \varepsilon_{2}, \varepsilon_{3}\right)
\left(\begin{array}{l}2 \\ 0 \\ 1\end{array}\right)$, 

~~~~
$\sigma \varepsilon_{2}
=\left(\varepsilon_{1}, \varepsilon_{2}, \varepsilon_{3}\right)
\left(\begin{array}{c}-1 \\ 1 \\ 0\end{array}\right), \quad
\sigma \varepsilon_{3}
=\left(\varepsilon_{1}, \varepsilon_{2}, \varepsilon_{3}\right)\left(\begin{array}{l}0 \\ 1 \\ 0\end{array}\right),$

则 $$\sigma\left(\varepsilon_{1}, \varepsilon_{2}, \varepsilon_{3}\right)=\left(\varepsilon_{1}, \varepsilon_{2}, \varepsilon_{3}\right)A, \quad \mbox{其中 }A=\left(\begin{array}{ccc}2 & -1 & 0 \\ 0 & 1 & 1 \\ 1 & 0 & 0\end{array}\right)$$


(2) $\left(\eta_{1}, \eta_{2}, \eta_{3}\right)=\left(\varepsilon_{1}, \varepsilon_{2}, \varepsilon_{3}\right)C$, \quad 其中$C=\left(\begin{array}{ccc}1 & 1 & 1 \\ 0 & 1 & 1 \\ 0 & 0 & 1\end{array}\right)_{}$,则 $\sigma$ 在基 $\eta_{1}, \eta_{2}, \eta_{3}$ 下的矩阵为 $C^{-1} A C=
\left(\begin{array}{ccc}2 & 0 & -1 \\ -1 & 0 & 1 \\ 1 & 1 & 1\end{array}\right)$
}
\end{frame}

\begin{frame}
	
(3) $\alpha=\left(\varepsilon_{1}, \varepsilon_{2}, \varepsilon_{3}\right)\left(\begin{array}{l}1 \\ 2 \\ 3\end{array}\right)$, 

$\sigma \alpha=\sigma\left(\varepsilon_{1}, \varepsilon_{2}, \varepsilon_{3}\right)\left(\begin{array}{l}1 \\ 2 \\ 3\end{array}\right)=\left(\varepsilon_{1}, \varepsilon_{2}, \varepsilon_{3}\right) A\left(\begin{array}{l}1 \\ 2 \\ 3\end{array}\right)
=\left(\varepsilon_{1}, \varepsilon_{2}, \varepsilon_{3}\right) \left(\begin{array}{l}0 \\ 5 \\ 1\end{array}\right)
$

$
\sigma \alpha
=\left(\eta_{1}, \eta_{2}, \eta_{3}\right) C^{-1} 
	\left(\begin{array}{l}0 \\ 5 \\ 1	\end{array}\right)
= \left(\eta_{1}, \eta_{2}, \eta_{3}\right)
	\left(\begin{array}{c} -5 \\ 4 \\ 1	\end{array}\right)
$
\end{frame}

\section{特征值与特征向量}
\begin{frame}{\S 4 特征值与特征向量}
\begin{itemize}
\item 我们知道,在有限维线性空间中,取了一组基之后,线性变换 就可以用矩阵来表示.

\item 为了利用矩阵来研究线性变换,对于每个给 定的线性变换,我们希望能找到一组基使得它的矩阵具有最简单 的形式.

\item 从现在开始,我们主要地就来讨论,在适当的选择基之后,  一个线性变换的矩阵可以化成什么样的简单形式.
\end{itemize}
\end{frame}

\begin{frame}
%$\begin{aligned} {P}^{-1} {A P} &=\Lambda \\ \text { 其中 } \Lambda &=\left(\begin{array}{ccc}\lambda_{1} & \\ & \lambda_{2} \\ & & \ddots \\ & & \lambda_{n}\end{array}\right) \\ &=\operatorname{diag}\left(\lambda_{1}, \lambda_{2}, \cdots, \lambda_{n}\right) \end{aligned}$


设 $A$ 是 $n$ 阶方阵, ${P}$ 为 ${n}$ 阶可逆阵, 
$${P}^{-1} {A P}=\Lambda$$
其中
$\Lambda =\operatorname{diag}\left\{\lambda_{1}, \lambda_{2}, \cdots, \lambda_{n}\right\}$

令$P=\left(p_{1}, p_{2}, \cdots, p_{n}\right)$
\begin{align*}
& {P}^{-1} {A} P=\Lambda \\
\Leftrightarrow & 	A \left(p_{1}, p_{2}, \cdots, p_{n}\right)
=\left(p_{1}, p_{2}, \cdots, p_{n}\right) 
\operatorname{diag}\left\{\lambda_{1}, \lambda_{2}, \cdots, \lambda_{n}\right\}\\
\Leftrightarrow	&	\left(A p_{1}, A p_{2}, \cdots, A p_{n}\right)	
=	\left(\lambda_{1} p_{1}, \lambda_{2} p_{2}, \cdots, \lambda_{n} p_{n}\right)\\
\Leftrightarrow &  A p_{i}=\lambda_{i} p_{i}, i=1,2, \cdots, n
\end{align*}
此过程的逆推在最后一步要求矩阵 P是可逆的。
\end{frame}

\begin{frame}
为了这个目的,  先介绍特征值和特征向量的概念,它们对于线性变换的研究具有 基本的重要性.  
\begin{defi}
设 $\A$ 是数域 $P$ 上线性空间 $V$ 的一个线性变换,如果 对于数域 $P$ 中一数 $\lambda_{0},$ 存在一个非零向量 ${\xi}$,使得
\[
\A \xi=\lambda_{0} \xi
\]
个特征向量. 
\end{defi}
从几何上来看,特征向量的方向经过线性变换后,保持在同一 条直线上,这时或者方向不变( $\lambda_{0}>0$ ) ,或者方向相反 $\left(\lambda_{0}<0\right),$ 至
于 $\lambda_{0}=0$ 时,特征向量就被线性变换变成 $\mathbf{0}$
\end{frame}



\begin{frame}
如果 $\xi$ 是线性变换$\A$的属于特征值$\lambda_0$的特征向量,那么 的任何一个非零倍数$k\xi$ 也是 $\A$的属于 $\lambda_{0}$ 的特征向量. 因为$\A \xi=\lambda_{0} \xi$, 所以
\[
\mathscr{A}(k \xi)=\lambda_{0}(k \xi)
\]
这说明特征向量不是被特征值所唯一确定定的.相反,特征值却是被 特征向量所唯一确定的,因为,一个特征向量只能属于一个特征值.
\end{frame}



\begin{frame}
现在来给出寻找特征值和特征向量的方法, 设 $V$ 是数域 $P$ 上 $n$ 维线性空间, ${\varepsilon}_{1}, {\varepsilon}_{2}, \cdots, {\varepsilon}_{n}$ 是它的一组基,线性变换 $\A$ 在这组基
下的矩 阵是 $A$. 设 $\lambda$是特征值, 它的一个特征向量 ${\xi}$ 在 ${\varepsilon}_{1}$ ${\varepsilon}_{2}, \cdots, {\varepsilon}_{n}$ 下的坐标是 $x_{01}, x_{02}, \cdots, x_{0 n}$, 则 $\A \xi$的坐标是
\[
A\left(\begin{array}{c}
x_{01} \\
x_{02} \\
\vdots \\
x_{0 n}
\end{array}\right)
\]


$\lambda_{0} \xi$ 的坐标是
\[
\lambda_{0}\left(\begin{array}{l}
x_{01} \\
x_{02} \\
\vdots \\
x_{0 n}
\end{array}\right)
\]

\end{frame}



\begin{frame}
因此, $\A \xi=\lambda_{0} \xi$, 相当于坐标之间的等式
\[
{A}\left(\begin{array}{c}
x_{01} \\
x_{02} \\
\vdots \\
x_{0 n}
\end{array}\right)=\lambda_{0}\left(\begin{array}{c}
x_{01} \\
x_{02} \\
\vdots \\
x_{0 n}
\end{array}\right)
\]
或
\[
\left(\lambda_{0} {E}-{A}\right)\left(\begin{array}{c}
x_{01} \\
x_{02} \\
\vdots \\
x_{0 n}
\end{array}\right)=\mathbf{0} \tag{*} \label{eq-*}
\]
\end{frame}


\begin{frame}
这说明特征向量 $\xi$ 的坐标$\left(x_{01}, x_{02}, \cdots, x_{0 n}\right)$ 满足齐次方程组
即
\[
\left\{\begin{array}{rrrrr}
\left(\lambda_{0}-a_{11}\right) x_{1}&-a_{12} x_{2}&-\quad \cdots& -a_{1 n} x_{n}&=0 \\
-a_{21} x_{1}& +\left(\lambda_{0}-a_{22}\right) x_{2}& - \quad \cdots& -a_{2 n} x_{n}&=0 \\
\cdots& \cdots &\cdots &\quad \cdots  & \\
-a_{n 1} x_{1} & -a_{n 2} x_{2} & - \quad \cdots & +\left(\lambda_{0}-a_{n n}\right) x_{n}& =0
\end{array}\right.
\]
由于 $\xi\neq \mathbf{0}$, 所以它的坐标  $x_{01}, x_{02}, \cdots, x_{0 n}$ 不全为零,即齐次方程 组有非零解.

我们知道, 齐次线性方程组$(\lambda E-A)X=\mathbf{0}$有非零解的充分必要 条件是 它的系数行列式为零,即
\[
\left|\lambda_{0} {E}-{A}\right|=\left|\begin{array}{cccc}
\lambda_{0}-a_{11} & -a_{12} &  \cdots & -a_{1 n} \\
-a_{21} & \lambda_{0}-a_{22} &  \cdots & -a_{2 n} \\
\vdots & \vdots & & \vdots \\
-a_{n 1} & -a_{n 2} & \cdots & \lambda_{0}-a_{n n}
\end{array}\right|=0
\]
\end{frame}


\begin{frame}
我们引入以下的定义. 
\begin{defi}
$\lambda {E}-{A}$ 的行列式
\[
|\lambda {E}-{A}|=\left|\begin{array}{cccc}
\lambda-a_{11} & -a_{12} &  \cdots &-a_{1 n} \\
-a_{21} & \lambda-a_{22} & \cdots & -a_{2 n} \\
\vdots & \vdots & & \vdots \\
-a_{n 1} & -a_{n 2} & \cdots & \lambda-a_{n n}
\end{array}\right|
\]
称为 $A$ 的特征多项式,这是数域 $P$ 上的一个 $n$ 次多项式.
\end{defi}


\end{frame}

\begin{frame}
\begin{itemize}
\item 上面的分析说明, 如果 $\lambda_0$是线 性变换 $\A$的特征值,那么 $\lambda_0$
一定是矩阵 A 的特征多项式的一个根;
\item 反过来,如果 $\lambda_{0}$ 是知阵 ${A}$ 的特征多项式在数域 P 中的一个根,即  $|\lambda_{0} {E}-{A} |=0,$ 那么齐 线性方程组$(\lambda E-A)X=\mathbf{0}$就有非零解. 
\item 这时, 如果$\left( x_{01} ; x_{02}, \cdots, x_{0 n}\right)$ 是方程
组$(\lambda E-A)X=\mathbf{0}$的一个非零解, 那么非零向量
\[
\xi=x_{01} \varepsilon_{1}+x_{02} \varepsilon_{2}+\cdots+x_{0 n} \varepsilon_{n}
\]
$\lambda_{0}$ 的一个特征向量. 
\end{itemize}





\end{frame}

\begin{frame}
求一个线性变换 $\A$的特征值与特征向量的方法可以
分成以下几步 :
\begin{enumerate}
	\item 在线性空间 $V$ 中取一组基 ${\varepsilon}_{1}, {\varepsilon}_{2}, \cdots, {\varepsilon}_{n},$ 写出 $\mathscr{A}$ 在这组基下的矩阵 $A$;
	\item 求出 $\A$ 的特征多项式| $\lambda {E}-{A} |$ 在数域 $P$ 中全部的根, 它们也就是线性变换 $\A$ 的全部特征值. 
	\item 把所求得的特征值逐个地代入方程组$(\lambda E-A)X=\mathbf{0}$,对于每一个特 征值,解方程组$(\lambda E-A)X=\mathbf{0}$,求出一组基础解系, 它们就是属于这个特征值 的几个线性无关的特征向量在基 ${\varepsilon}_{1}, {\varepsilon}_{2}, \cdots, {\varepsilon}_{n}$ 下的坐标,这样,我
	们也就求出了属于每个特征值的全部线性无关的特征向量. 
\end{enumerate}

\end{frame}

\begin{frame}{矩阵的特征值与特征向量}
\begin{defi}
\begin{itemize}
	\item 矩阵 $A$ 的特征多项式的根称为$A$ 的特征值,
	\item 相应的线性方程组$(\lambda E-A)X=\mathbf{0}$的解称为$A$的属于这个特征值的特征向量.  
\end{itemize}

\end{defi}
\end{frame}

\begin{frame}
\begin{exa}
在 $n$ 维线性空间中,数乘变换$\mathscr{K}$ 在任意一组基下的矩 阵都且 $kE$, 它的特征多项式是
\[
|\lambda {E}-k {E}|=(\lambda-k)^{n}
\]
因此,数乘变换 从的特征值只有 $k$. 由定义可知,每个非零向量都 是属于数乘变换 $\mathscr{K}$ 的特征向量.
\end{exa}
\end{frame}



\begin{frame}
%\begin{exa}
%设线性变换$\A$在基 ${\varepsilon}_{1}, {\varepsilon}_{2}, {\varepsilon}_{3}$ 下的矩阵是
%\[
%A=\left(\begin{array}{lll}
%1 & 2 & 2 \\
%2 & 1 & 2 \\
%2 & 2 & 1
%\end{array}\right)
%\]
%求 $\A$ 的特征值与特征向量. 
%\end{exa}

\begin{exa}
设 $A=\left(\begin{array}{ccc}1 & 2 & 2 \\ 2 & 1 & 2 \\ 2 & 2 & 1\end{array}\right)$,求 $A$ 的特征值与对应的特征向量.
\end{exa}

\sol
1~ 由矩阵 $A$ 的特征方程,求出特征值。

$$|\lambda E-A|=\left|\begin{array}{ccc}\lambda-1 & -2 & -2 \\ -2 & \lambda-1 & -2 \\ -2 & -2 & \lambda-1\end{array}\right|=(\lambda+1)^{2}(\lambda-5),$$
得
$$\lambda_{1}=\lambda_{2}=-1, \lambda_{3}=5.$$

\end{frame}


\begin{frame}
2、把每个特征值$\lambda$代入线性方程组 $(A-\lambda E) X=0$
求出基础解系。
\begin{itemize}
\item 再求属于 $\lambda_{1}=\lambda_{2}=-1$ 的特征向量,即求方程组 $(-E-A) X=0$ 的解.

$$\left\{\begin{array}{l}-2 x_{1}-2 x_{2}-2 x_{3}=0 \\ -2 x_{1}-2 x_{2}-2 x_{3}=0,\end{array}\right.$$
即 $$x_{1}+x_{2}+x_{3}=0,$$

得基础解系为 
$X_{1}=(1,-1,0)'$,
$X_{2}=(1,0,-1)'$,

$X_{1}, X_{2}$ 就是属于$\lambda_{1}=-1$ 的两个线性无关的特征向量,

属于 $\lambda_{1}=-1$ 的全部特征向量为 $k_{1} X_{1}+k_{2} X_{2},$ 其中 $k_{1}, k_{2}$ 不全为零. 
\end{itemize}
\end{frame}
\begin{frame}

\begin{itemize}
\item 最后求属于 $\lambda_{3}=5$ 的特征向量,即求方程组 $(5 E-A) X=0$ 的解.

$$\left\{\begin{aligned} 4 x_{1}-2 x_{2}-2 x_{3} &=0 \\-2 x_{1}+4 x_{2}-2 x_{3} &=0
\\-2 x_{1}-2 x_{2}+4 x_{3} &=0 \end{aligned}\right.,$$

基础解系为  $X_{3}=(1,1,1)'$

 $X_{3}$ 就是属于 $\lambda_{3}=5$ 的线性无关的特征向量,

属于 $\lambda_{3}=5$ 的全部特征向量为 $k_{3} X_{3},$ 其中 $k_{3} \neq 0$.
\end{itemize}

\begin{rem}
\begin{itemize}
\item 审题时注意:题目要的求是矩阵的特征值与特征向量,还是线性变换的特征值与特征向量.

\item 两者的特征值都是一样的,但线性变换的的特征向量是$V$中的向量,不是坐标;而矩阵的特征向量是$\mathbb{R}^n$中的向量.
\end{itemize}
\end{rem}
\end{frame}


\begin{frame}
\begin{exa}
求矩阵 $A=\left(\begin{array}{ccc}-1 & 1 & 0 \\ -4 & 3 & 0 \\ 1 & 0 & 2\end{array}\right)$ 的特征值和特征向量.
\end{exa}
\sol 1~ 由矩阵 $A$ 的特征多项式,求出特征值。
\[
\begin{aligned}
|A-\lambda E| &=\left|\begin{array}{ccc}
-1-\lambda & 1 & 0 \\
-4 & 3-\lambda & 0 \\
1 & 0 & 2-\lambda
\end{array}\right|=(2-\lambda)\left|\begin{array}{cc}
-1-\lambda & 1 \\
-4 & 3-\lambda
\end{array}\right| \\
&=(2-\lambda)(\lambda-1)^{2}=0
\end{aligned}
\]
特征值为 $\lambda=2,1$.
\end{frame}


\begin{frame}
2~ 把每个特征值 $\lambda$ 代入线性方程组 $(A-\lambda E) X=\mathbf{0}$
求出基础解系。
\begin{itemize}
	\item 当 $\lambda={2}$ 时, 解线性方程组 $(A-2 E) X=\mathbf{0}$
\[
(A-2 E)=\left(\begin{array}{ccc}
-3 & 1 & 0 \\
-4 & 1 & 0 \\
1 & 0 & 0
\end{array}\right) \rightarrow\left(\begin{array}{ccc}
1 & 0 & 0 \\
0 & 1 & 0 \\
0 & 0 & 0
\end{array}\right)
\]
解得
\[
\left\{\begin{array}{l}{x}_{1}={0} \\ {x}_{2}={0}\end{array}\right.
\]
得基础解系  
\[
{p}_{1}=\left(\begin{array}{l} {0} \\ {0} \\ {1}\end{array}\right).
\]
\end{itemize}
\end{frame}



\begin{frame}
\begin{itemize}
	\item 
当 $\lambda=1$ 时, 解线性方程组 $(A-E) X=\mathbf{0}$

\[
(A-E)=\left(\begin{array}{ccc}-2 & 1 & 0 \\ -4 & 2 & 0 \\ 1 & 0 & 1\end{array}\right) \rightarrow\left(\begin{array}{lll}1 & 0 & 1 \\ 0 & 1 & 2 \\ 0 & 0 & 0\end{array}\right)
\]
解得
\[
\left\{\begin{array}{l}x_{1}+x_{3}=0 \\ x_{2}+2 x_{3}=0\end{array}\right.
\]

得基础解系 
\[
p_{2}=\left(\begin{array}{c}1 \\ -2 \\ 1\end{array}\right).
\]
\end{itemize}


\end{frame}


\begin{frame}
\begin{exa}
在空间 $P[x]_n$ 中,线性变换
\[
\mathscr{D} f(x)=f^{\,\prime}(x)
\]
在基 $1, x, \frac{x^{2}}{2 !}, \ldots, \frac{x^{n-1}}{(n-1) !}$ 下的矩阵是
\[
{D}=\left(\begin{array}{ccccc}
0 & 1 & 0 & \cdots & 0 \\
0 & 0 & 1 & \cdots & 0 \\
\vdots & \vdots & \vdots & & \vdots \\
0 & 0 & 0 & & 1 \\
0 & 0 & 0 & \cdots & 0
\end{array}\right)
\]
\end{exa}
\end{frame}

\begin{frame}
$\mathscr{D}$ 的特征多项式是 $|\lambda E-D|=\left|\begin{array}{ccccc}\lambda & -1 & 0 & \cdots & 0 \\ 0 & \lambda & -1 & \cdots & 0 \\ \vdots & \vdots & \vdots & & \vdots \\ 0 & 0 & 0 & \cdots & -1 \\ 0 & 0 & 0 & \cdots & \lambda\end{array}\right|=\lambda^{n}$

$\mathscr{D}$ 的特征值只有0. 

通过解相应的
齐次线性方程组知道, 属于特征值0的线性
无关的特征向量只能是任一非零常数. 
%这表明微商为零的多项式只能是零或
%非零的常数. 

\end{frame}


\begin{frame}
\begin{exa}
平面上全体向量构成实数域上一个二
维线性空间, 第一节例3中旋转变换$\mathscr{I}_{\theta}$在直角坐
标系下的矩阵为
\[
\left(\begin{array}{cc}
\cos \theta & -\sin \theta \\
\sin \theta & \cos \theta
\end{array}\right)
\]
它的特征多项式为 $\left|\begin{array}{cc}\lambda-\cos \theta & \sin \theta \\ -\sin \theta & \lambda-\cos \theta\end{array}\right|=\lambda^{2}-2 \lambda \cos \theta+1$

当 $\theta \neq k \pi$ 时,这个多项式没有实根. 

因之,当 $\theta \neq k \pi$ 时, $\mathscr{I}_{\theta}$没有特征值.

%从几何上看, 这个结论是显然的.
\end{exa}
\end{frame}


\begin{frame}
\begin{itemize}
\item 属于同一特征值$\lambda_0$ 的特征向量全体连同零向量构成$V$的一个子空间$V_{\lambda_{0}}$, 称其为\alert{特征子空间}. 
用集合记号可号为 $$V_{\lambda_{0}}=\left\{\alpha |  \A \alpha=\lambda_{0} \alpha, \alpha \in V \right\}.$$
它的维数等于属于同一特征值$\lambda_{0}$ 的线性无关特征向量的最大个数, 称为特征值$\lambda_0$的\alert{几何重数}. 
\end{itemize}




\end{frame}

\begin{frame}{特征多项式的系数}{前两项}
{
$$|\lambda {E}-{A}|=\left|\begin{array}{cccc}\lambda-a_{11} & -a_{12} & & -a_{1n} \\ -a_{21} & \lambda-a_{22} & \dots & -a_{21} \\ \vdots & \vdots & & \vdots \\ -a_{n 1} & -a_{n 2} & \cdots & \lambda-a_{nn}\end{array}\right|$$

\begin{itemize}
	\item 展开式中有一项是主对角线上元素的连乘积 
	$$\left(\lambda-a_{11}\right)\left(\lambda-a_{22}\right) \cdots\left(\lambda-a_{nn}\right).$$
	\item 展开式中其余答项,至多包含 $n-2$ 个主对角线上的元素, 它对 $\lambda$ 的次数最多是 $n-2 .$
\end{itemize}
}
\end{frame}

\begin{frame}
\small{
 因此,特征多项式中含 $\lambda$ 的 $n$ 次与 $n-1$ 次的项只能出现在主对角线上元素的连乘积中, 它们是
\[
\lambda^{n}\alert{-\left(a_{11}+a_{2}+\dots+a_{nn}\right)} \lambda^{n-1}
\]
在$|\lambda {E}-{A}|$中, 令$\lambda=0$, 得到常数项系数为 $$\alert{|-{A}|=(-1)^n|A|}$$
\begin{prop}
	\begin{itemize}
		\item 如果只写出特征多项式的前两项和常数项, 就有
		$$|\lambda E-A|= \lambda^{n}{-\left(a_{11}+a_{22}+\cdots+a_{nn}\right)} \lambda^{n-1}+ \cdots+{(-1)^n|A|}.$$
		\item 由多项式的根与系数的关系可知
		\[
		\sum_{i=1}^{n} \lambda_{i}=\mbox{tr}({A}), \quad \prod_{i=1}^{n} \lambda_{i}=|{A}|, 
		\]
		其中
		$\mbox{tr}({A}):=\sum_{i=1}^{n} {a}_{i i}$称为矩阵$A$的迹.
	\end{itemize}
\end{prop} }
\end{frame}

\begin{frame}
%特征值自然是被线性变换所决定的.
%但是在有限维空间中,任取一组基之后,特征值就是线性变换在这组基下矩阵的特征多项 式的根.随着基的不同,线性变换的矩阵一般是不同的. 但这些 矩阵是相似的, 
%\item 相似矩阵有相同的特征多项式和相同的特征值.
%\item 线性变换的特征值与基的选择无关.
对于相似矩阵我们有 
\begin{thm}
相似的矩阵有相同的特征多项式.
\end{thm}
\pf 设 ${A} \sim {B},$ 即有可逆矩阵 ${X}$,使 ${B}={X}^{-1} {A X} .$ 于是
\[
\begin{aligned}
|\lambda {E}-{B}| &=\left|\lambda {E}-{X}^{-1} {A} {X}\right|=\left|{X}^{-1}(\lambda {E}-{A}) {X}\right| \\
&=\left|{X}^{-1}\right||\lambda {E}-{A}||{X}|=|\lambda {E}-{A}|.
\qquad \qquad  \hfill{\rule{4pt}{7pt}}
\end{aligned}
\]
\begin{itemize}
\item 线性变换的矩阵的特征多项式与基的选择
无关,它是直接被线性变换决定的.
%因此,以后就可以说线性变换 的特征多项式了.  
%\item 既然相似的矩阵有相同的特征多项式,当然特征多项式的各 项系数对于相似的矩阵来说都是相同的.觀如说,考虑特征多项式 的常数项,得到相似矩阵有相同的行列式.因此, 以后就可以说线 性变换的行列式了. 
\item 定理 6 的逆是不对的, 特征多项式相同的矩阵不一定是相似的.例如
\[
{A}=\left(\begin{array}{ll}
1 & 0 \\
0 & 1
\end{array}\right), {B}=\left(\begin{array}{ll}
1 & 1 \\
0 & 1
\end{array}\right)
\]
它们的特征多项式都是$( \lambda-1 )^{2}$,但 ${A}$ 和 ${B}$ 不相似.

这是因为与单位矩阵$A$相似的矩阵只能是$A$本身.
\end{itemize}

\end{frame}

\begin{frame}
\begin{itemize}
	\item 设 $A=\left(\begin{array}{cc}A_{1} & A_{2} \\ 0 & A_{3}\end{array}\right)$,则 $f_{A}(\lambda)=f_{A_{1}}(\lambda) f_{A_{3}}(\lambda),$ 即 $|\lambda E-A|=\left|\lambda E-A_{1} \| \lambda E-A_{3}\right|$
	\item 根据 $|A|=\prod_{i=1}^{n} \lambda_{i},$ 则 
	
	$|A|=0 \Leftrightarrow A$ 一定有零特征值.
	
	 $|A| \neq 0$ 	$\Leftrightarrow A$ 无零特征值,即特征值非零.
	 
	 例子: 设 $A=\left(\begin{array}{ll}1 & 1 \\ 2 & 2\end{array}\right)$,则 $|A|=0,$ 从而有一个零特征值,另一个为 $3 .$ 依据公式 $\sum_{i=1}^{n} a_{i i}=\sum_{i=1}^{n} \lambda_{i}$.
	 
	 \item  $f_{A}(\lambda)$ 在所考虑的数域范围内不一定有解,从而可以没有特征值.
\end{itemize}
\end{frame}


\begin{frame}
\begin{itemize}
\item 线性变换的矩阵的特征多项式与基的选择无关, $\sigma$ 在 的矩阵
$A$ 的特征多项式称为 $\sigma$ 的\alert{特征多项式}, 并记作 $f_{\sigma}(\lambda)$.
\item 相似矩阵有相同的行列式, 故可定义线性变换 $\sigma$ 的\alert{行列式}为 $\sigma$ 在任意一组
基下的矩阵的行列式.
\item 矩阵的\alert{特征值}、\alert{迹}、\alert{行列式}都是相似不变量.
\end{itemize}
\end{frame}


\begin{frame}
\begin{exa}
设 $A=\left(\begin{array}{ccc}3 & 2 & -1 \\ 2 & 3 & -1 \\ -4 & b & a\end{array}\right),$ 已知 $A$ 有特征值 $\lambda_{1}=\lambda_{2}=1, \lambda_{3}=5.$ 

求 $a, b$的值.
\end{exa} 
\sol  因为 $A$ 有特征值 $\lambda_{1}=\lambda_{2}=1, \lambda_{3}=5, \quad$ 

所以有 $\operatorname{Tr}(A)=6+a=7,$ 即
$a=1 .$ 

由 $|A|=b+1=\lambda_{1} \lambda_{2} \lambda_{3}$ 得 $b=4$.

\end{frame}


\begin{frame}{特征多项式的性质} 
首先介绍一个概念. 形如
\[
B(\lambda)=\lambda^{m} B_{0}+\lambda^{m-1} B_{1}+\cdots+B_{m}
\]
多项式, 其中 $B_{0}, B_{1}, \cdots, B_{m}$ 都是 $n \times n$ 数字矩阵, 叫做一个矩阵多项式, $n$ 叫做 的级数, 当 $B_{0} \neq 0$ 时, $m$ 叫做它的次数. 如
\begin{align*}
\left(\begin{array}{cc}\lambda^{3}+3 \lambda+1 & \lambda+2 \\ \lambda^{3}+2 \lambda & 3\end{array}\right)
& =\left(\begin{array}{cc}\lambda^{3} & 0 \\ \lambda^{3} & 0\end{array}\right)+\left(\begin{array}{cc}3 \lambda & \lambda \\ 2 \lambda & 0\end{array}\right)+\left(\begin{array}{cc}1 & 2 \\ 0 & 3\end{array}\right)\\
& =\lambda^{3}\left(\begin{array}{ll}
1 & 0 \\
1 & 0
\end{array}\right)+\lambda\left(\begin{array}{ll}
3 & 1 \\
2 & 0
\end{array}\right)+\left(\begin{array}{ll}
1 & 2 \\
0 & 3
\end{array}\right)
\end{align*}

\end{frame}

\begin{frame}{哈密顿 - 凯莱(Hamilton-Caylay)定理 }
%最后, 我们指出特征多项式的一个重要性质.  

\begin{itemize}
	\item 
设 $A$ 是数域 $P$ 一个 $n \times n$ 矩阵, $f(\lambda)=|\lambda {E}-{A}|$ 是 ${A}$ 的特征多项式,则
\[
f({A})={A}^{n}-\left(a_{11}+a_{22}+\cdots+a_{n n}\right) {A}^{n-1}+\cdots+(-1)^{n}|{A}| {E}=0
\]

\item 设$\A$是有限维线性空间 $V$ 的线性变换, $f(\lambda)$ 是 $\mathscr{A}$ 的特征 多项式, 那么 $$f(\A)=\O.$$

\end{itemize}
\end{frame}


\begin{frame}
\begin{prop}
若$\lambda$是 $A$ 的特征值,即 $A X=\lambda X( X \neq 0),$ 则
\begin{itemize}
	\item  $k \lambda$ 是 ${k} {A}$ 的特征值 $({k} \text { 是常数 }),$ 且 ${k} A {X}={k} \lambda {X}$.
	\item $\lambda^{m}$ 是 $A^{m}$ 的特征值 $\left(m \text { 是正整数) }, \text { 且 } A^{m} X=\lambda^{m} X\right.$
	\item 若 $A$ 可逆,则 
\begin{itemize}
	\item $\lambda^{-1}$ 是 $A^{-1}$ 的特征值, 且 $A^{-1} X=\lambda^{-1} X$, 
	\item $\lambda^{-1}|A|$ 是 $A^{*}$ 的特征值,且 $A^{*} X=\lambda^{-1}|A| X$.
\end{itemize}
	\item $\varphi(t)$ 为 $t$ 的多项式,则 $\varphi(\lambda)$ 是 $\varphi(A)$ 的特征值, 且 $\varphi(A) X=\varphi(\lambda) X$.
	\item 矩阵 $A$ 和 $A^{\mathrm{T}}$ 的特征值相同,特征多项式相同。	
\end{itemize}
\end{prop}
\end{frame}


\section{对角矩阵}
\begin{frame}{\S 5 对角矩阵}
本节的目的: 研究什么样的线性变换,在一组基下的矩阵形式是对角阵. 
\begin{qst}
设3维线性空间的一个线性变换 $\A$ 在一组基 $\varepsilon_{1}, \varepsilon_{2}, \varepsilon_{3}$ 下的矩阵为 $$A=\left(\begin{array}{ccc}1 & 2 & 2 \\ 2 & 1 & 2 \\ 2 & 2 & 1\end{array}\right).$$
是否存在在一组基,使得 $\A$ 在这组基下的矩阵形式是对角阵. 
\end{qst}



\end{frame}


\begin{frame}
\sol
\small{
令 $X=\left(\begin{array}{ccc}1 & 1 & 1 \\ -1 & 0 & 1 \\ 0 & -1 & 1\end{array}\right)$,可知 $X$ 可逆.

令 $$\left(\xi_{1}, \xi_{2}, \xi_{3}\right)=\left(\varepsilon_{1}, \varepsilon_{2}, \varepsilon_{3}\right) X, $$

即令
$$\xi_{1}=\varepsilon_{1}-\varepsilon_{2}, \xi_{2}=\varepsilon_{1}-\varepsilon_{3}, \xi_{3}=\varepsilon_{1}+\varepsilon_{2}+\varepsilon_{3},$$

则 $$\A \xi_{1}=-\xi_{1}, \A \xi_{2}=-\xi_{2}, \A \xi_{3}=5 \xi_{3},$$
且
$\xi_{1}, \xi_{2}, \xi_{3}$ 是线性空间的一组基.

因此, 
$\A$ 在这组基下的矩阵 
\[
X^{-1} A X = \left(\begin{array}{ccc}-1 & 0 & 0 \\ 0 & -1 & 0 \\ 0 & 0 & 5\end{array}\right)
\]
是对角阵.
}

\end{frame}


\begin{frame}
%可以认为是矩阵中最简单的一种.现在我们来考察,  究竟哪一些线性变换的矩阵在一组适当的基下可以是对角矩阵. 
\begin{thm}
设 $\A$是 $n$ 维线性空间 $V$ 的一个线性变换$\A$ 的矩阵.

 $\A$在某一组基下的矩阵为对角矩阵$\Longleftrightarrow$$\A$有 $n$ 个线性 无关的特征向量.
\end{thm}

\pf 
\small{
	\begin{itemize}
		\item 
	设 $\A$在基 ${\varepsilon}_{1}, {\varepsilon}_{2}, \cdots, {\varepsilon}_{n}$ 下具有对角矩阵
	$$\mbox{diag}\left(  \lambda_{1}, \lambda_{2}, \ldots, \lambda_{n} \right),$$
这就是说, 
\[
\mathscr{A} {\varepsilon}_{i}=\lambda_{i} {\varepsilon}_{i}, i=1,2, \cdots, n
\]
因此 , ${\varepsilon}_{1}, {\varepsilon}_{2}, \cdots, {\varepsilon}_{n}$ 就是 $\A$ 的 $n$ 个线性无关的特征向量.

\item 反过来,如果 $\A$有 $n$ 个线性无关的特征向量 ${\varepsilon}_{1}, {\varepsilon}_{2}, \cdots, {\varepsilon}_{n},$ 那
么就取 ${\varepsilon}_{1}, {\varepsilon}_{2}, \cdots, {\varepsilon}_{n}$ 为基, 在这组基下 $\mathscr{A}$ 的矩阵是对角矩阵. 
\end{itemize}}
\end{frame}


\begin{frame}
%为了进一步给出一些判别条件,我们来证 
\begin{thm}
属于不同特征值的特征向量是线性无关的.
\end{thm}
\pf
对特征值的个数作数学归纳法.
\begin{itemize}
	\item 由于特征向量是不为零的,所以单个的特征向量必然线性无关. 
	\item 现在设属于 $k$ 个不同 特征值伺特征向量线性无关.
	\item 我们证明属于 $k+1$ 个不同特征值 $\lambda_{1}, \lambda_{2}, \cdots, \lambda_{k+1}$ 的特征向量 ${\xi}_{1}, {\xi}_{2}, \cdots, {\xi}_{k+1}$ 也线性无关.
\end{itemize}
\setcounter{equation}{0}

假设有关系式
\begin{equation}
a_{1} \xi_{1}+a_{2} \xi_{2}+\cdots+a_{k} \xi_{k}+a_{k+1} \xi_{k+1}=0
\end{equation}

成立.等式两端乘以 $\lambda_{k+1}$,得
\begin{equation}
a_{1} \lambda_{k+1} \xi_{1}+a_{2} \lambda_{k+1} \xi_{2}+\cdots+a_{k} \lambda_{k+1} \xi_{k}+a_{k+1} \lambda_{k+1} \xi_{k+1}=0
\end{equation}
(1) 式两端同时施行变换 $\mathscr{A},$ 即有 
\begin{equation}
a_{1} \lambda_{1} \xi_{1}+a_{2} \lambda_{2} \xi_{2}+\cdots+a_{k} \lambda_{k} \xi_{k}+a_{k+1} \lambda_{k+1} \xi_{k+1}=0\end{equation}
\end{frame}


\begin{frame}


(3) 减去(2)得到 $$a_{1}\left(\lambda_{1}-\lambda_{k+1}\right) \xi_{1}+\cdots+a_{k}\left(\lambda_{k}-\lambda_{k+1}\right) \xi_{k}=\mathbf{0}$$
根据归纳法假设,  ${\xi}_{1}, {\xi}_{2}, \cdots, {\xi}_{k}$ 线性无关,于是
\[
a_{i}\left(\lambda_{i}-\lambda_{k+1}\right)=0, i=1,2, \cdots, k
\]
但 $\lambda_{i}-\lambda_{k+1} \neq 0(i \leqslant k),$ 所以 $a_{i}=0, i=1,2, \cdots, k$.

这时 (1) 式变成
$$a_{k+1} \xi_{k+1}=\mathbf{0}.$$
又因为 $\xi_{k+1} \neq 0,$ 所以只有 $a_{k+1}=0.$ 

这就证明了 $\xi_{1}$
${\xi}_{2}, \cdots, {\xi}_{k+1}$ 线性无关.

根据归纳法原理,定理得证. \qed

\end{frame}



\begin{frame}
从上面这两个定理就得到

\begin{coro}
	在 $n$ 维线性空间 $V$ 中, 线性变换 $\A$ 的特征多项式在数域 $P$ 中有 $n$ 个不同的根,即 $\A$有 $n$ 个不同的特征值, 那么 $\A$在某组基下的矩阵是对角形的.
\end{coro}

复数域中任一个 $n$ 次多项式都有 $n$ 个根. 因此, 上面的 论断可以改写为 
\begin{coro}
	在复数域上的线性空间中,如果线性变换$\A$的特征 多项式没有重根,那么$\A$在某组基下的矩阵是对角形的. 
\end{coro}
\end{frame}

\begin{frame}
 在一个线性变换没有 $n$ 个不同的特征值的情形,要判别这个 线性变换的矩阵能不能成为对角形,问题就要复杂些.
 
 为了利用定理 7,我们把定理 8 推广为 
\begin{thm}
	如果 $\lambda_{1}, \cdots, \lambda_{k}$ 是线性变换 $\A$ 的不同的特征值, 
	而 ${\alpha}_{i 1}, \cdots, {\alpha}_{i_{r_i}}$ 是属于特征值 $\lambda_{i}$ 的线性无关的特征向量 $i=1, \cdots, k$, 
	那么向量组 $${\alpha}_{11}, \cdots, {a}_{1 r_{1}}, \cdots, {a}_{k 1}, \cdots, {a}_{k r_{n}}$$ 也线性无关.
\end{thm}
\end{frame}


\begin{frame}
\begin{exa}
	在 $F^{\, 3}$ 中, 设线性变换 $\sigma$ 在基 $\varepsilon_{1}, \varepsilon_{2}, \varepsilon_{3}$ 下的矩阵为
	\[
	A=\left(\begin{array}{ccc}
	1 & 4 & 2 \\
	0 & -3 & 4 \\
	0 & 4 & 3
	\end{array}\right)
	\]
	其中 $\varepsilon_{1}, \varepsilon_{2}, \varepsilon_{3}$ 为 3 维单位向量组.	
	问: 线性变换 $\sigma$ 在此基下的矩阵是否为对 角形?
\end{exa}
\pause
\sol 特征多项式 $f_{A}(\lambda)=|\lambda E-A|=(\lambda-1)(\lambda-5)(\lambda+5),$

 所以 $\lambda_{1}=1, \lambda_{2}=5, \lambda_{3}=-5.$
%属于这些特征值的特征向量分为为
%\[
%\eta_{1}=\left(\begin{array}{l}
%1 \\
%0 \\
%0
%\end{array}\right), \eta_{2}=\left(\begin{array}{l}
%2 \\
%1 \\
%2
%\end{array}\right), \eta_{3}=\left(\begin{array}{c}
%1 \\
%-2 \\
%1
%\end{array}\right)
%\]
%易知 $\eta_{1}, \eta_{2}, \eta_{3}$ 是线性无关的, 从而构成一组基, 

由推论可知,线性变换 $\sigma$ 在此基下的矩阵为对 角形.
\end{frame}

\begin{frame}

根据这个定理,对于一个线性变换,求出属于每个特征值的线性无关的特征向量,把它们合在一起还是线性无关的.
\begin{itemize}
\item 如果它们的 个数等于空间的维数,那么这个线性变换在一组合适的基下的矩 阵是对角矩阵;
\item 如果它们的个数小于空间的维数,那么这个线性变 换在任何一组基下的矩阵都不能是对角形的.
\end{itemize}


\end{frame}

\begin{frame}
设 $\A$的全 部不同的特征值是 $\lambda_{1}, \cdots, \lambda_{s}$.
	
对每个 $1 \le  i \le  s$, 设 $\operatorname{dim} V_{\lambda_{i}}=m_{i},\left\{{\alpha}_{i 1}, \cdots, {\alpha}_{i_m}\right\}$ 是 $V_{i}$ 的一组基. 

则各特征子空间 $V_{\lambda_{i}}$ 的基 $M_{i}$ 所含向量共同组成的集合 $$\left\{ {\alpha}_{i j} \, | \, 1 \le i \le s , 1 \le j \le m_{i} \right\}$$
线性无关,它包含 $m_{1}+m_{2}+\cdots+m_{s}$ 个线性无关的特征向量, 是 $\A$ 的特征向量集合的一个极大线性无关量组.
 
 
\begin{prop}
	$\A$ 在某一组基下的矩阵成对角形$\Leftrightarrow  \A$ 的各特征子空间$V_{\lambda_{1}}, \ldots, V_{\lambda_{s}}$的维数之和等于空间的维数$\operatorname{dim}  V$ .
\end{prop}

\end{frame}


\begin{frame}

\begin{prop}
当线性变换$\A$在一组基下的矩阵 $A$ 是对角形
	$$\mbox{diag}\left(  \lambda_{1}, \lambda_{2}, \ldots, \lambda_{n} \right),$$
	时, $\A$ 的特征多项式为
	\[
	|\lambda{E}-{A}|=\left(\lambda-\lambda_{1}\right)\left(\lambda-\lambda_{2}\right) \cdots\left(\lambda-\lambda_{n}\right)
	\]
	因此,如果线性变换 
	线上的元素除排列次序外是确定的,它们正是 $\A$ 的特征多项式全 部的根(重根按重数计算). 
\end{prop}



 根据 $\S$ 3 定理 5,一个线性变换的矩阵能不能在某一组基下是 对角形的问题就相当于一个矩阵是不是相似于一个对角矩阵的问题.
 
 因此, 这一节的讨论也就解决了后一个问题.   
\end{frame}






\begin{frame}{总结:矩阵的可对角化问题}
\begin{itemize}
	\item 如果数域 $P$ 上的 $n$ 级矩阵 $A$ 可相似于对角矩阵,则称 $A$ 可对角化.
	
	\item 数域 $P$上 $n$ 级矩阵 $A$ 可对角化的条件如下:
	\begin{enumerate}
		\item (充分必要条件) $A$有 $n$ 个线性无关的特征向量.
		\item (充分条件) $A$ 有 $n$ 个互异的特征值.
		\item (充分必要条件) $A$ 的所有重特征值对应的线性无关特征向量的个数等于其重数.	
	\end{enumerate}
\end{itemize}
\end{frame}


\begin{frame}{总结:线性变换的可对角化问题}
设 $\A$ 是数域 $P$上 $n$ 维线性空间 $V$的一个线性变换,$\A$ 的矩阵可以在某
一组基下为对角矩阵有下列条件:
\begin{enumerate}
\item (充分必要条件)$\A$ 有 $n$ 个线性无关的特征向量.
\item (充分条件)$\A$在数域 $P$中有 $n$ 个不同的特征值.
\item (充分必要条件)设 $\A$的全部互异的特征值为$\lambda_1,\lambda_2,\ldots , \lambda_s$, 则 $\A$ 的特征子空间 $V_{1},  V_{2},  \cdots,  V_{s}$ 的维数之和等于 $n$.
\end{enumerate}
\end{frame}





\begin{frame}{代数重数与几何重数}

% $m_{1}+\cdots+m_{s}=\operatorname{dim} V$ 何时成立,会在本章后面几节继续研究.
 
对方阵 $A$,特征多项式 $$f_{A}(\lambda)=|\lambda E-A|=\left(\lambda-\lambda_{1}\right)^{n_{1}}\left(\lambda-\lambda_{2}\right)^{n_{2}} \cdots\left(\lambda-\lambda_{s}\right)^{n_{s}},$$ 其中$\lambda_{1}, \lambda_{2}, \cdots, \lambda_{s}$ 两两互异.

\begin{itemize}
	\item $n_{i}$ 称为特征值 $\lambda_{i}$ 的\alert{代数重数}.
	\item 再看特征值 $\lambda_{i}$ 的特征子空间 $V_{\lambda_{i}}$,其维数 $\operatorname{dim} V_{\lambda_{i}}=m_{i}$ 称为 $\lambda_{i}$ 的\alert{几何重数}. $\quad m_{i}=n-r\left(\lambda_{i} E-A\right)$
\end{itemize}

\begin{prop}
	矩阵的任一特征值的几何重数小于等于其代数重数.
\end{prop}
证明需要不变子空间的性质,会在本章\S 7给出。
\end{frame}

\begin{frame}
\begin{prop}
矩阵的任一特征值的几何重数小于等于其代数重数.
\end{prop}

\begin{prop}
方阵 $A$ 可相似对角化 $\Leftrightarrow$ 任一特征值的代数重数与几何重数相等.
\end{prop}

\begin{prop}
设 $A$ 是一 $n$ 阶方阵, $A$ 的特征多项式为 $$f_{A}(\lambda)=\left(\lambda-\lambda_{1}\right)^{n_{1}}\left(\lambda-\lambda_{2}\right)^{n_{2}} \cdots\left(\lambda-\lambda_{s}\right)^{n_{i}}$$
则 
\begin{align*}
A \mbox{ 可相似对角化 } & \Leftrightarrow n_{i}=n-r\left(\lambda_{i} E-A\right), \quad  i=1,2, \cdots, s.\\
& \Leftrightarrow r\left(\lambda_{i} E-A\right)=n-n_{i}, \quad i=1,2, \cdots, s.
\end{align*}
\end{prop}
\end{frame}


%矩阵的任一特征值的几何重数小于等于其代数重数.
%证明:对 $A$,任给线性空间 $V$ 及一组基 $\varepsilon_{1}, \varepsilon_{2}, \cdots, \varepsilon_{n},$ 存在线性变换 $\sigma$,使得 $\sigma-\frac{\varepsilon_{1}, \varepsilon_{2}, \cdots, \varepsilon_{n}}{\longrightarrow} A$
%假设 $\lambda_{0}$ 是 $A$ 的一个特征值,代数重数为 $n_{0} .$ 设 $\operatorname{dim} V_{\lambda_{0}}=m_{0},$ 不妨设 $\xi_{1}, \xi_{2}, \cdots, \xi_{m_{0}}$ 是 $V_{\lambda_{0}}$ 的一组基,扩充这一
%组基为 $V$ 的一组基, $\xi_{1}, \cdots, \xi_{m_{0}}, \xi_{m_{0}+1}, \cdots, \xi_{n},$ 则由于 $\sigma \xi_{i}=\lambda_{0} \xi_{i}, i=1,2, \cdots, m_{0} .$ 则
%$\sigma \frac{\xi_{1}, \cdots ; \xi_{m_{0}}, \xi_{m_{0}+1} \cdots, \xi_{n}}{\rightarrow}\left(\begin{array}{cc}\lambda_{0} E & A_{2} \\ 0 & A_{3}\end{array}\right) \cdot\left(\begin{array}{cc}\lambda_{0} E & A_{2} \\ 0 & A_{3}\end{array}\right)$ 与A是同一个线性变换 $\sigma$ 在不同基下的矩阵,则
%$\left(\begin{array}{cc}\lambda_{0} E & A_{2} \\ 0 & A_{3}\end{array}\right)$ 与 $A$ 相似,从而特征多项式相同. $\left|\lambda E-\left(\begin{array}{cc}\lambda_{0} E & A_{2} \\ 0 & A_{3}\end{array}\right)\right|=|\lambda E-A|$
%右边因式 $\lambda-\lambda_{0}$ 是 $n_{0}$ 重.左边中 $\lambda-\lambda_{0}$ 也应该是 $n_{0}$ 重.但是左边展开后
%$\left|\lambda E-\left(\begin{array}{cc}\lambda_{0} E & A_{2} \\ 0 & A_{3}\end{array}\right)\right|=\left|\begin{array}{cc}\left(\lambda-\lambda_{0}\right) E & A_{2} \\ 0 & \lambda E-A_{3}\end{array}\right|=\left(\lambda-\lambda_{0}\right)^{m_{0}}\left|\lambda E-A_{3}\right| \cdot$ 从而 $m_{0} \leq n_{0}$




\begin{frame}{判断相似对角化的方法}

%判断是否可以相似对角化,若可以,求可逆阵 X ,使得 $X^{-1} A X$ 为对角阵.


(1) 对 $A$,求所有特征值 $f_{A}(\lambda)=|\lambda E-A|=\left(\lambda-\lambda_{1}\right)^{n_{1}} \cdots\left(\lambda-\lambda_{s}\right)^{n_{s}}$


(2) 是否可以相似对角化 $\Leftrightarrow r\left(\lambda_{i} E-A\right)=n-n_{i}, i=1,2, \cdots, s$.
\begin{itemize}
	\item 若都成立,则可以;
	\item 若有一个不成立,则不能. 
	\begin{itemize}
		\item 对单根 $\lambda_{i}$ 自然成立, 不用判断 $r\left(\lambda_{i} E-A\right)=n-1$.
		\item 主要是对重根 $\lambda_{i},$ 来判断是否 $r\left(\lambda_{i} E-A\right)=n-n_{i}$.
	\end{itemize}
\end{itemize}


(3) 若可以,求方程组 $\left(\lambda_{i} E-A\right) X=0$ 的基础解系 $X_{i 1},  \cdots, X_{i n_{i}}$

令 $X=\left(X_{11}, \cdots, X_{1 n_{1}}, X_{21}, \cdots,  X_{s n_{s}}\right),$ 则 
$X^{-1} A X=
\left(\begin{array}{ccc}
\lambda_{1} E_{n_{1}} & & \\ 
& \ddots & \\ 
& & \lambda_{s} E_{n_{s}}
\end{array}\right)$

\end{frame}



\begin{frame}
\begin{exa}
判断 $\left(\begin{array}{cc}0 & 1 \\ 0 & 0\end{array}\right)$ 是否可以相似对角化. 

若可以,求可逆阵 X ,使得 $X^{-1} A X$ 为对角阵.

\end{exa}
\sol
$|\lambda E-A|=\left|\begin{array}{cc}\lambda & -1 \\ 0 & \lambda\end{array}\right|=\lambda^{2}=0,$ 则 $\lambda_{1}=\lambda_{2}=0$.

$0 E-A=-A$,秩为 $1 \neq 2-2$, 故不能相似对角化.
\end{frame}

\begin{frame}
\begin{exa}
判断$A=\left(\begin{array}{ccc}1 & 2 & 2 \\ 0 & 2 & 2 \\ 0 & 0 & 3\end{array}\right)$
是否可以相似对角化.

若可以,求可逆阵 $X$ ,使得 $X^{-1} A X$ 为对角阵.
\end{exa}
\sol 
$A$ 的特征值为 $\lambda_{1}=1, \lambda_{2}=2, \lambda_{3}=3$,有3个互异的特征值,可以相似对角化.

对 $\lambda_{1}=1$, 线性方程组$(E-A) X=0$ 的解: $X_{1}=(1,0,0)'$,

对 $\lambda_{2}=2$, 线性方程组$(2 E-A) X=0$ 的解: $X_{2}=(2,1,0)'$,

对 $\lambda_{3}=3$, 线性方程组$(3 E-A) X=0$ 的解: $X_{3}=(3,2,1)'$. 

令 $X=\left(\begin{array}{ccc}1 & 2 & 3 \\ 0 & 1 & 2 \\ 0 & 0 & 1\end{array}\right)$,

则 
$X^{-1} A X=\mbox{diag}(1,2,3).$
\end{frame}

\begin{frame}
\begin{exa}
判断$A=\left(\begin{array}{ccc}1 & 2 & 2 \\ 0 & 1 & 2 \\ 0 & 0 & 2\end{array}\right)$
是否可以相似对角化.

若可以,求可逆阵 $X$ ,使得 $X^{-1} A X$ 为对角阵.
\end{exa}
\sol
$A$ 的特征值为 $\lambda_{1}=\lambda_{2}=1, \lambda_{3}=2$

对 $\lambda_{1}=1, E-A=\left(\begin{array}{ccc}0 & -2 & -2 \\ 0 & 0 & -2 \\ 0 & 0 & -1\end{array}\right)$,秩为2, $2 \neq 3-2,$ 

不能相似对角化
\end{frame}



\begin{frame}
\begin{exa}
判断 $A=\left(\begin{array}{cccc}1 & 1 & 1 & 1 \\ 1 & 1 & 1 & 1 \\ 2 & 2 & 2 & 2 \\ 3 & 3 & 3 & 3\end{array}\right)$ 是否可以相似对角化
\end{exa} 
\sol 
$\left|\begin{array}{cccc}\lambda-1 & -1 & -1 & -1 \\ -1 & \lambda-1 & -1 & -1 \\ -2 & -2 & \lambda-2 & -2 \\ -3 & -3 & -3 & \lambda-3\end{array}\right| 
=\lambda^{3}(\lambda-7)$,

得到特征值 $\lambda_{1}=0$(3重), $\lambda_{2}=7$.

$r(-A)=1=4-3,$ 有 3 个.

可以相似对角化.
\end{frame}



\begin{frame}
\begin{exa}[*]
设 $n$ 阶方阵 $A$ 满足 $A^{2}=E,$ 判断 $A$ 是否可以相似对角化
\end{exa}
\sol $A^{2}-E=(A+E)(A-E)=0$.

首先 $|A+E \| A-E|=0$,则 $A$ 有特征值1或者 $-1 .$
$r(A+E)+r(A-E) \leq n$.同时 $n=r(A+E-A+E) \leq r(A+E)+r(A-E)$

则 $r(A+E)+r(A-E)=n .$ 设 $r(A+E)=r,$ 则 $r(A-E)=n-r$

对特征值1, 线性方程组$(E-A) X=0$ 的基础解系含有 $n-(n-r)=r$ 个向量 $\xi_{1}, \xi_{2}, \cdots, \xi_{r}$

对特征值-1, 线性方程组$(-E-A) X=0$ 的基础解系含有 $n-r$ 个向量 $\xi_{r+1}, \xi_{r+2}, \cdots, \xi_{n}$

又属于不同特征值的特征问量线性无关,从而 A 有 $n$ 个线性无关的特征向量 $\xi_{1}, \xi_{2}, \cdots, \xi_{n}$

令 $X=\left(\xi_{1}, \xi_{2}, \cdots, \xi_{n}\right)$ 则 $X^{-1} A X=\left(\begin{array}{cc}E_{r} & 0 \\ 0 & -E_{n-r}\end{array}\right).$
\end{frame}


\begin{frame}
\begin{exa}[*]
	设 $n$ 阶方阵 $A$ 满足 $A^{2}=A$,判断 $A$ 是否可以相似对角化.
\end{exa}
\sol
	$A^{2}-A=A(A-E)=0$,首先 $|A \| A-E|=0$,则 $A$ 有特征值 0 或者 1
	且 $r(A)+r(A-E)=n .$ 设 $r(A)=r,$ 
	
	则 $r(A-E)=n-r$
	
	对特征值1 , 线性方程组$(E-A) X=0$ 的基础解系含有 $n-(n-r)=r$ 个向量 $\xi_{1}, \xi_{2}, \cdots, \xi_{r}$
	
	对特征值0, 线性方程组$(-A) X=0$ 的基础解系含有 $n-r$ 个向量 $\xi_{r+1}, \xi_{r+2}, \cdots, \xi_{n}$
	
	又属于不同特征值的特征向量线性无关,从而 $A$ 有 $n$ 个线性无关的特征向量 $\xi_{1}, \xi_{2}, \cdots, \xi_{n}$
	
	令$X=\left(\xi_{1}, \xi_{2}, \cdots, \xi_{n}\right)$.则 $X^{-1} A X=\left(\begin{array}{cc}E_{r} & 0 \\ 0 & 0\end{array}\right)$.
\end{frame}






\section{线性变换的值域与核}
\begin{frame}{\S 6 线性变换的值域与核}
\begin{defi}
\begin{itemize}
	\item 设$\A$是线性空间 $V$ 的一个线性变换,
	则的全体像组成的集合称为A的\alert{值域},
	用$\A V$ 或者$\operatorname{Im} \A$表示.
	\item 
	所有被$\A$变成零向量的向量组成的集合
	称为$\A$的\alert{核}, 用$\A^{-1}(\0)$或 $\A ^{-1}(\0)$表示.
\end{itemize}


若用集合的集合, 则
\begin{align*}
\A V  & = \{ \A\xi \,| \, \xi \in V \},\\
\A^{-1}(\0) & = \{\xi \, | \,\xi \in V,  \A \xi = \0\}.
\end{align*}

\end{defi}

\end{frame}

\begin{frame}
\begin{prop}
线性变换的值域与核都是 $V$ 的子空间.
\end{prop}
\pf 
\begin{itemize}
\item 
首先$\A V$非空, 并且对于 $V$中任何向量 $\alpha, \beta$ 和 ${k} \in {P},$ 都有
\[
\A \alpha+ \A \beta= \A(\alpha+\beta), \quad k \A \alpha=\A(k \alpha)
\]
即$\A V$对$V$的加法和数乘封闭, 故$\A V$是$V$的子空间. 

\item 同样, 由于$\A(\0) =\0,$ 故 $\0 \in {\A}^{-1}(\0)$ ${\A}^{-1}(\mathbf{0})$ 非空.

设 $\alpha, \beta$ 是 $\A^{-1}(\0)$ 中任何向量和 ${k} \in {P}$
由
\[
\A \alpha=\0, \quad \A \beta=\0
\]
可知
\[
\A (\alpha+\beta)=\0, \quad \A(k \alpha)=\0
\]
所以, $\A^{-1}(\0)$对加法和数乘封闭,  故$\A^{-1}(\0)$是$V$的子空间. 
\end{itemize}
\end{frame}

\begin{frame}
\begin{defi}
\begin{itemize}
\item $\A V$的维数称为$\A$的\alert{秩}, 记为$rank (\A)$ 或$r(\A)$. 

\item $\A^{-1}(\0)$的维数称为$\A$的\alert{零度}, 记为$null (\A)$.
\end{itemize}

\end{defi}

\begin{exa}
在线性空间 $P[x]_{n}$ 中,令
\[
\mathscr{D}(f(x))=f^{\,\, \prime}(x)
\]
则 $\mathscr{D}$的值域就是 $P[ x]_{n-1}$, $\mathscr{D}$ 的核就是子空间 $P$.

从而$rank (\A) =n-1$, $null (\A) = 1$.
\end{exa}
\end{frame}


\begin{frame}
\begin{thm}
设$\A$是$n$维向量空间$V$的线性变换, ${\varepsilon}_{1},{\varepsilon}_{2}, \ldots, {\varepsilon}_{n}$是$V$的一组基, $\A$在这组基下的矩阵是$A$, 则

\begin{enumerate}
	\item  $\A$的值域 $\A V$ 是由基像组生成的子空间,即 $$\mathscr{A} V=L\left(\mathscr{d} {\varepsilon}_{1}, \mathscr{d} {\varepsilon}_{2}, \cdots, \mathscr{A} {\varepsilon}_{n}\right).$$
	\item $\A$的秩 = $A$ 的秩. 
\end{enumerate}
\end{thm}

\pf 
\small{(1) 集合相等. 
首先, $\A V \supseteq L\left(\A \varepsilon_{1}, \A \varepsilon_{2}, \cdots, \A \varepsilon_{n}\right)$.

%设 $\xi$ 是 $V$ 中任一元素, 可用基的线性组合表示为
%\[
%\xi=x_{1} \varepsilon_{1}+x_{2} \varepsilon_{2}+\cdots+x_{n} \varepsilon_{n}
%\]
其次, 任给 $\A \alpha \in \A V$,则 $\alpha \in V,$ 
从而 $\alpha=\left(\varepsilon_{1}, \varepsilon_{2}, \cdots, \varepsilon_{n}\right) X$,

则$\A \alpha=\left(\A \varepsilon_{1}, \A \varepsilon_{2}, \cdots, \A \varepsilon_{n}\right) X \in L\left(\A \varepsilon_{1}, \A \varepsilon_{2}, \cdots, \A \varepsilon_{n}\right),$ 

从而 $\A V \subseteq L\left(\A \varepsilon_{1}, \A \varepsilon_{2}, \cdots, \A \varepsilon_{n}\right).$ 

因此,
$\A V=L\left(\A \varepsilon_{1}, \A \varepsilon_{2}, \cdots, \A \varepsilon_{n}\right)$.


(2)
因为$\left(\A \varepsilon_{1}, \A \varepsilon_{2}, \cdots, \A \varepsilon_{n}\right)=\left(\varepsilon_{1}, \varepsilon_{2}, \cdots, \varepsilon_{n}\right) A,$ 所以$$\operatorname{dim} L \left(\A \varepsilon_{1}, \A \varepsilon_{2}, \cdots, \A \varepsilon_{n}\right)=r(A).$$
}
\end{frame}


\begin{frame}
{求值域与核}
\begin{itemize}
\item  $\A V=L\left(\A  \varepsilon_{1}, \A  \varepsilon_{2}, \cdots, \A  \varepsilon_{n}\right),$ 求 $\A  \varepsilon_{1}, \A  \varepsilon_{2}, \cdots, \A  \varepsilon_{n}$ 的极大无关组即可.
\item 任给 $\alpha \in \A^{-1}(\0)$,则 $\A  \alpha=0$,假设 $\A$在基$\varepsilon_{1}, \varepsilon_{2}, \cdots, \varepsilon_{n}$ 下的矩阵为$A$, 并设 $\alpha=\left(\varepsilon_{1}, \varepsilon_{2}, \cdots, \varepsilon_{n}\right) X$,则
$$0=\A  \alpha=\A \left(\varepsilon_{1}, \varepsilon_{2}, \cdots, \varepsilon_{n}\right) X=\left(\varepsilon_{1}, \varepsilon_{2}, \cdots, \varepsilon_{n}\right) A X,$$ 
从而 $A X=0$, 
则 $\alpha$ 在基 $\varepsilon_{1}, \varepsilon_{2}, \cdots, \varepsilon_{n}$ 下的坐标是齐次线
性方程组 $A X=0$ 的解.
从而求得 $A X=0$ 的一组基础解系 $$\eta_{1}, \eta_{2}, \cdots, \eta_{r},$$ 以基础解系为坐标的向量
$$\xi_{1}, \xi_{2}, \cdots, \xi_{r}$$ 就是 $\A^{-1}(\0)$ 的一组基,即 $\A^{-1}(\0) =L\left(\xi_{1}, \xi_{2}, \cdots, \xi_{r}\right).$
\end{itemize}
\end{frame}


\begin{frame}
\begin{exa}
设 $\varepsilon_{1}, \varepsilon_{2}, \varepsilon_{3}, \varepsilon_{4}$ 是线性空间V的一组基,已知
	线性变换 $\sigma$ 在此基下的矩阵为 $A=\left(\begin{array}{cccc}1 & 0 & 2 & 1 \\ -1 & 2 & 1 & 3 \\ 1 & 2 & 5 & 5 \\ 2 & -2 & 1 & -2\end{array}\right)$.
求 $\sigma $ 的值域与核的维数与基.
\end{exa}



\end{frame}


\begin{frame}


\sol 
先求 $\sigma(V) .$ 
对矩阵$A$做初等行变换,得
$$A \rightarrow  \left(
\begin{array}{cccc}
1 & 0 & 2 & 1 \\
0 & 1 & \frac{3}{2} & 2 \\
0 & 0 & 0 & 0 \\
0 & 0 & 0 & 0 \\
\end{array}
\right)$$

于是,
$\sigma$ 的秩为2,
即 $\sigma(V)$ 为2维的, 且
\[
\begin{array}{lr}
\sigma\left(\varepsilon_{1}\right)=&\varepsilon_{1}-\varepsilon_{2}+\varepsilon_{3}+2 \varepsilon_{4}, \\
\sigma\left(\varepsilon_{2}\right)=& \quad 2 \varepsilon_{2}+2 \varepsilon_{3}-2 \varepsilon_{4}.
\end{array}
\]
% $\sigma\left(\varepsilon_{1}\right), \sigma\left(\varepsilon_{2}\right)$ 
 是 $\sigma(V)$ 的一组基.
\end{frame}


\begin{frame}
%由于\sigma 的零度为2,所
求 $\sigma^{-1}(\0)$.  设 $\xi \in \sigma^{-1}(\0),$ 它在 $\varepsilon_{1}, \varepsilon_{2}, \varepsilon_{3}, \varepsilon_{4}$
下的坐标为 $\left(x_{1}, x_{2}, x_{3}, x_{4}\right)$
由于 $\sigma(\xi)=\mathbf{0},$ 有 $\sigma(\xi)$ 在 $\varepsilon_{1}, \varepsilon_{2}, \varepsilon_{3}, \varepsilon_{4}$ 下的坐标为
$({0}, {0}, {0}, {0})$.
解齐次线性方程组
$$\left(\begin{array}{cccc}1 & 0 & 2 & 1 \\ -1 & 2 & 1 & 3 \\ 1 & 2 & 5 & 5 \\ 2 & -2 & 1 & -2\end{array}\right)\left(\begin{array}{l}x_{1} \\ x_{2} \\ x_{3} \\ x_{4}\end{array}\right)=\left(\begin{array}{l}0 \\ 0 \\ 0 \\ 0\end{array}\right)$$
得它的一个基础解系
$(-2,-2/3, 1,  0)$, $(-1, -2, 0, 1)$.
%再求 $\sigma(V)$. 由于$\sigma$ 的零度为2,所以の 的秩为2,
%即 $\sigma(V)$ 为2维的. $\quad$ 又由矩阵$A$,有
于是,$\sigma^{-1}(\0)$的维数为2,且
\[\begin{array}{rrrr}
-2 \varepsilon_{1}& -\frac{2}{3}\varepsilon_{2}& +\varepsilon_{3},& \\
- \varepsilon_{1} & -2 \varepsilon_{2}&&+  \varepsilon_{4}
\end{array}\]
是
$\sigma^{-1}(\0)$的一组基. \qed
\end{frame}

\begin{frame}
\begin{exa}
设 $V=F^{\, 3},$ 变换: $\A \left(x_{1}, x_{2}, x_{3}\right)=\left(2 x_{1}+x_{2}, x_{2}+x_{3}, 2 x_{1}-x_{3}\right)$. 求 $\A $ 的值域与核.
\end{exa}
\sol 
\begin{itemize}
	\item 设 $X=\left( x_1, x_2, x_3 \right)'$, 
%则 $\A (X)=\left(\begin{array}{ccc}2 & 1 & 0 \\ 0 & 1 & 1 \\ 2 & 0 & -1\end{array}\right) X,$ 
取单位向量 $\varepsilon_{1}, \varepsilon_{2}, \varepsilon_{3},$ 则 
$\A$在基$\varepsilon_{1}, \varepsilon_{2}, \varepsilon_{3}$下的矩阵为
$A=\left(\begin{array}{ccc}2 & 1 & 0 \\ 0 & 1 & 1 \\ 2 & 0 & -1\end{array}\right).$

将$A$做初等行变换化成阶梯形矩阵:
$$A\rightarrow\left(\begin{array}{ccc}2 & 1 & 0 \\ 0 & 1 & 1 \\ 0 & 0 & 0\end{array}\right)$$
取 $\alpha_{1}=(2,0,2)', \alpha_{2}=(1,1,0)'$, 
则 $\A V=L\left(\alpha_{1}, \alpha_{2}\right)$.

	\item  解方程组 $A X=0$, 得基础解系 $\eta=(1,-2,2)'$,则 $\A^{-1}(\0) =L(\eta)$.
\end{itemize}
\end{frame}



\begin{frame}
\begin{thm}
设 $\A  \in L(V), \operatorname{dim} V=n$,则 
\begin{itemize}
\item $\A (V)$ 的一组基的原象及 $\A^{-1}(\0)$ 的一组基合起来是 $V$ 的一组基,
\item $\A  \mbox{的秩}+ \A \mbox{ 的零度 }=n.$
\end{itemize}
%$$\operatorname{dim} \A (V)+\operatorname{dim} \A ^{-1}(0)=\operatorname{dim} V=n,$$ 即 

\end{thm}

\begin{exa}
设 $F[x]_{n}$ 的微分变换 $\D$, 
$\D  F[x]_{n}=F[x]_{n-1}, \D ^{-1}(\0)=F,$ 从而 $\operatorname{dim} \A (V)=n-1, \operatorname{dim} \A ^{-1}(\0)=1 .$
% $\A  V \cap \A^{-1}(\0)=F$
\end{exa}

\begin{rem}
	$\A  V$ 与 $\A^{-1}(\0)$ 的维数之和为 $n,$ 但是 $\A  V+\A^{-1}(\0)$ 不一定是直和
\end{rem}
\end{frame}

\begin{frame}
\pf 设 $\A (V)$ 的一组基为 $\eta_{1}, \eta_{2}, \cdots, \eta_{r},$ 其原象设为 $\varepsilon_{1}, \varepsilon_{2}, \cdots, \varepsilon_{r},$ 即 $\A  \varepsilon_{i}=\eta_{i}$, $i=1,2, \cdots, r$.
设  $\A^{-1}(\0)$ 的一组基为 $\varepsilon_{r+1}, \varepsilon_{r+2}, \cdots, \varepsilon_{s},$ 下证$$\varepsilon_{1}, \varepsilon_{2}, \cdots, \varepsilon_{r}, \varepsilon_{r+1}, \varepsilon_{r+2}, \cdots, \varepsilon_{s}$$ 是 $V$ 的一组基.

1)~~ 先证$\varepsilon_{1}, \varepsilon_{2}, \cdots, \varepsilon_{r}, \varepsilon_{r+1}, \varepsilon_{r+2}, \cdots, \varepsilon_{s}$ 线性无关.
设$$k_{1} \varepsilon_{1}+\cdots+k_{r} \varepsilon_{r}+k_{r+1} \varepsilon_{r+1}+\cdots+k_{s} \varepsilon_{s}=\0$$
用 $\A $ 作用一下, $$k_{1} \A  \varepsilon_{1}+\cdots+k_{r} \A  \varepsilon_{r}=\0,$$ 即 $$k_{1} \eta_{1}+\cdots+k_{r} \eta_{1}=\0.$$
因为  $\eta_{1}, \eta_{2}, \cdots, \eta_{r}$是$\A (V)$ 的一组基, 所以 $$k_{1}=\cdots=k_{r}=0.$$

\end{frame}

\begin{frame}
于是,
$$k_{r+1} \varepsilon_{r+1}+\cdots+k_{s} \varepsilon_{s}=\0.$$
因为
$\varepsilon_{r+1}, \varepsilon_{r+2}, \cdots, \varepsilon_{s}$ 为$\A^{-1}(\0)$的一组基, 所以 $$k_{r+1}=\cdots=k_{s}=0.$$
因此,$\varepsilon_{1}, \varepsilon_{2}, \cdots, \varepsilon_{r}, \varepsilon_{r+1}, \varepsilon_{r+2}, \cdots, \varepsilon_{s}$ 线性无关.\\[10pt]


2)~~  再证$V$中任意向量都可由$\varepsilon_{1}, \varepsilon_{2}, \cdots, \varepsilon_{r}, \varepsilon_{r+1}, \varepsilon_{r+2}, \cdots, \varepsilon_{s}$ 线性表示.
任给 $\alpha \in V$,则 $\A  \alpha \in \A (V)=L\left(\eta_{1}, \eta_{2}, \cdots, \eta_{r}\right),$ 从而
\begin{align*}
\A  \alpha& =k_{1} \eta_{1}+k_{2} \eta_{2}+\cdots+k_{r} \eta_{r}\\
& =k_{1} \A  \varepsilon_{1}+k_{2} \A  \varepsilon_{2}+\cdots+k_{r} \A  \varepsilon_{r}\\
& =\A \left(k_{1} \varepsilon_{1}+k_{2} \varepsilon_{2}+\cdots+k_{r} \varepsilon_{r}\right)
\end{align*}
即 $$\A \left(\alpha-k_{1} \varepsilon_{1}-k_{2} \varepsilon_{2}-\cdots-k_{r} \varepsilon_{r}\right)=\0,$$ 

\end{frame}

\begin{frame}
则 $$\alpha-k_{1} \varepsilon_{1}-k_{2} \varepsilon_{2}-\cdots-k_{r} \varepsilon_{r} \in \A^{-1}(\0),$$ 
即有
$$\alpha-k_{1} \varepsilon_{1}-k_{2} \varepsilon_{2}-\cdots-k_{r} \varepsilon_{r}=k_{r+1} \varepsilon_{r+1}+k_{r+2} \varepsilon_{r+2}+\cdots+k_{s} \varepsilon_{s}$$
从而 $$\alpha=k_{1} \varepsilon_{1}+k_{2} \varepsilon_{2}+\cdots+k_{r} \varepsilon_{r}+k_{r+1} \varepsilon_{r+1}+k_{r+2} \varepsilon_{r+2}+\cdots+k_{s} \varepsilon_{s}.$$


综上, $$\underbrace{\varepsilon_{1}, \varepsilon_{2}, \cdots, \varepsilon_{r},}_{\small{\A (V) \mbox{的一组基的原象}}} \underbrace{\varepsilon_{r+1}, \varepsilon_{r+2}, \cdots, \varepsilon_{s}}_{\small{\A^{-1}(\0) \mbox{的一组基}}}$$ 是 $V$ 的一组基, 从而$s=n$, 且
$\operatorname{dim} \A (V)+\operatorname{dim} \A ^{-1}(\0)=\operatorname{dim} V=n.$
\qed
%另外的证明: $\A $ 的秩+ $\A $ 的零度 $=\operatorname{dim} V$ 由于 $\A $ 的秩等于 $\A $ 在一组基下矩阵 $A$ 的秩,而 $\A $ 的奥度等于核的维数,即等于 $A X=0$ 的基础解系所含向
%量的个数 $n-r(A),$ 从而 $\A $ 的秩+ $\A $ 的零度 $=r(A)+n-r(A)=n$



\end{frame}

\begin{frame}
\begin{coro}
	设 $V$ 是一个有限维线性空间, $\A  \in L(V)$,则 $\A $ 是单射 $\Leftrightarrow \A $ 是满射.
\end{coro}
\pf
\begin{align*}
\A  \mbox{ 是单射} & \Leftrightarrow  \A ^{-1}(\0)=\0  \Leftrightarrow \operatorname{dim} \A ^{-1}(\0)=0\\
\A  \mbox{ 是满射} & \Leftrightarrow  \A (V)=V   \,\,\,\,\, \Leftrightarrow \operatorname{dim} \A (V)=n
\end{align*}

\end{frame}

\begin{frame}
\small{
\begin{exa}[*]
设 $n$ 阶方阵 $A$, 满足 $A^{2}=A,$ 证明 $A$ 可以相似于对角阵 $\left(\begin{array}{cc}E_{r} & 0 \\ 0 & 0\end{array}\right)$
\end{exa}

\pf 取线性空间 $V$ 及一组基 $\varepsilon_{1}, \varepsilon_{2}, \cdots, \varepsilon_{n},$ 则存在线性变换 $\A$,
使得 $\A$ 在 $\varepsilon_{1}, \varepsilon_{2}, \cdots, \varepsilon_{n}$ 下的矩阵为
$A$,
则 $\A^{2}=\A$.

 对 $\A V,$ 取一组基 $\eta_{1}, \eta_{2}, \cdots, \eta_{r},$ 其原象设为 $\alpha_{1}, \alpha_{2}, \cdots, \alpha_{r},$ 
 
即 $\A \alpha_{i}=\eta_{i}, i=1,2, \cdots, r$
则 $\A \eta_{i}=\A^{2} \alpha_{i}=\A \alpha_{i}=\eta_{i}, i=1,2, \cdots, r,$ 

即 $\eta_{1}, \eta_{2}, \cdots, \eta_{r}$ 得原象就是 $\eta_{1}, \eta_{2}, \cdots, \eta_{r} .$ 

取 $\A ^{-1}(\0)$ 的一组基
$\eta_{r+1}, \eta_{r+2}, \cdots, \eta_{n},$ 

则 $\eta_{1}, \eta_{2}, \cdots, \eta_{r}, \eta_{r+1}, \eta_{r+2}, \cdots, \eta_{n}$ 是 $V$ 的一组基, 

$\A$ 在 $\eta_{1}, \eta_{2}, \cdots, \eta_{n}$ 下的矩阵为
$\A\left(\eta_{1}, \eta_{2}, \cdots, \eta_{n}\right)=\left(\eta_{1}, \cdots, \eta_{r}, 0, \cdots, 0\right)=\left(\eta_{1}, \eta_{2}, \cdots, \eta_{n}\right)\left(\begin{array}{cc}E_{r} & 0 \\ 0 & 0\end{array}\right)$

从而由于一个线性变换在不同基下的矩阵是相似的.

故 $A$ 与 $\left(\begin{array}{cc}E_{r} & 0 \\ 0 & 0\end{array}\right)$ 相似.
}


\end{frame}
\section{不变子空间}
\begin{frame}{\S 7 不变子空间}
%1. 不变子空间
\begin{defi}
取线性空间 $V, \A  \in L(V),$ 设 $W$ 是 $V$ 的一个子空间,若 $W$ 中的向量在 $\A $ 作用下的象仍在 $W$ 中,即
任给 $\alpha \in W$,则有 $\A  \alpha \in W$,则称 $W$ 是 $\A $ 的一个不变子空间,筒称为 $\A -$ 子空间.
\end{defi}


\begin{exa}{非不变子空间}
取3维平面 $V=\mathbb{R}^{3}$,取线性变换 $\A $ 为绕 $X$ 轴线从 $Y$ 轴线到 $Z$ 轴线旋转 $90^{\circ}$ 角的变换
即 $\A (x, y, z)=(x,-z, y),$ 取子空间 $W=X O Y$ 面,则 $W$ 不是不变子空间, 因为 $\A (0,1,0)=(0,0,1) \notin W$.
\end{exa}

\begin{exa}
	平凡子空间 $\0$ 与 $V$ 是任一线性变换的不变子空间.
\end{exa}

\end{frame}

\begin{frame}


\begin{exa}
任一线性变换 $\A $ 的值域 $\A  V$ 与核 $\A ^{-1}(\0)$ 是 $\A $ 的不变子空间.
\end{exa}
\pf
\begin{itemize}
\item 任给 $\A  \alpha \in \A  V$, 则 $\A (\A  \alpha) \in \A  V$.
\item 任给 $\alpha \in \A ^{-1}(\0) $,则 $\A  \alpha=\0,$ 从而 $\A (\A  \alpha)=\A (\0)=\0,$ 因此 $\A  \alpha \in \A ^{-1}(\0).$
\end{itemize}




\begin{exa}
取 $\A , \B \in L(V)$,满足 $\A  \B =\B \A $,则 $\B V$ 与 $\B^{-1}(\0)$ 都是 $\A$-子空间
$\B V$.
\end{exa}
\pf
\begin{itemize}
\item 任给 $\B \alpha \in \B V,$ 证 $\A (\B \alpha) \in \B V$.
$\A (\B \alpha)=(\A  \B) \alpha=(\B \A ) \alpha= \B (\A  \alpha) \in \B V$.
\item 任给 $\alpha \in \B^{-1}(\0),$ 证 $\A  \alpha \in \B ^{-1}(\0),$ 
即证 $\B(\A  \alpha)=\A (\B \alpha)=\0.$
\end{itemize}


\end{frame}


\begin{frame}
\begin{exa}
	任一子空间都是数乘变换的不变子空间, 这是因为子空间对数乘封闭.
\end{exa}
{\bf 注~~ }

\begin{itemize}
	\item 特征向量与一维不变子空间的关系.

\begin{itemize}
	\item 取 $\A  \xi=\lambda_{0} \xi,$ 设 $W=L(\xi),$ 则 $\operatorname{dim} W=1, \A (k \xi)=\lambda_{0} k \xi \in W,$ 从而 $W=L(\xi)$ 是 $\A -$ 子空间.
	\item 反之,取一维不变子空间 W = L( $\xi), \A  W \subseteq W$,则 $\A  \xi \in W=L(\xi),$ 从而存在一个数 $\lambda,$ 使得 $\A  \xi=\lambda \xi,$ 即生成元 $\xi$ 是 $\A $ 的属于 $\lambda$ 的一个特征向量.
\end{itemize}

\item $\A$的属于特征值 $\lambda_{0}$ 的特征子空间 $V_{\lambda_{0}}$ 也是 $\mathscr{A}$ 的不变子
空间.

\item 取 $\A$-子空间 $W_{1}, W_{2},$ 则 $W_{1}+W_{2}$ 与 $W_{1} \cap W_{2}$ 都是 $\A$-子空间.

\end{itemize}

\end{frame}


\begin{frame}
\begin{defi}
设 $\A $ -子空间 $W$,由于 $\A  W \subseteq W$,则 $\A $ 就可以看做是 $W$ 上的一个线性变换, 
从而定义 $\left.\A \right|_{m}$ 为 $\A$ \alert{限制}在不
变子空间 W 上的线性变换.
\end{defi}



注意 $\A$和$\A|_W$的异同: 
\begin{itemize}
\item $\A$是$V$的线性变
换, $V$中每个向量在$\A$下都有确定的像; 
\item $\A|_W$是不变子空间$W$上的线性变换. 
它的作用为
任给 $\alpha \in W,\left.\A \right|_{m}(\alpha)=\A (\alpha),$ 但是 $\A $ 对 $W$ 之外的向量没有作用.

\end{itemize}

例如, 对于任一线性变换$\A$,
\begin{itemize}
\item  $\A$在它的核空间上引起的变换是零变换, 即 $\A |_{\A^{-1}(\0)} = \O$,
\item $\A$在特征子空间$V_{\lambda_0}$上引起的变换是数乘变换$\A|_{V_{\lambda_0}}(\xi)=\lambda_0 \xi$.
\end{itemize}
\end{frame}


\begin{frame}
\begin{prop}
若  $W=L\left(\alpha_{1}, \alpha_{2}, \cdots, \alpha_{m}\right)$, 则 
\begin{align*}
W \mbox{ 是 }\, \A -\mbox{子空间} \, \Leftrightarrow \,  
\A  \alpha_{1}, \A  \alpha_{2}, \cdots, \A  \alpha_{m} \in W.
\end{align*}
\end{prop}

\pf  
$\Rightarrow$ 若 $\A  W \subseteq W$,则 $\A  \alpha_{1}, \A  \alpha_{2}, \cdots, \A  \alpha_{s} \in W$.

$\Leftarrow$ 反之,若 $\A  \alpha_{1}, \A  \alpha_{2}, \cdots, \A  \alpha_{s} \in W$,则任给
$\alpha \in W,$ 设 $\alpha=\left(\alpha_{1}, \alpha_{2}, \cdots, \alpha_{s}\right) X,$ 则 $\A  \alpha=\left(\A  \alpha_{1}, \A  \alpha_{2}, \cdots, \A  \alpha_{s}\right) X \in W,$ 从而 $\A  W \subseteq W$. 
$W$ 是 $\A$-子空间.
\end{frame}



\begin{frame}{不变子空间与线性变换的矩阵}
1) 设 $\A$是 $n$ 维线性空间 $V$ 的线性变换,$W$ 是 $V$ 的 $\A$-子空
间. 

在 $W$ 中取一组基 ${\varepsilon}_{1}, {\varepsilon}_{2}, \cdots, {\varepsilon}_{k},$ 并且把它扩充成 $V$ 的一组基
\begin{align}\label{eq-b}
{\varepsilon}_{1}, {\varepsilon}_{2}, \cdots, {\varepsilon}_{k}, {\varepsilon}_{k+1}, \cdots, {\varepsilon}_{n},
\end{align}
 那么 $\A$在这组基下的矩阵就具有下列形状
 \begin{align}\label{eq-utm}
\left(\begin{array}{cccccc}a_{11} & \cdots & a_{1 k} & a_{1, k+1} & \cdots & a_{1 n} \\ \vdots & & \vdots & \vdots & & \vdots \\ a_{k 1} & \cdots & a_{k k} & a_{k, k+1} & & a_{k n} \\ 0 & \cdots & 0 & a_{k+1, k+1} & \cdots & a_{k+1, n} \\ \vdots & & \vdots & \vdots & & \vdots \\ 0 & \cdots & 0 & a_{n, k+1} & \cdots & a_{n n}\end{array}\right)=\left(\begin{array}{cc}A_{1} & A_{3} \\ 0 & A_{2}\end{array}\right)
 \end{align}
且左上角的 $k$ 级矩阵 ${A}_{1}$ 就是 $\mathscr{A} |_W$ 在 $W$ 的基 ${\varepsilon}_{1}, {\varepsilon}_{2}, \cdots, {\varepsilon}_{k}$ 下的矩阵. 

\end{frame}


\begin{frame}
这是因为 $W$ 是 $\A$-子空间,所以像 $\mathscr{A} {\varepsilon}_{1}, \mathscr{A} {\varepsilon}_{2}, \cdots, \mathscr{A} {\varepsilon}_{k}$ 仍在
$W$ 中. 

它们可以通过 $W$ 的基 ${\varepsilon}_{1}, {\varepsilon}_{2}, \cdots, {\varepsilon}_{k}$ 线性表示
\[
\begin{array}{l}
\mathscr{A} {\varepsilon}_{1}=a_{11} {\varepsilon}_{1}+a_{21} {\varepsilon}_{2}+\cdots+a_{k 1} {\varepsilon}_{k} \\
\mathscr{A} {\varepsilon}_{2}=a_{12} {\varepsilon}_{1}+a_{22} {\varepsilon}_{2}+\cdots+a_{k 2} {\varepsilon}_{k} \\
\ldots \ldots \ldots . \\
\mathscr{A} {\varepsilon}_{k}=a_{1 k} {\varepsilon}_{1}+a_{2 k} {\varepsilon}_{2}+\cdots+a_{k k} {\varepsilon}_{k}
\end{array}
\]
从而 
\begin{itemize}
\item 
$\A$在基${\varepsilon}_{1}, {\varepsilon}_{2}, \cdots, {\varepsilon}_{k}, {\varepsilon}_{k+1}, \cdots, {\varepsilon}_{n}$下的矩阵具有形状~\eqref{eq-utm}, 
\item
$\A|_W$ 在 $W$ 的基 ${\varepsilon}_{1}, {\varepsilon}_{2}, \cdots$
${\varepsilon}_{k}$ 下的矩阵是 ${A}_{1}$.\\[10pt]
\end{itemize}

反之,如果 $\A$在基(1)下的矩阵是(2),那么可以证明,由 ${\varepsilon}_{1}$ ${\varepsilon}_{2}, \cdots, {\varepsilon}_{k}$ 生成的子空间 $W$ 是 $\mathscr{A}$ 的不变子空间.
\end{frame}

\begin{frame}
2) 设 $V$ 分解成若干个 $\A$-子空间的直和:
$$
V=W_{1} \oplus W_{2} \oplus \cdots \oplus W_{s}
$$
在每一个 $\A$-子空间 $W$ 中取基
\begin{align}\label{eq-7-3}
{\varepsilon}_{i \, 1}, {\varepsilon}_{i \, 2}, \cdots, {\varepsilon}_{i \, n_{i}} \quad(i=1,2, \cdots, s)
\end{align}
并把它们合并起来成为 V 的一组基,则在这组基下的矩阵为准对角形
\begin{align}\label{eq-7-4}
\mbox{diag} \left(A_1, A_2, \ldots, A_s \right) 
\end{align}
其中 ${A}_{i}(i=1,2, \cdots, s)$ 就是 $\mathscr{A} | W_{i}$ 在基\eqref{eq-7-3}下的矩阵. 

反之,如果线性变换 A在某组基下的矩阵是准对角形\eqref{eq-7-4}, 则由
\eqref{eq-7-3} 生成的子空间 $W_{i}$ 是 $\mathscr{A}$-子空间.

由此可知,\alert{矩阵分解为准对角形}与\alert{空间分解为不变子空间的直和}是相当的.
\end{frame}


\begin{frame}
\begin{defi}
	设 $\lambda_{i}$ 是线性变换 $\A$ 的一个特征值,则
	\begin{itemize}
		\item 特征子空间 $V_{i}$ 维数 称为 $\lambda_{i}$ 的\alert{几何重数}.
		\item 每个特征根 $\lambda_{i}$ 在特征多项式 $f_{\A}\left(\lambda_{i}\right)$ 中的重数称为 \alert{代数重数}.
	\end{itemize}
	
\end{defi}
同一个特征值的代数重数和几 重数之间有如下关系:
\begin{theo}
	设 $\lambda_{i}$ 是线性变换 $\A$ 的特征值,它的代数重数为 $n_{i},$ 几何重 数为 $m_{i},$ 则
	\begin{enumerate}
		\item 每个特征值的几何重数小于或等于它的代数重数, 即$1 \le m_{i} \le n_i$.
		\item  $\A$ 的矩阵可以在某
		一组基下为对角矩阵 $\Leftrightarrow$ 每个特征值的几何重数都等于代数重数.
	\end{enumerate}
\end{theo}
\end{frame}


\begin{frame}
\pf (1)~~特征值$\lambda_i$的特征子空间 $V_{\lambda_i}$的维数等于$m_i$, 取$V_{\lambda_i}$的一组基扩充充为 $V$ 的一组基,则 $\A$在这组基下的矩阵为上三角矩阵 
$$B=
\left(
\begin{array}{cc} 
\lambda_i E_{m_i} & A_{3} \\
 0 & A_{2}
\end{array}
\right)$$
它的特征多项式$(\lambda-\lambda_{i})^{m_i} \det \left(\lambda E - A_{2} \right) $, 含有因子 $(\lambda-\lambda_{i})^{m_i}$.  于是$\lambda_{i}$在其中的代数重数$n_i \ge m_i$.

(2) 设 $\lambda_{1}, \quad \cdots, \lambda_{t}$ 是 $\mathscr{A}$ 的全部不同的特征值. 由\S 5 中的推论  知道
$\A$ 可对角化的充分必要条件是
\[
m_{1}+\cdots+m_{s}=n=n_{1}+\cdots+n_{s}
\]
成立. 
但 $m_{i} \leqslant n_{i}$ 对 $1 \leqslant i \leqslant s$ 成立,故$\A$ 可对角化当且仅当 $m_{i}=n_{i}$ 对所有的 $1 \leqslant i \leqslant s$ 成立。
\end{frame}


\begin{frame}
 下面我们应用哈密尔顿 - 凯莱定理将空间 $V$ 按特征值分解 成不变子空间的直和.  
 \begin{thm}[准素分解]
 设线性变换 $\A$ 的特征多项式为 $f(\lambda)$,它可分解成
一次因式的乘积
\[
f(\lambda)=\left(\lambda-\lambda_{1}\right)^{r_{1}}\left(\lambda-\lambda_{2}\right)^{r_{2}} \cdots\left(\lambda-\lambda_{s}\right)^{r_{s}}
\]
则 V 可分解成不变子空间的直和
\[
V=V_{1} \oplus V_{2} \oplus \cdots \oplus V_{s},
\]
其中 
\[
V_{i}=\left\{\xi \, |\, \left(  \A -\lambda_{i} \mathscr{E} \right)^{r_i} \xi=0, \, \xi \in V\right\}.
\]
 \end{thm}
通常将$V_i$称为根子空间.
\end{frame}


\section{若尔当(Jordan)标准形介绍}
\begin{frame}{\S 若尔当(Jordan)标准形介绍}
一般来说, 并不是对于每一个线性变换 都有一组基,使它在这组基下的矩阵成为对角形.

这一节的目的就是考察一般的一个方阵相似最简形式是什么. 我们的讨论限制在\alert{复数域}中.
\begin{defi}
形式为
\[
{J}(\lambda, t)=
\left(\begin{array}{cccccc}
\lambda & 0 & \cdots & 0 & 0 & 0 \\
1 & \lambda & \cdots & 0 & 0 & 0 \\
\vdots & \vdots & & \vdots & \vdots & \vdots \\
0 & 0 & \cdots & 1 & \lambda & 0 \\
0 & 0 & \cdots & 0 & 1 & \lambda
\end{array}\right)_{t \times t}
\]
的矩阵称为Jordan块,其中$\lambda$是复数. 
%  用$J(\lambda,t)$表示$t$ 阶对角线上元素是$\lambda$的Jordan块.

\end{defi}
\end{frame}




\begin{frame}
\begin{defi}
由Jordan块组成的准对角矩阵称为Jordan形矩阵, 其一般形状如
\[
\left(\begin{array}{cccc}
J(\lambda_{1}, k_1) & & & \\
& J(\lambda_{2}, k_2)& & \\
& & \ddots & \\
& & &J(\lambda_{s}, k_s)
\end{array}\right)
\]
并且 $\lambda_{1}, \lambda_{2}, \cdots, \lambda_s$ 中有一些可以相等.
\end{defi}

\end{frame}



\begin{frame}
例如
\[
\left(\begin{array}{lll}
2 & 0 & 0 \\
1 & 2 & 0 \\
0 & 1 & 2
\end{array}\right),
\left(\begin{array}{cccc}
0 & 0 & 0 & 0 \\
1 & 0 & 0 & 0 \\
0 & 1 & 0 & 0 \\
0 & 0 & 1 & 0
\end{array}\right),
\left(\begin{array}{cc}
i & 0 \\
1 & i
\end{array}\right)
\]
都是若尔当块,而
\[
\left(\begin{array}{cccccc}
1 & 0 & 0 & 0 & 0 & 0 \\
1 & 1 & 0 & 0 & 0 & 0 \\
0 & 0 & 4 & 0 & 0 & 0 \\
0 & 0 & 0 & 4 & 0 & 0 \\
0 & 0 & 0 & 1 & 4 & 0 \\
0 & 0 & 0 & 0 & 1 & 4
\end{array}\right)
\]
是一个若尔当形矩阵.
\end{frame}

\begin{frame}
\begin{rem}
	$
	\left(\begin{array}{cccccc}
	\lambda & 0 & \cdots & 0 & 0 & 0 \\
	1 & \lambda & \cdots & 0 & 0 & 0 \\
	\vdots & \vdots & & \vdots & \vdots & \vdots \\
	0 & 0 & \cdots & 1 & \lambda & 0 \\
	0 & 0 & \cdots & 0 & 1 & \lambda
	\end{array}\right)_{t}
	$
	与
	$
	\left(\begin{array}{cccccc}
	\lambda & 1 & \cdots & 0 & 0 & 0 \\
	0 & \lambda & \cdots & 1 & 0 & 0 \\
	\vdots & \vdots & & \vdots & \vdots & \vdots \\
	0 & 0 & \cdots & 0 & \lambda & 1 \\
	0 & 0 & \cdots & 0 & 0 & \lambda
	\end{array}\right)_{t}
	$
	相似.	
\end{rem}
\pf 

令
$P=
\left(\begin{array}{cccccc}
0 & 0 & \cdots & 0 & 0 & 1 \\
0 & 0 & \cdots & 0 & 1 & 0 \\
\vdots & \vdots & & \vdots & \vdots & \vdots \\
0 & 1 & \cdots & 0 & 0 & 0 \\
1 & 0 & \cdots & 0 & 0 & 0
\end{array}\right)_{t}
$, 则 $P^{-1}{J}(\lambda, t)P$等于右边的矩阵.
\end{frame}

\begin{frame}
\begin{rem}
\begin{itemize}
	\item 上面的$\lambda_{1}, \lambda_{2}, \cdots, \lambda_s$ 中
	有一些可以相等.
	\item 一阶Jordan块就是一阶矩阵,因此Jordan 形矩阵中包括对角矩阵.
	\item 因为Jordan形矩阵是下三角形矩阵, 所以, Jordan标准形中, 主对角线上的元素正是特征多项式的全部的根(重根按重数计算).
\end{itemize}
\end{rem}


\end{frame}
\begin{frame}
\begin{thm}
设复数域上线性空间$V$ 的线性变换$\A$, 则存在$V$的一组基, 使得$\A$ 在此组基下的矩阵为Jordan形矩阵.
\end{thm}



\begin{thm}
	设$A$是复数域上的一个方阵,
则$A$相似于一个Jordan形矩阵. 
\end{thm}


\end{frame}



\section{最小多项式}
\begin{frame}{\S 9  最小多项式}
\begin{defi}
设$A$是数域$P$上$n$阶方阵,$f(x)\in P[x]$. 
\begin{itemize}
\item 如果$f(A)=0$, 那么称$f(x)$是$A$的\alert{零化多项式}.
\item 如果$f(x)$首1,且$f(x)$是A的零化多项式中
次数最低者, 那么称$A$的零化多项式$f(x)$是$A$的\alert{最小多项式}.
\end{itemize}
\end{defi}
\end{frame}


\begin{frame}
\begin{lem}
数域$P$上$n$阶方阵$A$的最小多项式唯一
\end{lem}
\pf 
设 $g_{1}(x)$ 和 $g_{2}(x)$ 都是 ${A}$ 的最小多项式,根据带余除
法,$g_1(x)$ 可表成
\[
g_{1}(x)=q(x) g_{2}(x)+r(x)
\]
其中 $r(x)=0$ 或 $\deg(r(x))<\deg \left(g_{2}(x)\right)$, 于是
\[
g_{1}({A})=q({A}) g_{2}({A})+r({A})={O}
\]
因此 $r({A})={O} .$ 由最小多项式的定义 $, r(x)=0,$ 即 $g_{2}(x) |$
$g_{1}(x)$. 

同样可证 $g_{1}(x) | g_{2}(x) .$ 因此 $g_{1}(x)$ 与 $g_{2}(x)$ 只能相差 一个非零常数因子.

又因 $g_{1}(x)$ 与 $g_{2}(x)$ 的首项系数都等于 1, 所以 $g_{1}(x)=g_{2}(x)$
\qed
\end{frame}


\begin{frame}
%应用同样的方法, 可以证明
\small{
\begin{lem}
	设$m(x)$是方阵$A$的最小多项
式,$f(x) \in P[x]$, 则
\begin{center}
$f(x)$是$A$的零化多项式 $\Leftrightarrow  m(x)|f(x)$.
\end{center}
\end{lem}
\pf 
$\Rightarrow$
根据带余除法,$f(x)$ 可表成
\[
f(x)=q(x) m(x)+r(x)
\]
其中 $r(x)=0$ 或 $\deg(r(x))<\deg \left(m(x)\right)$.
若$r(x)\neq 0$, 
则由$$r({A}) =f({A})-q({A}) m({A})={O}$$
可知$r(x)$也是$A$的零化多项式. 但$\deg(r(x))<\deg \left(m(x)\right)$与设$m(x)$是方阵$A$的最小多项
式矛盾. 因此,$r(x)=0$, 进而$m(x)|f(x)$.

$\Leftarrow$
 因为$m(x)|f(x)$, 所以存在$q(x) \in P[x]$使得$f(x)=q(x) m(x)$.
又因为$m(x)$是$A$的零化多项式, 所以$f(A)=q(A) m(A)={O}.$
即 $f(x)$是$A$的零化多项式.
\qed

\begin{coro}
最小多项式整除特征多项式. 
\end{coro}
}

\end{frame}


\begin{frame}
\begin{exa}
\begin{itemize}
\item 数量矩阵$kE$的最小多项式是
$x-k$.  反过来, 
\item 最小多项式是一次的矩阵
必是数量矩阵. 
\end{itemize}
\end{exa}


\begin{exa}
\[
{A}=\left(\begin{array}{ccc}
1 & 1 & \\
& 1 & \\
& & 1
\end{array}\right)
\]
求 A 的最小多项式.  
\end{exa}
\sol 
因为 A 的特征多项式为 $|x {E}-{A}|=(x-1)^{3}$,
所以 A 的最小多项式为$(x-1)^{3}$ 的因式.
因为$A-E \neq O$ 而 $({A}-{E})^{2}={O}$, 所以  ${A}$ 的最小多项式为 $(x-1)^{2}$.
\end{frame}


\begin{frame}
	content...
\end{frame}

\begin{frame}
\begin{prop}
如果矩阵 A 与B 相似: ${B}={T}^{-1} {A T},$ 那么对任一多项式
$f(x), f({B})={T}^{-1} f({A}) {T} .$ 因此 $f({B})={O}$ 当且仅当 $f({A})={O}$
这说明相似矩阵有相同的最小多项式.
\end{prop}
但是需要注意, 这个条件并 不是充分的, 即最小多项式相同的矩阵不一定是相似的.下面的例 子说明这个结论 例 3 设
\[
{A}=\left(\begin{array}{cccc}
1 & 1 && \\
& 1&& \\
&& 1&\\
&&&2
\end{array}\right), 
{B}=\left(\begin{array}{cccc}
1 & 1 &  &\\
& 1  &&\\
& & 2 &\\
& & &2
\end{array}\right)
\]
${A}$ 与 ${B}$ 的最小多项式都等于 $(x-1)^{2}(x-2),$ 但是它们的特征手 项式不同,因此 A 和 B 不是相似的.
\end{frame}

\begin{frame}
\small{
\begin{lem}
设 $A$ 是一个准对角矩阵
\[
{A}=\left[\begin{array}{ll}
{A}_{1} & \\
& {A}_{2}
\end{array}\right]
\]
并设 $A_{1}$ 的最小多项式为 $g_{1}(x), {A}_{2}$ 的最小多项式为 $g_{2}(x)$,那么 A 的最小多项式为 $g_{1}(x), g_{2}(x)$ 的最小公倍式 $\left[g_{1}(x), g_{2}(x)\right]$
\end{lem}
\pf 
记 $g(x)=\left[g_{1}(x), g_{2}(x)\right]$,首先
\[
g({A})=\left[\begin{array}{cc}
g\left({A}_{1}\right) & \\
& g\left({A}_{2}\right)
\end{array}\right]={O}
\]
因此 $g(x)$ 能被 $A$ 的最小各项式整除.其次, 如果 $h({A})={O}$,那么
\[
h({A})=\left[\begin{array}{cc}
h\left({A}_{1}\right) & \\
& h\left({A}_{2}\right)
\end{array}\right]={O}
\]
所以 $h\left({A}_{1}\right)={O}, h\left({A}_{2}\right)={O} .$ 因而 $g_{1}(x)\left|h(x), g_{2}(x)\right| h(x)$ 并
由此得 $g(x) | h(x)$.这样就证明了 $g(x)$ 是 $\mathbf{A}$ 的最小多项式.
\qed}
\end{frame}

\begin{frame}
这个结论可以推广到 $A$ 为若干个矩阵组成的准对角矩阵的 情形.
\begin{coro}
如果
\[
{A}=\left(\begin{array}{cccc}
{A}_{1} & & & \\
& {A}_{2} & & \\
& & \ddots & \\
& & & {A}
\end{array}\right)
\]
${A}_{i}$ 的最小多项式为 $g_{i}(x), i=1,2, \cdots, s,$ 那么 ${A}$ 的最小多项式为
$\left[g_{1}(x), g_{2}(x), \cdots, g_{s}(x)\right]$
\end{coro}
\end{frame}


\begin{frame}
\begin{lem}
$k$级Jordan块
\[
{J}=\left(\begin{array}{cccc}
a & & & \\
1 & \ddots & & \\
& \ddots & \ddots & \\
& & 1 & a
\end{array}\right)
\]
的最小多项式为 $(x-a)^{k}$.
\end{lem}
\pf 
${J}$ 的特征多项式为 $(x-a)^{k},$ 而
所以 $J$ 的最小多项式为( $x-a)^{k}$. 
\qed
\end{frame}


\begin{frame}

\begin{thm}
数域 $P$ 上 $n$ 级矩阵$A$ 与对角矩阵相似  $\Leftrightarrow A$ 的最小多项式是 $P$ 上互素的一次因式的乘积. 
\end{thm}
%\pf  根据引理 3 的推广的情形,条件的必要性是显然的. 
%现在证明充分性.  根据矩阵和线性变换之间的对应关系,我们可定义任意线性 变换.
%
%$\A$的最小多项式,它等于其对应矩阵 A 的最小多项式.我们 只要证明,若数域 P 上某线性空间 V 上的线性变换 . 1的最小多
%项式 $g(x)$ 是 $P$ 上互素的一次因式的乘积 $: g(x)=\prod_{i=1}^{1}\left(x-a_{i}\right)$ 则 \&有一组特征向量做成 V 的基.  实际上,由于 $g(\mathscr{N}) V=\mathbf{0},$ 用定理 12 中同样步骤可证 $V=$ $V_{1} \oplus \cdots \oplus V_{i},$ 其中 $V_{i}=\left\{\xi |\left(\emptyset /-a_{i} \mathscr{E}\right) \xi=\mathbf{0}, \xi \in V\right\} .$ 把 $V_{1}, \cdots$
%$V,$ 各自的基合起来就是 $V$ 的基. 而 每个基向量都属于某个 $V_{i}$
%因而是 A 的特征向量

\begin{coro}
复数矩阵 $A$ 与对角矩阵相似$\Leftrightarrow  A$的最小多项式没有重根.
\end{coro}
\end{frame} 


\begin{frame}
\begin{exa}
设复矩阵$A$满足$A^\ell=E$, 其中$\ell$为正整数. 证明$A$相
似于对角形. 
\end{exa} 
\pf 因为$A$的最小多项式是$x^\ell-1$ 的因式,
从而无重根,由推论知A相似于对角形. 
\qed

\begin{exa}
	设矩阵$A$满足$A^{2}=A$. 证明$A$相
	似于对角形. 
\end{exa} 
\pf 因为$A$的最小多项式是$x^2-x$ 的因式,
从而无重根,可知A相似于对角形. 
\qed

\end{frame}


\end{document} 