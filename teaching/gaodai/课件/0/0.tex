% !Mode:: "TeX:UTF-8"
\documentclass[13pt]{beamer}
\usepackage[utf8]{inputenc}


\usepackage{amsmath,amssymb,amsthm}             % AMS Math
\usepackage[T1]{fontenc}

\usepackage{graphicx}
\usepackage{epstopdf}
\usepackage{tikz}
\linespread{1.3}

\usepackage{mathrsfs}  %花写字母
 

%%%=== theme ===%%%
% \usetheme{Warsaw}
%\usetheme{Copenhagen}
%\usetheme{Singapore}
\usetheme{Madrid}
%\usefonttheme{professionalfonts}
%\usefonttheme{serif}
% \usefonttheme{structureitalicserif}
%%\useinnertheme{rounded}
%%\useinnertheme{inmargin}
\useinnertheme{circles}
%\useoutertheme{miniframes}
\setbeamertemplate{navigation symbols}{}
\setbeamertemplate{footline}[page number]




\usepackage{ctex}
% \usepackage{CJK,CJKnumb,CJKulem}
\usepackage{minitoc}


\setbeamertemplate{theorems}[numbered]
\newtheorem{thm}{定理}
\newtheorem{lem}{引理}
\newtheorem{exa}{例}
\newtheorem*{theo}{定理}
\newtheorem*{conj}{猜想}
\newtheorem*{defi}{定义}
\newtheorem*{coro}{推论}
\newtheorem*{ex}{练习}
\newtheorem*{rem}{注}
\newtheorem*{prop}{性质}
\newtheorem*{qst}{问题}

\def\qed{\nopagebreak\hfill{\rule{4pt}{7pt}}\medbreak}
\def\pf{{\bf 证明~~ }}
\def\sol{{\bf 解~~ }}



\def\R{\mathbb{R}}
\def\Rn{\mathbb{R}^n}
\def\A{\mathscr{A}}
\def\B{\mathscr{B}}
\def\D{\mathscr{D}}
\def\E{\mathscr{E}}
\def\O{\mathscr{O}}

\def\rank{\operatorname{rank}}
\def\dim{\operatorname{dim}}
\def\0{\mathbf{0}}
\def\a{\alpha}
\def\b{\beta}
\def\r{\gamma}

\usepackage{color}
\definecolor{linkcol}{rgb}{0,0,0.4}
\definecolor{citecol}{rgb}{0.5,0,0}

\definecolor{blue}{rgb}{0,0.08,1}
\newcommand{\blue}{\textcolor{blue}}

  \usepackage{graphicx}
  \DeclareGraphicsExtensions{.eps}
%   \usepackage[a4paper,pagebackref,hyperindex=true,pdfnewwindow=true]{hyperref}


\begin{document}



\title[]{高等代数}
\author[]{{\large 张彪}\\  }
\institute[]{{\normalsize
		天津师范大学\\[6pt]
		zhang@tjnu.edu.cn}}

\date{}


\AtBeginSection[]
{
\setcounter{exa}{0}
\setcounter{equation}{0}
}


\begin{frame}
\maketitle
\end{frame}

\begin{frame}{Outline}
\tableofcontents
\end{frame}





\section{引言}
\begin{frame}
大约在 1637 年,法国数学家 Fermat 断言,
\begin{conj}
	对于大于 2 的整数 $n$,三个未知量 $x, y, z$ 的代数方程 $x^{n}+y^{n}=z^{n}$ 没有正整数解. 
\end{conj}
\pause

%这个问题中,只牵涉到正整数的加法与乘 法(乘方)运算,可说是再简单不过了,具有初中一年级代数知识的人 都能看明白.

%但是
它历经 350 余年,无数第一流的数学家为之纹尽脑汁, 才于 1994 年被 Princeton 大学的数学家 Wiles 使用现代最深奥 的数学理论得出解答. 

\end{frame}



\begin{frame}
\begin{itemize}
\item 从生产实践和自然科学理论中,自然地产生了\alert{求解代数方
程}的问题,它就是代数学的经典课题.
\item 
例如,根据牛顿第二运动定律, 物体所受的力 F,它的质量 $m$ 和产生的加速度 $a$ 之间存在关系 $F=$ ma. 
如果已知物体的质量 $m$ 和所受的力 $F$,求加速度 $a,$ 这就是\alert{一元 一次方程}的求解问题.
\item  又比如,一个以初速 $v_{0}$ 在水平面上作匀加速 运动的物体,它的加速度 a,运动时间 $t$ 和移动的距离 S 满足
\[
S=v_{0} t+\frac{1}{2} a t^{2}
\]
如果已知 $S, v_{0}, a,$ 求运动时间 $t,$ 这就是求\alert{一元二次方程}的根.
\end{itemize}


 

\end{frame} 


\begin{frame}
\begin{itemize}
\item 数学 史表明,早在中世纪人们就已经找到解一元一次、二次代数方程的一 般方法. 
\item 到欧洲的文艺复兴时代,又找到一元三次、四次方程的求根 公式. 
\item 但是随后数学家们就碰到难题了. 在数百年时间内,他们苦苦 寻求五次以上代数方程的求根公式,却总是遭到失败. 
\pause
\item 直到 1832 年, 法国数学家 Galois 才找到了一个高次代数方程有根式解(即用该方 程的系数经加、减、乘、除及开方运算表示它的全部根)的判别准则, 完满地解决了高次代数方程根的理论课题. 
\end{itemize}
\end{frame}


\begin{frame}
根据 Galois 的理论,五 次以上的一般代数方程没有求根公式. 

Galois 的工作中最值得注意 的是,他不是局限在数的四则运算的范围内考查问题. 

他跳出这个圈 子,考查 $n$ 次方程的 $n$ 个根的某些\alert{置换}所组成的集合 $G,$ 规定 $G$ 内两 个置换的“乘积”是对根的集合逐次进行这两个置换. 

他在一个 并非由数组成的集合 G 内定义了一种新的代数运算:\alert{乘法}(它完全 不同于数的乘法). 他发现这种乘法也具有与数的乘法相类似的某些 运算法则(例如满足结合律等等). 这个新的具有乘法运算的集合我 们现在把它称为该高次代数方程的 \alert{Galois 群}. 

Galois 证明:
\begin{center}
高次代 数方程有没有根式解取决于它的 Galois 群的结构. 
\end{center}

\end{frame}

\begin{frame}
这样,人们的认 识发生了一个质的飞跃,那就是为了研讨数及其代数运算中所包含 的深刻规律,我们必须跳出数及其四则运算的框框,去研究一个更一 般的集合及其中应有的代数运算.

 这样 ,代数学发生了一个革命性的 变化:从研究代数方程的求根这一经典课题解脱出来,变成研究一 个一般的\alert{集合}(其元素可以完全抽象,没有具体内容),在其中存在一 种或若干种\alert{代数运算}(这种运算不同于数的四则运算,甚至可以是抽 象定义的),同时要求这些运算要满足一定的运算法则. 
\end{frame}


\begin{frame}{Outline}
\alert{第一学期}
\begin{itemize}
\item[1]多项式
\item[2] 行列式
\item[3] 线性方程组
\end{itemize}

\alert{第二学期}
\begin{itemize}
\item[4] 矩阵
\item[5] 二次型
\item[6] 线性空间
\item[7] 线性变换
\item[9] 欧几里得空间
\end{itemize}

\end{frame}


\section{充分必要条件}

\begin{frame}{充分条件和必要条件}
设A与B为两命题,
\begin{itemize}
	\item A的\alert{充分条件}是B
	
	如果B成立,那么A成立,即$A\Leftarrow B$(箭头表示能够推导出)
	
	\item A的\alert{必要条件}是B
	
	如果A成立,那么B成立,即$A\Rightarrow B$.
	
	\item A的\alert{充分必要条件}是B
	\begin{itemize}
		\item 充分性 $A\Leftarrow B$
		\item 必要性 $A\Rightarrow B$
	\end{itemize}
	
\end{itemize}


\end{frame}
\begin{frame}{当且仅当}
当且仅当(英文:If and only if, 或者:iff),在数学、哲学、逻辑学以及其他一些技术性领域中广泛使用,在英语中的对应标记为iff。

设A与B为两命题,在证明
\begin{center}
A当且仅当B
\end{center}
时,这相当于去同时证明陈述
\begin{itemize}
\item 如果A成立,则B成立
\item 如果B成立,则A成立
\end{itemize}


%另外,也可以证明“如果A成立,则B成立”和“如果A不成立,则B不成立”,后者作为对偶,等价于“如果B成立,则A成立”。

公认的其他同样说法还有
\begin{center}
B是A的充分必要条件(或称为充要条件).
\end{center}
%
%一般而言,当我们看到“A当且仅当B”,我们可以知道“如果A成立时,则B一定成立”、“如果B成立时,则A也一定成立”、“如果A不成立时,则B也一定不成立”、“如果B不成立时,则A也一定不成立”。
%
%当且仅当A(命题)成立时,B(命题)成立。
%
%也可表示成:B(命题)成立时,A(命题)成立 ;A(命题)成立时,B(命题)成立。即B(命题)等价于A(命题)。
%
%通俗一点来说,就是“在这些情况下,并且仅仅在这些情况下”。

\end{frame}



\section{数学归纳法}

\begin{frame}{第一数学归纳法}


如果你有一排很长的直立着的多米诺骨牌那么如果你可以确定:

第一张骨牌将要倒下. 

只要某一个骨牌倒了,与他相临的下一个骨牌也要倒. 

那么你就可以推断所有的的骨牌都将要倒. \\[15pt]

\pause

第一数学归纳法可以概括为以下三步:
\begin{enumerate}
\item 归纳奠基:证明$n=1$时命题成立.
\item 归纳假设:假设$n=k$时命题成立.
\item 归纳递推:由归纳假设推出$n=k+1$时命题也成立.
\end{enumerate}
\end{frame}

\begin{frame}
\begin{exa}
证明对于任意自然数$n$,下面的公式都成立
\[
1+2+3+\cdots+n=\frac{n(n+1)}{2}.
\]
\end{exa}
%这是用于计算前 $n$ 个自然数之和的公式. 

\pf 
\begin{itemize}
\item 这个公式在 $n=1$ 时成立.  左边 = 1,右边 = 1(1+ 1) / 2 = 1. 

所以这个公式在 $n=1$ 时成立. 
\item 我们假设 $n=m$ 时公式成立,即
\[
1+2+\cdots+m=\frac{m(m+1)}{2}.
\]
\item 在上式等号两边分别加上 $m+1$ 得到
\[
1+2+\cdots+m+(m+1)=\frac{m(m+1)}{2}+(m+1) = \frac{(m+1)(m+2)}{2}.
\]
这就是 $n=m+1$ 时的等式. 

因此,对于任意自然数等式都成立. \qed
\end{itemize}
\end{frame}

\begin{frame}
\begin{exa}
对于任意自然数$n$证明$3^n−1$ 是 $2$ 的倍数.
\end{exa}
\pf 

\begin{itemize}
	\item  $3^1−1 = 3−1 = 2$是$ 2 $的倍数.  所以, 当 $n=1$ 时命题成立. 
	\item 我们假设 $n=k$ 时命题成立, 即 $3^{3k}−1$ 是 $2$ 的倍数.
	\item 接下来证明$n=k+1$ 时命题也成立. 
	\begin{align*}
		3^{3k+1}-1 & = 2 \cdot 3^{3k}+(3^{3k}-1 )
	\end{align*}
	$2 \cdot 3^{3k}$是 $2$ 的倍数.
	由归纳假设,$3^{3k}−1$ 是 $2$ 的倍数.
	又因为$2 \cdot 3^{3k}$也是 $2$ 的倍数, 
	所以	$3^{3k+1}-1$是 $2$ 的倍数.
	
	因此,对于任意自然数$n$,都有 $3^n−1$ 是 $2$ 的倍数.
	 \qed
\end{itemize}
\end{frame}

\begin{frame}{第二数学归纳法}
有些命题用第一归纳法证明不大方便,可以用第二归纳法证
明. 

第二数学归纳法的证明步骤是:
\begin{enumerate}
\item 证明当 $n=1$时命题成立.
\item 假设$n\le k$ 时命题都成立.
\item 由归纳假设推出 $n=k+1$时命题也成立.
\end{enumerate}
\end{frame}

\section{复数}
\begin{frame}
高中的时候,定义了
\[
i=\sqrt{-1}
\]
然后形如:
\[
a+b i \quad(a, b \in \mathbb{R})
\]
这样的数就是复数。
全体复数的集合记为
\[
\mathbb{C}=\{a+b i \, | \, a, b \text { 取所有实数 }\}
\]
%集合 $\mathbb{C}$包含实数集合 $\mathbb{R}$. 


有了复数之后,开方运算就不再局限于大于0的数了,这样一元二次方 程:
\[
a x^{2}+b x+c=0 \quad(a \neq 0)
\]
就总是有解了:
\[
x=\frac{-b \pm \sqrt{b^{2}-4 a c}}{2 a}
\]


\end{frame}

\begin{frame}
\begin{itemize}
\item 定义$\mathbb{C}$内的加法
$$(a+b i)+(c+d i)  : =(a+c)+(b+d) i $$

\item 定义 $a+b {i}$的负数 $-(a+b {i})${ 是 } $(-a)+(-b) {i}$

\item 定义  $\mathbb{C}$内的减法 $$(a+b i)-(c+d i) =(a-c)+(b-d) i $$
\end{itemize}

\end{frame}

\begin{frame}
\begin{itemize}
	\item 定义$\mathbb{C}$内的乘法
\[
(a+b i)(c+d i) =(a c-b d)+(a d+b c) i
\]

\item 定义 $a+b {i}$ 的倒数或逆
\[
\frac{1}{a+b i}=\frac{1}{a^{2}+b^{2}}(a-b i)=\frac{a-b i}{a^{2}+b^{2}}
\]
\item  $\mathbb{C}$ 内的除法是(设 $c+d {i} \neq 0$ )
\[
\frac{a+b i}{c+d i}=(a+b i) \frac{1}{c+d i}=(a+b i) \frac{c-d i}{c^{2}+d^{2}}
\]
\end{itemize}
\end{frame}

\begin{frame}{复数的表示:实部、虚部、共轭、模}

\begin{defi}
对于复数$z=a+b i,$ 其中 $a, b$ 是实数.

\begin{itemize}
\item $a$ 称为$z$的\alert{实部}, 记为$\text{Re} z$

\item $b$ 称为$z$的\alert{虚部}, 记为$\text{Im}  z$

\item 复数$z=a+b i$的\alert{共轭} $\bar{z}:=a-bi$ 
\item  $|z|=\sqrt{a^{2}+b^{2}}$ 称为 $a+b i$ 的\alert{模}或绝对值。
\end{itemize}
\end{defi}
\begin{prop}
\begin{itemize}
	\item $z  \bar{z}=(a+b i)(a-b i)=a^{2}+b^{2}$.
	\item $z+\bar{z}=(a+b i)+(a-b i)=2 a$.
	\item $z-\bar{z}=(a+b i)-(a-b i)=2 b i$.
\end{itemize}
\end{prop}
\end{frame}



\begin{frame}
将 $Ox$ 轴正方向沿反时针方向旋转到直线 $OA$ 的旋转角 $\varphi$ 称为复数 $\alpha=a+b {i}$ 的\alert{辐角}. 辐角的值不是唯一确定的,可以加上 $2 \pi$ 的任意整数倍. 

因为 $a=|\alpha| \cos \varphi, b=|\alpha| \sin \varphi,$ 故有
\[
\alpha=a+b {i}=|\alpha|(\cos \varphi+{i} \sin \varphi)
\]
上式称为复数的三角表示. 
\begin{center}
\begin{tikzpicture}
\draw [fill=black](0, 0) [radius=0.03cm] circle;   % 原点
\draw [fill=black](2, 1) [radius=0.03cm] circle;   % 原点
\draw[->](0,0)--(2.5,0);
\draw[->](0,0)--(0,1.8);
\draw[-](0,0) node[below left]{$O$}--(2,1) node[above right]{$\alpha$};
\draw[->](-1/2,0)--(2.5,0) node[right]{x};
\draw[->](0,-1/2)--(0,1.8) node[above]{y};
\draw (2/3, 0) node[above right]{$\varphi$} arc (0 : 18 : 1);
\end{tikzpicture}
\end{center}
\end{frame}

\begin{frame}
如果又有复数
\[
\beta=c+d {i}=|\beta|(\cos \psi+\operatorname{isin} \psi)
\]
那么
\[
\begin{aligned}
\alpha \beta &=|\alpha||\beta|(\cos \varphi+i \operatorname{sin} \varphi)(\cos \psi+ i \operatorname{sin} \psi) \\
&=|\alpha||\beta|(\cos \varphi \cos \psi-\sin \varphi \sin \psi)+(\sin \varphi \cos \psi+\cos \varphi \sin \psi) {i}) \\
&=|\alpha||\beta|(\cos (\varphi+\psi)+i \operatorname{sin}(\varphi+\psi))
\end{aligned}
\]
上式表示,两个复数相乘时,其模为这两个复数的模相乘,其辐角相 加(因为三角函数以 2 $\pi$ 为周期,故把相差 2 $\pi$ 的整数倍的角认为是相 同的).
\end{frame}

\begin{frame}{欧拉公式}
令
\[
{e}^{{i} \varphi}=\cos \varphi+{i} \sin \varphi
\]
上式表示的复数模为 1,称为复数的欧拉公式. 

因而位于以坐标原点 O 为中心的单位圆上, 其辐角为 $\varphi$. 于是
\[
{e}^{{i} \varphi} {e}^{{i} \psi}={e}^{{i}(\varphi+\psi)}
\]


当$\varphi$为$\pi$时, $$ e^{i\pi }=-1$$
将数学内4个极重要的数$e, i, \pi, -1 $连起来.
\end{frame}



\begin{frame}
给定正整数 $n$,考查下列 $n$ 个复数
\[
{e}^{\frac{2 k \pi_{{i}}}{n}}=\cos \frac{2 k \pi}{n}+i\operatorname{sin} \frac{2 k \pi}{n}
\]
其中 $k=0,1,2, \cdots, n-1.$ 这 $n$ 个复数就是以坐标原点 $O$ 为中心的单 位圆的内接正 $n$ 边形的 $n$ 个顶点.
于是
\[
\left({e}^{\frac{2 {k\pi i}}{n}}\right)^{n}=\left(\cos \frac{2 k \pi}{n}+i \operatorname{sin} \frac{2 k \pi}{n}\right)^{n}=\cos 2 k \pi+{i} \sin 2 k \pi=1
\]
因此,上面 $n$ 个复数 ${e}^{\frac{2 k \pi_{{i}}}{n}}=\cos \frac{2 k \pi}{n}+$ isin $\frac{2 k \pi}{n}$ 恰为 $n$ 次代数方程 $$x^{n}-1=0$$ 在复数系 $\mathbb{C}$ 内的 $n$ 个根,称为 $n$ 次单位根,它们是很有用的工
具,在许多问题中都会用到.
\end{frame}


\section{整数的可除性理论}
\begin{frame}{整数的可除性理论}
%整数及其运算是大家熟悉的. 整数包括正整数,零及负整数.

%用 $\mathbb{N}$
%表示全体正整数组成的数集,而

用 $\mathbb{Z}$ 表示全体整数组成的数集.

%这一附录引导读者学习整数的可除性理论.这是学习高等代数以及其它 课程必须知道的数学知识. 我们列出了所有的结论,但没有证明.读者完全可 以仿照多项式理论中有关结论的论证补出这些证明.这是留给读者的绝好的
%练习

整数有加法, 减法和乘法等运算, 减法是加法的逆运算. 

%加法和乘法满足 下面的八条规律.
%以下用 $a, b, c, d, \ldots$ 表示任意整数.
%\begin{enumerate}
%\item 加法交换律: $a+b=b+a$
%\item 加法结合律: $(a+b)+c=a+(b+c)$
%\item $\mathbb{Z}$ 中有零元素 0,满足
%$a+0=0+a=a$
%\item 每个整数 $a $都有唯一的负元素 $- a$, 使得
%$a+(-a)=(-a)+a=0$
%\item 乘法交换律: $a \cdot b=b \cdot a$
%\item  乘法结合律: $(a \cdot b) \cdot c=a \cdot(b \cdot c)$
%\item 分配律 : $a \cdot(b+c)=a \cdot b+a \cdot c$
%\item 消去律: 如果 $a \cdot b=a \cdot c, a \neq 0,$ 则 $b=c$
%\end{enumerate}
%根据第 4 条, 减法可看成是加法的逆运算$a-b:=a+(-b)
%$
%\end{frame}
%
%
%\begin{frame}{整 数 的可除 性 理 论}
\begin{itemize}
\item 带余除法
\item 整除
\item 最大公因数
\item 辗转相除法
\item 互素
\item 素数
\item 因数分解定理
\item 最小公倍数
\end{itemize}
\end{frame}
\begin{frame}{带余除法}
在 $\mathbb{Z}$ 中不能作除法,但是有以下的\alert{带余除法}.  
\begin{thm}
对于任意两个整数 $a ,b,b\neq 0$,存在一对整数 $q, r$ 满足
\[
a=q \cdot b+r, \quad 0 \leqslant r<|b|
\]
而且满足这个条件的整数 $q, r$ 是唯一的. 
\end{thm}
\begin{defi}
\begin{itemize}
\item $q$ 称为b 除 $a$ 的\alert{商}, 
\item $r$ 称为 $b$ 除 $a$ 的\alert{余数}. 
\end{itemize}
\end{defi}
 
\end{frame}

\begin{frame}
\begin{defi}
	对于整数 $a ,b$,如果存在一个整数 $c$ 使得 $a=b c,$ 则称
\begin{itemize}
\item $b$ 是 $a$ 的\alert{因数}, 
\item $a$ 是 $b$ 的\alert{倍数}.
\end{itemize}
\end{defi}
\begin{rem}
在定义中我们并不要求 $b\neq0$. 
\end{rem}


\end{frame}

\begin{frame}

\begin{prop}
	当 $b\neq 0$ 时,$b$ 是 $a$ 的因数的充 分必要条件是 $b$ 除 $a$ 所得的余数为 $0$.
\end{prop}
因此 $b$ 是 $a$ 的因数, 也称 $b$ \alert{整除} $a,$ 记作$b | a$. 

关于整除,有以下一些性质:
\begin{prop}
\begin{enumerate}
\item 如果 $a|b, b| a,$ 则 $a=\pm b$
\item 如果 $a|b, b| c,$ 则 $a | c$
\item  如果 $a|b, a| c,$ 则对任意整数 $k, l$ 都有 $a | k b+l c$
\end{enumerate}
\end{prop}
\begin{rem}
\begin{itemize}
\item 如果 $a | b,$ 则有 $-a | b$ 及 $a |(-b),$ 因此以后我们只讨论\alert{非负整数}的\alert{非 负因数}和\alert{非负倍数},不再加以说明.

\item 根据定义,每个整数都是 0 的因数,但是 0 不是任何非零整数的因数.
\end{itemize}
\end{rem}

\end{frame}




\begin{frame}
\begin{defi}
如果 $a$ 既是 $b$ 的因数,又是 $c$ 的因数,则称 $a$ 是 $b$ 和 $c$ 的一个\alert{公因数}.
\end{defi}
公因数中最重要的是最大公因数.

\begin{defi}
设 $d$ 是 $a$ 和 $b$ 的一个公因数. 如果 $a, b $的任一个因数都是 $d$ 的
因数, 则称 $d$ 是 $a, b$ 的一个\alert{最大公因数}.
\end{defi}

\begin{rem}
	\begin{itemize}
\item 根据定义,如果 $d_{1}, d_{2}$ 都是 $a, b$ 的最大公因数,那么 $d_{1}\left|d_{2}, d_{2}\right| d_{1} .$ 从
而 $d_{1}=\pm d_{2} .$ 
按规定 $d_{1}, d_{2}$ 皆非负,故 $d_{1}=d_{2}$.

\item 当 $b | a$ 时, $b$ 是 $a$ 与 $b$ 的最大公因数.

\item 特别地当 $a=0$ 时, $b$ 是 $a$ 与 $b$ 的一 个最大公因数.
 
\item 当 $a, b$  不全为零时, $a,b$ 的最大公因数不为0,  这时我们规定:以$(a, b)$ 表示 $a, b$ 的正的最大公因数. 在这个规定下,$(a, b)$是唯一的. 
\end{itemize}
\end{rem}
\end{frame}

\begin{frame}{辗转相除法}

设 $b \neq 0,$ 即 $b>0 .$ 反复应用 带余除法. 
	\[
	\begin{array}{cc}
	a=q_{1} b+r_{1}, & 0<r_{1}<b \\
	b=q_{2} r_{1}+r_{2}, & 0<r_{2}<r_{1} \\
	\cdots  & \cdots  \\
	r_{k-2}=q_{k} r_{k-1}+r_{k}, & 0<r_{k}<r_{k-1} \\
	r_{k-1}=q_{k+1} r_{k}+0
	\end{array}
	\]
	直到出现余数为零而终止. 则有
	\[
	(a, b)=\left(b, r_{1}\right)=\left(r_{1}, r_{2}\right)=\cdots=\left(r_{k-1}, r_{k}\right)=r_{k}
	\]
	从上面的算法中还可以找到整数 $u, v$ 使得
	\[
	(a, b)=u a+v b
	\]
	这是最大公因数的重要性质.
\end{frame}


\begin{frame}
\begin{defi}
	如果整数 $a, b$ 的最大公因数等于 1,则称 $a, b$ \alert{互素} (亦称互质).
\end{defi}
例如,3 与5互素, 21 与 40 互素.

互素有以下一些重要性质:
\begin{enumerate}
\item $a ,b$ 互素的充分必要条件是存在整数 $u, v$ 使
\[
u a+v b=1
\]
\item 如果 $a | b c,$ 月 $(a, b)=1,$ 则 $a | c$.
\item 如果 $a|c, b| c$ 而且 $(a, b)=1,$ 则 $a b | c$
\item 如果 $(a, c)=1,(b, c)=1,$ 则 $(a b, c)=1$
\end{enumerate}
这些性质说明了互素的重要性. 
\end{frame}


 
 \begin{frame}
%最后介绍素数(亦称质数)的概念.  

%一个整数 $a(a>1)$ 至少有两个因数: 1 和 $a$ 本身. 不等于 1 和 $a$ 的因数 叫做 $a$ 的真因数.  

\begin{defi}
	$a$ 是一个大于 1 的整数.如果除去 1 和本身外,$a$ 没有其它因数, 那么称$a$是一个\alert{素数}.
\end{defi}
例如 2, 3, 5, 23 等都是素数.  

从定义可知, 如果$p$ 表示成 $p=a \cdot b,$ 则必有 $a=1, b=p$ 或 $a=p, b=1$ 

%素数有下述性质:
\begin{prop}
\begin{enumerate}
\item  一个 素数$p$ 和任一个整数 $a$ 都有:或者 $p | a,$ 或者 $(p, a)=1$.
\item 如果 素数 $p | a \cdot b$ 则 $p | a$ 或$p| b$.
\item  如果一个>1 的整数 $p$ 和任何整数 $a$ 都有 : $p | a \text { 或( } p, a)=1,$ 则 $p$
是一个素数.
\item 如果大于 1 的整数$p$ 具有下述性质:对任何整数 $a , b$: 从 $p | a b$ 可推 出 $p | a$ 或 $p | b,$ 则 $p$ 是一个素数.  
\end{enumerate}
\end{prop}
如果一个素数  $p$ 是整数 $a$ 的一个因数,则 $p$ 称为 $a$ 的一个素因数.
 \end{frame}


 \begin{frame}
根据互素及素数的性质, 应用数学归纳法可以证明整数的一个基本
定理. 
\begin{thm}[因数分解及唯一性定理]
任一个大于 1 的整数 a 可以分解 成有限多个 素因数的乘积:
\[
a=p_{1} p_{2} \cdots p_{s}
\]
而且分解法是唯一的,即如果有两种分解法:
\[
a=p_{1} p_{2} \cdots p_{s}= q_{1} q_{2} \cdots q_{t}
\]
其中 $p_{1}, \cdots, p_{s} ; q_{1}, \cdots, q_{t}$ 都是素数,那么有 $s=t$,  并且重新将 $q_{1}, \cdots, q_{t}$ 适当排序后,可得
$p_{i}=q_{i}, \quad i=1,2, \cdots, s$.
\end{thm}
\end{frame}


\begin{frame}
在 $a$ 的分解式中,将同一个素因数合并写成方幂, 并且将素因数按大小排列,得到
\[
a=p_{1}^{\ell_{1}} p_{2}^{\ell_{2}} \cdots p_{r}^{\ell_r},  \quad p_{1}<p_{2}<\cdots<p_{r}, l_{i}>0, i=1, \cdots, r.
\]
这种表示法称为 $a$ 的\alert{标准分解式}. 

可以应用整数的分解式来判断整除性及计算最大公因数. 

现在将整数 $a$ , $b$ 的因数合在一起, 设为 $p_{1}, p_{2}, \cdots, p_{t},$ 并设
\begin{equation}\label{eq-1}
\left\{\begin{array}{ll}
a=p_{1}^{\ell_{1}} p_{2}^{\ell_{2}} \cdots p_{t}^{\ell_t}, & \ell_{i} \geqslant 0,  \quad i=1,2, \cdots, t \\
b=p_{1}^{d_{1}} p_{2}^{d_{2}} \cdots p_{t}^{d_t}, & d_{i} \geqslant 0,  \quad i=1,2, \cdots, t
\end{array}\right.
\end{equation}
则
\begin{enumerate}
\item $a$ 能整除 $b$ 的充分必要条件为
$
\ell_{i} \leqslant d_{i}, i=1,2, \cdots, t
$
\item 
$
(a, b)=p_{1}^{\min \left(\ell_{1}, d_{1}\right)} p_{2}^{\min \left(\ell_{2}, d_{2}\right)}\cdots p_{t}^{\min \left(\ell_{t}, d_{t}\right)}
$
\end{enumerate}

\end{frame}



\begin{frame}
%最后我们介绍最小公倍数的概念.  

\begin{defi}
设 $a ,b$ 是两个非负整数. $m$ 是 $a, b$ 的一个公倍数(按前面约定, 也是非负的). 
如果 $a , b$ 的任一个公倍数都是 $m$ 的倍数, 则 $m$ 称为 $a, b$ 的一 个\alert{最小公倍数}.  
\end{defi}
\begin{rem}
\begin{itemize}
\item 由定义可看出 $a ,b$ 的最小公倍数是唯一的,记作$[a, b]$.
\item 当 $a, b$ 是正整数时, 从它们的标准分解式可以求出最小公倍数.

设 a , b 的分解 如 \eqref{eq-1},则
\[
[a, b]=p_{1}^{\max \left(l_{1}, d_{1}\right)} p_{2}^{\max \left(l_{2}, d_{2}\right)} \cdots p_{t}^{\max \left(p_{t}, d_{t}\right)}
\]
\item 由此还可看出
\[
a  b=(a, b) \cdot[a, b]
\]
\end{itemize}
\end{rem}




\end{frame}
%可以把最大公因数及最小公倍数的概念推广到有限多个整数 $a_{1}, a_{2}$


%$a,$ 的楠形.类似地规定 $\left(a_{1}, a_{2}, \cdots, a_{s}\right)$ 和 $\left[a_{1}, a_{2}, \cdots, a_{s}\right] .$ 
%
%特别地,当 $a_{1}, a_{2}$
%$\cdots, a,$ 全为正整数时有
%\[
%\begin{array}{l}
%\left(a_{1}, a_{2}, \cdots, a_{s}\right)=\left(\left(a_{1}, a_{2}, \cdots, a_{s-1}\right), a_{s}\right) \\
%{\left[a_{1}, a_{2}, \cdots, a_{s}\right]=\left[\left[a_{1}, a_{2}, \cdots, a_{s-1}\right], a_{s}\right]}
%\end{array}
%\]
%并且存在整数 $u_{1}, u_{2}, \cdots, u,$ 使得
%\[
%u_{1} a_{1}+u_{2} a_{2}+\cdots+u_{s} a_{s}=\left(a_{1}, a_{2}, \cdots, a_{s}\right)
%\]


\end{document} 