% !Mode:: "TeX:UTF-8"
\documentclass[13pt,fontset=mac]{ctexbeamer}
\usepackage[utf8]{inputenc}


\usepackage{amsmath,amssymb,amsthm}             % AMS Math
%\usepackage[T1]{fontenc}
\usepackage{graphicx}
\usepackage{epstopdf}
\usepackage{tikz}
\linespread{1.3}

\usepackage{mathrsfs}  %花写字母
 

%%%=== theme ===%%%
% \usetheme{Warsaw}
%\usetheme{Copenhagen}
%\usetheme{Singapore}
\usetheme{Madrid}
%\usefonttheme{professionalfonts}
%\usefonttheme{serif}
% \usefonttheme{structureitalicserif}
%%\useinnertheme{rounded}
%%\useinnertheme{inmargin}
\useinnertheme{circles}
%\useoutertheme{miniframes}
\setbeamertemplate{navigation symbols}{}
\setbeamertemplate{footline}[page number]



%\usepackage[fontset=mac]{ctex}
%\usepackage{ctex}



\setbeamertemplate{theorems}[numbered]
\newtheorem{thm}{定理}
\newtheorem{lem}{引理}
\newtheorem{exa}{例}
\newtheorem*{theo}{定理}
\newtheorem*{conj}{猜想}
\newtheorem*{defi}{定义}
\newtheorem*{coro}{推论}
\newtheorem*{ex}{练习}
\newtheorem*{rem}{注}
\newtheorem*{prop}{性质}
\newtheorem*{qst}{问题}

\def\qed{\nopagebreak\hfill{\rule{4pt}{7pt}}\medbreak}
\def\pf{{\bf 证明~~ }}
\def\sol{{\bf 解~~ }}



\def\R{\mathbb{R}}
\def\Rn{\mathbb{R}^n}
\def\A{\mathscr{A}}
\def\B{\mathscr{B}}
\def\D{\mathscr{D}}
\def\E{\mathscr{E}}
\def\O{\mathscr{O}}

\def\rank{\operatorname{rank}}
\def\dim{\operatorname{dim}}
\def\0{\mathbf{0}}
\def\a{\alpha}
\def\b{\beta}
\def\r{\gamma}

\usepackage{color}
\definecolor{linkcol}{rgb}{0,0,0.4}
\definecolor{citecol}{rgb}{0.5,0,0}

\definecolor{blue}{rgb}{0,0.08,1}
\newcommand{\blue}{\textcolor{blue}}

  \usepackage{graphicx}
  \DeclareGraphicsExtensions{.eps}
%   \usepackage[a4paper,pagebackref,hyperindex=true,pdfnewwindow=true]{hyperref}


\begin{document}



\title[]{论文写作指导}
\author[]{{\large 张彪} }
\institute[]{{\normalsize
		天津师范大学\\[6pt]
		zhang@tjnu.edu.cn}}

\date{}


%
%\AtBeginSection[]
%{
%\begin{frame}
%	\frametitle{Outline}
%	\tableofcontents[currentsection]
%\end{frame}
%\setcounter{exa}{0}
%\setcounter{equation}{0}
%}



\begin{frame}
\maketitle
\end{frame}



\begin{frame}{参考书目}

	\begin{itemize}
	\item 汤涛,丁玖,数学之英文写作,高等教育出版社,2019. 

	\item  数学论文写作(原书第二版), [英] 尼古拉斯,J.海厄姆 著,贾志刚,常亮,李建波 译, 科学出版社, 2016. 

	\item \alert{数学论文英文写作实用模板(汉英对照)}, [波兰] 耶日 泰锡斯克(Jerzy Trzeciak) 著,张文彪,刘晓敏 译, 机械工业出版社, 2016. 

	\item LaTeX入门, 刘海洋 著, 电子工业出版社, 2013.
	\end{itemize}

	文档密码:po21
\end{frame}

\begin{frame}
	通过
		\begin{itemize}
		\item  系统的训练,
		\item 用心去探索\alert{规律}
		\item 并吸取教训,反复提高
	\end{itemize}
研究者完全可以写出语言表达上乘的学术论文。


\vspace{15pt}

\begin{itemize}
	\item 在多阅读、多练习的基础上,
	\item 掌握\alert{技巧},
	\item 熟练一些\alert{典型句型}和\alert{结构}
\end{itemize}
是写出好的科技论文的第一步。
\end{frame}


\begin{frame}
	数学在语言表达上的特点
	\begin{itemize}
		\item 数学词汇的意义经久不衰,不为时代所动,
		\item 数学概念的定义严密准确,无懈可击,
		\item 数学定理的证明服从逻辑规律,以三段论推理为其宗旨,
		\item 数学写作的方式技巧,有章可循
	\end{itemize}
\vspace{15pt}
\pause
	数学写作的要求
\begin{itemize}
	\item 如何能让我们的写作体现数学之美?
	\item 如何能让我们的文章结构、遣词造句、思想流动、动机结论让人读之犹如行云流水?
	\item 如何能让表达之美和推导之美并驾齐驱、相辅相成?
\end{itemize}
\end{frame}




\begin{frame}{第一章~数学文章的结构}
	写文章前的准备工作及写作中应该遵循的几项基本原则
	\begin{itemize}
		\item  写作目的是让别人清楚知道你想叙述和表达的东西。
		\item 要收集一些与写作有关的材料,特别要有几篇关键的文献。
		\item 文章要精确、清楚、简洁地表达你想说的东西。
		\item 初稿完成后,要反复修改,不要冀望一次完成。
		\item 平生第一篇英文文章写作时,应该找有英文写作经验的人修改。
	\end{itemize}
\end{frame}

\begin{frame}{基本原则}

\begin{itemize}
	\item  写作帮助学习。写作通过强迫你将思想集中于有可能忽略的每一步,从而提出理解上的间断。
	\item 优秀的写作反映清晰的思想。好的框架来源于清晰的思考。
	\item  写作时困难的。一个在某些程度上很有效的解决方法是强迫自己先写出来。
	\item 写作的一个最基本的原则是保持你的论文简单和直接。
	\item 改进写作的最好的方法可能是接受建设性的批评并从中学习。
	\item 另外一个改进写作的办法是带着批判的眼光去读,越多越好。
\end{itemize}
\end{frame}

\begin{frame}{Halmos's principle}
	\begin{itemize}
		\item  In order to say something well you must have something to say. 
		\item When you decide to write something, ask yourself who it is that you want to reach. (为谁而写)
	\end{itemize}

\vspace{8pt}
The basic problem in writing mathematics is the same as in writing biology, writing a novel, or wrting directions for assembling a harpsichord(大键琴): 
\begin{center}
the problem is \alert{to communicate an idea}.
\end{center}
To do so, and to do it clearly, 
\begin{itemize}
	\item you must have somthing to say, and 
	\item you must arrange it in the order that you want to it said in, 
	\item you must write it, rewrite it, and re-rewrite it  several times, and 
	\item you must be willing to think hard about and work hard on mechanical details such as diction, notation, and punctuation.
	\end{itemize}
That is' all there is to it.
\end{frame}
	
	
	
\begin{frame}
一篇数学论文一般由如下几个部分组成:
\begin{itemize}
\item 题目
\item 摘要
\item 引言
\item 主体
\item 结论
\item 致谢
\item 文献
\item 附录
\end{itemize}


\end{frame}

\begin{frame}{1.1~题目}
题目是文章的一句广告语。
\begin{itemize}
	\item 题目不宜太长,也不宜太短。
	\item 题目应具体、明确,反映文章的主要贡献。
	\item 不要泛指一个过大的方向。
\end{itemize}
\end{frame}


\begin{frame}{1.1~题目}
	题目是文章的一句广告语。
	\begin{itemize}
		\item 题目不宜太长,也不宜太短。
		\item 题目应具体、明确,反映文章的主要贡献。
		\item 不要泛指一个过大的方向。
	\end{itemize}

题目是文章的脸面。文章的题目需:
\begin{itemize}
	\item 含有一定的信息量,使读者课通过搜索引擎得到你的文章;
	\item 能够吸引读者的注意力;
	\item  言简意赅地表述文章的研究成果;
	\item 和现有的标题有一定的区别。
\end{itemize}
\end{frame}


\begin{frame}{1.2~摘要}
摘要是一篇有足够信息的微型文章。	文章的摘要应该:
	\begin{itemize}
		\item 概括文章的主要目的、思想和结果。
		\item 极可能简明扼要,但又要有足够的内容。
		\item 用词精确、意思明确,尽可能让更多的人读懂你的叙述。
	\end{itemize}
	
摘要是文章的心脏。它一般要回答
	\begin{itemize}
		\item 本文要干什么?本文涉及的问题是什么?
		\item 研究的问题如何解决?研究方法是什么?
		\item 主要的结果是什么?问题彻底解决还是部分解决了?
		\item 研究结果的意义?对科学或对读者有多大帮助?
	\end{itemize}

\end{frame}


\begin{frame}{1.3~引言}
	引言应该比较全面、准确并且客观地介绍文章中将要讨论
	\begin{itemize}
		\item 问题的背景材料和简要发展,
		\item 人们对此已做的相关贡献,
		\item 写这篇文章的动机,
		\item 本文的主要结果。
	\end{itemize}
需要注意:
	\begin{itemize}
	\item 引言的客观性非常重要,切记自我吹嘘。\\
	~~~~~~~~~~~~``文章在自己~~评价在别人''~~~~(华罗庚)
	\item 不要随意贬低别人的研究。
	\item 不要把别人的甚至自己以前的文章句子原封不动地直接移植过来。
\end{itemize}
\end{frame}

\begin{frame}{1.3.1~引言的开首}
		\begin{itemize}
		\item ``好的开始是成功的一半'

		\item 最糟糕的开头可能是先给出一堆数学符号和定义。

		\item  最好的引言应该这样开头:
	\begin{itemize}
		\item 先做一些较为通俗浅显的描述,这样会给读者一个容易的起头,以便他们轻松自如、有条不紊地登堂入室,进入角色,并能提高继续读下去的兴趣。
		\item  把你所关心的问题提出来,开门见山、直奔主题,不要不着边际地叙述与文章的中心关系不大的食物。要用直截了当、浅显易懂的语音把问题指出来。
	\end{itemize}
	\end{itemize}
\end{frame}


\begin{frame}{1.3.2~引言的中间}
	\begin{itemize}
		
		\item 定义问题
		
		\item 解释准备解决什么问题
		\item 总结以前去的的主要成果及不足之处
		\item 简介一下你解决问题主要的手法
		\item 指出写作的动机和目的是必要的。
	\end{itemize}
\end{frame}

\begin{frame}{1.3.3~引言的结尾}
	\begin{itemize}
		
		\item 简单叙述文章的组成部分
		
		\item 对于文章后面的每一节内容写上一句话,概括地告诉读者这一节是干什么的
		\item 用不用的语句给出每一节大意一个简单的小结。
	\end{itemize}
\end{frame}


\begin{frame}{1.4~主体}
	Preliminaries(预备知识)
	\begin{itemize}
		\item  引进符号说明和概念定义。
		\item  有些已知的但对这篇文章有用的结果也需要单独引出来。这样做是为下一节作准备,并能区别什么是你文章的主要新结论,什么是以前的结果。
	\end{itemize}
	Main results(主要结果)
		\begin{itemize}
		\item  1-2个主要定理,最多3-4个
		\item 定理的叙述中药包含完整的条件和结论
	\end{itemize}

Proofs(证明)
\end{frame}


\begin{frame}{1.5~结论}
	
	\begin{itemize}
		\item  非常简短地叙述文章的主要贡献
		\item  解释你在引言里提到的问题和疑问。经过整篇文章的论证,也许可以谈论部分答案了
		\item 指出由于某些原因,文章没有考虑的其他方面或更广泛的问题,并说明本文的论证方法是否可以推广到这些情形
		\item  展望下一步,后续研究可以做什么?
		\item 如果合适,讨论一些和本研究相关的猜想
	\end{itemize}
\end{frame}


\begin{frame}{1.6~致谢}
	
	\begin{itemize}
		\item  致谢的对象是主要包括给文章写作提供过意见或者帮助的人或机构
		\item  很多研究基金规定获资助者在所发表的\alert{相关}论文中必须清楚表明获其资助,有时需列出所获资基金的号码。
		\item 不必提及匿名审稿人。但如果审稿人对改进文章的质量有很大的共享,应向他们致谢。
	\end{itemize}
\end{frame}


\begin{frame}{1.7~文献}
	
	\begin{itemize}
		\item  文献引用
		\item  文献格式
	\end{itemize}
\end{frame}


\begin{frame}{1.8~附录}
	
	\begin{itemize}
		\item  附录记载的主要是作者不想放在正文内的冗长复杂的定理证明。
		\item  另一种附录防止的是正文定理证明中所要的的一些预备证明或一些标准引理,甚至一些有关概念的定义等。
		\item 附录可以收入任何作者不便或不拟放在文章不要部分、但有不想随意丢掉的东西。
	\end{itemize}
\end{frame}


\begin{frame}{1.9.3~关键词和学科分类}
	关键词keywords
	\begin{itemize}
		
		\item 关键词的设置是为了让数学论著归档服务机构(如美国《数学评论》)能把你的论文放到其门类齐全的数据库中的正确位置。
		
		\item 科学引用指标检索工具(SCI)利用关键词给文章分门别类,而读者根据关键词检索与它研究或兴趣相关的文献。
		\item 最好只写单数形式,但是写成复数也无伤大雅
			\end{itemize}
\end{frame}


\begin{frame}{1.9.3~关键词和学科分类}
数学学科分类 Mathematics Subject Classification  (MSC)
	\begin{itemize}
		
		\item 分类系统是由美国《数学评论》Math Reviews(MR)和德国《数学文摘》(zbMath)共同制定的。2010,2020
		
	\end{itemize}
\end{frame}



\begin{frame}
{CTeX 套装}


\begin{itemize}
	\item 网站  http://www.ctex.org/
%	\item CTeX 套装是科学院吴凌云研究员的个人作品。
	\item 在 CTeX 套装刚刚问世之时,因其解决了繁琐的中文字体安装工作,所以广受欢迎。
\end{itemize}
 但是,
\begin{itemize}
	\item  一方面 CTeX 套装已经很久不更新(最新的稳定版本	v2.9.2.164 -- 2012.03.22),内里的宏包、工具陈旧;
	\item 另一方面,随着 XeLaTeX 的发展,以及 xeCJK 等技术的成熟,上述这些繁琐的工作已经没有必要而失去意义;
\end{itemize}

	因此,\alert{现在不推荐使用 CTeX 套装}。
	


%不要安装和使用 CTeX 套装!
%如\documentclass{ctexart}
\end{frame}



\begin{frame}
{CTeX 宏集}
\begin{itemize}
	\item 
虽然它的名字也是「CTeX」,但是 CTeX 宏集和 CTeX 套装是两个不同的东西。
\item  CTEX 宏集支持LATEX、pdfLATEX、XƎLATEX、LuaLATEX、upLATEX 等多种不同的编译方式,并为它们提供了统一的界面。主要功能由宏包ctex 以及中文文档类ctexart、ctexrep、ctexbook 和 ctexbeamer 实现。
\item 我们推荐\alert{优先使用 CTeX 宏集处理中文}。
\item 中文的文档可以直接使用ctex 文档类。
%也就是ctexart、ctexrep、ctexbook、ctexbeamer 这些。
\end{itemize}

 
%请在任何情况下优先使用 CTeX 宏集在 LaTeX 中处理中文!
\end{frame}



\begin{frame}
{ 编译器: TeX Live}

\begin{itemize}
	\item TeX Live 是 TUG (TeX User Group) 维护和发布的 TeX 系统,可说是「官方」的 TeX 系统。
	\item 我们推荐任何阶段的 TeX 用户,都尽可能使用 TeX Live,以保持在跨操作系统平台、跨用户的一致性。
	\item TeX Live 的官方站点是 https://tug.org/texlive/。
	\item 建议尝试使用国内大学的镜像站下载。
	
	
\end{itemize}
\end{frame} 

\begin{frame}{编辑器}
\begin{itemize}
\item  TeXStudio (推荐使用)
\item TeXmaker
\item TeXworks (TexLive自带)
\item TexPad (for Mac users)
\end{itemize}
\end{frame}
\end{document} 