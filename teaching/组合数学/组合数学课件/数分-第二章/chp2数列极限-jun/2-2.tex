% !Mode:: "TeX:UTF-8" 
% !TEX program = xelatex
\documentclass [t,12pt,mathserif] {beamer} 

\usepackage[english]{babel}
\usepackage[utf8]{inputenc}
\usepackage[T1]{fontenc}
\usepackage{csquotes}
\usepackage{expl3,biblatex}
\usepackage{booktabs}
\usepackage{color,xcolor}
\usepackage{graphicx,caption,wrapfig,setspace}
\usepackage{amsthm,thmtools,amsmath,amsfonts,amssymb,dsfont} 
\usepackage{soul}
\usepackage{xeCJK}
\usepackage{newtxtext, newtxmath}

\usetheme{Madrid}
\usecolortheme{crane}
% set font
\usefonttheme{serif}

\definecolor{titlecolor}{RGB}{227, 169, 5}

\addtobeamertemplate{block begin}{%
  \setlength{\textwidth}{0.9\textwidth}%
}{}

\addtobeamertemplate{block alerted begin}{%
  \setlength{\textwidth}{0.9\textwidth}%
}{}

\addtobeamertemplate{block example begin}{%
  \setlength{\textwidth}{0.9\textwidth}%
}{}


\makeatletter
\let\HL\hl
\renewcommand\hl{%
  \let\set@color\beamerorig@set@color
  \let\reset@color\beamerorig@reset@color
  \HL}
\makeatother


\addbibresource{bibliography.bib}

% \usefonttheme[onlymath]{serif}
\mathversion{bold}
\setbeamerfont{normal text}{family=\sffamily,series=\mdseries}
\setbeamerfont{alerted text}{series=\bfseries}
\setbeamerfont{frametitle}{series=\bfseries}
\AtBeginDocument{\usebeamerfont{normal text}}

\renewcommand\baselinestretch{1.5}
\renewcommand\arraystretch{1.3}
\allowdisplaybreaks
\everymath{\displaystyle}

\beamertemplatenavigationsymbolsempty
\setbeamertemplate{footline}[page number]{} 
 \addtobeamertemplate{frametitle}{\vspace*{-0.5em}}{}
 

\newtheoremstyle{mydef}%
{3pt}{3pt}{\bfseries}{}{\bfseries\color{red}}{}{.5em}{}
\theoremstyle{mydef}
\newtheorem{dfn}{定义}
\newtheorem{thm}{定理}[section] 

\newtheoremstyle{myliti}%
{3pt}{3pt}{\bfseries}{}{\bfseries\color{blue}}{}{.5em}{}
\theoremstyle{myliti}
\newtheorem{ex}{例}

\newtheoremstyle{mystarli}%
{3pt}{3pt}{\bfseries}{}{\bfseries\color{blue}}{}{.5em}{}
\theoremstyle{mystarli}
\newtheorem{starli}{*例}

\setbeamertemplate{theorems}[numbered]
\setbeamertemplate{caption}[numbered]

\newcommand{\liang}[1]{\textcolor{blue}{#1}}
\newcommand{\tuchu}[1]{\textcolor{red}{#1}}
\setbeamertemplate{blocks}[rounded][shadow=false]

\newcommand{\blue}{\textcolor{blue}}
\newcommand{\red}{\textcolor{red}}
\newcommand{\magenta}{\textcolor{magenta}}
\newcommand{\teal}{\textcolor{teal}}
% \newcommand{\bena}{\vspace{-0.7\baselineskip}\begin{eqnarray}\begin{array}{l}}
% \newcommand{\eena}{\end{array}\end{eqnarray}\vskip -0.3\baselineskip}
% \newcommand{\benas}{\vspace{-0.7\baselineskip}\begin{eqnarray*}\begin{array}{l}}
% \newcommand{\eenas}{\end{array}\end{eqnarray*}\vskip -0.6\baselineskip}
\newcommand{\bena}{\begin{eqnarray}\begin{array}{l}}
\newcommand{\eena}{\end{array}\end{eqnarray}}
\newcommand{\benas}{\begin{eqnarray*}\begin{array}{l}}
\newcommand{\eenas}{\end{array}\end{eqnarray*}}
\newcommand{\ben}{\begin{eqnarray}}
\newcommand{\een}{\end{eqnarray}}
\newcommand{\bea}{\begin{array}}
\newcommand{\eea}{\end{array}}
\newcommand{\beq}{\begin{equation}}
\newcommand{\eeq}{\end{equation}}
\newcommand{\bec}{\setstretch{1.3}\begin{cases}}
\newcommand{\eec}{\end{cases}}
\newcommand{\beu}{\begin{enumerate}}
\newcommand{\eeu}{\end{enumerate}}
\def\rd{\ensuremath{ \mathrm{~d} }}
\def\re{\ensuremath{ \mathrm{e} }}
\def\vsp{\vskip 0.3\baselineskip}
\newcommand{\zhu}[1]{
\begin{alertblock}{}
 \textcolor{blue}{注 } #1
        \par  
\end{alertblock}
}
\def\jie{\textcolor{blue}{解} }
\def\zheng{\textcolor{blue}{证} }

\makeatletter
\setbeamertemplate{theorem begin}
{%  
\begin{\inserttheoremblockenv}
  {}{
%   \usebeamerfont*{block title}
  \usebeamercolor[fg]{block title}%
  \inserttheoremheadfont
  \inserttheoremname
  \inserttheoremnumber
  \ifx \inserttheoremaddition \empty \else\ \inserttheoremaddition\fi
%   \inserttheorempunctuation
  }
%   \normalfont
  }
  \setbeamertemplate{theorem end}{\end{\inserttheoremblockenv}}
\makeatother

% \declaretheoremstyle[
%     headfont=\normalfont
% ]{definition}

\AtBeginDocument{%
  \addtolength\abovedisplayskip{-0.5\baselineskip}%
  \addtolength\belowdisplayskip{-0.5\baselineskip}%
} 

\setbeamercolor{alerted text}{fg=titlecolor!70!blue}

% \setbeamercolor{titlelike}{fg=titlecolor}

\setbeamerfont{titlelike}{family=\sffamily,series=\bfseries}

\setbeamerfont{section number projected}{%
  family=\rmfamily,series=\bfseries,size=\normalsize}
\setbeamercolor{section number projected}{bg=titlecolor!50!red,fg=white}
\setbeamerfont{section in toc}{family=\sffamily,series=\bfseries,size=\normalsize}
\setbeamerfont{section in toc shaded}{size=\normalsize}

\setbeamerfont{title}{family=\sffamily,series=\bfseries,size=\LARGE}

\title[数列极限]{数列极限}
\institute[]{天津师范大学,数学科学学院}
% \titlegraphic{\includegraphics[height=2.0cm]{imologo.png}}
\author[Jun Li]{
	李~君 }
\date{2023年6月}


\begin{document}
\begin{frame}
\maketitle
\end{frame}

\begin{frame}
\frametitle{目~~录}
\setcounter{tocdepth}{2}
\tableofcontents
\end{frame}

%%%%%%%%%%%%%%%%%%%%%%%%%%%%%%%%%%%%%
\setcounter{section}{1}
\section{第二章~~数列极限}

\subsection{数列极限的概念}
\subsection{收敛数列的性质}
\begin{frame}{$\S 2$ 收敛数列的性质}
\alert{本节首先考察收敛数列这个新概念有哪 些优良性质? 然后学习怎样运用这些性质.}\\
一、惟一性\\
二、有界性\\
三、保号性\\
四、保不等式性\\
五、迫敛性(夹逼原理)\\
六、极限的四则运算\\
七、一些例子
\end{frame}

\begin{frame}{一、惟一性}
\addtocounter{thm}{1}
\begin{thm}
若 $\left\{a_n\right\}$ 收敛, 则它只有一个极限.
\end{thm}
\zheng 设 $a$ 是 $\left\{a_n\right\}$ 的一个极限.下面证明对于任何 定数 $b \neq a, b$ 不能是 $\left\{a_n\right\}$ 的极限.\\
若 $a, b$ 都是 $\left\{a_n\right\}$ 的极限, 则对于任何正数 $\varepsilon>0$, $\exists N_1$, 当 $n>N_1$ 时, 有
\bena
\left|a_n-a\right|<\varepsilon
\eena
$\exists N_2$, 当 $n>N_2$ 时, 有
\bena
\left|a_n-b\right|<\varepsilon . 
\eena
令 $N=\max \left\{N_1, N_2\right\}$, 当 $n>N$ 时 (1), (2) 同时成立, 从而有

  
\end{frame}

\begin{frame}{}
$$
|a-b| \leq\left|a_n-a\right|+\left|a_n-b\right|<\mathbf{2} \varepsilon .
$$
   因为 $\varepsilon$ 是任意的, 所以 $\boldsymbol{a}=\boldsymbol{b}$.   
\end{frame}

\begin{frame}{二、有界性}%%%%
 \begin{thm}
若数列 $\left\{a_n\right\}$ 收敛, 则 $\left\{a_n\right\}$ 为有界数列, 即存在 $M>0$, 使得 $\left|a_n\right| \leq M, n=1,2, \cdots$.
\end{thm}
\zheng 设 $\lim _{n \rightarrow \infty} a_n=a$, 对于正数 $\varepsilon=1, \exists N, n>N$ 时, 有\\ $\left|a_n-a\right|<1$, 即 $a-1<a_n<a+1$.\\
若令 $M=\max \left\{\left|a_1\right|,\left|a_2\right|, \cdots,\left|a_n\right|,|a-1|,|a+1|\right\}$,\\
则对一切正整数 $n$, 都有 $\left|a_n\right| \leq M$.
\begin{alertblock}{}
\liang{注} 数列 $\left\{(-1)^n\right\}$ 是有界的, 但却不收敛. 这就说 明有界只是数列收敛的必要条件, 而不是充分条 件.
\end{alertblock}
\end{frame}


\begin{frame}{  三、保号性}%%%%
\begin{thm}
设 $\lim _{n \rightarrow \infty} a_n=a$, 对于任意两个实数 $b, c$, $b<a<c$, 则存在 $N$, 当 $n>N$ 时, $b<a_n<c$.
\end{thm}
\zheng 取 $\varepsilon=\min \{a-b, c-a\}>0, \exists N$, 当 $n>N$ 时, $b \leq a-\varepsilon<a_n<a+\varepsilon \leq c$, 故 $b<a_n<c$. 
\begin{alertblock}{}
\liang{注} 若 $a>0$ (或 $a<0)$, 我们可取 $b=\frac{a}{2}$ (或 $c=\frac{a}{2}$ ), 则 $a_n>\frac{a}{2}>0$ (或 $a_n<\frac{a}{2}<0$).
\end{alertblock}
这也是为什么称该定理为保号性定理的原因.  
\end{frame}

\begin{frame}{}%%%%
\begin{ex}
证明 $\lim _{n \rightarrow \infty} \frac{1}{\sqrt[n]{n !}}=0$.
\end{ex}
\zheng 对任意正数 $\varepsilon$, 因为 $\lim _{n \rightarrow \infty} \frac{(1 / \varepsilon)^n}{n !}=0$, 所以由 定理 2.4, $\exists N>0$, 当 $n>N$ 时,
$$
\frac{(1 / \varepsilon)^n}{n !}<1, \text { 即 } \frac{1}{\sqrt[n]{n !}}<\varepsilon . 
$$
这就证明了 $\lim _{n \rightarrow \infty} \frac{1}{\sqrt[n]{n !}}=0$.
\end{frame}


\begin{frame}{ 四、保不等式性}%%%%
\begin{thm}
设 $\left\{a_n\right\},\left\{b_n\right\}$ 均为收敛数列, 如果存在正 数 $N_0$, 当 $n>N_0$ 时, 有 $a_n \leq b_n$, 则 $\lim _{n \rightarrow \infty} a_n \leq \lim _{n \rightarrow \infty} b_n$.
\end{thm}
\zheng 设 $\lim _{n \rightarrow \infty} a_n=a, \lim _{n \rightarrow \infty} b_n=b$. 若 $b<a$, 取 $\varepsilon=\frac{a-b}{2}$, 由保号性定理, 存在 $N>N_0$, 当 $n>N$ 时,
$$
a_n>a-\frac{a-b}{2}=\frac{a+b}{2},~ b_n<b+\frac{a-b}{2}=\frac{a+b}{2},
$$
故 $\boldsymbol{a}_{\boldsymbol{n}}>\boldsymbol{b}_{\boldsymbol{n}}$, 导致矛盾. 所以 $\boldsymbol{a} \leq \boldsymbol{b}$.   
\end{frame}



\begin{frame}{}%%%%
\zhu{若将定理 2.5 中的条件 $a_n \leq b_n$ 改为 $a_n<b_n$, 也只能得到 $\lim _{n \rightarrow \infty} a_n \leq \lim _{n \rightarrow \infty} b_n$.}
\alert{这就是说, 即使条件是严格不等式, 结论却不一定 是严格不等式.}\\
\vskip 0.5\baselineskip
例如, 虽然 $\frac{1}{n}<\frac{2}{n}$, 但 $\lim _{n \rightarrow \infty} \frac{1}{n}=\lim _{n \rightarrow \infty} \frac{2}{n}=0$.
    
\end{frame}


\begin{frame}{ 五、迫敛性(夹逼原理)}%%%%
\begin{thm}
设数列 $\left\{a_n\right\},\left\{b_n\right\}$ 都以 $a$ 为极限, 数列 $\left\{c_n\right\}$
满足:存在 $N_0$, 当 $n>N_0$ 时, 有 $a_n \leq c_n \leq b_n$, 则 $\left\{c_n\right\}$ 收敛, 且 $\lim _{n \rightarrow \infty} c_n=a$.
\end{thm}
\zheng  对任意正数 $\varepsilon$, 因为 $\lim _{n \rightarrow \infty} a_n=\lim _{\substack{n \rightarrow \infty}} b_n=a$, 所以分 别存在 $N_1, N_2$, 使得\\
当 $n>N_1$ 时, $a-\varepsilon<a_n$;
当 $n>N_2$ 时, $b_n<a+\varepsilon$. \\
取 $N=\max \left\{N_0, N_1, N_2\right\}$,
当 $n>N$ 时, $a-\varepsilon<a_n \leq c_n \leq b_n<a+\varepsilon$.\\
这就证得  
$$
\lim _{n \rightarrow \infty} c_n=a. 
$$

\end{frame}


\begin{frame}{}%%%%
\begin{ex}
求数列 $\{\sqrt[n]{n}\}$ 的极限.   
\end{ex}
\jie  设 $\boldsymbol{h}_n=\sqrt[n]{\boldsymbol{n}}-\mathbf{1} \geq \mathbf{0}$, 则有
$$
n=\left(1+h_n\right)^n \geq \frac{n(n-1)}{2} h_n^2(n \geq 2),
$$
故 $1 \leq \sqrt[n]{n}=1+h_n \leq 1+\sqrt{\frac{2}{n-1}}$. 又因
$$
\lim _{n \rightarrow \infty} 1=\lim _{n \rightarrow \infty}\left(1+\sqrt{\frac{2}{n-1}}\right)=1 \text {, }
$$
所以由迫敛性, 求得 $\lim _{n \rightarrow \infty} \sqrt[n]{n}=1$. 
\end{frame}


\begin{frame}{   六、四则运算法则}%%%%
\begin{thm}
若 $\left\{a_n\right\}$ 与 $\left\{b_n\right\}$ 为收敛数列, 则 $\left\{a_n+b_n\right\}$, $\left\{a_n-b_n\right\},\left\{a_n \cdot b_n\right\}$ 也都是收敛数列, 且有
\begin{enumerate}
    \item[{\color{black}$(1)$}] $\lim _{n \rightarrow \infty}\left(a_n \pm b_n\right)=\lim _{n \rightarrow \infty} a_n \pm \lim _{n \rightarrow \infty} b_n$;
    \item[{\color{black}$(2)$}]  $\lim _{n \rightarrow \infty}\left(a_n \cdot b_n\right)=\lim _{n \rightarrow \infty} a_n \cdot \lim _{n \rightarrow \infty} b_n$, 当 $b_n$ 为常数 $c$ 时, $\lim _{n \rightarrow \infty} c b_n=c \lim _{n \rightarrow \infty} b_n;$
    \item[{\color{black}$(3)$}] 若 $b_n \neq 0, \lim _{n \rightarrow \infty} b_n \neq 0$, 则 $\left\{\frac{a_n}{b_n}\right\}$ 也收敛, 且 $\lim _{n \rightarrow \infty} \frac{a_n}{b_n}=\lim _{n \rightarrow \infty} a_n / \lim _{n \rightarrow \infty} b_n$. 
\end{enumerate}

\end{thm}

\end{frame}


\begin{frame}{}%%%%
\zheng \liang{(1)} 设 $\lim _{n \rightarrow \infty} a_n=a, \lim _{n \rightarrow \infty} b_n=b, \forall \varepsilon>0$, 存在 $N$, 当 $n>N$ 时, 有 $\left|a_n-a\right|<\varepsilon,\left|b_n-b\right|<\varepsilon$, 所以
$$
\left|a_n \pm b_n-(a \pm b)\right| \leq\left|a_n-a\right|+\left|b_n-b\right|<2 \varepsilon,
$$
由 $\varepsilon$ 的任意性, 得到
$$
\lim _{n \rightarrow \infty}\left(a_n \pm b_n\right)=a \pm b=\lim _{n \rightarrow \infty} a_n \pm \lim _{n \rightarrow \infty} b_n .
$$
\zheng \liang{(2)}  因 $\left\{b_n\right\}$ 收敛, 故 $\left\{b_n\right\}$ 有界, 设 $\left|b_n\right| \leq M$.
对于任意 $\varepsilon>0$, 当 $n>N$ 时, 有
$$
\left|a_n-a\right|<\frac{\varepsilon}{M+1},\left|b_n-b\right|<\frac{\varepsilon}{|a|+1},
$$
    
\end{frame}


\begin{frame}{}%%%%
  于是
$$
\begin{aligned}
\left|a_n b_n-a b\right| & =\left|a_n b_n-a b_n+a b_n-a b\right| \\
& \leq\left|b_n\right|\left|a_n-a\right|+|a|\left|b_n-b\right|<2 \varepsilon
\end{aligned}
$$
由 $\varepsilon$ 的任意性, 证得
$$
\lim _{n \rightarrow \infty} a_n b_n=a b=\lim _{n \rightarrow \infty} a_n \lim _{n \rightarrow \infty} b_n .
$$
\zheng \liang{(3)} 因为 $\frac{a_n}{b_n}=a_n \cdot \frac{1}{b_n}$, 由(2), 只要证明
$$
\lim _{n \rightarrow \infty} \frac{1}{b_n}=\frac{1}{\lim\limits_{n \rightarrow \infty} b_n} .
$$
由于 $b \neq 0$, 据保号性, $\exists N_1$, 当 $n>N_1$ 时,  
\end{frame}


\begin{frame}{}%%%%
 $$
\left|b_n\right|>\frac{|b|}{2} \text {. }
$$
又因为 $\lim _{n \rightarrow \infty} b_n=b, \exists N_2$, 当 $n>N_2$ 时,
$$
\left|b_n-b\right|<\frac{|b|^2}{2} \varepsilon
$$
取 $N=\max \left\{N_1, N_2\right\}$, 当 $n>N$ 时,
$$
\left|\frac{1}{b_n}-\frac{1}{b}\right|=\left|\frac{b_n-b}{b_n b}\right| \leq \frac{2}{|b|^2}\left|b_n-b\right| \leq \varepsilon,
$$
即 $\lim _{n \rightarrow \infty} \frac{1}{b_n}=\frac{1}{b}$. 所以 $\lim _{n \rightarrow \infty} \frac{a_n}{b_n}=\frac{\lim \limits_{n \rightarrow \infty} a_n}{\lim \limits_{n \rightarrow \infty} b_n}$.   
\end{frame}


\begin{frame}{ 七、一些例子}%%%%
\begin{ex}
用四则运算法则计算
$$
\lim _{n \rightarrow \infty} \frac{a_m n^m+a_{m-1} n^{m-1}+\cdots+a_1 n+a_0}{b_k n^k+b_{k-1} n^{k-1}+\cdots+b_1 n+b_0}
$$
其中 $m \leq k, a_m b_k \neq 0$.
\end{ex}
\jie 依据 $\lim _{n \rightarrow \infty} \frac{1}{n^\alpha}=0(\alpha>0)$, 分别得出:\\
(1) 当 $\boldsymbol{m}=\boldsymbol{k}$ 时,有  

\end{frame}


\begin{frame}{}%%%%
\vskip -0.5\baselineskip
 $$
\begin{aligned}
& \lim _{n \rightarrow \infty} \frac{a_m n^m+a_{m-1} n^{m-1}+\cdots+a_1 n+a_0}{b_k n^k+b_{k-1} n^{k-1}+\cdots+b_1 n+b_0} \\
& \quad=\lim _{n \rightarrow \infty} \frac{a_m+a_{m-1} \frac{1}{n}+\cdots+a_1 \frac{1}{n^{m-1}}+a_0 \frac{1}{b^m}}{b_m+b_{m-1} \frac{1}{n}+\cdots+b_1 \frac{1}{n^{m-1}}+b_0 \frac{1}{n^m}} \\
& \quad=\frac{a_m}{b_m} .
\end{aligned}
$$
(2) 当 $m<k$ 时, 有  
$$
\begin{aligned}
& \lim _{n \rightarrow \infty} \frac{a_m n^m+a_{m-1} n^{m-1}+\cdots+a_1 n+a_0}{b_k n^k+b_{k-1} n^{k-1}+\cdots+b_1 n+b_0} \\
& =\lim _{n \rightarrow \infty} \frac{1}{n^{k-m}} \cdot \lim _{n \rightarrow \infty} \frac{a_m+a_{m-1} \frac{1}{n}+\cdots+a_1 \frac{1}{n^{m-1}}+a_0 \frac{1}{n^m}}{b_k+b_{k-1} \frac{1}{n}+\cdots+b_1 \frac{1}{n^{k-1}}+b_0 \frac{1}{n^k}} \\
& =0 \cdot \frac{a_m}{b_k}=0 .
\end{aligned}
$$

\end{frame}

\begin{frame}{}%%%%
  所以
$$
\text { 原式 }=\bec
\frac{\boldsymbol{a}_m}{\boldsymbol{b}_m}, & \boldsymbol{m}=\boldsymbol{k}, \\
\mathbf{0}, & \boldsymbol{m}<\boldsymbol{k} .
\eec.
$$  
\end{frame}


\begin{frame}{}%%%%
\begin{ex}
 设 $a_n \geq 0, \lim _{n \rightarrow \infty} a_n=a$, 求证 $\lim _{n \rightarrow \infty} \sqrt{a_n}=\sqrt{a}$. 
\end{ex}
\zheng 由于 $\boldsymbol{a}_n \geq \mathbf{0}$, 根据极限的保不等式性, 有 $\boldsymbol{a} \geq \mathbf{0}$. 对于任意 $\varepsilon>0, \exists N$, 当 $n>N$ 时, $\left|a_n-a\right|<\varepsilon$. 于 是可得:\\
(1) $a=0$ 时, 有 $\left|\sqrt{a_n}-\mathbf{0}\right|=\sqrt{a_n}<\sqrt{\varepsilon}$;\\
(2) $a>0$ 时, 有
$$
\left|\sqrt{a_n}-\sqrt{a}\right|=\frac{\left|a_n-a\right|}{\sqrt{a_n}+\sqrt{a}} \leq \frac{\left|a_n-a\right|}{\sqrt{a}} \leq \frac{\varepsilon}{\sqrt{a}} .
$$
故 $\lim _{n \rightarrow \infty} \sqrt{a_n}=\sqrt{a}$ 得证.   
\end{frame}

\begin{frame}{}%%%%
\begin{ex}
 设 $a_n \geq 0, \lim _{n \rightarrow \infty} a_n=a>0$, 求证 $\lim _{n \rightarrow \infty} \sqrt[n]{a_n}=1$.
\end{ex}
\zheng 因为 $\lim _{n \rightarrow \infty} a_n=\boldsymbol{a}>\mathbf{0}$, 根据极限的保号性, 存在 $N$, 当 $n>N$ 时, 有 $\frac{a}{2}<a_n<\frac{3 a}{2}$, 即
$$
\sqrt[n]{\frac{a}{2}}<\sqrt[n]{a_n}<\sqrt[n]{\frac{3 a}{2}}.
$$
又因为 $\lim _{n \rightarrow \infty} \sqrt[n]{\frac{a}{2}}=\lim _{n \rightarrow \infty} \sqrt[n]{\frac{3 a}{2}}=1$, 所以由极限的迫
敛性, 证得 $\lim _{n \rightarrow \infty} \sqrt[n]{a_n}=1$.   
\end{frame}

\begin{frame}{}%%%%
\begin{ex}
 求极限 $\lim _{n \rightarrow \infty} \frac{a^n}{1+a^n}(a \neq-1)$.
\end{ex} 
\jie (1) $|a|<1$, 因为 $\lim _{n \rightarrow \infty} a^n=0$, 所以由极限四则 运算法则, 得 $\lim _{n \rightarrow \infty} \frac{a^n}{1+a^n}=\frac{\lim _{n \rightarrow \infty} a^n}{1+\lim _{n \rightarrow \infty} a^n}=0$.\\
(2) $a=1, \lim _{n \rightarrow \infty} \frac{a^n}{1+a^n}=\lim _{n \rightarrow \infty} \frac{1}{2}=\frac{1}{2}$.\\
(3) $|a|>1$, 因 $\lim _{n \rightarrow \infty}\left(1 / a^n\right)=0$, 故得
$$
\lim _{n \rightarrow \infty} \frac{a^n}{1+a^n}=\lim _{n \rightarrow \infty} \frac{1}{1+1 / a^n}=\frac{1}{1+\lim _{n \rightarrow \infty}\left(1 / a^n\right)}=1 .
$$   
\end{frame}

\begin{frame}{}%%%%
\begin{ex}
 设 $a_1, a_2, \cdots, a_m$ 为 $m$ 个正数, 证明
$$
\lim _{n \rightarrow \infty} \sqrt[n]{a_1{ }^n+a_2{ }^n+\cdots+a_m{ }^n}=\max \left\{a_1, a_2, \cdots, a_m\right\} .
$$
\end{ex}    
\zheng 设 $a=\max \left\{a_1, a_2, \cdots, a_m\right\}$. 由
$$
\begin{gathered}
a \leq \sqrt[n]{a_1{ }^n+a_2{ }^n+\cdots+a_m{ }^n} \leq \sqrt[n]{m} a, \\
\lim _{n \rightarrow \infty} \sqrt[n]{m} a=\lim _{n \rightarrow \infty} a=a,
\end{gathered}
$$
以及极限的迫敛性, 可得
$$
\lim _{n \rightarrow \infty} \sqrt[n]{a_1{ }^n+a_2{ }^n+\cdots+a_m{ }^n}=a=\max \left\{a_1, a_2, \cdots, a_m\right\} .
$$
\end{frame}

\begin{frame}{}%%%%
\begin{dfn}
 设 $\left\{a_n\right\}$ 为数列, $\left\{n_k\right\}$ 为 $\mathbf{N}_{+}$的无限子集, 且
$$
\boldsymbol{n}_1<\boldsymbol{n}_{\mathbf{2}}<\cdots<\boldsymbol{n}_k<\cdots,
$$
则数列
$$
a_{n_1}, a_{n_2}, \cdots, a_{n_k}, \cdots
$$
称为 $\left\{a_n\right\}$ 的子列, 简记为 $\left\{a_{n_k}\right\}$.
\end{dfn}
\zhu{由定义, $\left\{a_n\right\}$ 的子列 $\left\{a_{n_k}\right\}$ 的各项均选自 $\left\{a_n\right\}$, 且保持这些项在 $\left\{a_n\right\}$ 中的先后次序. $\left\{a_{n_k}\right\}$ 中的第 $k$ 项是 $\left\{a_n\right\}$ 中的第 $\boldsymbol{n}_{\boldsymbol{k}}$ 项, 故总有 $\boldsymbol{n}_{\boldsymbol{k}} \geq \boldsymbol{k}$. }
  
\end{frame}

\begin{frame}{}%%%%
 \begin{thm}
 若数列 $\left\{a_n\right\}$ 收敛到 $a$, 则 $\left\{a_n\right\}$ 的任意子列 $\left\{\boldsymbol{a}_{n_k}\right\}$ 也收敛到 $\boldsymbol{a}$.
 \end{thm}   
\zheng 设 $\lim _{n \rightarrow \infty} a_n=a$. 则 $\forall \varepsilon>0, \exists N$, 当 $n>N,\left|a_n-a\right|<\varepsilon$. 设 $\left\{a_{n_k}\right\}$ 是 $\left\{a_n\right\}$ 的任意一个子列. 由于 $\boldsymbol{n}_k \geq \boldsymbol{k}$, 因此 $\boldsymbol{k}>\boldsymbol{N}$ 时, $\boldsymbol{n}_k \geq \boldsymbol{k}>\boldsymbol{N}$, 亦有 $\left|a_{n_k}-a\right|<\varepsilon$. 这就证明了
$$
\lim _{k \rightarrow \infty} a_{n_k}=\boldsymbol{a} . 
$$
\zhu{ 由定理 $\mathbf{2. 8}$ 可知, 若一个数列的两个子列收敛 于不同的值,则此数列必发散.}
\end{frame}

\begin{frame}{}%%%%
\begin{ex}
 求证 $\lim _{n \rightarrow \infty} a_n=a$ 的充要条件是
$$
\lim _{n \rightarrow \infty} a_{2 n-1}=\lim _{n \rightarrow \infty} a_{2 n}=a . 
$$
\end{ex}
\zheng (必要性) 设 $\lim _{n \rightarrow \infty} a_n=a$, 则 $\forall \varepsilon>0, \exists N, n>N$ 时,
\benas
\left|a_n-a\right|<\varepsilon
\eenas
因为 $2 n>N, 2 n-1 \geq N$, 所以
\benas
\left|a_{2 n-1}-a\right|<\varepsilon, \quad\left|a_{2 n}-a\right|<\varepsilon .
\eenas
$$a+b$$

\end{frame}

\begin{frame}{}%%%%
(充分性) 设 $\lim _{k \rightarrow \infty} a_{2 k+1}=\lim _{k \rightarrow \infty} a_{2 k}=a$, 则 $\forall \varepsilon>0, \exists N$, 当 $k>N$ 时,     $$
\left|a_{2 k-1}-a\right|<\varepsilon, \quad\left|a_{2 k}-a\right|<\varepsilon .
$$
令 $N=2 K$, 当 $n>N$ 时, 则有
$$
\left|a_n-a\right|<\varepsilon
$$
所以 $\lim _{n \rightarrow \infty} a_n=a$.
\end{frame}

\begin{frame}{}%%%%
\begin{ex}
 若 $a_n=(-1)^n\left(1-\frac{1}{n}\right)$. 证明数列 $\left\{a_n\right\}$ 发散.
\end{ex}
\zheng 显然
$$
\begin{gathered}
\lim _{k \rightarrow \infty} a_{2 k-1}=\lim _{k \rightarrow \infty}-\left(1-\frac{1}{2 k-1}\right)=-1 ; \\
\lim _{k \rightarrow \infty} a_{2 k}=\lim _{k \rightarrow \infty}\left(1-\frac{1}{2 k}\right)=1 .
\end{gathered}
$$
因此, 数列 $\left\{a_n\right\}$ 发散.   
\end{frame}

\begin{frame}{ 复习思考题}%%%%
1. 极限的保号性与保不等式性有什么不同?\\
2. 仿效例题5的证法, 证明: 若 $\left\{a_n\right\}$ 为正有界数列, 则
$$
\lim _{n \rightarrow \infty} \sqrt[n]{a_1^n+a_2^n+\cdots+a_n^n}=\sup \left\{a_n\right\} .
$$   
\end{frame}


\end{document}


\begin{frame}{}%%%%
    
\end{frame}