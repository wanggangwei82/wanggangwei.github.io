\documentclass{ctexbook}
%\usepackage{ctex}
\usepackage{amsmath}
\usepackage{amssymb}
\usepackage{amsthm}
\usepackage{graphicx}
\usepackage{geometry}
\geometry{left=3cm,right=3cm,top=3cm,bottom=2cm}


\newtheorem{thm}{定理}[section]
\newtheorem{lem}[thm]{引理}
\newtheorem{prop}[thm]{性质}
\newtheorem{cor}[thm]{推论}
\newtheorem{conj}[thm]{猜想}
\newtheorem{rem}[thm]{注}
\newtheorem{proposition}[thm]{命题}



\begin{document}
\setcounter{chapter}{1}
\chapter{符号计数}

\begin{center}
    (朱瑶瑶 \quad  翻译)
\end{center}
%$$
%C(n)=\frac{1}{n+1}\left(\begin{array}{c}
%    2 n \\
%    n
%\end{array}\right)
%$$


	在前一章中, 我们集中讨论了所有项都为正的计数公式. 但是当其中一项也允许为负数时, 有一些有趣的事要说. 本章专门介绍在这种情况下可以使用的主要技术.

\section{容斥原理}
	容斥原理, 简称PIE, 是符号计数的经典方法之一. 在介绍了该原理本身之后, 我们将应用于错排, 即无不动点的排列.

	在加法原则中, 引理1.1.1(a), 我们假设集合$S, T$是不相交的. 当然, 很容易看出, 对于任何有限集合$S, T$, 我们有
\begin{equation}
|S \cup T|=|S|+|T|-|S \cap T|
\end{equation}
%
%$$
%|S \cup T|=|S|+|T|-|S \cap T|
%$$
%
%\begin{equation}
%|S \cup T|=|S|+|T|-|S \cap T|
%\end{equation}
%
%\begin{align*}
%    |S \cup T|=|S|+|T|-|S \cap T|
%\end{align*}

	事实上, $|S|+|T|$ 计算了两次$S \cap T$, 所以为了只计算一次, 我们必须减去相交部分的基数. 但有人可能会问, 对于任意数量的集合的并集, 是否有类似的公式. 事实证明, 将这些集合视为某些全集$S$的子集, 并计算$S$中不在任何子集中的元素个数, 这通常更有用. 这类似于模式避免中所使用的观点. 我们设置符号, 令$S$是一个集合, 并且令$S_{1},...,S_{n} \subseteq S$. 我们希望找到一条针对$|S-\bigcup _{i}S_{i}|$的公式. 当$n=1$时, 我们显然有
\[
	|S-S_{1}|=|S|-|S_{1}|
\]
	并且对于$n=2$, 由等式(2.1)得到
   $$	|S-(S_{1} \cup S_{2})|=|S|-|S_{1}|-|S_{2}|+|S_{1} \cap S_{2}|$$
\begin{figure}[htbp]
    \centering
    \includegraphics*[width=0.95\textwidth]{./fig2/1.jpg}
    \caption{$n=1,2$时的PIE}
\end{figure}


    图2.1给出了两种情况下计算阴影区域的韦恩图. 读者可能已经猜到对于任意$n$的归纳. 这种基数加减交替的枚举有时也被称为筛.
    \begin{thm}
    	(容斥原理, PIE)假设$S$是有限集合, 有子集$S_{1},...,S_{n}$ , 则
    	\begin{equation}
    	|S-\bigcup_{i=1}^{n} S_{i}|=|S|-\sum_{\substack{1\le i\le n}}|S_{i}|+\sum_{\substack{1\le i<j\le n}}|S_{i}\cap S_{j}|-...+(-1)^n|\bigcap_{i=1}^{n} S_{i}|
    	\end{equation}
    \end{thm}
    \begin{proof}
    	对于任意集合S我们有$|S|=\sum_{s\in S}1$. 我们将会使用符号$|S|=\sum_{s\in S}1_{s}$, 即$1_{s}$表示s对总和的贡献. 因此, 只要证明若$s\notin \cup_{i} S_{i}$, 则交替和中$1_{s}$的系数为1, 否则为0. 在第一种情况中, $1_{s}$只出现在|S|,给出了所希望的系数. 在第二种情况中, 假设$s\in S_{i}$
    	, $m\geq 1$, 指标为$i$. 现在, 当$S_{i_{1}},\ldots,S_{i_{k}}$是包含$m$个包含$s$的子集中的$k$个, 则有$s\in S_{i_{1}}\cap \ldots \cap S_{i_{k}}$. 因此, 在$k$次交和中, 求和数$1_{s}$的个数为$\binom{m}{0}$. 因此, (2.2)右侧的$1_{s}$系数为
    	\[
    	\binom{m}{0}-\binom{m}{1}+\binom{m}{2}-...=0
    	\]
    	根据定理1.3.3(d). 证明结束.
    \end{proof}

	为了简化符号, 我们通常简写$\bigcup_{i=1}^n S_{i}$为$\bigcup S_{i}$,我们也将用$S_{I}$代替$\bigcap_{i\in I} S_{i}$.

	我们将会计算没有不动点的排列, 来作为PIE的一个应用. 这个问题有时候伴随着下面的故事. 假设在餐厅用餐前, 有$n$个开心的狂欢者(他们是开心的, 这很重要), 将他们的$n$顶相同的圆顶礼帽放在帽架上. 在用餐时, 帽架会被掀翻(我告诉过你他们很开心), 因此当狂欢者离开时, 没有识别标记的帽子会被随意重新翻转. 没有人能拿回自己帽子的可能性有多大?

	如果把男人编号为$1,...,n$, 同样地对帽子进行编号, 帽子$i$属于男人$i$, 则一个拿回帽子的方法只是一个排列$\pi =\pi_{1} ...\pi_{n} \in \mathfrak{S}_{n}$, 其中$\pi_{i} =j$意味着男人$i$拿到了帽子$j$. 所以这种情况下没有男人拿回来自己的帽子, 意味着$\pi_{i} \neq i$对于所有的$i$;也就是说$\pi$ 没有不动点. 这种排列被称为错排, 标记$\mathfrak{S}_{n}$中的错排数为$D(n)$,并称为第$n$错排数.

	我们现在希望解决一个问题, 以便于我们可以使用PIE. 特别是, 我们想要定义S和子集$S_{1},...,S_{n}$, 使得$D(n)=|S-\cup S_{i}|$. 要做到这一点, 我们将这个问题视为计算一个元素受到某些限制的集合的数量, 并且令

	(i) S是一个元素没有受到限制的集合, 以及

	(ii) $S_{1},...,S_{n}$是S的子集, 则从S中删去$S_{i}$,即施加第$i$个限制

	\noindent 如果(2.2)右边式子的基数可以被算出, 我们将正确选择S和$S_{i}$. 在考虑这种情况时, 我们想要计算没有不动点的排列的个数. 所以我们应该令$S=\mathfrak{S}_{n}$, 为所有不动点没有任何限制的排列的集合. 我们也将令$S_{i}$为$\pi \in \mathfrak{S}_{n}$, $\pi_{i} =i$的集合, 即我们删去那些以i作为不动点的排列. 请注意, 我们不选择子集$S_{i}^{\prime}$,其中$S_{i}^{\prime}$被定义为有不动点i的$\pi \in \mathfrak{S}_{n}$集合. 因为我们这样做, 那么$S_{i}^{\prime}$将会是不相交的, 则$|S-\cup S_{i}^{\prime}|=|S|-|S_{1}^{\prime}|-...-|S_{n}^{\prime}|$.因此, 计算$S_{i}^{\prime}$的基数与直接计算差集的基数是一样困难的, 我们什么也得不到. 但是, 我们最初选择的子集将非常好.

	我们现在计算必要的基数. 当然, $|S|=|\mathfrak{S}_{n}|=n!$. 接下来, 如果$\pi\in S_{1}$,则$\pi =1\pi_{1} \pi_{2} ... \pi_{n}$, 其中$\pi_{2} ... \pi_{n}$构成2,...,n的排列. 所以$|S_{1}|=(n-1)\text{!}$. 显然, 相同的参数可以应用于任何$S_{i}$, 所以
\[
	\sum_{i}\left|S_{i}\right|=n\cdot (n-1)!=n!
\]
	类似地, $S_{1}\cap S_{2}\cap ...\cap S_{k}$是所有形式为$\pi =12...k\pi_{k+1}...\pi_{n}$的集合, 则有$(n-k)!$种方法选择$\pi_{k+1}...\pi_{n}$. 事实上, 所有$k$次交和中的项有相同的值, 并且$\binom{n}{k}$这样的项给出了总数
\[
	(n-k)!\binom{n}{k}=\frac{n!}{k!}
\]
	总而言之, 我们已经证明了以下几点.
\begin{thm}
	对于任意正整数$n$, 有
	\[
	D(n)=n!\left(1-\frac{1}{1!}+\frac{1}{2!}-\cdots+(-1)^{n}\frac{1}{n!}\right).
	\]
\end{thm}

	读者应该认识到上一个结果中的序列是$1/\mathit{e}$序列的截断. 因为没有人能拿回帽子的概率是归还帽子的排列总数中, 这种情况可能发生的方式总数, 或者$D(n)/n!$. 对于最初提出的问题, 我们得到了一个非常漂亮的答案.
\begin{cor}
	取极限, 即$n\rightarrow \infty$, 则没有人能拿回帽子的概率为1/$\mathit{e}$.
\end{cor}
	
	令人惊讶的是, $\mathit{e}$是一个典型的超越数, 出现在一个最开始只涉及整数的组合问题的解决方案.

\section{符号相反的对合}
	符号相反的对合是证明包含符号的恒等式甚至是没有明确符号的恒等式的一种有效方法. 正如我们看到的, 这些映射可以用来证明PIE, 并且在Garsia-Milne对合原则中发挥重要作用, 我们将在下一节中学习.

	令$S$是一个(不一定有限的)集合. 函数$\iota:S\rightarrow S$是一个对合, 如果$\iota^2$是一个$S$上的恒等映射. 等价地, 如果$\iota^{-1}=\iota$, 则$\iota$是一个双射. 一旦我们引入符号, 对合的另一种很好的刻画将是至关重要的. 对于任意$f:S\rightarrow S$, 它的不动点集为
\[
	\operatorname{Fix}f=\{s\in S\mid f(s)=s\}
\]
	我们还说, 不同的元素$s,t\in S$形成了一个$f$的2-圈, 如果$f(s)=t,f(t)=s$.
	在这种情况下, 我们用$(s,t)$或者$s\leftrightarrow t$来表示2-圈.

\begin{lem}
	考虑$\iota:S\rightarrow S$. 该映射$\iota$是一个对合当且仅当$S$是不动点的不相交集, 并且是$\iota$的2-圈.
\end{lem}
\begin{proof}
	对于正向, 只要证明如果$s\in S$不是一个不动点, 那么它在一个2-圈中. 因此假设$\iota(s)=t$, 那么$\iota(t)=\iota(s)^2=s$.
	
	相反地, 假设$S$是一个不相交的集合, 并选择$s\in S$. 如果$s\in Fix\iota$, 那么$\iota(s)^2=\iota(s)=s$. 否则, $s$在2-圈$(s,t)$中, 使得$\iota(s)^2=\iota(t)=s$. 所以$\iota^2$是一个恒等映射. 结论得证.
\end{proof}

	有符号集合是集合$S$与符号函数$\operatorname{sgn}:S\rightarrow\{+1,-1\}$的结合. 在这种情况下, 我们令
\[
	S^{+}=\{s\in S\mid \operatorname{sgn}s=+1\}
\]
	类似的有$S^{-}$. 如果$\iota :S\rightarrow S$是$S$上的一个对合, 那么我们说$\iota$是符号相反的, 如果$\operatorname{sgn}\iota (s)=-\operatorname{sgn}s$对于每个在$\iota$的2-圈中的$s$. 这种情况的图示将在图2.2中找到. 现在假设$S$是有限的. 因此
\begin{equation}
\sum_{s\in S}\operatorname{sgn}s=\sum_{s\in \operatorname{Fix}\iota}\operatorname{sgn}s .
\end{equation}
\begin{figure}[htbp]
	\centering
	\includegraphics*[width=0.7\textwidth]{./fig2/2.jpg}
	\caption{集合S上的一个符号相反对合}
\end{figure}
	事实上, 如果$s$在2-圈$(s,\iota(s))$中, 那么在等式左边, 我们有$\operatorname{sgn}s+\operatorname{sgn}\iota(s)=0$. 所以所有2-圈中的元素都会从求和中抵消, 只剩下$\operatorname{Fix}\iota$中的一项. 如果右边和式比左边少得多, 这个公式将会非常有用. 如果$\iota$中的所有不动点有相同的符号, 使得等式(2.3)的右边为$\pm|\operatorname{Fix}\iota|$, 那么我们可以得到更多的信息. 尝试证明关于符号求和$\sum_{k\geq 0}(-1)^{k}a_{k}$对于正整数$a_{k}$的事实的一般方法如下:

	(i)找到一个由正数和$\sum_{k}a_{k}$枚举得到的集合$S$.

	(ii)给$S$标记符号, 使得等式(2.3)左边等于$\sum_{k}(-1)^{k}a_{k}$.

	(iii)构造一个在$S$上的具有多个2-圈的符号相反对合$\iota $

	作为符号相反对合的首次应用, 我们将重新证明定理(1.3.3)(d)中二项式系数的交替和公式. 事实上, 最初的证明隐晦地运用了这种技巧, 但是现在我们可以充分展示对合证明. 为了便于参考, 我们重新写出这个等式:
\begin{equation}
\sum_{k}(-1)^k\binom{n}{k}=\delta_{n,0}.
\end{equation}
\begin{proof}
	通常我们假设$n\geq 1$, 因为$n=0$时是平凡的. 从去掉符号的和来看, 很显然我们应该令$S=2^{[n]}$. 从$k$在最初的和式里的使用方法来看, 我们倾向于令$\operatorname{sgn}s=(-1)^{\#S}$对于$s\subseteq[n]$.我们现在需要检查(2.3)和(2.4)的左边是否一致. 我们将使用的技术, 把单和变成二重和, 然后分组项, 是枚举组合中常见的一种技术. 在这种情况下
	\[
	\begin{aligned}
	\sum_{s\in S}\operatorname{sgn}s&=\sum_{s\subseteq[n]}(-1)^{\# S}\\
	&=\sum_{k}\sum_{s\in \binom{[n]}{k}}(-1)^{k}\\
	&=\sum_{k}I(-1)^{k} \binom{n}{k}
	\end{aligned}
	\]
	这就是要证明的.
	
	至于符号相反对合, 我们已经在这个结果的最初证明中看到了. 定义$\iota:2^{[n]}\rightarrow 2^{[n]}$, $\iota(s)=s\Delta\{n\}$. 如前所述, 这是一个对合. 为了证明这是符号相反的, 我们有$|s\Delta\{n\}|=|s|\pm 1$. 因此, $\operatorname{sgn}\iota(s)=(-1)^{|s|\pm 1}=-\operatorname{sgn}s$. 最后我们只需证明$\operatorname{Fix}\iota$. 但是$s\Delta\{n\} \neq s$,对于所有的$s\subseteq[n]$. 因此(2.3)的右边是空和. 因为这等于零, 这个证明就完成了. 
\end{proof}
	
	考虑到(2.4)是证明PIE的一个关键工具, 所以可以用符号相反对合来证明这个原理本身也就不足为奇了. 我们在这里重新陈述PIE, 是为了不与我们为符号相反对合设置的符号相冲突. 给定一个有限集$A$和子集$A_{1},...,A_{n}$, 我们希望证明
\begin{equation}
|A-\bigcup_{i=1}^{n} A_{i}|=|A|-\sum_{\substack{1\le i\le n}}|A_{i}|+\sum_{\substack{1\le i<j\le n}}|A_{i}\cap A_{j}|-...+(-1)^n|\bigcap_{i=1}^{n} A_{i}|.
\end{equation}
\begin{proof}
	在证明结束后我们将找到一个例子来说明这个证明. 我们不能取$S=A$, 因为(2.5)右边的许多项都计算了$A$中的相同元素. 为了处理这些重复, 令
	\begin{equation}
	S=\left\{(a,I)\in A\times 2^{[n]}\mid a\in A_{I}\right\}
	\end{equation}
	回忆这个符号
	\begin{equation}
		A_{I}=\bigcap_{i\in I}A_{i}
	\end{equation}
	注意, 他们在其中发挥了作用, 尽管从证明结果的原始陈述中并不明显, 就像定理1.9.3的证明一样. 注意到$A_{\emptyset}=A$. 所以$(a,\emptyset)$是一对, 对于所有的$a\in A$, 并且如果$a\notin \bigcup A_{i}$, 这是唯一出现$a$的一对. 因为(2.5)中的符号来自交集中的子集个数, 我们定义
	\[
	\operatorname{sgn}(a,I)=(-1)^{\# I}
	\]
	因此
	\[
	\begin{aligned}\sum_{s\in S}\operatorname{sgn}s
	&=\sum_{(a,I)\in S}(-1)^{\# I}\\
	&=\sum_{I\in 2^{[n]}}\sum_{a\in A_{I}}(-1)^{\#I}\\
	&=\sum_{k=0}^{n}\sum_{I\in \binom{[n]}{k}} \sum_{a\in A_{I}}(-1)^{k}\\
	&=\sum_{k=0}^{n}(-1)^{k}\sum_{I\in\binom{[n]}{k}}\left|A_{I}\right|\end{aligned}
	\]
	这就是要证明的.
	
	为了构造一个对合, 定义每一个$a\in \bigcup A_{i}$, 指标
	\[
	m(a)=\max \left\{i \mid a \in A_{i}\right\}
	\]
	最后, 我们令
	\[
	\begin{array}{c}
	\iota(a, I)=\left\{\begin{array}{ll}
	(a, I \Delta\{m(a)\}) & \text {若} a \in \bigcup A_{i} \\
	(a, I) & \text {其他}
	\end{array}\right.
	\end{array}
	\]
	从定义中可以清楚地看出, 这是一个对合, 其不动点含有元素$A-\bigcup A_{i}$的双射中, 并且他的2-圈包含了相反符号的元素. 因为$A-\bigcup A_{i}$中的元素每个出现在一对里, 因此(2,3)右边的是这个集合的基数. 结论得证.
\end{proof}

	为了说明这个证明, 假设$A={a,b,c,d},A_{1}={a,b},以及A_{2}={b,c}$. 为了可读性, 在索引集省略了花括号和逗号,
\[
	S=\{(a, \emptyset),(a, 1),(b, \emptyset),(b, 1),(b, 2),(b, 12),(c, \emptyset),(c, 2),(d, \emptyset)\}
\]
	并且$m(a)=1,m(b)=m(c)=2$, 因此对合产生以下2-圈:
\[
(a, \emptyset) \leftrightarrow(a, 1),(b, \emptyset) \leftrightarrow(b, 2),(b, 1) \leftrightarrow(b, 12),(c, \emptyset) \leftrightarrow(c, 2).
\]
	唯一的不动点为$(d, \emptyset)$并且有$A-(A_{1} \cup A_{2})={d}$

	在使用我们的新技术之前, 如果能证明一些我们从未见过的东西, 那就太好了. 这里有一个涉及第二类斯特林数的恒等式.
\begin{thm}
	对于$n\geq 0$, 我们有
	\[
	\sum_{k \geq 0}(-1)^{k} k ! S(n, k)=(-1)^{n}.
	\]
\end{thm}
\begin{proof}
	首要任务是对求和进行组合解释. 集合T的组成是一个非空子集$\rho=\left(B_{1}, \ldots, B_{k}\right)$的序列, 使得$\uplus_{i} B_{i}=T$. 在这种情况下, 我们写作$\rho \vDash T$. 因此有$k$块的$\rho \vDash [n]$的个数为$k!S(n,k)$, 因为我们从$S([n],k)$中任意的$S(n,k)$分拆开始, 他的块的排列方式有$k!$种. 读者应该充分体会到了带有符号的集合, 在这种情况下, 我们想要令$S$为所有$\rho \vDash [n]$, 并且$\operatorname{sgn} \rho=(-1)^k$如果$\rho$有$k$块. 这是一个正确的交替和的证明留给读者作为练习.
	
	这个对合将会非常有趣. 我们将把它分成两个相反的情况. 通常情况下, 证明后面会有一个例子. 给定$\rho =(B_{1}, \ldots, B_{k})\vDash [n]$, 我们说$B_{j}$是可拆分的, 如果$\# B_{j}\geq 2$. 在这种情况下, 将$B_{j}$上的可拆分的函数定义为
	\[
	\sigma\left(B_{1}, \ldots, B_{k}\right)=\left(B_{1}, \ldots, B_{j-1},\{b\}, B_{j}-\{b\}, B_{j+1}, \ldots, B_{k}\right)
	\]
	其中$b=\operatorname{min}B_{j}$. 换句话说, $B_{j}$被一对块替代, 第一个包含其最小值, 其他则为他的剩余元素. 虽然符号$\rho$并不表示要分拆哪个块, 但这将从文章中明确. 我们将定义对合中撤销分拆的部分. 给定$\rho$, 我们说$B_{j}$可以和$B_{j+1}$合并, 如果
	
	(1)$B_{j}=\{b\}$, 对某些元素$b\in [n]$, 以及
	
	(2)$b<\operatorname{min}B_{j+1}$
	
	\noindent 在这种情况下, $B_{j}$上的合并映射定义为
	\[
	\mu(B_{1},\ldots,B_{k})=(B_{1},\ldots,B_{j-1},B_{j}\cup B_{j+1},B_{j+2},\ldots,B_{k}).
	\]
	应该清楚的是, 如果$B_{i}$可以被拆分为$B_{i}^{'}$和$B_{i+1}^{'}$, 那么素块可以合并到$B_{i}$, 反之亦然. 为了定义对合$\iota$, 假设给定
	$\rho (B_{1},\ldots,B_{k})$. 我们从左到右观察$\rho$, 直到我们找到第一个指标$j$, 如果有的话, 使得$B_{j}$可以拆分或者与$B_{j}$合并. (显然, 我们不能两者兼而有之. 因为拆分意味着$\# B_{j}\geq 2$, 合并意味着$\# B_{j}=1$. )现在定义
	\[
	\iota (\rho)=\left\{\begin{array}{cc}
	\sigma(\rho) & \text{若} B_{j}\text{可以被拆分, }\\
	\mu(\rho) & \text{若} B_{j}\text{可以被合并. }
	\end{array}\right.
	\]
	如果不存在这样的指标, 那么$\rho$将会是$\iota$的一个不动点.
	
	我们需要做一些工作来验证$\iota$是一个对合. 具体来说, 我们必须证明如果$\iota(\rho)=\rho^{'}$是通过从指标$j$处拆分$\rho$得到的, 那么$\iota(\rho)^{'}$将通过在同一指标处合并得到, 反之亦然. 我们将证明第一种情况, 将第二个留给读者. 首先要注意的是, 因为在$\rho$中没有$B_{i},i<j$可以被拆分, 我们必须有$B_{i}=\{b_{i}\}$对每个在范围中的$i$的$b_{i}$. 此外, 由于没有$B_{i}$可以被合并到$B_{i+1}$, 我们必须有$b_{1}>b_{2}>\ldots>b_{j-1}>b_{j}=\operatorname{min}B_{j}$. 现在在$\rho^{'}$中我们有$B_{i}^{'}=\{b_{i}\},i\leq j,b_{1}>\ldots>b_{j}$. 因此, 当$i<j$没有$B_{i}^{'}$可能被拆分或合并, 所以$\iota(\rho^{'})$将合并$B_{J}^{'}$到$B_{j+1}^{'}$. 因此$\iota(\rho^{'})=\rho$, 结论得证.
	
	显然, $\iota$是符号相反的, 因为$\iota(\rho)$比$\rho$多了一个或少了一个块. 所以我们只要找到不动点. 但是如果$\rho\in \operatorname{Fix}\iota$, 那么所有$\rho$的块都只包含一个元素;否则会有一个被拆分. 因此$\rho=(\{b_{1}\},\ldots,\{b_{n}\})$. 此外, 没有一个块可以被合并, 因此$b_{1}>\ldots >b_{n}$. 这迫使我们的集合组成为$\rho=(\{n\},\{n-1\},
	\ldots,\{1\})$, 并且$\operatorname{sgn}\rho=(-1)^{n}$, 证明完成. 
\end{proof}

为了举例说明,假设$n=8$. 正如我们前面所做的那样,我们将在集合中省去括号和逗号. 考虑$\rho=(\{B_{1}\},\ldots,\{B_{5}\})=(5,3,147,2,68)$.那么$B_{3}$是可拆分的,拆分后得到$\sigma(\rho)=(5,3,147,2,68)$. 同样,$B_{4}$可以合并到$\rho$中的k$B_{5}$中,因为$B_{4}=\{2\}$和$2<\min B_{5}=6$. 合并这两个块得$\mu(\rho)=(5,3,147,268)$.为了决定使用哪个操作,我们从$B_{1}$开始. 它不能拆分,因为只有一个元素. 而且它不能与$B_{2}$合并,因为$5>\min B_{2}=3$. 
类似地,$B_{2}$不能与$B_{3}$拆分或合并. 但我们已经看到$B_{3}$可以拆分,以便$\iota(\rho)=(5,3,1,47,2,68)=\rho^{'}$检查$\iota(\rho^{'})=\rho$是否相似. 

在寻找Hopf代数中的对极公式时,经常会出现涉及合并和分裂的对合. 例如,可以查阅Benedetti Bergeron【7】、Benedetti Hallam Machacek【8】、Benedetti Sagan【9】或Bergeron Ceballos【12】的论文. 
\section{Garsia-Milne 对合原理}
	到目前为止, 我们已经使用符号相反对合来解释了交替和中的化简. 但是他们也能提供一个双射来证明两个给定集合具有相同的基数吗?在某些情况下, 答案是对的, 而实现这一点的标准技术是Garsia-Milne对合原理. Garsia-Milne介绍了这种方法, 给出了Rogers-Rnmanujan恒等式的首个双射证明, Rogers-Rnmanujan恒等式是涉及某些整数集划分的著名公式. 从那时起, 对合原理有了许多其他应用. 例如, 参见Remmel[73]或Wilf[100]的文章.

	为了证明Garsia-Milne的结果, 我们需要引理1.9.5的一个适用于稍大一点的有向图的版本. 由于下一个结果的证明类似于前一个结果的证明, 我们很高兴向读者证明这一点.
\begin{lem}
	设$D=(V,A)$为有向图, 我们有$\operatorname{odeg}v,\operatorname{ideg}v\leq1$对所有的$v\in V$, 当且仅当$D$为有向路径和有向圈的不交并.
\end{lem}

	对合原理的基本思想是, 在适当的条件下, 如果一个图有两个符号集合, 每个符号集合都有自己的符号相反对合, 那么我们可以在这些集合之间使用双射, 建立他们的不动点集之间的双射. 所以, 令$S,T$为不相交的符号集合, 其符号相反对合$\iota:S\rightarrow S$和$\kappa:T\rightarrow T$, 使得$\operatorname{Fix}\iota \subseteq S^{+}$, $\operatorname{Fix}\kappa \subseteq T^{+}$. 此外, 我们假设有一个双射$f:S\rightarrow T$, 其保持前后符号一致, 即$\operatorname{sgn}f(s)=\operatorname{sgn}s$,对所有的$s\in S$. 图2.3中可以找到此设置的一个图. 注意, 尽管所有箭头实际上都是双向, 但由于即将发生的事情, 我们仅在一个方向上证明. 指向不动点的方向被忽略了. 我们现在构建一个映射$F:\operatorname{Fix}\iota \rightarrow \operatorname{Fix}\kappa$. 为了定义$F(s),s\in \operatorname{Fix}\iota$, 我们首先计算$f(s)\in T^{+}$. 如果$f(s)\in \operatorname{Fix}\kappa$, 那么我们令$F(S)=f(s)$. 如果不是, 我们将应用复合映射$\phi=f\circ \iota \circ f^{-1} \circ \kappa$到$f(s)$. 记住我们从右往左构造, 为$f(s)$, $T^{-}$, $S^{-}$, $S^{+}$, $T^{+}$的顺序. 取$\operatorname{Fix}\kappa$中的一个元素, 那么我们令$F(s)=\phi(f(s))$. 否则我们可以根据需要多次应用$\phi$, 例如需使用$m$次才能到达$\operatorname{Fix}\kappa$中的一个元素, 定义
\begin{equation}
F(s)=\phi^{m}(f(s))
\end{equation}
	继续图2.3中的例子, 我们可以看到$f(s)=u\notin \operatorname{Fix}\kappa$. 所以我们应用$\phi$, 把$u$依次带到$v,r,q,t$. 由于$t\in \operatorname{Fix}\kappa$, 我们令$F(s)=t$. 当然, 我们必须担心这是否是定义良好的;例如, $m$是否总是存在?我们也需要证明$F$是一个双射. 这是下一个定理所关心的.

\begin{thm}
	(Garsia-Milne对合原理)符号与之前相同, 映射$F:\operatorname{Fix}\iota \rightarrow \operatorname{Fix}\kappa$是良好定义的双射.
\end{thm}
\begin{proof}
	回忆第1.9节定理1.5.1中使用的函数有向图的概念. 通过限制定义域定义以下函数:
	\[
	\bar{f}=\left.f\right|_{S^{+}},\quad\bar{g}=\left.f^{-1}\right|_{T^{-}},\quad\bar{\iota}=\left.\iota\right|_{S^{-}},\quad\bar{\kappa}=\left.\kappa\right|_{T^{+}-\operatorname{Fix}\kappa}
	\]
	\begin{figure}[htbp]
		\centering
		\includegraphics*[width=0.95\textwidth]{./fig2/3.jpg}
		\caption{一个Garsia-Milne构造的例子}
	\end{figure}
	考虑$D$为函数有向图$\bar{f},\bar{g},\bar\iota,\bar\kappa$的集合. 从定义很容易证明, $x\in V(D)$, 其在包含$x$的$S\cup T$的子集中的出度, 入度如表所示:
	\[
	\begin{tabular}{c|c|c}
	subset & $\operatorname{odeg}x$ & $\operatorname{ideg}x$ \\
	\hline
	$\operatorname{Fix}\iota$ & 1 & 0 \\
	$\operatorname{Fix}\kappa$ & 0 & 1 \\
	$(S-\operatorname{Fix}\iota)\cup (T-\operatorname{Fix}\kappa)$ & 1 & 1 \\
	\end{tabular}
	\]
	例如, 如果$x\in \operatorname{Fix}\iota$, 那么只有一条从$\bar{f}$出来的包含$x$的弧, 所以$\operatorname{odeg}x=1$,$\operatorname{ideg}x=0$. 另一方面, 如果$x\in S^{+}-\operatorname{Fix}\iota$, 那么$x$有一条从$\bar{f}$出来的弧, 以及一条进入$\bar{\iota}$的弧, 所以$\operatorname{odeg}x=\operatorname{ideg}x=1$.
	
	现在$D$满足了引理2.3.1向前方向上的假设. 即$D$是有向路经和有向圈的不交并. 每条有向路经必须从一个出度为1, 入度为0的顶点出发, 在出度为0, 入度为1的顶点结束. 此外, 所有其他顶点的出度, 入度都为1. 从这些观察结果和图表可以看出, 这些路径定义了一个$\operatorname{Fix}\iota$与
	$\operatorname{Fix}\kappa$间的一一对应. 此外, 从$D$的定义中, 我们可以看到, 每条路径都精确地对应于一个复合函数$\phi^{m}f(s)$, 其中$s\in \operatorname{Fix}\iota,m\geq 0$. 所以$F$是由这些路径定义的双射.
\end{proof}

	在我们给出上述定理的一个应用前, 我们应该提及一个在构造必要的集合与双射上非常有用的方法. 这里有一种尝试找到两个有限集$X,Y$之间的双射$F:X\rightarrow Y$的方法.

	(i)和PIE一样, 构造一个集合$A$, 有子集$A_{1},...,A_{n}$, 使得$X=A-\bigcup A_{i}$. 类似构造$Y$中的$B$以及$B_{1},...,B_{N}$.

	(ii)利用PIE中的第二个证明方法, 去构造一个(2.6)给定集合$S$上的符号相反对合$\iota$. 类似构造集合$T$上的$\kappa$.

	(iii)找到一个双射$f:S\rightarrow T$, 其形式为
\[
	f(a,I)=(B,I)
\]

	$a\in A_{I}$当且仅当$b\in B_{I}$, 这是定义良好的.

	\noindent 回忆$\operatorname{Fix}\iota =(a,\emptyset)$, 其中$a\in A-\bigcup A_{i}$. 因此$\operatorname{Fix}\iota \in S^{+}$根据需要应用于对合原则, 这里自然有一个$\operatorname{Fix}\iota$和$X$间的双射. 并且注意, 由于$\operatorname{sgn}(a,I)=(-1)^{\# I}=\operatorname{sgn}(b,I)$, $f$是自动符号保持的. 所以当这三步完成后, 定理2.3.2保证了我们得到一个双射$X\rightarrow Y$.

	如前所述, 对合原则对证明整数分拆恒等式非常有用. 如果$\lambda_{1}>\lambda_{2}>\ldots >\lambda_{k}$, 那么说分拆$\lambda=(\lambda_{1},\lambda_{2},...,\lambda_{k})$有不同的部分(与通常的弱减情况相反). 另一方面, 如果所有$\lambda_{i}$为奇数, 那么说$\lambda$有奇部. 下一个结论是非常有名的欧拉原理. 按照惯例, 证明后面会有一个例子.
\begin{thm}
	(欧拉规则)设$P_{d}(n)$为有不同部分的$n$的分拆的集合, 并且令$\# P_{o}(n)$为有奇部的$n$的分拆的集合. 当$n\geq 0$, 我们有
	\[
	\# P_{d}(n)=\# P_{o}(n).
	\]
\end{thm}
\begin{proof}
	只要证明$P_{d}(n)\rightarrow P_{o}(n)$是一个双射. 在$P_{d}(n)$上应用PIE, 我们可以得到$A=P(n)$, 是所有$n$分拆的集合, 有子集$A_{1},...,A_{n}$, 其中
	\[
	A_{i}=\{\lambda \vdash n | \lambda \text{ (至少)有部分i中的两项}\}
	\]
	注意, 如果$i>n/2$, 则$A_{i}=\emptyset$, 但是这不会造成什么危害, 并能保持符号简单. 从定义能够清楚看出, $P_{d}(n)=A-\bigcup A_{i}$. 类似地, 对于$P_{o}(n)$, 我们令$B=P(n)$, 有子集
	\[
	B_{i}=\{\mu \vdash n | \mu \text{ 有一个形式为2i的部分}\}
	\]
	对于$1\leq i\leq n$. 再次, 能轻易地看到$P_{o}(n)=B-\bigcup B_{i}$
	
	$S,\iota,T,\kappa$的构造与PIE的第二个证明相同. 因此只要构造一个合适的双射$f:S\rightarrow T$. 给定$(\lambda,I)\in S$, 我们用$2i$部分代替在$\lambda$中$i$的一对, 其中$i\in I$, 形成$\mu$. 所以如果$\lambda \in A_{i}$, 那么$\mu\in B_{i}$, 其中$i\in I$, 并且映射$f(\lambda,I)=(\mu,I)$是定义良好的. 构造$f^{-1}$也很容易, 在$\mu$中取一个偶数部分$2i$. 当$i$取遍$I$时, 用$i$中的两项替换它, 形成$\lambda$. 借助定理2.3.2完成证明.
\end{proof}

	为了说明这个证明, 假设我们从$(6,2,1)\in P_{d}(9)$开始. 对于$S$和$T$中的对, 我们通常不使用分隔符和逗号. 所以
\[
\begin{aligned}
	(621,\emptyset) &\stackrel{f}{\mapsto}(621,\emptyset)\stackrel{\kappa}{\mapsto}(621,3)\stackrel{f^{-1}}{\mapsto}(3321,3)\stackrel{\iota}{\mapsto}(3321,\emptyset) \\
	&\stackrel{f}{\mapsto}(3321,\emptyset)\stackrel{\kappa}{\mapsto}(3321,1)\stackrel{f^{-1}}{\mapsto}(33111,1)\stackrel{\iota}{\mapsto}(33111,13) \\
	&\stackrel{f}{\mapsto}(621,13)\stackrel{\kappa}{\mapsto}(621,1)\stackrel{f^{-1}}{\mapsto}(6111,1)\stackrel{\iota}{\mapsto}(6111,\emptyset) \\
	&\stackrel{f}{\mapsto}(6111,\emptyset)\stackrel{\kappa}{\mapsto}(6111,3)\stackrel{f^{-1}}{\mapsto}(33111,3)\stackrel{\iota}{\mapsto}(33111,\emptyset) \\
	&\stackrel{f}{\mapsto}(33111,\emptyset).
\end{aligned}
\]
	因此, 我们应该映射$(6,2,1)\stackrel{F}{\mapsto}(3,3,1,1,1)$. 显然, 如果存在双射, 人们可能会希望找到一种更有效的双射. 这一问题将在练习中进一步探讨.


\section{反射原理}
	反射原理是一种几何方法, 用于处理涉及格路的某些组合问题. 特别是, 他将允许我们对卡特兰数的二项式系数公式给出一个非常简单的证明. 他也有助于证明单峰性, 这是实数序列的一个有趣性质.
\begin{figure}[htbp]
	\centering
	\includegraphics*[width=0.95\textwidth]{./fig2/4.jpg}
	\caption{映射$Y_{L}$}
\end{figure}

	考虑整数格$\mathbb{Z}$以及该格路中的东北路径. 假设对于某些$b\in \mathbb{Z}$, 给定平面上一条形式为$\operatorname{L}:\operatorname{y}=x+b$的直线. 注意, 任意东北路径的反射仍为东北路径. 如果$P$是从$u$到$v$的一条路, 那么我们记作$\operatorname{P}:u\rightarrow v$或者$u\stackrel{P}{\rightarrow} v$. 假设$\operatorname{P}:u\rightarrow v$与$\operatorname{L}$相交, 令$x$为最后(最东北方向)的交点. 那么$P$可以写成串联形式
\[
	Y_{L}(P):u\stackrel{P_{1}}{\rightarrow} x\stackrel{P_{2}^{'}}{\rightarrow} v^{'}
\]
	其中$P_{2}^{'},v^{'}$分别是$P_{2},v$在$L$中的反射. 回到我们的例子, $P_{2}^{'}=EENE$, 这是通过简单地交换东北方向而得到的, 因此如图2.4中右图所示, $Y_{L}(P)=EEENNEENE$. 这是使用反射原理的基本映射. 精确地说, 让$\mathcal{N} \mathcal{E} (u;v)$表示从$u$到$v$的东北路径的集合, 并让$\mathcal{N} \mathcal{E}_{L} (u;v)$表示与$L$相交路径的子集. 如果省略$u$, 那么可以假定$u=(0, 0)$. 此外, 为确保将从$u$到$v$的东北路径的符号$\mathcal{N} \mathcal{E} (u;v)$与从$(0,0)$到$(m,n)$的东北路径的符号$\mathcal{N} \mathcal{E} (m,n)$区分开来. 前者包含分号, 后者包含逗号.
\begin{thm}
	(映射原则)给定直线${L}:y=x+b$, 其中$b\in \mathbb{Z},v\in \mathbb{Z}^{2}$, 我们令$v^{'}$为$v$在$L$中的反射. 那么映射${Y_{L}}:\mathcal{N} \mathcal{E}_{L} (u;v) \rightarrow \mathcal{N} \mathcal{E}_{L} (u;v^{'})$是一个双射.
\end{thm}
\begin{proof}
	事实上, 我们可以证明$Y_{L}$是$\mathcal{N} \mathcal{E}_{L} (u;v) \cup \mathcal{N} \mathcal{E}_{L} (u;v^{'})$的一个对合. 那么根据事实, 在$L$上的反射为一个对合, 并且从$P\cap L$传递到$Y_{L}(P)\cap L$时, 交点的集合不变.
\end{proof}
	
	作为定理2.4.1的第一个应用, 我们将给出定理1.11.3的一个更简单的证明, 尽管这不是纯粹的组合证明. 我们在此引用公式:
\begin{proof}
	回忆一下, $C(n)$是对从$(0,0)$到$(n,n)$, 东北方向的Dyck路的集合$\mathcal{D}(n)$进行计数. 从定理1.11.1, 我们知道从原点到$(n,n)$的所有东北路径P的总数是
	\[
	\# \mathcal{N} \mathcal{E} (n,n)=\binom{2n}{n}
	\]
	注意, 当且仅当P与直线$\operatorname{L}:y=x-1$相交时, P不会保持在$y=x$的上方. 根据反射原理, 由于$(n+1,n-1)$是(n,n)在L中的反射, 则这类路径$\mathcal{N} \mathcal{E}_{L} ((0,0);(n+1,n-1))$是双射的. 但是由于这两个点位于直线的两侧, 故所有从$(0,0)$到$(n+1,n-1)$的路径都会穿过L. 因此, 再次利用定理1.11.1,
	\[
	\# \mathcal{N} \mathcal{E}_{L}((0,0);(n+1,n-1))=\#\mathcal{N} \mathcal{E} (n+1,n-1)=\binom{2n}{n+1}
	\]
	因此从$\mathcal{N} \mathcal{E} (n,n)$中的路径总数减去非Dyck路径的数量, 得到
	\[
	\begin{aligned} C(n)
	&=\binom{2n}{n}-\binom{2n}{n+1}\\
	& =\frac{(2n)!}{n!n!}-\frac{(2n)!}{(n+1)!(n-1)!} \\
	& =(1-\frac{n}{n+1})\binom{2n}{n}\\
	& =\frac{1}{n+1} \binom{2n}{n}
	\end{aligned}
	\]
	这是我们要证明的. 
\end{proof}

	反射原理也可用于证明某些序列具有单峰性. 称实数序列$a_{0},a_{1},...,a_{n}$是单峰的, 若指标m如下所示
\[
	a_{0}\leq a_{1}\leq \ldots \leq a_{m}\geq a_{m+1}\geq \ldots a_{n}
\]
	这是继弱增, 弱减后的第二个复杂的行为. 事实上, 后者是单峰的特例, 其中$m=n$或者$m=0$.
	组合数学, 代数学和几何学中出现的许多序列都是单峰序列. 更多详情请参见Stanley[89], Brenti[20], Br$\ddot{a}$nd$\acute{e}$n[19]的调查文章. 术语“单峰”来自概率和统计学, 人们认为$a_{i}$给出了随机变量的分布, 取值为$\{0,1,...,n\}$. 并且, 单峰分布只有一个驼峰.

	我们已经得到了一列单峰序列, 尽管我们没有证明这个事实, 但这是非常简单的.
\begin{thm}
	当$n \geq 0$时, 序列
	\[
	\binom{n}{0},\binom{n}{1},\ldots,\binom{n}{n}
	\]
	是单峰的.
\end{thm}
\begin{proof}
	由于二项式系数是对称的, 由定理1.3.3(b),只要证明这个序列递增到它的中间点. 因此我们首先证明
	\[
	\binom{n}{k} \leq \binom{n}{k+1}
	\]
	当$k<\lfloor n/2 \rfloor$. 从定理1.11.1, 我们知道
	\[
	\binom{n}{k}=\# \mathcal{N} \mathcal{E}(k,n-k) ,
	\binom{n}{k+1}=\# \mathcal{N} \mathcal{E}(k+1,n-k-1)
	\]
	因此只要找到单射$\operatorname{\mathit{i}}:\mathcal{N} \mathcal{E}(k,n-k) \rightarrow \mathcal{N} \mathcal{E}(k+1,n-k-1)$. 令L为从$(k,n-k)$到$(k+1,n-k-1)$的线段垂直平分线. 这很容易发现L的形式为$y=kx+b$, 其中$b\in \mathbb{Z}$. 从反射原理中, 我们有双射$\operatorname{Y}_{L}:\mathcal{N} \mathcal{E}_{L}(k,n-k) \rightarrow \mathcal{N} \mathcal{E}_{L}(k+1,n-k-1)$. 但是由于$k<\lfloor n/2 \rfloor$, 格点$(0,0),(k,n-k)$在L的两侧, 使得$\mathcal{N} \mathcal{E}_{L}(k,n-k)=\mathcal{N} \mathcal{E}(k,n-k)$. 此外, $\mathcal{N} \mathcal{E}_{L}(k+1,n-k-1) \subseteq \mathcal{N} \mathcal{E}(k+1,n-k-1)$. 因此扩大$\operatorname{Y}_{L}$的范围, 我们得到想要的单射.
\end{proof}


	事实证明, 斯特林数序列
\[
	c(n,0),c(n,1),\ldots,c(n,n) \quad \text{ 和 }  \quad S(n,0),S(n,1),\ldots,S(n,n)
\]
	也是单峰的. 但是直接证明这一点并不容易. 其中一个原因是, 这些序列不像二项式系数序列那样是对称的. 对于指标m在哪里达到最大值, 没有已知的简单的表达式. 相反, 这最好利用实数序列的另一个性质, 称为对数凹性, 这可以表明单峰性. 这是下一节的动机之一.


\section{Lindstr$\ddot{o}$m-Gessel-Viennot引理}
	所讨论的引理是处理某些确定的恒等式的强大技术. 他最早由Lindstr$\ddot{o}$m发现, 然后被Gessel和Viennot以及许多其他作者使用, 产生了巨大影响. 与反射原理一样, 该方法使用定向路径. 另一方面, 它使用多种路径, 并不局限于整数格路. 特别地, 当这里有两条路径时, 可以得到对数凹性的结果.

	一实数序列$a_{0},a_{1},\ldots,a_{n}$被称为对数凹的, 如果对所有的$0<k<n$, 我们有
\begin{equation}
a_{k}^{2} \geq a_{k-1}a_{k+1}
\end{equation}
	通常, 我们可以通过令$a_{k}=0$当$k<0$或$k>n$, 将其扩展到所有$k\in \mathbb{Z}$. 对数凹序列, 例如单峰序列, 在组合数学, 代数学和几何学中普遍存在. 详见之前引用的Stanley, Brenti, Br$\ddot{a}$nd$\acute{e}$n的调查文章. 例如, 一排帕斯卡三角形或任一斯特林三角形是对数凹的.

	对数凹的名字来源于以下场景. 假设我们有一个函数$\operatorname{\mathit{f}}:\mathbb{R} \rightarrow \mathbb{R}$是向下凹的. 所以在$\mathit{f}$的图上任意取两个点, 则该两点连线在$\mathit{f}$的下方. 取点$(k-1,f(k-1))$和$(k+1,f(k+1))$, 将相应线段中点的$y$坐标与$\mathit{f}$上的坐标比较, 得到$(f(k-1)+f(k+1))/2\leq f(k)$. 现在, 如果对于所有的$x$, $f(x)>0$, 并且函数$\operatorname{log}f(x)$是向下凹的, 那么代入之前的不等式并求幂, 得到$f(k-1)f(k+1)\leq f(k)^{2}$, 就像序列对数凹的定义一样.

	事实证明, 对数凹性和单峰性是相关的.
\begin{proposition}
	假设$a_{0},a_{1},\ldots,a_{n}$是一个正实数序列. 如果序列是对数凹的, 那么他是单峰的.
\end{proposition}
\begin{proof}
	为了证明一个序列是单峰的, 只要证明他在第一次严格减少之后, 继续减少. 但是, $a_{k-1}>a_{k}$等价于$a_{k-1}/a_{k}>1$, 其中$a_{k}$为正数. 将(2.9)重新写为$a_{k}/a_{k+1}\geq a_{k-1}/a_{k}$, 我们可以看到, 如果$a_{k-1}/a_{k}>1$, 那么$a_{l-1}/a_{l}>1$, 对于所以$l\geq k$. 因此这个序列是单峰的. 
\end{proof}


	尽管对数凹性意味着正数序列是单峰的, 但从对位角度来看, 证明对数凹性通常比证明单峰性更容易. 这在一定程度上是因为对数凹的条件对所有$k$是一致的, 相对于单峰性, 其必须知道序列的最大值出现在哪里.

	我们可以将(2.9)改写作$a_{k}^{2}-a_{k-1}a_{k+1}\geq 0$, 或者写作行列式
\begin{equation}
\left|\begin{array}{cc}a_{k} & a_{k+1} \\
a_{k-1} & a_{k}
\end{array}\right| \geq 0
\end{equation}

	为了证明该行列式是非负的, 我们可以证明它对某些事物进行计数, 这就是Lindstr$\ddot{o}$m-Gessel-Viennot引理所要做的. 我们将首先考虑2$\times$2行列式的情况, 并在本节末尾说明如何处理一般情况. 在这个例子中, 我们将展示如何证明定理2.4.2中的二项式系数序列的对数凹性.

	设$D$是无环的有向图, 因为它不包含有向圈. 给定两个顶点$u,v\in V(D)$, 我们让$\mathcal{P}(u;v)$表示从$u$到$v$有向路径的集合. 我们假设$u,v$总是被选择, 使得$p(u;v)=\# \mathcal{P}(u;v)$是有限的, 即使$D$本身不是有限的. 为了说明, 设$D$为有向图, 有顶点$\mathbb{Z}^{2}$, 以及从$(m,n)$到$(m+1,n)$, 再到$(m,n+1)$的弧对于所有$m,n\in \mathbb{Z}$. 那么$\mathcal{P}(u;v)$就是从$u$到$v$东北格路的集合, 这在上一节中表示为$\mathcal{N} \mathcal{E}(u;v)$. 我们将继续使用该节中一般路径的符号来表示任何无环有向图. 我们也将该符号扩展如下. 给定有向路径$\operatorname{P}:u\rightarrow v$并且在$P$上顶点$x$出现在$y$的前面, 我们假设$x\stackrel{P}{\rightarrow} y$为$P$在$x$和$y$之间的部分.

	继续一般的解释, 假设我们给出$u_{1},u_{2}\in V$称为起点, 以及$v_{1},v_{2}\in V$称为终点. 我们希望考虑如下形式的决定因素
\begin{equation}
\left|\begin{array}{cc} p(u_{1};v_{1}) & p(u_{1};v_{2}) \\
p(u_{2};v_{1}) & p(u_{2};v_{2})
\end{array}\right| =p(u_{1};v_{1})p(u_{2};v_{2})-p(u_{1};v_{2})p(u_{2};v_{1})
\end{equation}
	注意, $p(u_{1};v_{1})p(u_{2};v_{2})$对路径对进行计数
\[
	(P_{1},P_{2})\in \mathcal{P}(u_{1};v_{1}) \times \mathcal{P}(u_{1};v_{1}):=\mathcal{P}_{12}
\]
	类似有$p(u_{1};v_{2})p(u_{2};v_{1})$, 以及
\[
	\mathcal{P}(u_{1};v_{2}) \times \mathcal{P}(u_{2};v_{1}):=\mathcal{P}_{21}
\]
	回到我们的例子, 我们希望证明
\[
	\binom{n}{k}^{2}-\binom{n}{k-1}\binom{n}{k+1} \geq 0
\]
	那么我们可以得到
\[
	u_{1}=(1,0),u_{2}=(0,1),v_{1}=(k+1,n-k),v_{2}=(k,n-k+1)
\]
	根据定理1.11.1, $p(u_{1};v_{1})=p(u_{2};v_{2})=\binom{n}{k}$, 当$p(u_{1};v_{2})=\binom{n}{k-1}$以及$p(u_{2};v_{1})=\binom{n}{k+1}$. 更具体地说, 如果$n=7,k=3$, 那么在图2.5中我们有一对$\mathcal{P}_{21}$中的路径在左边, 按$\binom{7}{2} \binom{7}{4}$计算, 有另一对$\mathcal{P}_{12}$中的路径在右边, 按$\binom{7}{3}^{2}$计算. 为了可读性, 去掉整数格路上的网格, 只留下$\mathbb{Z}^{2}$上的顶点.

	为了证明行列式(2.11)是非负的, 我们将构造一个在集合$\mathcal{P}:=\mathcal{P}_{12}\cup \mathcal{P}_{21}$上的符号逆对合$\Omega$, 其中
\[
\operatorname{sgn} \left(P_{1},P_{2}\right)=\left\{\begin{array}{cc}
	+1 & \text{若}\left(P_{1},P_{2}\right) \in \mathcal{P}_{12} \\
	-1 & \text{若}\left(P_{1},P_{2}\right) \in \mathcal{P}_{21}
\end{array}\right.
\]
	我们将构造$\Omega$使得$\mathcal{P}_{21}$中每一对都在有一对在$\mathcal{P}_{12}$的2-圈中. 此外, 剩下$\mathcal{P}_{12}$中的不动点就是$\mathcal{P}$中不相交的路径对. 因此, (2.11)只是$\mathcal{P}$中不相交路径对的数量, 因此必须是非负的.

	为了定义$\Omega$, 考虑一个路径对$(P_{1},P_{2})\in \mathcal(P)$. 如果$P_{1}\cap P_{2}$是空的, 那么这对在$\mathcal{P}_{21}$中, 因为在$\mathcal(P)_{21}$中每个路径对都相交. 所以在这个例子中, 我们令$\Omega(P_{1},P_{2})=(P_{1},P_{2})$, 为不动点. 如果$P_{1}\cap P_{2}\not=\emptyset$, 那么考虑交点$x_{1},\ldots,x_{t}$, 按照他们在$P_{1}$上遇到的顺序. 我们声称他们必须按照在$P_{2}$中的顺序相遇. 如果存在交点$x,y$, 使得在$P_{1}$中$x$出现在$y$前面, 在$P_{2}$中y出现在$x$前面, 那么可以找到一条有向路$x\stackrel{P_{1}}{\rightarrow} y\stackrel{P_{2}}{\rightarrow} x$, 包含一有向环, 就像读者在练习中被要求做的那样. 这与$D$是无环的假设相矛盾. 所以第一个交点$x=x_{1}$是定义良好的. 我们现在令$\Omega(P_{1},P_{2})=(P_{1}^{'},P_{2}^{'})$其中
\[
\begin{aligned}
	P_{1}^{'}=u_{1}\stackrel{P_{1}}{\rightarrow} x\stackrel{P_{2}}{\rightarrow} v_{2}\\
	P_{2}^{'}=u_{2}\stackrel{P_{2}}{\rightarrow} x\stackrel{P_{1}}{\rightarrow} v_{1}
\end{aligned}
\]
	如果$(P_{1},P_{2})\in \mathcal{P}_{12}$, 类似地如果$(P_{1},P_{2})\in \mathcal{P}_{21}$, 则$v_{1},v_{2}$相反. 图2.5展示了该映射.
\begin{figure}[htbp]
	\centering
	\includegraphics*[width=0.95\textwidth]{./fig2/5.jpg}
	\caption{$Lindstr\ddot{o}m-Gessel-Viennot$对合}
\end{figure}

	因为在$(P_{1},P_{2})$的交点集与$(P_{1}^{'},P_{2}^{'})$的相同, 第一个交点保持不变, 这就使得$\Omega$成为一个对合. 从他的定义中也可以清楚地看出, 他改变了符号. 我们已经证明了下面的引理和推论.
\begin{lem}
	设$D$为无环有向图, 设$u_{1},u_{2},v_{1},v_{2}\in V(D)$, 使得每对路径$(P_{1},P_{2})\in \mathcal{P}_{21}$相交. 那么
	\[
	\left|\begin{array}{cc} p(u_{1};v_{1}) & p(u_{1};v_{2}) \\
	p(u_{2};v_{1}) & p(u_{2};v_{2})
	\end{array}\right|=\text{不相交的路径对}(P_{1},P_{2})\in \mathcal{P}_{21}\text{的个数}
	\]
	特别是,该行列式是非负的.
\end{lem}
\begin{cor}
	当$n\geq 0$, 序列
	\[
	\binom{n}{0},\binom{n}{1},\ldots,\binom{n}{n}
	\]
	是对数凹的.
\end{cor}

	如下所示, 引理2.5.2可以扩展到$n\times n$行列式. 令$u_{1},...,u_{n}$,$v_{1},...,v_{n}$为无环图中不同顶点的n元组. 对于$\pi \in \mathfrak{S}_{n}$, 我们令
\[
	\mathcal{P}_{\pi}=\{(P_{1},\ldots,P_{n})|P_{i}:u_{i}\rightarrow v_{\pi(i)},i\in [n]\}
\]
	以及
\[
	\mathcal{P}=\bigcup_{\pi \in \mathfrak{S}_{n}}\mathcal{P}_{\pi}
\]
	为了使得$\mathcal{P}$成为符号集, 回忆抽象代数中$\pi \in \mathfrak{S}_{n}$的符号为
\[
	\operatorname{sgn}\pi =(-1)^{n-k}
\]
	如果$\pi$在他的不相交环的分拆中有$k$个环. 还有其他定义$\operatorname{sgn}\pi$的方法, 但他们都是等价的. 这个符号函数的一个关键特性是, 如果$A=[a_{i,j}]$是一个矩阵, 那么
\[
	\operatorname{det}A=\sum_{\pi \in \mathfrak{S}_{n}}(\operatorname{sgn}\pi)a_{1,\pi(1)}a_{2,\pi(2)}...a_{n,\pi(n)}
\]
	现在如果$(P_{1},...,P_{n})\in \mathcal{P}_{\pi}$, 那么我们令$\operatorname{sgn}(P_{1},...,P_{n})=\operatorname{sgn}\pi$

	为了扩展对合$\Omega$, 称$\operatorname{P}=(P_{1},...,P_{n})$相交, 如果有一对$P_{i},P_{j}$相交. 给定一个相交的$P$, 我们找到最小的$i$, 使得$P_{i}$与$P$的另一条路径相交, 并且令$x$为$P_{i}$与$P$的另一条路径的第一个交点. 现在取最小的$j>i$, 使得$P_{j}$穿过$x$. 我们现在令$\Omega(P)=P^{'}$, 其中$P^{'}$为$P$中$P_{i},P_{j}$分别由
\begin{equation}
\begin{aligned}
\operatorname{P_{i}^{'}}& =u_{i}\stackrel{P_{i}}{\longrightarrow} x\stackrel{P_{j}}{\longrightarrow} v_{\pi(j)},\\
\operatorname{P_{j}^{'}}& =u_{j}\stackrel{P_{j}}{\longrightarrow} x\stackrel{P_{i}}{\longrightarrow} v_{\pi(i)}
\end{aligned}
\end{equation}
	代替. 现在需要证明$\Omega$是符号逆对合. 和之前一样, 非相交路径族$P$是$\Omega$的不动点. 根据$\Omega$的相关细节, 我们现在已经证明了以下几点.
\begin{lem}
	(Lindstr$\ddot{o}$m-Gessel-Viennot)设$D$为无环有向图. 考虑两个顶点序列$u_{1},...,u_{n},v_{1},...v_{n}\in V(D)$, 使得每个$P\in \mathcal{P}_{\pi}$是相交的, 其中$\pi \not=id$, 恒等排列. 我们有
	\[
	\operatorname{det}[p(u_{i};v_{j})]_{1\leq i,j\leq n}=\text{不相交}P\in \mathcal{P}_{id}\text{的个数}
	\]
	特别地, 该行列式是非负的. 
\end{lem}

	这个定理对实数序列也有一些说明. 任意序列$a_{0},...,a_{n}$有一个相应的托普利兹矩阵, 为无限矩阵$A=[a_{j-i}]_{i,j\geq 0}$. 因此
\[
	A=\left[\begin{array}{ccccccccc}
		a_{0} & a_{1} & a_{2} & \ldots & a_{n} & 0 & 0 & 0 & \ldots \\
		0 & a_{0} & a_{1} & a_{2} & \ldots & a_{n} & 0 & 0 & \ldots \\
		0 & 0 & a_{0} & a_{1} & a_{2} & \ldots & a_{n} & 0 & \ldots \\
		\vdots & \vdots & \vdots & \vdots & \vdots & \vdots & \vdots & \vdots & \vdots
	\end{array}\right]
\]
	如果A的每个子方阵都有一个非负行列式, 则称该序列为波利亚频率, 简称$PF$. 注意, 特别是, 我们在(2.10)中得到行列式, 因此$PF$意味着对数凹. 引理2.5.4可以用来证明一个序列是$PF$, 就像引理2.5.2可以用来证明它是对数凹的一样. 读者现在应该不难证明以下结果.
\begin{thm}
	当$n\geq 0$, 序列
	\[
	\binom{n}{0},\binom{n}{1},\ldots,\binom{n}{n}
	\]
	是$PF$.
\end{thm}

\section{矩阵树定理}
	我们以行列式的另一个应用来结束本章. 在计数组合学中, 有很多地方生活着动物, Krattenthaler[54,55]的文章中有一个很好的调查. 在这里, 我们将使用著名的基尔霍夫矩阵树定理来计算生成树.

	如果$V(H)=V(G)$, 则称子图$H\subseteq G$为生成图. 因此生成子图完全由他的边集决定. $G$的生成树$T$是生成子图, 他是一棵树. 显然, 生成树要存在, $G$必须是连通的. 设$\mathcal{S}T(G)$是$G$的生成树集. 如果考虑图2.6中左侧的图$G$, 以及他的八个生成树的列表位于图的中间(缩小到一半大小, 使得他们适合放在图中, 并且没有顶点和边的标记). 为了开发证明我们的主要定理所需的工具, 我们首先需要对组合矩阵做一些注释.
\begin{figure}[htbp]
	\centering
	\includegraphics*[width=0.95\textwidth]{./fig2/6.jpg}
	\caption{ 图G, 他的生成树, 以及一个方向}
\end{figure}

	我们将经常有机会创建矩阵, 其行列由集合索引而不是数字. 如果$S,T$是集合, 则$S\times T$矩阵$M$的构造方法是:对$S$和$T$的元素给定一个线性顺序, 并使用他们分别对$M$的行和列进行索引. 因此, 如果$(s,t)\in S\times T$, 那么$m_{s,t}$是矩阵$M$中由$s$索引的行和由$t$索引的列中的项. 读者可能已经注意到, 这样的矩阵不仅取决于$S,T$, 还取决于他们的线性顺序. 然而, 只是更改$M$中的行列顺序, 这通常不会影响我们希望从中提取到的信息.

	如果$G=(V,E)$是一个图, 那么有几个重要的矩阵与之相关. $G$的邻接矩阵是$V\times V$矩阵$A=A(G)$, 其中
\[
\operatorname{a}_{v,w} =\left\{\begin{array}{cc}
	1 & \text{若} vw\in E,\\
	0 & \text{其他.}
\end{array}\right.
\]
	使用顺序$v,w,x,y$, 图2.6左侧的图有邻接矩阵
\[
	\includegraphics*[width=0.35\textwidth]{./fig2/7.jpg}
\]

	邻接矩阵总是对称的, 由于$vw$和$wv$表示同一条边. 他在对角线上也有零, 因为我们的图(通常)是无环的.

	与$G$相关的第二个矩阵是他的关联矩阵, $B=B(G)$, 为$V\times E$矩阵, 其中项为
\[
\operatorname{b}_{v,e} =\left\{\begin{array}{cc}
	1 & \text{若v为e的端点, }\\
	0 & \text{其他.}
\end{array}\right.
\]
	回到我们的例子, 这个图有
\[
	\includegraphics*[width=0.4\textwidth]{./fig2/8.jpg}
\]
	通过构造, $B$的第$v$行包含$\operatorname{deg}v$个, 每一列包含两个. 我们还需要对角$V\times V$矩阵$C(G)$, 其对角线上的项$c_{v,v}=\operatorname{deg}v$. 这三个矩阵有很好的相关性.
\begin{proposition}
	对任意图$G$, 我们有
	\[
	BB^{t}=A+C
	\]
\end{proposition}
\begin{proof}
	$BB^{t}$的$(v,w)$项为$B$的第$v$行和第$w$行的内积. 如果$v=w$, 那么, 使用(1.9)中的符号,
	\[
	\sum_{e}b_{v,e}^{2}=\sum_{e}\delta(\text{v为e的端点})^{2}=\operatorname{deg}v=c_{v,v}
	\]
	$0^{2}=0,1^{2}=1$. 类似地, 如果$v\not=w$, 那么项为
	\[
	\begin{aligned}
	\sum_{e}b_{v,e}b_{w,e}
	&=\sum_{e}\delta(\text{v为e的端点})\cdot \delta(\text{w为e的端点})\\
	&=\delta(vw\in E)\\
	&=a_{v,w}
	\end{aligned}
	\]
	证明完成. 
\end{proof}

	有趣的是, 为了计算$G$的生成树的个数, 我们必须将$G$转化为一个有向图. $G$的方向是有向图$D$, 其$V(D)=V(G)$, 每条边$vw\in E(G)$, 每条弧$\stackrel{\rightarrow}{vw}$或者$\stackrel{\rightarrow}{wv}$在$A(D)$上. 在这种情况下, $G$被称为$D$的基础图. 图2.6中右侧的有向图是例子中的图$G$的一个方向. 有向图的邻接矩阵的定义与图一样, 这不是我们关心的. 但是我们需要有向关联矩阵, $B=B(D)$, 定义为
\[
\operatorname{b}_{v,a} =\left\{\begin{array}{cc}
	-1 & \text{若} a=\stackrel{\rightarrow}{vw},\\
	1 & \text{若} a=\stackrel{\rightarrow}{wv},\\
	0 & \text{其他.}
\end{array}\right.
\]
	对于图2.6中的有向图, 我们有
\[
	\includegraphics*[width=0.45\textwidth]{./fig2/9.jpg}
\]
	这里有两个关于$B(D)$的命题, 这对我们很重要.
\begin{proposition}
	设$D$为有向图, $B=B(D)$.
	
	(a)如果$B$的行为$\mathbf{b_{1}},\ldots,\mathbf{b_{n}}$, 那么
	\[
	\mathbf{b_{1}}+\ldots+\mathbf{b_{n}}=\mathbf{0},
	\]
	其中$\mathbf{0}$为零向量.
	
	(b)如果$D$为图$G$的一个方向, 那么
	\begin{equation}
	BB^{t}=C(G)-A(G)
	\end{equation}
\end{proposition}
\begin{proof}
	(a), 只需注意$B$的每一列都包含一个1和一个-1, 这将在总数中抵消. (b)的证明与命题2.6.1类似, 因此留给读者.
\end{proof}

	值得注意的是, 虽然(2.13)左侧的矩阵$B$依赖于$D$, 但右侧仅依赖于基础图$G$. 矩阵$L(G)=C(G)-A(G)$被称为$G$的拉普拉斯算子, 并控制图的许多组合方面. 回到我们的例子, 我们有
\[
	L(G)=\left[\begin{array}{rrrr}
		3&-1&-1&-1 \\
		-1&2&-1&0 \\
		-1&-1&3&-1 \\
		-1&0&-1&2
	\end{array}\right].
\]
	注意到, $L=L(G)$的行之和为零, 由于对所有的$v\in V$第$v$列包含了$\operatorname{deg}v$在对角线上, 并且其他非零项均为-1. 因此$\operatorname{det}L=0$. 但是去掉前面显示的矩阵的最后一行和最后一列, 并取行列式, 有
\[
	det\left[\begin{array}{ccc}
		3&-1&-1\\
		-1&2&-1\\
		-1&-1&3
	\end{array}\right] =8.
\]
	读者可能还记得, 8也是$G$的生成树的个数. 这并不是巧合!但在证明隐含定理之前, 我们还需要一个结果.

	设$M$是一个$S\times T$矩阵, 设$I\subseteq S,J\subseteq T$. 设$M_{I,J}$表示$M$的子矩阵, 其行由$I$索引, 列由$J$索引. 在$B(G)$中, 对于我们例子中的图$G$, 其$I=\{v,x\},J=\{f,g,i\}$, 我们有
\[
	B_{I,J}\left[\begin{array}{ccc}
		1&1&0\\
		0&1&1
	\end{array}\right].
\]
	如果对某些$s\in S$有$I=S-\{s\}$, 对某些$t\in T$有$J=T-\{t\}$, 那么我们对$M_{I,J}$使用缩写$M_{\hat{s},\hat{t}}$. 在这种情况下, 当$S=T=[n]$, $M$的第$(i,j)$余子式为
\[
	m_{\hat{i},\hat{j}}=(-1)^{i+j}\operatorname{det}M_{\hat{i},\hat{j}}.
\]
	我们将需要以下关于行列式的著名结论, 叫做柯西-比奈定理. 由于这实际上是一个关于线性代数而非组合数学的陈述, 我们将在练习中概述他的证明.
\begin{thm}
	(柯西-比奈定理)设$Q$是一个$[m]\times [n]$矩阵, 设$R$是一个$[n]\times [m]$矩阵. 那么
	\[
	\operatorname{det}QR=\sum_{K\in \binom{[n]}{m}}\operatorname{det}Q_{[m],K}\cdot\operatorname{det}R_{K,[m]}.
	\]
\end{thm}

	注意, 在特例$m=n$中, 这将简化为众所周知的表述$\operatorname{det}QR=\operatorname{det}Q\cdot \operatorname{det}R$.
\begin{thm}
	(矩阵树定理)设$G$为一个图, 并且$V=[n],E=[m]$, 设$L=L(G)$. 对任意$i,j\in [n]$有
	\[
	\#\mathcal{S}T(G)=\ell_{\hat{i},\hat{j}}
	\]
\end{thm}
\begin{proof}
	我们将在$i=j=n$的情况下进行计算, 因为其他情况类似. 所以
	\[
	\ell_{\hat{n},\hat{n}}=(-1)^{n+n}\operatorname{det}L_{\hat{n},\hat{n}}=\operatorname{det}L_{\hat{n},\hat{n}}.
	\]
	设$D$是$G$的任意方向, 并且$B=B(D)$. 根据命题2.6.2(b), 我们有$L=C(G)-A(G)=BB^{t}$. 因此
	\[
	L_{\hat{n},\hat{n}}=B_{W,E}(B_{W,E})^{t}
	\]
	其中$w=[n-1]$. 应用柯西-比奈定理, 我们得到
	\[
	\ell_{\hat{n},\hat{n}}=\sum_{F\in \binom{E}{n-1}}\operatorname{det}B_{W,E}\operatorname{det}(B_{W,E})^{t}=(\operatorname{det}B_{W,E})^{2}.
	\]
	这个定理将会被证明, 如果我们能证明
	\begin{equation}
	\operatorname{det}B_{W,E}=\left\{\begin{array}{cc}
	\pm1 & \text{若F的边为G的生成树}\\
	0 & \text{其他.}
	\end{array}\right.
	\end{equation}
	注意, $B_{W,F}$是图$D_{F}$的关联矩阵, 其中$V(D_{F})=V,A(D_{F})=F$, 但除去了顶点$n$所在的行. 如果他的基本图是一棵树, 我们就说$D_{F}$是一棵树.
	
	我们首先考虑的情况是$D_{F}$不是一棵树. 我们知道$\# F=n-1$, 所以根据定理1.10.2, $D_{F}$必须是非连通的. 因此$D_{F}$有一个不含顶点$n$的分量. 根据命题2.6.2(a), $B_{W,F}$行向量的和对应分量为零向量. 因此在这种情况下$\operatorname{det}B_{W,F}=0$.
	
	现在假设$D_{F}$是一棵树. 为了证明(2.14)的这种情况, 只需排列$B_{W,F}$的行和列就足够了, 使得矩阵变成下三角形, 对角线元素为$\pm 1$. 这种排列对应于$D_{F}$的顶点和边的重新标记. 如果$n=1$, 那么$B_{W,F}$是行列式为1的空矩阵. 如果$n>1$, 那么, 根据引理1.10.1, $D_{F}$至少有两片叶子. 特别是, $W=[n-1]$中有一片叶子. 通过重新标记$D_{F}$, 我们可以假设$v=1$是叶子, $a=1$是包含$v$的唯一弧. 由此得出$B_{W,F}$的第一行的(1,1)位置为$\pm$1, 其他位置为0. 现在我们考虑$D_{F}-v$和递归来完成矩阵的构造.
\end{proof}

	我们可以用这个定理来重新推导Cayley的结果, 定理1.10.3, 枚举给定顶点集上的所有树. 为此, 考虑顶点集为$V=[n]$的完全图$K_{n}$. 显然, 有$n$个顶点的树与$K_{n}$的生成树的个数相同. 拉普拉斯算子$L(K_{n})$由$n-1$沿对角线向下构成, 其他地方为$-1$. 所以$L_{\hat{n},\hat{n}}$是同一个矩阵, 但维数为$(n-1)\times (n-1)$. 将矩阵的所有行添加到第一行. 结果是第一行的元素都是同一个数, 因为每列都由$n-1$和$n-2$减1按一定顺序组成. 接下来, 将第一行加到其他每一行. 这将消去这些行中的负数, 并将每个对角线项从$n-1$改为$n$. 现在矩阵是上三角形的, 因为矩阵的初等行变换不会改变行列式, 我们得到$\ell_{\hat{n},\hat{n}}$是对角线项的乘积, 其包含了$n$中的一项或$n-2$项. Cayley定理得证.
	
\section{习题}

\begin{enumerate}
	\item[(1)]
	设$n$为正整数,$p_1,\dots,p_k$为不同素数. 证明$1$到$n$之间不能被任何$p_i$整除的整数的个数是
	$$
	n-\sum_{1\le i\le k}\left\lfloor\frac{n}{p_i}\right\rfloor+
	\sum_{1\le i<j\le k}\left\lfloor\frac{n}{p_ip_j}\right\rfloor-\dots+
	(-1)^k\left\lfloor\frac{n}{p_1p_2\dots p_k}\right\rfloor.
	$$
\end{enumerate}


\begin{enumerate}
	\item[(2)]设$A(n)$为$\rho\vdash[n]$的个数,使得对于任何$i\in[n-1]$,$i,i+1$永远不会出现在$\rho$的同一块中.
	\begin{enumerate}
		\item[(a)] 证明
		$$
		A(n)=\sum_{i=0}^{n-1}(-1)^i\binom{n-1}{i}B(n-i),
		$$
		其中$B(n)$是第$n$个贝尔数. 
		\item[(b)] 找到并证明涉及第二类斯特林数的类似恒等式. 
		\item[(c)] 表明(a)部分遵循(b)部分. 
	\end{enumerate}
\end{enumerate}

	
\begin{enumerate}
		\item[(3)]
		固定正整数$k\le n$. 使用包含和排除原理找到一个组合数$\alpha=[\alpha_1,\dots,\alpha_k]\models n$的公式,其性质是对所有的$i\in[k]$有$\alpha_i\ge 2$. 
\end{enumerate}
	
\begin{enumerate}
	\item[(4)]用两种方法证明对$n\ge 3$有
	$$
	D(n)=(n-1)(D(n-1)+D(n-2))
	$$
	\begin{enumerate}
		\item[(a)] 利用定理2.1.2,
		\item[(b)] 通过组合论证. 
	\end{enumerate}
\end{enumerate}	
	

\begin{enumerate}
	\item[(5)]
    证明对$n\ge 1$有
    $$
    D(n)=nD(n-1)+(-1)^n.
    $$
\end{enumerate}


\begin{enumerate}
	\item[(6)]
	如果$gcd(k,n)=1$,其中$gcd$是最大公约数,则调用两个相对素数的正整数$k$,$n$.Euler Toticent函数,也称为Euler phi函数,是
	$$
	\phi(n)=\#\{k\in[n]|gcd(k,n=1)\}.
	$$
	使用PIE,证明乘积覆盖所有素数$p$除以$n$的
	$$
	\phi(n)=n\prod_p\left(1-\frac{1}{p}\right).
	$$
\end{enumerate}

\begin{enumerate}
	\item[(7)]
     利用定理1.5.1给出了当$S$是有限的时引理2.2.1的另一个证明. 
\end{enumerate}

\begin{enumerate}
	\item[(8)]  固定集合$A$和子集$A_1,\dots,A_n\subseteq A$.用(2.7)定义$I\subseteq[n]$的$A_I$.证明$$A_\emptyset=A.$$
\end{enumerate}


\begin{enumerate}
	\item[(9)] 证明对于第一类(有符号)斯特林数
     $$
     \sum_ks(n,k)=\begin{cases}
     1 & \text { if }n=0 \text { or }1, \\
     0 & \text { if }n\ge2,
     \end{cases} 
     $$
	使用符号可逆对合.
\end{enumerate}

\begin{enumerate}
	\item[(10)]  填写定理2.2.2的证明细节. 
\end{enumerate}

\begin{enumerate}
	\item[(11)]  
	考虑排列$\pi\in P(s)$和$\sigma\in P(T)$,其中$S\cap T=\emptyset$.$\pi$和$\sigma$的无序排列是
	$$
	\pi!\sigma=\{
	\tau\in P(S\uplus T)|\pi\text{和}\sigma\text{是}\tau text{的子字}
	\}
	$$
	例如,$31!24=\{
	3124,3214,3241,2314,2341,2431
	\}.$
	我们把排列的线性组合当作向量. 例如
	$$6(3124)-7(3241)-9(3124)+(3241)=-3(3124)-6(3241).$$
	一组排列表示集合中所有元素的和,系数为1. 所以我们也会写
	$$31!24=3124+3214+3241+2314+2341+2431$$以及让上下文确定$31!24$是指集合还是和. 证明
	$$
	\sum_{k\ge1}(-1)^k\sum_{w_1\cdot w_2\cdot \dots w_k=12\dots n}w_1!w_2!\dots !w_k=(-1)^n(n\dots21),
	$$
	其中,总和覆盖所有串联$w_1\cdot w_2\cdot \dots w_k=12\dots n$. 例如,当$n=3$时,则串联为$$123=1\cdot2\cdot3=1\cdot23=12\cdot3=123.$$
	提示:考虑一个包含在无序排列
	$w_1!w_2!\dots !w_k$中的排列$v$. 找到最大索引$j\ge0$(如果有),这样
	(i)$|w_1|=|w_2|=\dots=|w_j|=1$(这意味着$i\in[j]$代表$w_i=i$)以及
	(ii)$j\dots21$是$v$的子字.
	使用$j$和$j+1$在$v$中的相对位置以及合并和拆分,在另一个相反符号的无序排列中找到$v$的副本. 
\end{enumerate}

\begin{enumerate}
	\item[(12)]  证明引理2.3.1. 
\end{enumerate}


\begin{enumerate}
	\item[(13)]下面是一种获得直接双射$g:P_d(n)\to P_o(n)$的方法.考虑$\lambda\in P_d(n)$.对于某些奇数h和整数x,$\lambda$的每个部分$p$都可以唯一地写为$p=q2^r$. 将$p$替换为$2^r$的$q$个副本以获得分区$\mu=g(\lambda)$. 例如,如果$\lambda=(6,4,1)$那么$6=3\cdot 2^1=3+3,4=1\cdot 2^2=1+1+1+1$,以及$1=1\cdot2^0=1.$因此$g(6,4,1)=(3,3,1,1,1,1,1)$.	
	\begin{enumerate}
		\item[(a)] 证明$g$是双射
		\item[(b)] 证明$g$与定理2.3.3证明中使用对合原理得到的双射相同. 
	\end{enumerate}
\end{enumerate}


\begin{enumerate}
	\item[(14)]如果每个组件都有一个称为该组件根的可分辨顶点,则调用$G$根. 如果有一个双射$f:V_1\to V_2$同时保持根($r$是$G_1$的根,当且仅当$f(r)$是$G_2$的根)和边($vw\in E_1$当且仅当$f(v)f(w)\in E_2$),则两个无标号有根图$G_1=(V_1,E_1)$和$G_2=(V_2,E_2)$相等. 调用有根树$T$,即使有一些边$rv$,其中$r$是根,这样从$T$中删除该边并使$v$成为其分量的根,就会得到一个具有两个相等分量的图. 如果根林的所有组件树都不相等,则将其称为不同的. 
	\begin{enumerate}
		\item[(a)] 使用Garsia–Milne对合原理,在没有均匀分量树的$n$顶点上的有根族和不同$n$顶点上的有根族之间找到一个双射. 
		\item[(b)] 使用练习13中的想法描述(a)的双射. 
		\item[(c)] 证明(a)和(b)中的双射实际上是相同的. 
	\end{enumerate}
\end{enumerate}

\begin{enumerate}
	\item[(15)]可以用以下方式推广定理2.3.3. 固定正整数$m$.设$P_{<m}(n)$为$\lambda \mapsto n $的集合,其中每个部分的重复次数少于$m$次. 设$P_{\neq !m}(n)$是$\lambda \mapsto n $的集合,使得所有部分都不能被$m$整除.
	\begin{enumerate}
		\item[(a)] 证明$P_{<2}(n)=P_{d}(n)$和$P_{\neq !2}(n)=P_o(N)$.
		\item[(b)] 通过推广前面练习的双射来证明$\#P_{<m}(n)=\#P_{\neq !m}(n)$. 
		\item[(c)] 用对合原理重新计算$\#P_{<m}(n)=\#P_{\neq !m}(n)$. 
		\item[(d)] 证明(b)和(c)中的双射是相同的. 
	\end{enumerate}
\end{enumerate}


\begin{enumerate}
	\item[(16)]设$\mathcal{S}=(S;S_1,\dots,S_n)$,其中$\mathcal{S}$是有限集并且$S_1,\dots,S_n$是子集.同样地定义$\mathcal{J}=(T;T_1,\dots,T_n)$.若对所有的$I\subseteq[n]$有$\#S_I=\#T_I$,那么就叫$\mathcal{S}$和$\mathcal{J}$是筛当量. 
	\begin{enumerate}
		\item[(a)] 使用PIE证明如果$\mathcal{S}$和$\mathcal{J}$是筛当量,就有
		$$
		\left|S-\bigcup_{i=1}^nS_i\right|=\left|T-\bigcup_{i=1}^nT_i	\right|.
		$$
		\item[(b)] 如果$\mathcal{S}$和$\mathcal{J}$是筛当量,那么可以利用对合原理构造双射证明(a). 
	\end{enumerate}
\end{enumerate}


\begin{enumerate}
	\item[(17)]
	\begin{enumerate}
		\item[(a)] 检查定理2.4.2证明中使用的线$L$的形式是否正确. 	
		使用此方程式验证$(0,0)$和$(k,n-k)$位于$L$的两侧. 
		\item[(b)] 利用二项式系数的阶乘表达式,给出该定理的第二个证明. 
		\item[(c)] 用归纳法给出这个定理的第三个证明. 
	\end{enumerate}
\end{enumerate}


\begin{enumerate}
	\item[(18)]  考虑长度为$n$的格路,从原点开始,到$(x,y)$结束,使用步骤$N,E,S,W$,其中$S=[0,-1],W=[-1,0]$.令$r=(n-x-y)/2,s=(n+x-y)/2$.
	\begin{enumerate}
		\item[(a)] 证明此类路径的数目由
		$$\binom{n}{r}\binom{n}{s}.$$
		给出. 	
		提示:找到一个具有$EW$-格路对的双射,这些路径在第一章练习33中定义. 
		\item[(b)]证明在$x$轴上方保持弱的路径数为
		$$\binom{n}{r}\binom{n}{s}-\binom{n}{r-1}\binom{n}{s-1}.$$
		\item[(c)] 证明对于正整数$n,r$,序列
		$$\binom{n}{r}\binom{n}{0},\binom{n}{r-1}\binom{n}{1},\dots,\binom{n}{0}\binom{n}{r}$$	
		是单峰的. 
	\end{enumerate}
\end{enumerate}



\begin{enumerate}
	\item[(19)]设$D$为有向图. 
	\begin{enumerate}
		\item[(a)] 证明从$u$到$v$的任意有向行走,其中$u\ne v$包含从$u$到$v$的有向路径.
		\item[(b)] 证明从$u$到$v$长度至少为2的带$u=v$的任何有向游动都包含一个有向循环. 
	\end{enumerate}
\end{enumerate}


\begin{enumerate}
	\item[(20)]使用因子的二项式系数公式证明对于$n\in\mathbb{N}$,序列
		$$\binom{n}{0},\binom{n}{1},\dots,\binom{n}{n}$$是对数凹的.
\end{enumerate}


\begin{enumerate}
	\item[(21)]设$a_0,a_1,\dots,a_n$是一个正实数序列.证明序列是对数凹的当且仅当所有$0<k\le l<n$都有$$a_ka_l\ge a_{k-1}a_{l+1}.$$
	提示:使用命题2.5.1证明中的想法. 
\end{enumerate}

\begin{enumerate}
	\item[(22)]
	\begin{enumerate}
		\item[(a)] 设$t(n,k)$是$0\le k\le n$的实数三角形数组. 如果序列$t(n,0),\dots,t(n,n)$对所有$n$都是对数凹的,则称$k$中的数组为对数凹. 
		假设$t(n,k)$满足递归
		$$t(n,k)=a(n,k)t(n-1,k-1)+b(n,k)t(n-1,k),$$其中$n\ge1$,$a(n,k),b(n,k),t(n,k)$是非负实数,并且对于$k<0$或者$k>n$有$a(n,k)=b(n,k)=t(n,k)=0$.还假设
		\begin{enumerate}
			\item[(i)]$a(n,k)$和$b(n,k)$是$k$中的对数凹以及
		    \item[(ii)]$n\ge1$,时有$a(n,k-1)b(n,k+1)+a(n,k+1)b(n,k-1)\le 2a(n,k)b(n,k)$. 
		\end{enumerate}
        证明$t(n,k)$是$k$中的对数凹.
        \item[(b)]使用第(a)部分证明$\binom{n}{k},c(n,k)$((第一类无符号斯特林数),和$S(n,k)$ (第二类斯特林数)是$k$中的对数凹.
     \end{enumerate}
\end{enumerate}

\begin{enumerate}
	\item[(23)]设$0\le k<n$.通过使用因子的二项式系数表达式和使用格路径这两种方法证明
	$${\binom{n}{k}}^2\ge\binom{n-1}{k}\binom{n+1}{k}.
	$$
\end{enumerate}


\begin{enumerate}
	\item[(24)]检查一般路径族$P=(P_1,\dots,P_n)$定义的$\Omega$是否为反向对合符号.
\end{enumerate}	

\begin{enumerate}
	\item[(25)]证明定理2.5.5. 
\end{enumerate}	

\begin{enumerate}
	\item[(26)]考虑第一类无符号斯特林数的序列$c(n,0),\dots,c(n,n)$.
	\begin{enumerate}
		\item[(a)]用引理2.5.2证明这个序列是对数凹的. 提示:尝试用$V={\mathbb{Z}}^2$构造$D$,使从(0,0)到$(n,k)$的路径数为$c(n,k)$. 使用多个但可区分的弧将很有帮助. 
		\item[(b)] 用引理2.5.4证明,实际上,这是一个PF序列. 
	\end{enumerate}
\end{enumerate}

\begin{enumerate}
	\item[(27)]
	\begin{enumerate}
		\item[(a)]找到一个单峰但不是对数凹的正实数序列. 
		\item[(b)] 找到一个对数凹但不是PF的正实数序列. 
	\end{enumerate}
\end{enumerate}

\begin{enumerate}
	\item[(28)]
	\begin{enumerate}
		\item[(a)]表明$A{(G)}^n$的$(v,w)$项是长度$n$从$v$到$w$的行走次数. 
		\item[(b)] 表明类似的结果适用于有向图. 
	\end{enumerate}
\end{enumerate}


\begin{enumerate}
	\item[(29)]使用矩阵$B(G)$证明Handshaking引理,定理1.9.3. 
\end{enumerate}


\begin{enumerate}
	\item[(30)]证明命题2.6.2(b). 
\end{enumerate}
\begin{figure}[htbp]
	\centering
	\includegraphics*[width=0.4\textwidth]{./fig2/10.png}
	\caption{图$G_6$}
\end{figure}

\begin{enumerate}
	\item[(31)]给出定理2.6.3的两个证明,如下所示. 
	\begin{enumerate}
		\item[(a)] 使用Lindström–Gessle–Vienno引理给出一个证明. 
		\item[(b)]根据下面的大纲进行第二次演示. 
		\begin{enumerate}
			\item[(i)]证明如果$m>n$,那么两边都是零. 
			\item[(ii)]假设$m\le n$,写出$QR$,并使用多线性展开乘积的列,以显示
			$$\operatorname{det}QR=\sum_{\pi\in P(([n],m))}(\operatorname{det}Q_{\cdot,\pi})r_{\pi_1,1}r_{\pi_2,2}\dots r_{\pi_m,m},$$其中,$Q_{\cdot,\pi}$是矩阵,其第$j$列是$Q$的$\pi_{j}$列.
			\item[(iii)]表明在上一个求和中,如果$\pi$包含一个重复条目,则为$\operatorname{det}Q_{\cdot,\pi}=0$. 
			\item[(iv)]证明如果$k\in\binom{[n]}{m}$,那么$\operatorname{det}Q_{[m],K}$可以从和中的所有项中分解出来,其中$\pi$是$K$的置换,剩下的和是$\operatorname{det}R_{K,[m]}$. 
		\end{enumerate}	
	\end{enumerate}
\end{enumerate}



\begin{enumerate}
	\item[(32)]证明定理2.6.4的情况,其中$i=1$和$j=2$.
\end{enumerate}

\begin{enumerate}
	\item[(33)]设$G_n$是具有顶点集$V=[n]$和边集
	$$
	E=\{12,13,14,\dots,1n,23\}.
	$$
	图$G_6$如图2.7所示. 用两种方法求$G_n$的生成树数:直接计数法和矩阵树定理. 
\end{enumerate}

\begin{enumerate}
	\item[(34)]完整的双向图,$K_{m,n}$,有$v=\{v_1,\dots,v_m,w_1,\dots,w_n\}$和由$v_iw_j$组成的边集,对所有$i,j$(没有其他边)成立. 证明
	$$\#\mathcal{S}T(K_{m,n})=m^{n-1}n^{m-1}.$$
\end{enumerate}




\end{document}
