\documentclass{report}
\usepackage{amsthm}
\usepackage{amsmath}
\usepackage{newtxmath}
\usepackage{ctex} %显示中文的宏包
\usepackage{graphicx}
\usepackage{geometry}
\usepackage{multirow}
%\numberwithin{equation}
\geometry{left=3cm,right=3cm,top=3cm,bottom=2cm}

\linespread{1.5}
%\renewcommand\theequation{\thechapter-\arabic{equation}}

\begin{document}
	第 6 章 递 推 关 系
	递推关系几乎在所有的数学分支中都有重要作用, 对于组合数学更是如此. 这 是因为每个组合问题都有它的组合结构, 而在许多情况下递推关系是刻画组合结 构的最合适的工具. 如何建立递推关系, 已给的递推关系有何性质, 以及如何求解 递推关系等, 是递推关系中的几个基本问题.
	本章首先讨论递推关系的建立问题, 然后对一些常见的递推关系做比较深人 的讨论, 并给出其解法.
	6. 1 递推关系的建立
	在 4.3.2 小节中讨论集合 $\{1,2, \cdots, n\}$ 的错排数 $D_n$ 时, 我们建立了关于 $D_n$ 的 递推关系
	$$
	\left\{\begin{array}{l}
	D_n=(n-1)\left(D_{n-1}+D_{n-2}\right) \quad(n \geqslant 3), \\
	D_1=0, \quad D_2=1,
	\end{array}\right.
	$$
	并由此推出了
	$$
	\left\{\begin{array}{l}
	D_n=n D_{n-1}+(-1)^n \quad(n \geqslant 2), \\
	D_1=0 .
	\end{array}\right.
	$$
	等式 (6.1.1) 和等式 (6.1.2) 都是递推关系的例子. 等式 (6.1.1) 给出了 $n$ 元错排 数 $D_n$ 同 $n-1$ 元错排数及 $n-2$ 元错排数 $D_{n-2}$ 之间的关系, 这样, 由初值 $D_1$ 和 $D_2$ 就可以计算出 $D_3$, 由 $D_2$ 和 $D_3$ 又可以计算出 $D_4$, 如此可以逐个计算出错排数序列 $D_1, D_2, D_3, \cdots$. 而等式 (6.1.2) 给出了 $n$ 元错排数 $D_n$ 同 $n-1$ 元错排数 $D_{n-1}$ 之间 的关系, 这样由初始值 $D_1$ 就唯一地确定了错排数序列.
	
	定义 6.1.1 给定一个数的序列 $H(0), H(1), \cdots, H(n), \cdots$. 若存在整数 $n_0$, 使当 $n \geqslant n_0$ 时,可以用等号 (或大于号,小于号) 将 $H(n)$ 与其前面的某些项 $H(i)$ $(0 \leqslant i<n)$ 联系起来, 这样的式子就叫作递推关系.
	下面通过几个例子来看看如何建立递推关系, 至于递推关系的求解,将在后面 的几节中讨论.
	例 1 (Hanoi 塔问题) 现有 $A, B, C$ 三根立柱以及 $n$ 个大小不等的中空圆盘, 这些圆盘自小到大套在 $A$ 柱上形成塔形, 如图 6. $1.1$ 所示. 要把 $n$ 个圆盘从 $A$ 柱上 搬到 $C$ 柱上,并保持原来的顺序不变, 要求每次只能从一根立柱上拿下一个圆盘放 在另一根立柱上,且不允许大盘压在小盘上. 问至少要搬多少次?
	图 6.1.1
	解 记 $f(n)$ 为 $n$ 个圆盘从 $A$ 柱搬到 $C$ 柱所需的最小次数. 整个搬动过程可以人 分成三个阶段:
	(1) 将套在 $A$ 柱上面的 $n-1$ 个圆盘从 $A$ 柱按要求搬到 $B$ 柱,搬动次数为 $f(n-1)$;
	(2)把 $A$ 柱上最下面的那个圆盘搬到 $C$ 柱上,搬动次数为 1 ;
	(3)把 $B$ 柱上的 $n-1$ 个圆盘按要求搬到 $C$ 柱上, 搬动次数为 $f(n-1)$. 由加法原则知
	$$
	f(n)=2 f(n-1)+1,
	$$
	又显然 $f(1)=1$, 所以有如下带有初值的递推关系
	$$
	\left\{\begin{array}{l}
	f(n)=2 f(n-1)+1, \\
	f(1)=1 .
	\end{array}\right.
	$$
	例 2 在信道上传输由 $a, b, c$ 三个字母组成的长为 $n$ 的字符串, 若字符串中 有两个 $a$ 连续出现, 则信道就不能传输. 令 $f(n)$ 表示信道可以传输的长为 $n$ 的字 符串的个数, 求 $f(n)$ 满足的递推关系.
	解 信道上能够传输的长度为 $n(n \geqslant 2)$ 的字符串可分成如下四类:
	(1) 最左字符为 $b$;
	(2) 最左字符为 $c$;
	(3) 最左两个字符为 $a b$;
	(4) 最左两个字符为 $a c$.
	如图 6.1.2 所示, 前两类字符串分别有 $f(n-1)$ 个, 后两类字符串分别有 $f(n-2)$ 个. 容易求出 $f(1)=3, f(2)=8$, 从而得到
	$$
	\left\{\begin{array}{l}
	f(n)=2 f(n-1)+2 f(n-2) \quad(n \geqslant 3), \\
	f(1)=3, \quad f(2)=8 .
	\end{array}\right.
	$$
	图 6.1.2
	例 3 考虑 0,1 字符串中“ 010 ” 子串的相继出现问题.例如, 在 110101010101 中, 我们说“010” 在第 5 位和第 9 位出现, 而不是在第 7 位和第 11 位出现, 在整个字 符串中“ 010 ” 共出现两次. 计算 $n$ 位 0,1 字符串中“ 010 ” 子串在第 $n$ 位出现的字符 串有多少?
	解 设“ 010 ” 子串在第 $n$ 位出现的长为 $n$ 的 0,1 字符串的个数为 $f(n)$, 则显然 $f(3)=1, f(4)=2, f(5)=3$. 符串有 $f(n)$ 个, 而 “ 010 ” 不在第 $n$ 位出现, 当且仅当最后 5 位形如“ 01010 ”, 并且
	
	
	解 将所有满足要求的着色方案分成两类 $(n \geqslant 4)$ :
	(1) $D_1$ 与 $D_{n-1}$ 同色. 此时, $D_n$ 有 $k-1$ 种着色方案. 可将 $D_1$ 与 $D_{n-2}$ 看成相邻 区域, $D_1, D_2, \cdots, D_{n-2}$ 的着色方案数为 $f(n-2)$. 故此类着色方案数为 $(k-1)$ - $f(n-2)$.
	(2) $D_1$ 与 $D_{n-1}$ 异色. 此时, $D_n$ 有 $k-2$ 种着色方案. 此时, 可将 $D_1$ 与 $D_{n-1}$ 看 成相邻的区域. 又 $D_1, D_2, \cdots, D_{n-1}$ 用 $k$ 种颜色着色的方案数为 $f(n-1)$, 故此类着 色方案数为 $(k-2) f(n-1)$.
	而容易求得 $f(2)=k(k-1), f(3)=k(k-1)(k-2)$, 从而有
	$$
	\left\{\begin{array}{l}
	f(n)=(k-1) f(n-2)+(k-2) f(n-1) \quad(n \geqslant 4), \\
	f(2)=k(k-1), \quad f(3)=k(k-1)(k-2) .
	\end{array}\right.
	$$
	例 5 设 $X$ 是一具有乘法运算的代数系统,乘法不满足结合律,用 $x y$ 表示 $x$ 对 $y$ 之积. 如果
	$$
	x_1, x_2, \cdots, x_n \in X,
	$$
	而且这 $n$ 个元素依上面列出的顺序所能作出的一切可能的积彼此不同, 其个数记 为 $f(n)$, 求 $f(n)$ 满足的递推关系.
	解 例如, 对于 $x_1, x_2, x_3 \in X$, 符合题意的积有 2 个:
	$$
	\left(x_1 x_2\right) x_3, \quad x_1\left(x_2 x_3\right),
	$$
	所以 $f(3)=2$.
	如果在 $x_1 x_2 \cdots x_n$ 的某些字母间加上括号, 但不改变字母间的相互位置关系, 使得这 $n$ 个字母间的乘法可以按所加括号指明的运算方式进行运算,那么 $f(n)$ 就 是加括号的方法的个数.
	最外层的两对括号形如
	$$
	\left(x_1 \cdots x_r\right)\left(x_{r+1} \cdots x_n\right) \quad(1 \leqslant r \leqslant n-1),
	$$
	当 $r=1$ 或 $n-1$ 时,通常简记为
	$$
	\begin{aligned}
	&x_1\left(x_2 \cdots x_n\right)=\left(x_1\right)\left(x_2 \cdots x_n\right), \\
	&\left(x_1 \cdots x_{n-1}\right) x_n=\left(x_1 \cdots x_{n-1}\right)\left(x_n\right) .
	\end{aligned}
	$$
	在前一个括号中有 $f(r)$ 种加括号的方法, 在后一个括号中又有 $f(n-r)$ 种加括号 的方法, 当 $r$ 遍历 $1,2, \cdots, n-1$ 时, 就得到
	$$
	\begin{aligned}
	f(n)=& f(1) f(n-1)+f(2) f(n-2)+\cdots \\
	&+f(n-2) f(2)+f(n-1) f(1) \\
	=& \sum_{i=1}^{n-1} f(i) f(n-i) \quad(n>1) .
	\end{aligned}
	$$
	初始值为
	$$
	f(1)=1, \quad f(2)=1
	$$
	6. 2 常系数线性齐次递推关系的求解
	定义 6.2.1 设 $k$ 是给定的正整数, 若数列 $f(0), f(1), \cdots, f(n), \cdots$ 的相邻 $k+1$ 项间满足关系
	$$
	f(n)=c_1(n) f(n-1)+c_2(n) f(n-2)+\cdots
	$$
	$$
	+c_k(n) f(n-k)+g(n),
	$$
	对 $n \geqslant k$ 成立, 其中 $c_k(n) \neq 0$, 则称该关系为 $\{f(n)\}$ 的 $k$ 阶线性递推关系. 如果 $c_1(n), c_2(n), \cdots, c_k(n)$ 都是常数, 则称之为 $k$ 阶常系数线性递推关系. 如果 $g(n)=0$, 则称之为齐次的.
	如果有一个数列代人递推关系 (6.2.1), 使得其对任何 $n \geqslant k$ 都成立, 则称这 个数列是递推关系 $(6.2 .1)$ 的解.
	常系数线性齐次递推关系的一般形式为
	$$
	f(n)=c_1 f(n-1)+c_2 f(n-2)+\cdots
	$$
	$$
	+c_k f(n-k) \quad\left(n \geqslant k, c_k \neq 0\right) \text {. }
	$$
	定义 6.2.2 方程
	$$
	x^k-c_1 x^{k-1}-c_2 x^{k-2}-\cdots-c_k=0
	$$
	叫作递推关系 (6.2.2) 的特征方程. 它的 $k$ 个根 $q_1, q_2, \cdots, q_k$ (可能有重根) 叫作该 递推关系的特征根, 其中, $q_i(i=1,2, \cdots, k)$ 是复数.
	引理 6.2.1 设 $q$ 是非零复数, 则 $f(n)=q^n$ 是递推关系 $(6.2 .2)$ 的解, 当且 仅当 $q$ 是它的特征根.
	证明 设 $f(n)=q^n$ 是递推关系 (6.2.2) 的解, 即
	$$
	q^n=c_1 q^{n-1}+c_2 q^{n-2}+\cdots+c_k q^{n-k} \quad(n \geqslant k) .
	$$
	因为 $q \neq 0$, 所以
	$$
	q^k=c_1 q^{k-1}+c_2 q^{k-2}+\cdots+c_k,
	$$
	即 $q$ 是递推关系 (6.2.2) 的特征根. 反之亦然.
	引理 6.2.2 如果 $h_1(n), h_2(n)$ 都是递推关系 (6.2.2) 的解, $b_1, b_2$ 是常数, 则 $b_1 h_1(n)+b_2 h_2(n)$ 也是递推关系 (6.2.2) 的解.
	证明 因为 $h_1(n), h_2(n)$ 都是递推关系 (6.2.2) 的解, 所以
	$$
	b_1 h_1(n)+b_2 h_2(n)=b_1\left[c_1 h_1(n-1)+\cdots+c_k h_1(n-k)\right]
	$$
	$$
	\begin{aligned}
	&+b_2\left[c_1 h_2(n-1)+\cdots+c_k h_2(n-k)\right] \\
	=& c_1\left[b_1 h_1(n-1)+b_2 h_2(n-1)\right]+\cdots \\
	&+c_k\left[b_1 h_1(n-k)+b_2 h_2(n-k)\right],
	\end{aligned}
	$$
	从而 $b_1 h_1(n)+b_2 h_2(n)$ 也是递推关系 (6.2.2) 的解.
	由引理 6.2.1 和引理 $6.2 .2$ 知, 若 $q_1, q_2, \cdots, q_k$ 是递推关系 (6.2.2) 的特征 根, $b_1, b_2, \cdots, b_k$ 是常数,那么
	$$
	f(n)=b_1 q_1{ }^n+b_2 q_2{ }^n+\cdots+b_k q_k{ }^n
	$$
	也是递推关系 $(6.2 .2)$ 的解.
	定义 6.2.3 如果对于递推关系 (6.2.2) 的每个解 $h(n)$, 都可以选择一组常 数 $c_1{ }^{\prime}, c_2{ }^{\prime}, \cdots, c_k{ }^{\prime}$, 使得
	$$
	h(n)=c_1{ }^{\prime} q_1{ }^n+c_2{ }^{\prime} q_2{ }^n+\cdots+c_k{ }^{\prime} q_k{ }^n
	$$
	成立, 则称 $b_1 q_1{ }^n+b_2 q_2{ }^n+\cdots+b_k q_k{ }^n$ 是递推关系 (6.2.2) 的通解, 其中, $b_1, b_2$, $\cdots, b_k$ 为任意常数.
	定理 6.2.1 设 $q_1, q_2, \cdots, q_k$ 是递推关系 (6.2.2) 的 $k$ 个互不相等的特征根, 则
	$$
	f(n)=b_1 q_1{ }^n+b_2 q_2{ }^n+\cdots+b_k q_k{ }^n
	$$
	是递推关系 (6.2.2) 的通解.
	证明 由前面的分析可知, 对任意一组 $b_1, b_2, \cdots, b_k, f(n)$ 是递推关系 (6.2.2) 的解.
	下面证明: 递推关系 (6.2.2) 的任意一个解 $h(n)$ 都可以表示成 $(6.2 .4)$ 的形 式. $h(n)$ 是 (6.2.2) 的解, 故 $h(n)$ 由 $k$ 个初值 $h(0)=a_0, h(1)=a_1, \cdots$, $h(k-1)=a_{k-1}$ 唯一地确定. 若 $h(n)$ 可以表示成式(6.2.4) 的形式, 则有
	$$
	\left\{\begin{array}{l}
	b_1+b_2+\cdots+b_k=a_0, \\
	b_1 q_1+b_2 q_2+\cdots+b_k q_k=a_1, \\
	\cdots, \\
	b_1 q_1{ }^{k-1}+b_2 q_2{ }^{k-1}+\cdots+b_k q_k{ }^{k-1}=a_{k-1} .
	\end{array}\right.
	$$
	如果方程组 (6.2.5) 有唯一解 $b_1{ }^{\prime}, b_2{ }^{\prime}, \cdots, b_k{ }^{\prime}$, 这说明可以找到 $k$ 个常数 $b_1{ }^{\prime}, b_2{ }^{\prime}$, $\cdots, b_k{ }^{\prime}$, 使得
	$$
	h(n)=b_1{ }^{\prime} q_1{ }^n+b_2{ }^{\prime} q_2{ }^n+\cdots+b_k{ }^{\prime} q_k{ }^n
	$$
	成立, 从而 $b_1 q_1{ }^n+b_2 q_2{ }^n+\cdots+b_k q_k{ }^n$ 是该递推关系的通解. 考察方程组 (6.2.5), 它的系数行列式为$$
	\left|\begin{array}{cccc}
	1 & 1 & \cdots & 1 \\
	q_1 & q_2 & \cdots & q_k \\
	\vdots & \vdots & \vdots & \vdots \\
	q_1{ }^{k-1} & q_2{ }^{k-1} & \cdots & q_k{ }^{k-1}
	\end{array}\right|=\prod_{1 \leqslant i<j \leqslant k}\left(q_j-q_i\right)
	$$
	这是著名的 Vandermonde 行列式. 因为 $q_1, q_2, \cdots, q_k$ 互不相等, 所以该行列式不 等于零,这也就是说方程组 (6.2.5) 有唯一解. 所以, $h(n)$ 可以表示成式 (6.2.4) 的形式.
	故式 (6.2.4) 是递推关系 (6.2.2) 的通解.
	例 1 求解 6.1 节例 2 中的递推关系
	$$
	\left\{\begin{array}{l}
	f(n)=2 f(n-1)+2 f(n-2), \\
	f(1)=3, \quad f(2)=8 .
	\end{array}\right.
	$$
	解 先求这个递推关系的通解. 它的特征方程为
	解这个方程,得
	$$
	x^2-2 x-2=0
	$$
	所以,通解为
	$$
	x_1=1+\sqrt{3}, \quad x_2=1-\sqrt{3} .
	$$
	$f(n)=c_1(1+\sqrt{3})^n+c_2(1-\sqrt{3})^n$.
	$c_2$ ,得
	$$
	\left\{\begin{array}{l}
	c_1(1+\sqrt{3})+c_2(1-\sqrt{3})=3, \\
	c_1(1+\sqrt{3})^2+c_2(1-\sqrt{3})^2=8 .
	\end{array}\right.
	$$
	求解这个方程组, 得
	$$
	c_1=\frac{2+\sqrt{3}}{2 \sqrt{3}}, \quad c_2=\frac{-2+\sqrt{3}}{2 \sqrt{3}} .
	$$
	因此, 所求的字符串个数为
	$$
	f(n)=\frac{2+\sqrt{3}}{2 \sqrt{3}}(1+\sqrt{3})^n+\frac{-2+\sqrt{3}}{2 \sqrt{3}}(1-\sqrt{3})^n \quad(n=1,2, \cdots) .
	$$
	例 2 核反应堆中有 $\alpha$ 和 $\beta$ 两种粒子,每秒钟内一个 $\alpha$ 粒子可反应产生三个 $\beta$ 粒子,而一个 $\beta$ 粒子又可反应产生一个 $\alpha$ 粒子和两个 $\beta$ 粒子. 若在时刻 $t=0$ 时反应 堆中只有一个 $\alpha$ 粒子, 问 $t=100$ 秒时反应堆中将有多少个 $\alpha$ 粒子?多少个 $\beta$ 粒子? 共有多少个粒子?
	解 设在 $t$ 时刻的 $\alpha$ 粒子数为 $f(t), \beta$ 粒子数为 $g(t)$, 根据题设,可以列出下面 的递推关系
	$$
	\left\{\begin{array}{l}
	g(t)=3 f(t-1)+2 g(t-1) \quad(t \geqslant 1) \\
	f(t)=g(t-1) \quad(t \geqslant 1) \\
	g(0)=0, \quad f(0)=1
	\end{array}\right.
	$$
	由式(6.2.7) 得到
	$f(t-1)=g(t-2)$,
	$$
	\left\{\begin{array}{l}
	g(t)=3 g(t-2)+2 g(t-1) \quad(t \geqslant 2), \\
	g(0)=0, \quad g(1)=3 .
	\end{array}\right.
	$$
	递推关系 (6.2.8) 的特征方程为
	其特征根为
	$$
	x^2-2 x-3=0,
	$$
	所以,该递推关系的通解为
	$$
	g(t)=c_1 \cdot 3^t+c_2 \cdot(-1)^t .
	$$
	代人初值 $g(0)=0, g(1)=3$, 得
	$$
	\left\{\begin{array}{l}
	c_1+c_2=0, \\
	3 c_1-c_2=3 .
	\end{array}\right.
	$$
	解这个方程组, 得
	$$
	c_1=\frac{3}{4}, \quad c_2=-\frac{3}{4} .
	$$
	所以,该递推关系的解为
	$$
	g(t)=\frac{3}{4} \cdot 3^t-\frac{3}{4} \cdot(-1)^t .
	$$
	从而求得
	$$
	\begin{aligned}
	&f(t)=g(t-1)=\frac{3}{4} \cdot 3^{t-1}-\frac{3}{4} \cdot(-1)^{t-1}, \\
	&f(t)+g(t)=\frac{3}{4} \cdot 3^{t-1}-\frac{3}{4} \cdot(-1)^{t-1}+\frac{3}{4} \cdot 3^t-\frac{3}{4} \cdot(-1)^t=3^t .
	\end{aligned}
	$$
	因此
	$$
	\begin{aligned}
	f(100) &=\frac{3}{4} \cdot 3^{99}-\frac{3}{4} \cdot(-1)^{99} \\
	&=\frac{3}{4}\left(3^{99}+1\right), \\
	g(100) &=\frac{3}{4} \cdot 3^{100}-\frac{3}{4} \cdot(-1)^{100}=\frac{3}{4}\left(3^{100}-1\right),
	\end{aligned}
	$$故
	$$
	f(100)+g(100)=3^{100} .
	$$
	对于 $k$ 阶常系数线性齐次递推关系, 当特征根 $q_1, q_2, \cdots, q_k$ 互不相等时, 我们 已经给出了求通解的方法. 但是, 当 $q_1, q_2, \cdots, q_k$ 中有重根时, 这种方法就不再适 用, 换句话说, $c_1 q_1{ }^n+c_2 q_2{ }^n+\cdots+c_k q_k{ }^n$ 就不再是原递推关系的通解. 请看下面 的例子.
	例 3 求解递推关系
	$$
	\left\{\begin{array}{l}
	f(n)=4 f(n-1)-4 f(n-2), \\
	f(0)=1, \quad f(1)=3 .
	\end{array}\right.
	$$
	解 递推关系 (6.2.9) 的特征方程为
	其特征根为
	$$
	x^2-4 x+4=0,
	$$
	$x_1=x_2=2$. 生特根为
	由引理 6. 2.1, 可知 $2^n$ 是递推关系 (6.2.9) 的解(不考虑初值). 我们不妨试试 $n 2^n$, 把它代人式 (6.2.9), 得
	$$
	\begin{aligned}
	n 2^n-4(n-1) 2^{n-1}+4(n-2) 2^{n-2} &=n 2^n-(n-1) 2^{n+1}+(n-2) 2^n \\
	&=2^n[n-2(n-1)+(n-2)] \\
	&=0,
	\end{aligned}
	$$
	这说明 $n 2^n$ 也是递推关系 (6.2.9) 的解. 易知 $n 2^n$ 与 $2^n$ 线性无关, 所以原递推关系 的通解为
	代人初值 $f(0)=1, f(1)=3$, 得
	$$
	\left\{\begin{array}{l}
	c_1=1, \\
	2 c_1+2 c_2=3 .
	\end{array}\right.
	$$
	解这个方程组, 得
	$$
	c_1=1, \quad c_2=\frac{1}{2} .
	$$
	所以,原递推关系的解为
	$$
	f(n)=2^n+2^{n-1} n .
	$$
	下面分析, 当特征根有重根时, 常系数线性齐次递推关系 (6.2.2) 的通解的一 般形式.
	设递推关系
	$$
	f(n)=c_1 f(n-1)+c_2 f(n-2)+\cdots+c_k f(n-k) \quad\left(n \geqslant k, c_k \neq 0\right)
	$$的特征方程为
	$$
	\text { 令 } \begin{aligned}
	x^k-c_1 x^{k-1}-c_2 x^{k-2}-\cdots-c_k=0 . \\
	P(\cdot x) &=x^k-c_1 x^{k-1}-c_2 x^{k-2}-\cdots-c_k, \\
	P_n(x) &=x^{n-k} \cdot P(x) \\
	&=x^n-c_1 x^{n-1}-c_2 x^{n-2}-\cdots-c_k x^{n-k} .
	\end{aligned}
	$$
	如果 $q$ 是 $P(x)=0$ 的二重根, 则 $q$ 也是 $P_n(x)=0$ 的二重根, 那么 $q$ 是 $P_n{ }^{\prime}(x)=$ 0 的根. 其中, $P_n{ }^{\prime}(x)$ 是 $P_n(x)$ 的微商, 即
	$$
	P_n{ }^{\prime}(x)=n x^{n-1}-c_1(n-1) x^{n-2}-c_2(n-2) x^{n-3}
	$$
	因此, $q$ 是 $x P_n^{\prime}(x)=0$ 的根. 而
	$$
	\begin{aligned}
	x P_n{ }^{\prime}(x)=& n x^n-c_1(n-1) x^{n-1}-c_2(n-2) x^{n-2} \\
	&-\cdots-c_k(n-k) x^{n-k},
	\end{aligned}
	$$
	代人 $x=q$, 得
	$$
	n q^n-c_1(n-1) q^{n-1}-c_2(n-2) q^{n-2}-\cdots-c_k(n-k) q^{n-k}=0 .
	$$
	这说明 $n q^n$ 是原递推关系的解.
	类似地可以证明, 如果 $q$ 是 $P(x)=0$ 的三重根, 那么 $q$ 就是 $x P_n{ }^{\prime}(x)=0$ 的二 重根, 即 $q$ 是 $x P_n{ }^{\prime}(x)=0$ 和 $x\left[x P_n{ }^{\prime}(x)\right]^{\prime}=0$ 的根, 从而证明 $n q^n, n^2 q^n$ 也是原 递推关系的解.
	一般地, 可以证明以下的结论: 如果 $q$ 是 $P(x)=0$ 的 $e$ 重根, 则 $q^n, n q^n$, $n^2 q^n, \cdots, n^{c-1} q^n$ 都是原递推关系的解.
	定理 6.2.2 设 $q_1, q_2, \cdots, q_{\mathrm{t}}$ 是递推关系 (6.2.2) 的全部不同的特征根, 其重 数分别为 $e_1, e_2, \cdots, e_t\left(e_1+e_2+\cdots+e_t=k\right)$, 那么递推关系 (6.2.2) 的通解为
	$$
	f(n)=f_1(n)+f_2(n)+\cdots+f_1(n),
	$$
	其中
	$$
	f_i(n)=\left(b_{i_1}+b_{i_2} n+\cdots+b_{i_c} n^{e^e-1}\right) \cdot q_i{ }^n \quad(1 \leqslant i \leqslant t) .
	$$
	证明 由前面的讨论知
	$$
	f(n)=f_1(n)+f_2(n)+\cdots+f_t(n)
	$$
	是递推关系 (6.2.2) 的解. 再由初值 $f(0)=a_0, f(1)=a_1, \cdots, f(k-1)=a_k$, 得 到关于 $b_{i j}$ , $1 \leqslant i \leqslant t, 1 \leqslant j \leqslant e_i$ ) 的线性方程组,其系数行列式的值为 (证明略)
	$$
	\prod_{i=1}^t\left(-q_i\right)^{\left(e_i^i\right)} \prod_{1 \leqslant i<j<t}\left(q_j-q_i\right)^{e_i \cdot e_i} \neq 0,
	$$
	故可由初值唯一地确定 $b_i$, 这说明递推关系 (6.2.2) 的任意解均可写成
	$$
	f(n)=\sum_{i=1}^t f_i(n)
	$$
	的形式,其中, $f_i(n)$ 如前所示.
	例 4 求解递推关系
	$$
	\left\{\begin{array}{l}
	f(n)=-f(n-1)+3 f(n-2)+5 f(n-3)+2 f(n-4), \\
	f(0)=1, \quad f(1)=0, \quad f(2)=1, \quad f(3)=2 .
	\end{array}\right.
	$$
	解 该递推关系的特征方程为
	$$
	x^4+x^3-3 x^2-5 x-2=0,
	$$
	其特征根为
	$$
	x_1=x_2=x_3=-1, \quad x_4=2 \text {. }
	$$
	由定理 $6.2 .2$, 对应于 $x=-1$ 的解为
	$$
	f_1(n)=c_1(-1)^n+c_2 n(-1)^n+c_3 n^2(-1)^n \text {, }
	$$
	对应于 $x=2$ 的解为
	$$
	f_2(n)=c_4 2^n .
	$$
	因此,递推关系的通解为
	$$
	\begin{aligned}
	f(n) &=f_1(n)+f_2(n) \\
	&=c_1(-1)^n+c_2 n(-1)^n+c_3 n^2(-1)^n+c_4 2^n .
	\end{aligned}
	$$
	代人初始值,得到方程组
	$$
	\left\{\begin{array}{l}
	c_1+c_4=1, \\
	-c_1-c_2-c_3+2 c_4=0 \\
	c_1+2 c_2+4 c_3+4 c_4=1, \\
	-c_1-3 c_2-9 c_3+8 c_4=2,
	\end{array}\right.
	$$
	解这个方程组,得
	$$
	c_1=\frac{7}{9}, \quad c_2=-\frac{1}{3}, \quad c_3=0, \quad c_4=\frac{2}{9} .
	$$
	所以,原递推关系的解为
	$$
	f(n)=(-1)^n \frac{7}{9}-(-1)^n \frac{1}{3} n+\frac{2}{9} \cdot 2^n .
	$$
	6.3 常系数线性非齐次递推关系的求解
	$k$ 阶常系数线性非齐次递推关系的一般形式为
	$$
	f(n)=c_1 f(n-1)+c_2 f(n-2)+\cdots+c_k f(n-k)+g(n) \quad(n \geqslant k),
	$$
	其中, $c_1, c_2, \cdots, c_k$ 为常数, $c_k \neq 0, g(n) \neq 0$. 递推关系 (6.3.1) 对应的齐次递推 关系为
	$$
	f(n)=c_1 f(n-1)+c_2 f(n-2)+\cdots+c_k f(n-k) . \quad \text { (6.3.2) }
	$$
	定理 6.3.1 $k$ 阶常系数线性非齐次递推关系 (6.3.1) 的通解是递推关系 (6.3.1) 的特解加上其相应的齐次递推关系 (6.3.2) 的通解.
	证明 设 $f^{\prime}(n)$ 是递推关系 $(6.3 .1)$ 的特解, $f^{\prime \prime}(n)$ 是递推关系 $(6.3 .2)$ 的解,
	则
	$$
	\begin{aligned}
	f^{\prime}(n)+f^{\prime \prime}(n)=& {\left[c_1 f^{\prime}(n-1)+c_2 f^{\prime}(n-2)+\cdots\right.} \\
	&\left.+c_k f^{\prime}(n-k)+g(n)\right] \\
	&+\left[c_1 f^{\prime \prime}(n-1)+c_2 f^{\prime \prime}(n-2)+\cdots+c_k f^{\prime \prime}(n-k)\right] \\
	=& c_1\left[f^{\prime}(n-1)+f^{\prime \prime}(n-1)\right]+\cdots \\
	&+c_k\left[f^{\prime}(n-k)+f^{\prime \prime}(n-k)\right]+g(n) .
	\end{aligned}
	$$
	所以, $f^{\prime}(n)+f^{\prime \prime}(n)$ 是递推关系 $(6.3 .1)$ 的解.
	反之, 任给递推关系 (6.3.1) 的一个解 $f(n)$, 与上类似, 可以证明 $f(n)-$ $f^{\prime}(n)$ 是递推关系 (6.3.2) 的解. 而 $f(n)=\left[f(n)-f^{\prime}(n)\right]+f^{\prime}(n)$, 所以 $f(n)$ 可 以表示成 $f^{\prime}(n)$ 与递推关系 $(6.3 .2)$ 的解之和.
	综合以上分析, 定理成立.
	对于一般的 $g(n), k$ 阶常系数线性非齐次递推关系 (6.3.1) 没有普遍的解法, 只有在某些简单的情况下, 可以用待定系数法求出递推关系 (6.3.1) 的特解. 表 6.3.1 对于几种 $g(n)$ 给出了递推关系 (6.3.1) 的特解 $f^{\prime}(n)$ 的一般形式. 在 $6.5$ 节, 我们将用生成函数的方法证明表 6.3.1 中特解的正确性.
	表 $6.3 .1$
	\begin{tabular}{c|c|c}
		\hline$g(n)$ & 特征多项式 $P(x)$ & 特解 $f^{\prime}(n)$ 的一般形式 \\
		\hline \multirow{2}{*}{$\beta^n$} & $P(\beta) \neq 0$ & $a \beta^n$ \\
		\cline { 2 - 3 } & $\beta$ 是 $P(x)=0$ 的 $m$ 重根 & $a n^m \beta^n$ \\
		\hline \multirow{2}{*}{$n^s$} & $P(1) \neq 0$ & $b_s n^s+b_{s-1} n^{s-1}+\cdots+b_1 n+b_0$ \\
		\cline { 2 - 3 } & 1 是 $P(x)=0$ 的 $m$ 重根 & $n^m\left(b_s n^s+b_{s-1} n^{s-1}+\cdots+b_1 n+b_0\right)$ \\
		\hline \multirow{2}{*}{$n^s \beta^n$} & $P(\beta) \neq 0$ & $\left(b_s n^s+b_{s-1} n^{s-1}+\cdots+b_1 n+b_0\right) \beta^n$ \\
		\cline { 2 - 3 } & $\beta$ 是 $P(x)=0$ 的 $m$ 重根 & $n^m\left(b_s n^s+b_{s-1} n^{s-1}+\cdots+b_1 n+b_0\right) \beta^n$ \\
		\hline
	\end{tabular}
例 1 求解递推关系
$$
\left\{\begin{array}{l}
f(n)=2 f(n-1)+4^{n-1}, \\
f(0)=1 .
\end{array}\right.
$$
解 因为 4 不是特征方程的根, 所以该递推关系的非齐次特解为 $4^{n-1} a$. 将其 代人递推关系, 得
比较等式两边 $4^{n-1}$ 的系数, 得
$$
a=\frac{1}{2} a+1,
$$
从而
而相应齐次递推关系的通解为 $2^n c$, 由定理 6.3.1 知, 非齐次递推关系的通解为 由初值 $f(0)=1$, 得
$$
f(n)=2^n c+4^{n-1} \cdot 2 .
$$
$$
2^0 c+4^{0-1} \cdot 2=1,
$$
从而
$$
c=\frac{1}{2},
$$
故
$$
f(n)=\frac{1}{2}\left(2^n+4^n\right) .
$$
从例 1 可以看出, 求解常系数线性非齐次递推关系的基本步聧是: 首先用待定 系数法通过递推关系 (不带初值) 求出特解, 然后用常系数线性齐次递推关系的通 解与初值解出递推关系的解.
例 2 求和 $1^4+2^4+\cdots+n^4$.
解 令
$$
f(n)=1^4+2^4+\cdots+n^4,
$$
它满足递推关系
$$
\left\{\begin{array}{l}
f(n)=f(n-1)+n^4, \\
f(0)=0 .
\end{array}\right.
$$
因为 1 是特征方程的一重根, 所以该递推关系的特解为
$$
n\left(b_4 n^4+b_3 n^3+b_2 n^2+b_1 n+b_0\right) \text {. }
$$
将其代人递推关系, 并比较等号两边 $n^i(0 \leqslant i \leqslant 4)$ 的系数, 得到

$$
\left\{\begin{array}{l}
	-5 b_4+b_3+1=b_3, \\
	10 b_4-4 b_3+b_2=b_2, \\
	-10 b_4+6 b_3-3 b_2+b_1=b_1, \\
	5 b_4-4 b_3+3 b_2+2 b_1+b_0=b_0, \\
	-b_4+b_3-b_2+b_1-b_0=0 .
\end{array}\right.
$$
解得
$$
b_0=\frac{1}{30}, b_1=0, b_2=\frac{1}{3}, b_3=\frac{1}{2}, b_4=\frac{1}{5},
$$
即非齐次特解为
$$
f^{\prime}(n)=\frac{n}{30}\left(6 n^4+15 n^3+10 n^2-1\right) .
$$
而相应齐次递推关系的通解为
由定理 6.3.1 知, 非齐次通解为
$$
\begin{aligned}
	f(n) &=f^{\prime}(n)+f^{\prime \prime}(n) \\
	&=c+\frac{n}{30}\left(6 n^4+15 n^3+10 n^2-1\right) .
\end{aligned}
$$
又由 $f(0)=0$ 可求得 $c=0$, 故
$$
\begin{aligned}
	f(n) &=\frac{n}{30}\left(6 n^4+15 n^3+10 n^2-1\right) \\
	&=\frac{n}{30}(n+1)(2 n+1)\left(3 n^2+3 n-1\right) .
\end{aligned}
$$
例 3 求解递推关系
$$
\left\{\begin{array}{l}
	f(n)-4 f(n-1)+4 f(n-2)=n \cdot 2^n, \\
	f(0)=0, \quad f(1)=1 .
\end{array}\right.
$$
解 由于 2 是特征方程的二重根, 所以该递推关系的特解为
$$
f^{\prime}(n)=n^2\left(b_1 n+b_0\right) \cdot 2^n .
$$
将它代人递推关系, 并比较等号两边 $n$ 的系数及常数项,得到
$$
\left\{\begin{array}{l}
	6 b_1=1, \\
	-6 b_1+2 b_0=0,
\end{array}\right.
$$
解得
$$
b_0=\frac{1}{2}, \quad b_1=\frac{1}{6} .
$$
而相应齐次递推关系的通解为 $\left(c_0+c_1 n\right) \cdot 2^n$, 从而非齐次递推关系的通解为
$$
f(n)=\left[\left(c_0+c_1 n\right)+n^2\left(\frac{n}{6}+\frac{1}{2}\right)\right] \cdot 2^n .
$$

再由初值 $f(0)=0, f(1)=1$, 求得 $c_0=0, c_1=-\frac{1}{6}$.于是
$$
f(n)=\frac{1}{6}\left(n^3+3 n^2-n\right) \cdot 2^n .
$$
例 4 求解 Hanoi 塔问题满足的递推关系
$$
\left\{\begin{array}{l}
	f(n)=2 f(n-1)+1, \\
	f(0)=0 .
\end{array}\right.
$$
解 相应的特征方程为 $x=2$, 故齐次通解为 $2^n c$. 设非齐次特解为 $b$, 代人原 递推关系, 得
$$
b-2 b=1 \text {, }
$$
所以特解为 $b=-1$. 根据前面的分析, 可知该递推关系的通解为
$f(n)=2^n c-1$.
$$
f(n)=2^n-1 .
$$
6.4 用迭代归纳法求解递推关系
迭代归纳法也是求解递推关系的一种方法, 尤其对于某些非线性的递推关系, 不存在求解的一般性方法和公式, 不妨用这种方法来试一试. 下面通过几个例子来 说明.
例 1 求解递推关系
$$
\left\{\begin{array}{l}
	f(n)=f(n-1)+n^3, \\
	f(0)=0 .
\end{array}\right.
$$
解 先用迭代法求解该递推关系, 得
$$
\begin{aligned}
	f(n) &=f(n-1)+n^3 \\
	&=f(n-2)+(n-1)^3+n^3 \\
	&=\cdots \\
	&=f(1)+2^3+\cdots+(n-1)^3+n^3 \\
	&=f(0)+1^3+2^3+\cdots+(n-1)^3+n^3 \\
	&=1^3+2^3+\cdots+(n-1)^3+n^3 .
\end{aligned}
$$
能否找到 $f(n)$ 的一个简单表达式呢?为此,我们考察该数列的前 5 项,得
$$
\begin{aligned}
	f(0) &=0=0^2 \\
	f(1) &=1^3=1=1^2 \\
	f(2) &=1^3+2^3=9=3^2 \\
	&=(1+2)^2, \\
	f(3) &=1^3+2^3+3^3=36=6^2 \\
	&=(1+2+3)^2 \\
	f(4) &=1^3+2^3+3^3+4^3=100=10^2 \\
	&=(1+2+3+4)^2
\end{aligned}
$$
由此,我们得出 $f(n)$ 的前 5 项满足
$$
\begin{aligned}
	f(n) &=(0+1+\cdots+n)^2 \\
	&=\frac{n^2(n+1)^2}{4} .
\end{aligned}
$$
要证明上式的确是原递推关系的解,我们用归纳法.
当 $n=0$ 时, $f(0)=0$,上式成立.
假设 $n=k$ 时上式成立, 即
$$
f(k)=\frac{1}{4} k^2(k+1)^2,
$$
则由递推关系, 有
$$
\begin{aligned}
	f(k+1) &=f(k)+(k+1)^3 \\
	&=\frac{1}{4} k^2(k+1)^2+(k+1)^3 \\
	&=\frac{1}{4}(k+1)^2(k+2)^2 .
\end{aligned}
$$
故由归纳法知
$$
f(n)=\frac{1}{4} n^2(n+1)^2
$$
是原递推关系的解.
迭代法并不仅仅局限于如例 1 所示的直接导出 $f(n)$ 的表达式. 利用迭代法, 还可以先将原递推关系化简, 然后再求解.
下面介绍解递推关系常用的几个技巧.
1. 将非齐次递推关系齐次化
例 2 求解递推关系
$$
\left\{\begin{array}{l}
	f(n)=2 f(n-1)+4^{n-1} \\
	f(1)=3
\end{array}\right.
$$

解 由递推关系 (6.4.1) 可以得到
$$
f(n-1)=2 f(n-2)+4^{n-2},
$$
将上式乘以 $-4$ 后再与式 (6.4.1) 相加, 得
$$
f(n)=6 f(n-1)-8 f(n-2) .
$$
如此我们得到了二阶齐次递推关系 (6.4.2), 它需要两个初值才能确定解. 将 $f(1)$ $=3$ 代人递推关系 (6.4.1), 得
所以有
$$
f(2)=2 f(1)+4^{2-1}=10 \text {. }
$$
$$
\left\{\begin{array}{l}
	f(n)=6 f(n-1)-8 f(n-2), \\
	f(1)=3, \quad f(2)=10 .
\end{array}\right.
$$
它的特征方程为
$$
x^2-6 x+8=0,
$$
解得两个特征根为
于是, 通解为
$$
x_1=2, \quad x_2=4,
$$
$$
f(n)=A \cdot 2^n+B \cdot 4^n .
$$
由初值 $f(1)=3, f(2)=10$, 求得 $A=B=\frac{1}{2}$. 故
$$
f(n)=\frac{1}{2}\left(2^n+4^n\right) .
$$
2. 将变系数的一阶线性递推关系化为常系数线性递推关系
例 3 求解递推关系
$$
\left\{\begin{array}{l}
	f(n)=\frac{n+1}{2 n} f(n-1)+1, \\
	f(0)=1 .
\end{array}\right.
$$
解 令
$$
\begin{aligned}
	f(n) &=\frac{n+1}{2 n} \cdot \frac{n}{2(n-1)} \cdot \cdots \cdot \frac{2}{2 \cdot 1} \cdot h(n) \\
	&=\frac{n+1}{2^n} h(n),
\end{aligned}
$$
代人上述递推关系并化简, 即得到关于 $h(n)$ 的递推关系
$$
\left\{\begin{array}{l}
	h(n)=h(n-1)+\frac{2^n}{n+1}, \\
	h(0)=1 .
\end{array}\right.
$$

解得
$$
h(n)=\sum_{k=0}^n \frac{2^k}{k+1},
$$
$$
f(n)=\frac{n+1}{2^n} \sum_{k=0}^n \frac{2^k}{k+1} .
$$
一般地,一阶线性递推关系可以表示成
$$
f(n)=c(n) f(n-1)+g(n) .
$$
令
$$
\begin{aligned}
	&f(n)=c(n) c(n-1) \cdots c(1) h(n), \\
	&(n)=h(n-1)+\frac{g(n)}{c(n) c(n-1) \cdots c(1)},
\end{aligned}
$$
它把变系数化为常系数.
3. 将一阶高次递推关系通过变量代换化为一阶线性递推关系
例 4 求解递推关系
$$
\left\{\begin{array}{l}
	f(n)=3 f^2(n-1), \\
	f(0)=1 .
\end{array}\right.
$$
解 对递推关系 (6.4.3) 两边取自然对数, 得
$$
\ln f(n)=\ln 3+2 \ln f(n-1) .
$$
令 $h(n)=\ln f(n)$, 得
$$
\left\{\begin{array}{l}
	h(n)=2 h(n-1)+\ln 3, \\
	h(0)=0
\end{array}\right.
$$
解得
$$
h(n)=\left(2^n-1\right) \ln 3,
$$
从而
$$
f(n)=3^{3^n-1}
$$

$$
\left\{\begin{array}{l}
	f(n)=4 f(n-1)-4 f(n-2) \quad(n \geqslant 2) \\
	f(0)=0, \quad f(1)=1 .
\end{array}\right.
$$
解 令
$$
A(x)=\sum_{n=0}^{\infty} f(n) x^n,
$$
则有
$$
\begin{aligned}
	A(x)-f(0)-f(1) x &=\sum_{n=2}^{\infty} f(n) x^n \\
	&=\sum_{n=2}^{\infty}[4 f(n-1)-4 f(n-2)] x^n \\
	&=4 x \sum_{n=1}^{\infty} f(n) x^n-4 x^2 \sum_{n=0}^{\infty} f(x) x^n \\
	&=4 x \cdot[A(x)-f(0)]-4 x^2 A(x) .
\end{aligned}
$$
将 $f(0)=0, f(1)=1$ 代人上式并整理, 得
$$
A(x)=\frac{x}{4 x^2-4 x+1}=\frac{x}{(1-2 x)^2} .
$$
设
$$
A(x)=\frac{C_1}{1-2 x}+\frac{C_2}{(1-2 x)^2},
$$
其中, $C_1, C_2$ 为待定系数. 通过比较等式两边分子的常数项与 1 次项系数, 可得
$$
\left\{\begin{array}{l}
	C_1+C_2=0, \\
	-2 \cdot C_1=1,
\end{array}\right.
$$
所以, $C_1=-\frac{1}{2}, C_2=\frac{1}{2}$.
$$
\begin{aligned}
	A(x) &=-\frac{1}{2} \cdot \frac{1}{1-2 x}+\frac{1}{2} \cdot \frac{1}{(1-2 x)^2} \\
	&=-\frac{1}{2} \sum_{n=0}^{\infty} 2^n x^n+\frac{1}{2} \sum_{n=0}^{\infty}\left(\begin{array}{c}
		n+1 \\
		n
	\end{array}\right) \cdot 2^n \cdot x^n .
\end{aligned}
$$
故
$$
f(n)=-\frac{1}{2} \cdot 2^n+\frac{1}{2} \cdot(n+1) \cdot 2^n=\frac{1}{2} \cdot n \cdot 2^n .
$$

例 2 求解递推关系
$$
\left\{\begin{array}{l}
	f(n)=f(n-1)+n^4, \\
	f(0)=0 .
\end{array}\right.
$$
薢 令
$$
A(x)=\sum_{n=0}^{\infty} f(n) x^n,
$$
代人递推关系, 得
$=x A(x)+\sum_{n=1}^{\infty} n^4 x^n$,
解得
$$
A(x)=\frac{1}{1-x} \sum_{n=1}^{\infty} n^4 x^n .
$$
利用
$$
G\left\{k^3\right\}=\frac{x\left(1+4 x+x^2\right)}{(1-x)^4},
$$
求得
所以
$$
G\left\{k^4\right\}=\frac{x\left(1+11 x+11 x^2+x^3\right)}{(1-x)^5}
$$
$$
\begin{aligned}
	A(x) &=\frac{x\left(1+11 x+11 x^2+x^3\right)}{(1-x)^6} \\
	&=\left(x+11 x^2+11 x^3+x^4\right) \sum_{i=1}^{\infty}\left(\begin{array}{c}
		i+5 \\
		i
	\end{array}\right) x^i .
\end{aligned}
$$
于是, $x^n$ 的系数 $f(n)$ 为

于是, $x^n$ 的系数 $f(n)$ 为
$$
\begin{aligned}
	f(n) &=\left(\begin{array}{c}
		n-1+5 \\
		n-1
	\end{array}\right)+11\left(\begin{array}{c}
		n-2+5 \\
		n-2
	\end{array}\right)+11\left(\begin{array}{c}
		n-3+5 \\
		n-3
	\end{array}\right)+\left(\begin{array}{c}
		n-4+5 \\
		n-4
	\end{array}\right) \\
	&=\frac{1}{30} n(n+1)\left(6 n^3+9 n^2+n-1\right) .
\end{aligned}
$$
例 3 求解递推关系
$$
\left\{\begin{array}{l}
	f(n)=2 f(n-1)+4^{n-1} \quad(n \geqslant 2), \\
	f(1)=3 .
\end{array}\right.
$$
解 令
$$
A(x)=\sum_{n=1}^{\infty} f(n) x^n,
$$
代人递推关系, 得
$$
\begin{aligned}
	A(x)-f(1) x &=\sum_{n=2}^{\infty}\left[2 f(n-1)+4^{n-1}\right] x^n \\
	&=2 x A(x)+4 x^2 \sum_{n=0}^{\infty}(4 x)^n,
\end{aligned}
$$
解得
- 168 .
$$
A(x)=\frac{(3-8 x) x}{(1-2 x)(1-4 x)}=x\left(\frac{1}{1-2 x}+\frac{2}{1-4 x}\right) .
$$
所以, $x^n$ 的系数 $f(n)$ 为
$$
f(n)=\frac{1}{2}\left(2^n+4^n\right) .
$$
\end{document}