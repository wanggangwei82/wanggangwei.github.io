\chapter{基本原理}
\label{chap1} \minitoc

\section{加法原理与乘法原理}
在计数问题中,加法原理与乘法原理是两个最基本也是最常用的原理。以下假设$A$和$B$是两类不同的、互不关联的事件。
\subsection{加法原理}
\textbf{加法原理}\
设事件$A$有$m$种选取方式,事件$B$有$n$种选取方式,则选$A$或$B$共有$m+n$种方式。

例如,大于$0$小于$10$的偶数有$4$个,即$2,4,6,8$;大于$0$小于$10$的奇数有$5$个,即$1,3,5,7,9$,则大于$0$小于$10$的整数有
$9$个,即$1,2,3,4,5,6,7,8,9.$\
这里,事件$A$指的是大于$0$小于$10$的偶数,事件$B$指的是大于$0$小于$10$的奇数。而大于$0$小于$10$的整数不外乎是偶数或奇数,
即属于$A$或$B$.

用集合的语言可将加法原理叙述成以下定理:
\begin{thm}
设$A$, $B$为有限集,$A\cap B=\varnothing$, 则$$|A\cup B|=|A|+|B|.$$
\end{thm}
\pf 当$A$, $B$中有一个是空集时,定理的结论是平凡的。

设$A\neq \varnothing,\ B\neq \varnothing$, 记$$A=\{a_1, a_2, \ldots,
a_m\},$$
$$B=\{b_1, b_2, \ldots,
b_n\},$$ 并作映射$$\varphi :\ a_i\rightarrow i\ (1\leq i\leq m),$$
$$\ \ \ \ \ \ \ \ \ \ \ b_j\rightarrow m+j\ (1\leq j\leq n).$$
因为$$\varphi(a_i)\neq \varphi (b_j)\ (1\leq i\leq m,\ 1\leq j\leq n
),$$ 所以$\varphi$是从$A\cup
B$到集合$\{1,2,\ldots,m+n\}$上的一一映射,因而定理成立。
\begin{coro}
设$n$个有限集合$A_1,A_2,\ldots,A_n$满足$$A_i\cup A_j=\varnothing\
(1\leq i \neq j\leq n),$$ 则$$|\bigcup_{i=1}^n A_i|=\sum_{i=1}^n
|A_i|.$$
\end{coro}

\begin{exa}
在所有六位二进制数中,至少有连续$4$位是$1$的有多少个?
\end{exa}

\textbf{解} 把所有满足要求的二进制数分为如下$3$类:

(i)恰有$4$位连续的$1$. 它们可能是$*01111, 011110, 11110*$,
其中“$*$”可能取$0$或$1$. 故此种情况共有$5$个不同的六位二进制数。

(ii)恰有$5$位连续的$1$, 它们可能是$011111, 111110$, 共有$2$个。

(iii)恰有$6$位连续的$1$, 即$111111$, 只有$1$个。

于是综合以上分析,由加法原理知共有$5+2+1=8$个满足题意要求的六位二进制数。

\subsection{乘法原理}

\textbf{乘法原理}
设事件$A$有$m$种选取方式,事件$B$有$n$种选取方式,那么选取$A$以后再选取$B$共有$m\cdot
n$种方式。

用集合论的语言可将上述乘法原理叙述成如下的定理:
\begin{thm}
设$A,\ B$是两个有限集合,$|A|=m,\ |B|=n,$ 则$$|A\times
B|=|A|\times|B|=m\cdot n.$$
\end{thm}
\pf 若$m=0$或$n=0$, 则上面的等式两边均为$0$, 故等式成立。

设$m>0,n>0$, 并且记$$A=\{a_1, a_2, \ldots, a_m\},$$
$$B=\{b_1, b_2, \ldots,
b_n\}.$$ 定义映射$$\varphi:\ (a_i,b_j)\rightarrow (i-1)n+j\ (1\leq
i\leq m,\ 1\leq j\leq n),$$ 则$\varphi$是$A\times
B$到集合$\{1,2,\ldots,mn-1,mn\}$上的一一映射,所以等式成立。
\begin{coro}
设$A_1,A_2,\ldots,A_n$为$n$个有限集合,则$$|A_1\times A_2\times
\cdots \times A_n|=|A_1|\times |A_2|\times \cdots \times |A_n|.$$
\end{coro}

\begin{exa}
设从$A$到$B$有$3$条不同的道路,从$B$到$C$有$2$条不同的道路,如下图所示,则从$A$经$B$到$C$的道路数为$$n=3\times
2=6.$$ \vspace*{0.2cm} \hspace*{5.5cm}
\begin{picture}(20,20)
\setlength{\unitlength}{2.5mm}
\put(-3,0){\circle*{0.3}}\put(5,0){\circle*{0.3}}\put(13,0){\circle*{0.3}}
\put(-3,0){\line(1,0){8}} \qbezier(-3,0)(0.6,4)(5,0)
\qbezier(-3,0)(0.6,-4)(5,0) \qbezier(5,0)(8.6,4)(13,0)
\qbezier(5,0)(8.6,-4)(13,0)
\put(-4.5,-2){$A$}\put(4,-2){$B$}\put(13.5,-2){$C$}
\end{picture}

\end{exa}
\begin{exa}
从$5$位先生、$6$位女士、$2$位男孩、$4$位女孩中选取$1$位先生、$1$位女士、$1$位男孩、$1$位女孩,共有$5\times
6\times 2\times
4=240$种方式(由乘法原理)。而从中选取一个人的方式共有$5+ 6+ 2+
4=17$种方式(由加法原理)。
\end{exa}

\begin{exa}
在$1000$到$9999$之间有多少个各位数字不同的奇数?
\end{exa}
\textbf{解}
将一个千位数的千位、百位、十位、个位分别记为第$1$、$2$、$3$、$4$位,则满足条件的数字第$4$位必须是奇数,可取$1,3,5,7,9$,
共有$5$种选择。第$1$位不能取$0$,也不能取第$4$位已选定的数字,所以在第$4$位选定后第$1$位有$8$种选择。
第$2$位不能取第$1$位和第$4$位已选定的数字,共有$8$种选择。类似地,第$3$位有$7$种选择。从而由乘法原理,满足题意的
数共有$5\times 8\times 8\times 7=2240$个。

\section{鸽巢原理}
\subsection{引言}

鸽巢原理又称抽屉原理,这个原理最早是由Dirichlet提出的。鸽巢原理是解决组合论中一些存在性问题的基本而又有力的工具。
它是组合数学中最简单也是最基本的原理之一,从这个原理出发,可以导出许多有趣的结果,而这些结果常常是令人惊奇的。

Ramsey理论对组合数学发展产生过重要的影响。$1928$年,年仅$24$岁的英国杰出数学家Ramsey发表了著名的论文《论形式逻辑中的一个问题》,
他在这篇文论中,提出并证明了关于集合论的一个重大研究成果,现称为Ramsey定理。尽管两年后他不幸去世,但他开拓的这一新领域至今
仍十分活跃,而且近年来在科技领域获得了成功的应用。

本章主要介绍鸽巢原理、Ramsey数及性质、Ramsey定理。
\subsection{鸽巢原理}
\begin{thm}
若有$n+1$只鸽子飞回$n$个鸽巢,则至少有两个鸽子飞入了同一鸽巢。
这个原理也可以表述为:\\如果把$n+1$件东西放入$n$个盒子中,则至少有一个盒子里面有不少于两件的东西。
\end{thm}
这个原理的证明非常容易,只要使用反证法马上就可以得到结论。

鸽巢原理不能用来寻找究竟是哪个盒子含有两件或更多的东西。该原理
只能证明某种安排或某种现象存在,而并未指出怎样构造这种安排或怎样寻找这种现象出现的场合。利用鸽巢原理解决实际问题关键是要
看出这是一个鸽巢问题,建立“鸽巢”寻找“鸽子”。下面我们给出几个利用鸽巢原理解决问题的例子。
\begin{exa}
如果鞋架上放$10$双鞋,从总任意取11只,其中至少有两只恰好是配对的。
\end{exa}
\begin{exa}
从整数$1,2,\ldots,100$中任选$51$个数,证明在所选的数中间必然存在两个整数,其中之一可以被另一个整除。

\pf 对于任何一个整数$x$,总可以把$x$写成$x=2^n\cdot
a$的形式,其中$a$是奇数,$n\geq 0$.
知$1$到$100$之间共有$50$个奇数,由所选的$51$个数利用上述表达方式可以得到$51$个奇数,由鸽巢原理知,其中必然有两个奇数相同,
设对应这两个奇数的整数为$x=2^ra,\ y=2^sa$,如果$r\leq s$那么$x|y$;如果$r>s$那么$y|x$.\\
注:本例中:鸽子$=$去掉$2$因子所得到的奇数;鸽巢$=$$1$到$100$之间的奇数。这个例子可以推广到从$1,2,\ldots,2n$中任意取
$n+1$个数,其中必然存在两个数,其中一个整除另外一个,证法类似。
\end{exa}

利用鸽巢原理解决问题的关键在于:辨认问题,建立鸽巢,寻找鸽子。
\begin{exa}
在\{1,2,\ldots,2n\}中任取$n+2$个数,其中必有两个数,其和为$2n$.
\\分析:鸽子为集合中的元素,鸽巢为$$\{1,2n-1\},\{2,2n-2\},\ldots,\{n-1,n+1\},\{n\},\{2n\}.$$
\end{exa}

\begin{exa}
任取$11$个正整数,其中必有两个数,其差被$10$整除。
\\分析:鸽子为选出的$11$个数,鸽巢为$A_i=\{\mbox{个位数为}i\},\ i=0,1,\ldots,9.$
\end{exa}

\subsection{一般形式的鸽巢原理}

已经知道了简单的鸽巢原理,现在我们来了解一下推广的鸽巢原理。
\begin{thm}
设$m_1,m_2,\ldots,m_n$均为正整数,如果有$m_1+m_2+\cdots+m_n-n+1$只鸽子飞回$n$个鸽巢,则或者第$1$个鸽巢至少有$m_1$只鸽子,
或者第$2$个鸽巢至少有$m_2$只鸽子,$\ldots$,或者第$n$个鸽巢至少有$m_n$只鸽子。
\end{thm}
\pf
用反证法。假若第$1$鸽巢少于$m_1$只鸽子,第$2$鸽巢少于$m_2$只鸽子,$\ldots$,第$n$鸽巢少于$m_n$只鸽子,则鸽子
总数之多为:$(m_1-1)+(m_2-1)+\cdots+(m_n-1)=m_1+m_2+\cdots+m_n-n$,
这比假定的鸽子数少了一个,矛盾。\qed

从定理二可以得到以下推论:
\begin{prop}
如果$m_1=m_2=\cdots=m_n=r$,
即若将$n(r-1)+1$个球放入$n$个盒子里,则至少有一个盒子含有不少于$r$个球。
\end{prop}
\begin{prop}
如果$n$个正整数$m_1,m_2,\cdots,m_n$的平均数$(m_1+m_2+\cdots+m_n)/n>r-1$,
则$m_1,m_2,\ldots,m_n$中至少有一个正整数不会小于$r$.
\end{prop}
\begin{prop}
有$m$个球放入$n$个盒子,则至少有一个盒子中有不少于$[(m-1)/n]+1$个球。
\end{prop}

下面我们来看一些利用这些定理和推论的例子。
\begin{exa}
随意给一个正十边形的十个顶点标上号码$1,2,\ldots,10$,求证:必然有一个顶点,该顶点及与之相邻的两个顶点的标号之和不小于$17$.
\\ \pf 设$v_1,v_2,\ldots,v_{10}$是正十边形的$10$个顶点,$a_i$表示顶点$v_i$及与$v_i$相邻的两个顶点的标号之和,则
$a_1+a_2+\cdots+a_{10}=(1+2+\cdots+10)\times 3=165>(17-1)\times
10+1$, 这样必然有某个$a_k\geq 17$.
\end{exa}

\begin{exa}
将$N_5=\{1,2,3,4,5\}$分为两组,则必有某个数,它是同组中的一个数的$2$倍,或者是同一组中另两个数之和。\\
\pf 若存在一个分组方法,$A\bigcup B=N_5,\ A\bigcap B=\emptyset $,
使得性质不成立,即$a,b\in A \Rightarrow a-b,b-a\in A$和$a,b\in B
\Rightarrow a-b,b-a \in
B$都不成立。由鸽巢原理,不妨设$a_1,a_2,a_3\in A, a_1>a_2>a_3$,
由反证法,
$$\left.\begin{array}{c}
  b_1 = a_1-a_2\notin A \\
  b_2 = a_1-a_3\notin A
\end{array}\right\}\Rightarrow\left. \begin{array}{c}
                                  b_1\in B \\
                                    b_2 \in B
                                \end{array}\right\}\Rightarrow
   b_2-b_1 \notin B \Rightarrow b_2-b_1\in A$$
但 $b_2-b_1=(a_1-a_3)-(a_1-a_2)=a_2-a_3\notin A$矛盾。
\end{exa}

通过上面的定理我们还可以得到以下结论:
\begin{thm}{(Erd\"{o}s)}
由$n_2-1$个不同实数构成的序列中,至少存在由$n+1$个实数组成一个单调递增子序列和单调递减子序列。
\end{thm}
\pf
设原序列为:$a_1,a_2,\ldots,a_{n_2-1}$,令$m_i$表示从$a_i$开始最长递增子序列的长度,若有某个$m_i\geq
n+1$,
则定理得证。因为给定的序列有$n^2+1$个实数,顾可产生$n^2+1$个长度:$m_1,m_2,\ldots,m_{n^2+1}$.
如果全部的$m_i<n+1$,
则这些整数必定在$1$到$n$之间,相对于把$n^2+1$个球放入$n$个盒子。由定理$2$的推论$1$可知,这是$r=n+1$的特殊情况,这$n^2+1$
个$m_i$中至少有$n+1$个数
相等。不妨设$m_{i_1}=m_{i_2}=\cdots=m_{i_{n+1}}=m$. 且$1\leq
i_1<i_2<\cdots<i_{n+1}\leq n^2+1$,
则可以得到下面的长度为$n+1$的递减序列:$a_{i_1}>a_{i_2}>\cdots>a_{i_{n+1}}$.
\qed

\section{容斥原理}
粗略地讲,组合计数学中的筛法是一种首先从一个较大的集合入手,
以某种方式减掉或删去一些不需要的元素,从而计算出集合 基数的方法。

容斥原理是组合计数学的主要工具之一。从抽象的角度来讲,容斥原理
就是计算某个矩阵的逆,因此,它只不过是线性代数中一个
很平凡的结果,但这个原理的优美之处远不在结果本身,而在于它应用的
广泛性。我们将给出一些可以利用容斥原理解决的问题的例子,其中一些
采用了相当巧妙的构思。在此之前,我们用最抽象的形式来陈述这个原理:

\subsection{容斥原理}
\begin{thm}
\label{theorem2.11} 设$S$是一个$n$元集,$V$是一个由函数$f\colon
2^S\rightarrow
K$组成的$2^n$维向量空间$(K$为某个域$)$。定义线性变换$\phi\colon
V\rightarrow V$如下:

\begin{equation}\phi f(T)=\sum_{Y\supseteq T}
f(Y),\quad \mbox{对任意的$T\subseteq S$},
\end{equation}

 则$\phi^{-1}$存在,且

\begin{equation} \phi^{-1}f(T)=\sum_{Y\supseteq T}(-1)^{|Y-T|}f(Y),\quad
\mbox{对任意的$T\subseteq S$}.\end{equation}
\end{thm}

{\bf 证明:}定义$\psi\colon V\rightarrow V$为$\psi
f(T)=\sum_{Y\supseteq T}
(-1)^{|Y-T|} f(Y).$则我们只需要证明$\phi\circ\psi$为恒等变换。\\
由定义,\allowdisplaybreaks
\begin{align*}
\phi\circ\psi f(T)&=\sum_{Y\supseteq T} (-1)^{|Y-T|}\phi f(Y)\\[5pt]
&=\sum_{Y\supseteq T} (-1)^{|Y-T|}\sum_{Z\supseteq Y} f(Z)\\[5pt]
&=\sum_{Z\supseteq T}\left(\sum_{Z\supseteq Y\supseteq
T}(-1)^{|Y-T|}\right)f(Z)
\end{align*}
设$m=|Z-T|,$则:
\[\sum_{Z\supseteq Y\supseteq T \atop
\mbox{\tiny{$Z,\,T$固定}}}(-1)^{|Y-T|}=\sum_{i=0}^m (-1)^i {m\choose
i}=\delta_{0m},
\]
所以,$\phi\circ\psi f(T)=f(T),$因此,$\phi\circ\psi
f=f.$从而可证$\psi=\phi^{-1}.$\qed

容斥原理通常的组合描述为:设$S$为由不同性质组成的集合,
$A$为一些元素的集合,且这些元素或者有或者没有$S$中的某些性质。
对$S$的任一个子集$T$,令
$f_=(T)$表示$A$中恰好具有$T$中性质的元素个数(易知这些元素
没有$\bar{T}=S-T$中的性质)。(更一般的,若$\omega\colon A\rightarrow
K$是$A$上的赋权函数,且赋权值在数域(或Abel群)$K$中,
则可令$f_{=}(T)=\sum_x \omega(x),$
这里$x$取遍$A$中恰好具有$T$中性质的元素。)
令$f_{\geq}(T)$表示$A$中至少具有$T$中性质的元素的个数,于是
\begin{equation}\label{equation3}
f_{\geq}(T)=\sum_{Y\supseteq T}f_{=} (Y).
\end{equation}
从而由定理\ref{theorem2.11}易知,
\begin{equation}\label{equation4}
f_{=}(T)=\sum_{Y\supseteq T}(-1)^{|Y-T|}f_{\geq}(Y).
\end{equation}
特别的,不具有$S$中任何性质的元素个数为:
\begin{equation}\label{equation5}
f_{=}(\emptyset)=\sum_Y (-1)^{|Y|} f_{\geq}(Y),
\end{equation}

接下来我们来了解一下容斥原理的运用。

\subsection{错排}

\begin{defi}
$[n]$上的置换$\pi=\pi_1,\pi_2,\ldots,
\pi_n$称为错排,是指对$\forall\  i\in [n]$均满足$\pi_i\ne
i$。即任何一个数字都不在其自然序时所处的位置上。同时我们称满足$\pi_i=i$的点为不动点。
例如:$[3]$上的错排有$2,3,1$和$3,1,2$。
\end{defi}


Pierre de Montmort在1708年首先提出了如何计算$[n]$上错排的个数问题,
然后在1713年由他本人解决了这一问题。Nicholas
Bernoulli几乎在同时运用容斥原理解决了这一问题。


\begin{defi}
令$a_0,a_1,a_2,\ldots$为一无穷序列,则
$$f(x)=\sum_{n\geq0}a_n\frac{x^n}{n!}$$
称为该无穷序列的指数型生成函数。例如:$\{0!,1!,2!,3!,\ldots\}$的指数型生成函数为:
$$f(x)=\sum_{n\ge0}n!\frac{x^n}{n!}=\frac{1}{1-x}.$$
$\{1,1,1,\ldots\}$的指数型生成函数为:
$$f(x)=\sum_{n\ge0}\frac{x^n}{n!}=e^x.$$
\end{defi}
由定义可得:

\begin{thm}
$f(x)=\sum a_n\frac{x^n}{n!}$和$g(x)=\sum b_n\frac{x^n}{n!}$
是两个生成函数,则:
$$f(x)g(x)=\sum c_n\frac{x^n}{n!},$$
其中$c_n=\sum{n\choose k}a_kb_{n-k}.$
\end{thm}

生成函数的乘法有严格的组合意义。假设$a_n$
所计数的组合结构称为A-结构,$b_n$
计数的是B-结构,那么$c_n$计数的则是由部分A-结构和部分B-结构组合而成的结构。


我们已经给出了错排及生成函数的定义,现在我们具体讨论一下。设$d_n$计数了$[n]$上的错排个数。现在讨论关于$d_n$
的计算及其生成函数。在这里会用到组合中一个常用方法:容斥原理。

\begin{thm}
\label{theorem2.11} 设$S$是一个$n$元集,$V$是一个由函数$f\colon
2^S\rightarrow
K$组成的$2^n$维向量空间$(K$为某个域$)$。定义线性变换$\phi\colon
V\rightarrow V$如下:

\begin{equation}\phi f(T)=\sum_{Y\supseteq T}
f(Y),\quad \mbox{对任意的$T\subseteq S$},
\end{equation}

 则$\phi^{-1}$存在,且

\begin{equation} \phi^{-1}f(T)=\sum_{Y\supseteq T}(-1)^{|Y-T|}f(Y),\quad
\mbox{对任意的$T\subseteq S$}.\end{equation}
\end{thm}

{\bf 证明:}定义$\psi\colon V\rightarrow V$为$\psi
f(T)=\sum_{Y\supseteq T}
(-1)^{|Y-T|} f(Y).$则我们只需要证明$\phi\circ\psi$为恒等变换。\\
由定义,\allowdisplaybreaks
\begin{align*}
\phi\circ\psi f(T)&=\sum_{Y\supseteq T} (-1)^{|Y-T|}\phi f(Y)\\[5pt]
&=\sum_{Y\supseteq T} (-1)^{|Y-T|}\sum_{Z\supseteq Y} f(Z)\\[5pt]
&=\sum_{Z\supseteq T}\left(\sum_{Z\supseteq Y\supseteq
T}(-1)^{|Y-T|}\right)f(Z)
\end{align*}
设$m=|Z-T|,$则:
\[\sum_{Z\supseteq Y\supseteq T \atop
\mbox{\tiny{$Z,\,T$固定}}}(-1)^{|Y-T|}=\sum_{i=0}^m (-1)^i {m\choose
i}=\delta_{0m},
\]
所以,$\phi\circ\psi f(T)=f(T),$因此,$\phi\circ\psi
f=f.$从而可证$\psi=\phi^{-1}.$\qed

容斥原理通常的组合描述为:设$S$为由不同性质组成的集合,
$A$为一些元素的集合,且这些元素或者有或者没有$S$中的某些性质。
对$S$的任一个子集$T$,令
$f_=(T)$表示$A$中恰好具有$T$中性质的元素个数(易知这些元素
没有$\bar{T}=S-T$中的性质)。(更一般的,若$\omega\colon A\rightarrow
K$是$A$上的赋权函数,且赋权值在数域(或Abel群)$K$中,
则可令$f_{=}(T)=\sum_x \omega(x),$
这里$x$取遍$A$中恰好具有$T$中性质的元素。)
令$f_{\geq}(T)$表示$A$中至少具有$T$中性质的元素的个数,于是
\begin{equation}\label{equation3}
f_{\geq}(T)=\sum_{Y\supseteq T}f_{=} (Y).
\end{equation}
从而由定理\ref{theorem2.11}易知,
\begin{equation}\label{equation4}
f_{=}(T)=\sum_{Y\supseteq T}(-1)^{|Y-T|}f_{\geq}(Y).
\end{equation}
特别的,不具有$S$中任何性质的元素个数为:
\begin{equation}\label{equation5}
f_{=}(\emptyset)=\sum_Y (-1)^{|Y|} f_{\geq}(Y),
\end{equation}

从抽象的角度来讲,容斥原理
就是计算某个矩阵的逆,因此,它只不过是线性代数中一个
很平凡的结果,但这个原理的优美之处远不在结果本身,而在于它应用的
广泛性。现在我们用容斥原理来计算一下$d_n$的大小。

\begin{thm}
设$d_n$计数了$[n]$上的错排个数,则$$d_n=\sum_{i=0}^n{n\choose
i}(-1)^{n-i}i!.$$
\end{thm}
{\bf
证明:}将条件$\pi(i)=i$视为第$i$个条件。不动点集至少包含集合$T\subseteq
[n]$的排列的个数等于 $f_{\geq}(T)=b(n-i)=(n-i)!,$其中$|T|=i$
(固定$T$中的元素,其余$n-i$个元素任意排列)。因此通过\ref{equation5}可以得到没有不动点的排列的个数
$f_{=}(\emptyset)=a(n)=d_n$为$\sum_{i=0}^n{n\choose
i}(-1)^{n-i}i!.$\qed

由$d_n$的计数可以很容易得到其生成函数如下:
\begin{thm}
设$d_n$计数了~$[n]$上的错排个数,则
$$\sum_{n\ge0}d_n\frac{x^n}{n!}=\frac{e^{-x}}{1-x}.$$
\end{thm}


\noindent{\bf 练习:}证明如下等式
$$d_n=nd_{n-1}+(-1)^n,$$

提示:设$\pi_n=i,$由于$\pi$为错排,$i\neq
n,$考虑$\pi_i$的不同取值就可以得到上述递推关系式。

$$n!=\sum_{k=0}^n{n\choose k}d_k.$$

提示:反着运用一下容斥原理即可。

\subsection{下降集}
对于$\pi\in\mathfrak{G}_n$,
如果$\pi=\pi_1\pi_2\ldots\pi_n$那么定义下降集为:
$$D(\pi)=\{i:\pi_i>\pi_{i+1}\}.$$
如果$S\subseteq[n-1],$
那么用$\alpha(S)$表示下降集包含在$S$中的排列$\pi\in\mathfrak{G}_n$的个数,用
$\beta(S)$表示下降集等于$S$的排列的个数。用符号表示,即为:
\begin{eqnarray*}
\alpha(S)=card\{\pi\in\mathfrak{G}_n:D(\pi)\subseteq S\};\\
\beta(S)=card\{\pi\in\mathfrak{G}_n:D(\pi)=S\}.
\end{eqnarray*}
显然
$$\alpha(S)=\sum_{T\subseteq S}\beta(S).$$
\begin{prop}
设$S=\{s_1,s_2,\ldots,s_k\}_<\subseteq[n-1],$ 则
$$\alpha(S)={n\choose s_1,s_2-s_1,\ldots,n-s_k}.$$
\end{prop}
\pf 为了得到一个满足$D{\pi}\subseteq
S$的排列$\pi=\pi_1\pi_2\ldots\pi_n$,
首先选择$\pi_1<\pi_2<\cdots<\pi_{s_1},$
 有${n\choose s_1}$种选择,然后选择$\pi_{s_1+1}<\pi_{s_1+2}<\cdots<\pi_{s_2},$ 有${n-s_1\choose s_2-s_1}$种选择,依此类推,可得:
 \begin{eqnarray*}
 \alpha(S)&=&{n\choose s_1}{n-s_1\choose s_2-s_1}\cdots{n-s_k\choose n-s_k}\\
 &=&{n\choose s_1,s_2-s_1,\ldots,n-s_k}
 \end{eqnarray*}

 接下来我们利用一下容斥原理计算$\beta(S)$
由于
$$\alpha(S)=\sum_{T\subseteq S}\beta(S),$$ 所以
$$\beta(S)=\sum_{T\subseteq S}(-1)^{|S-T|}\alpha(T).$$
而我们已知$\alpha(S)={n\choose s_1,s_2-s_1,\ldots,n-s_k},$ 因此
$$\beta(S)=\sum_{1\leq i_1\leq i_2\leq\cdots \leq i_j\leq k}(-1)^{k-j}{n\choose s_{i_1},s_{i_2}-s_{i_1},\ldots,n-s_{i_j}}.$$
通过变化上述式子可以转换为
$$\beta(S)=n!\det[1/(s_{j+1}-s_i)!].$$




