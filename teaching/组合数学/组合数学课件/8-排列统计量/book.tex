\documentclass[a4paper,11pt]{article}
\usepackage{ctex}



\usepackage{amsmath,amssymb}             % AMS Math
\usepackage[T1]{fontenc}



\usepackage{graphicx}
% \usepackage{epstopdf}
\usepackage{tikz}
\usepackage[left=1.5in,right=1.3in,top=1.1in,bottom=1.1in,includefoot,includehead,headheight=13.6pt]{geometry}
\renewcommand{\baselinestretch}{1.05}



\usepackage{minitoc}
\newtheorem{thm}{定理}[section]
\newtheorem{prop}[thm]{命题}
\newtheorem{coro}[thm]{推论}
\newtheorem{defi}[thm]{定义}
\newtheorem{lem}[thm]{引理}
\newtheorem{exa}[thm]{例}
\newtheorem{ex}[thm]{习题}
\newtheorem{conj}[thm]{猜想}

\def\qed{\nopagebreak\hfill{\rule{4pt}{7pt}}\medbreak}
\def\pf{{\bf 证明~~ }}
\def \sg{\sigma}
\def \asc{\mathrm{asc}}
\def \des{\mathrm{des}}
\def \fix{\mathrm{fix}}
\def \lef{\mathrm{lef}}
\def \one{\mathrm{one}}
\def \Des{\mathrm{Des}}
\def \maj{\mathrm{maj}}
\def \exc{\mathrm{exc}}
\def \inv{\mathrm{inv}}
\def \roots{\mathrm{roots}}
\def \sgn{\mathrm{sgn}}
% Table of contents for each chapter

\usepackage{color}
\definecolor{linkcol}{rgb}{0,0,0.4}
\definecolor{citecol}{rgb}{0.5,0,0}

  \usepackage{graphicx}
  \DeclareGraphicsExtensions{.eps}
  \usepackage[a4paper,pagebackref,hyperindex=true,pdfnewwindow=true]{hyperref}

% \usepackage{chapterbib}
\begin{document}


\section{排列统计量}

排列的各种统计量是组合数学研究的一个重要课题,对排列统计量的研究可以使我们
更清楚的了解排列的内部结构。下面我们就介绍一些在排列上十分熟知的统计量。

位置$i$($1\leqslant i<n$)称为是$\pi$的一个{下降位}(descent)
如果$\pi_i>\pi_{i+1}$;反之则称为$\pi$的{上升位}(acscent).
定义所有下降位构成的集合

$$\Des(\pi)=\{i|\pi_i>\pi_{i+1}\}$$
为$\pi$的下降集(descent set),
定义该集合的个数为$\des(\pi)=|\Des(\pi)|$为
$\pi$的下降数。由定义$n\notin \Des(\pi)$.
同时我们定义一个排列的主指标(major index)为
\[\maj(\pi)=\sum \limits_{i \in \Des(\pi)}i.\]

如果位置$i$满足$\pi_i>i$, 则称$i$是一个{胜位}(excedance), 若$i$满足
$\pi_i\geq i$, 则称$i$是{弱胜位}(weak excedance). 我们记 $\pi$
的所有胜位的个 数为 $\exc(\pi)$.

一对元素$(i,j)$
称为是一个{逆序}(inversion),如果满足$i<j$且$\pi_i>\pi_j$, 称$\pi$的
所有逆序的个数为$\pi$的逆序数,记作$\inv(\pi)$.

对于排列$\pi=\pi_1\pi_2\cdots \pi_n$, 定义其逆为其作为映射的逆,即
$\pi^{-1}=\pi^{-1}(1)\pi^{-1}(2)\cdots \pi^{-1}(n)$;
定义其反为$\pi^r=\pi_n\pi_{n-1}\ldots\pi_1$; 定义
其补为$\pi^c=(n+1-\pi_1)(n+1-\pi_2)\cdots(n+1-\pi_n)$,
显然它们三个都是$S_n$上自然的一一映射。

\begin{exa}
	对于$[5]$上的排列$\pi=43521$, 以上的统计量分别为:
	$\Des(\pi)=\{1,3,4\}$, $\des(\pi)=3$, $\maj(\pi)=1+3+4=8$,
	$\exc(\pi)=3$, $\inv(\pi)=7$.
\end{exa}

\subsection{下降数与胜位的等分布性质}
我们称两个统计量$u,v$在某个集合$S$上是{等分布的}(equidistribute),
若对于任意的自然数$k$, 有
$\#\{x\in S|u(x)=k\}=\#\{x\in S|v(x)=k\}$.

\begin{thm} \label{exc_des}
	$\exc$ 与 $\des$在$S_n$上是等分布的。
\end{thm}

一般而言,证明两个统计量的等分布性有两个主要的思路:
一个是组合证明,即寻找所在集合的一个到
自身的双射;另一个是代数证明,即证明二者有相同的生成函数。

\pf 组合证明:

在证明之前,先引入排列的另一种表示形式——圈表示。对于任意$x\in
[n]$, 考虑 序列$x,\pi(x),\pi^2(x),\ldots$,
最终一定形成一个圈(因为$\pi$是双射且$[n]$是有限集)。对所有的元素寻
找这样的圈,我们可以把排列$\pi$写成若干个不交圈的并的形式。
这种形式显然不是唯一的,首先,圈之间的顺
序可以任意,其次,圈内部的圈排列也有不同的表示。
为保证其唯一性,我们定义如下标准圈表示形式:

\begin{itemize}
	\item [a.]每个圈的最大元素放在首位;
	\item [b.]圈按照其最大元从小到大排列。
\end{itemize}


可以证明,以上的标准圈表示形式存在且唯一的。

对于任意一个排列$\pi\in S_n$,我们考虑其标准圈表示,
并将标准圈表示的圈去掉,这样就得到$[n]$上的一个新
的排列$\pi'$,可以证明$\pi\rightarrow \pi'$必然是$S_n$上的双射。
事实上,对于任意$\pi\in S_n$, 取其自左
向右极大元(即满足对于任意$j<i$, $\pi_j>\pi_i$的元素$\pi_i$)。
在相应位置加括号就可以得到上述映射的逆映射。

我们利用以上映射证明我们的结论,只需要证明对于任意的$\pi\in S_n$,
$\exc(\pi)=\des(\pi')$.
事实上,考虑$\pi$的补排列$\pi^c$的标准圈表示形式,
$\pi$的每一个胜位恰好对应到$(\pi^c)'$的一个下降位。命题得证。\qed

一般地,称与$\des$在$S_n$上等分布的统计量为Eulerian的.


\subsection{逆序数与主指标}

首先我们用代数的方法来给出逆序数的生成函数。

\begin{thm}
	\begin{equation}
	\sum_{\pi\in
		S_n}q^{\inv(\pi)}=(1+q)(1+q+q^2)\cdots(1+q+q^2+\cdots+q^{n-1}).
	\end{equation}
\end{thm}

\pf 对任意的 $\pi \in S_n$, 定义其对应的逆序表(inversion table)为
$I(\pi)=(a_1,a_2,\cdots,a_n)$,
其中$a_i$为在$i$左边且比$i$大的元素的个数。
例如 $\pi=417396285$, 则$I(\pi)=(1,5,2,0,4,2,0,1,0)$.
由定义容易看出
\[I(n)=\{(a_1,a_2,\cdots,a_n): 0\leq a_i \leq n-i\}=
[0,n-1]\times [0,n-2] \times [0,1]\times [0,n].\]
且$I(n)$与$S_n$是一一对应的。

因此从上面的分析可知
\begin{eqnarray*}
	\sum_{\pi\in S_n}q^{\inv(\pi)}
	&=&\sum_{a_1=0}^{n-1}\sum_{a_2=0}^{n-2}\cdots
	\sum_{a_n=0}^{0}q^{a_1+a_2+\cdots+a_n}\\
	&=& \sum_{a_1=0}^{n-1}q^{a_1}\sum_{a_2=0}^{n-2}q^{a_2} \cdots
	\sum_{a_n=0}^{0}q^{a_n}\\
	&=&(1+q)(1+q+q^2)\cdots(1+q+q^2+\cdots+q^{n-1}).
\end{eqnarray*}
\qed

下面是置换与其逆之间的逆序数的一个关系。
\begin{prop}
	对任意的 $\pi \in S_n$, 我们有 $\inv(\pi)=\inv(\pi^{-1})$.
\end{prop}

逆序数的生成函数从它的定义中就很容易得到,然而另一个定义方式截然不同的统计量——
主指标却和它有着非常紧密的联系,下面的定理告诉我们,二者是同分布的。

\begin{thm}
	\begin{equation}
	\sum_{\pi\in S_n}q^{\inv(\pi)}=\sum_{\pi\in S_n}q^{\maj(\pi)}.
	\end{equation}
\end{thm}

\pf
我们寻找$S_n$到自身的一个双射来证明它。下面我们就给出由
Foata给出的这个经典的双射,一般称为Foata双射。

双射$\varphi$是递归的定义的。对$w=w_1w_2\cdots w_n \in S_n$,
我们首先令$r_1=w_1$. 现在假设$r_k$($k\leq 1$)已经定义了,
则$r_{k+1}$的定义是这样的:

如果$r_k$的最后一个字母大于(或小于)$w_{k+1}$,
则我们就在$r_k$中每个大于(或小于)$w_{k+1}$ 的字母后面画一条竖线,
这样就把$r_k$中的元素分成了一些块,
然后我们对每个块中的字母向右循环移动
一位,此时每个块中的最后一个元素就变成该块中第一个元素了,
最后我们再把$w_{k+1}$接到变换后的序
列后面,就得到了$r_{k+1}$. 令$\varphi(w)=r_n$.

由$\varphi$的构造可知在每一步变换后都能保证
$\maj(w_1w_2\cdots w_k)=\inv(r_k)$. 

要说明$\varphi$是双射,我们只需给出其逆映射。
从$\varphi$的定义我们可以类似的定义$\varphi^{-1}$如下:

假设$\sigma=\varphi(w)$, 则$\varphi^{-1}$ 的定义为:
若$\sg_n>\sg_1$, 则在小于$\sg_n$的数字之前加一 条竖线,并且在$\sg_n$
的前面也加;若$\sg_n<\sg_1$,
则在大于$\sg_n$的数字之前加一条竖线,并且在 $\sg_n$
的前面也加。然后我们把每个块中的元素向左循环移动一位,
去掉竖线就得到了一个新的置换,此时
我们就把最后一个元素固定下来作为$\varphi^{-1}$的最后一个元素。
接下来用同样的方法确定最后第二个元
素,$n$步以后就得到了$\varphi^{-1}(\sg)$,
且有$\inv(\sg)=\maj(\varphi^{-1}(\sg)$. \qed


我们给出一个例子以便读者更好的理解。

\begin{exa}
	若 $w=417396285$, 我们有:
	\begin{eqnarray*}
		r_1  &=& w_1=4;\\
		r_2  &=& 4|1;\\
		r_3  &=& 4|1|7;\\
		r_4  &=& 4|71|3;\\
		r_5  &=& 4|7|1|3|9;\\
		r_6  &=& 74|913|6;\\
		r_7  &=& 7|4|9|31|6|2;\\
		r_8  &=& 7|4|39|1|6|2|8;\\
		r_9  &=& 7|934|61|82|5.
	\end{eqnarray*}
	且 $\maj(w)=1+3+5+6+8=23, \inv(\varphi(w))=23$.
\end{exa}

一般地,与$\mathrm{maj}$在$S_n$上等分布的统计量称为Mohonian的。



\end{document} 
