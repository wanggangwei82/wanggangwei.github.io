\documentclass[a4paper,12pt]{ctexart}

%\setCJKmainfont[BoldFont={SimHei},ItalicFont={KaiTi}]{SimSun}

\usepackage{fontspec}
\usepackage{xunicode}

%\setCJKmainfont{SimSun}
%\setCJKsansfont{KaiTi}
%\setCJKmonofont{SimHei}



\usepackage{amssymb,amsfonts,amsmath,amsthm,cite}
\usepackage{epsfig}
\usepackage{enumerate}
\usepackage{amsmath,tikz}



\parskip=7pt
%--------------------------------------------------------------------
\hoffset -25truemm \oddsidemargin=25truemm \evensidemargin=25truemm
\textwidth=160truemm \voffset -25truemm \topmargin=20truemm
\headheight=5truemm \headsep=5truemm \textheight=240truemm
\baselineskip=16pt
%--------------------------------------------------------------------
\newtheorem{theo}{定理}[section]
\newtheorem{cor}{推论}[section]
\newtheorem{lem}{引理}[section]
\newtheorem{prop}{命题}[section]
\newtheorem{defi}{定义}[section]
\newtheorem{rem}{注}[section]
\newtheorem{exa}{例}[section]
\def\theequation{\thesection.\arabic{equation}}
\makeatletter \@addtoreset{equation}{section}
%--------------------------------------------------------------------

\def \e{\mathrm{e}}
\def\pf{\noindent {\bf 证明\ }}




\begin{document}





\title{二项式系数}
\author{张彪}
\maketitle

\tableofcontents



\section{课后习题}
7.  利用 \begin{align*}
	m^{2}=2\binom{m}{2}+\binom{m}{1},
\end{align*}
求 $1^{2}+2^{2}+\cdots+n^{2}$ 的值.

 \textbf{解} \quad
\begin{align*}
		\sum_{m=1}^{n} m^{2} &  =\sum_{m=1}^{n} \left( 2 \binom{m}{2}+\binom{m}{1}\right)
		=2 \sum_{m=1}^{n} \binom{m}{2}+\sum_{m=1}^{n} \binom{m}{1} \\[6pt]
		&=2\binom{n+1}{3}+\binom{n+1}{2}=2 \cdot \frac{(n+1) \cdot n(n-1)}{3 \times 2}+\frac{(n+1) n}{2} \\[6pt]
		& =\frac{1}{6} n(n+1)(2 n+1)
\end{align*}

8.求整数 $a, b, c$, 使得
\begin{align}\label{eq-m3}
	m^{3}=a\binom{m}{3}+ b \binom{m}{2}+ c \binom{m}{1},
\end{align}
求计算 $1^{3}+2^{3}+\cdots+n^{3}$ 的值.


 \textbf{解} \quad
将 $m=1$ 代入\eqref{eq-m3},  得 $c=1$.

将 $m=2$ 代入\eqref{eq-m3},  得 $b+2 c=8$, 解得 $b=6$.

 将 $m=3$ 代入\eqref{eq-m3},  得 $a+3 b+3 c=27$,  解得  $a=6$.

 因此,
\begin{align*}
	\sum_{m=1}^{n} m^{3} &=\sum_{m=1}^{n}\left(6\binom{m}{3}+ 6 \binom{m}{2}+  \binom{m}{1}\right) \\ &=b \sum_{m=1}^{n} \binom{m}{3} +6 \sum_{m=1}^{n}\binom{m}{2}+\sum_{m=1}^{n}\binom{m}{1}\\ &=6\binom{n+1}{4}+6\binom{n+1}{3}+\binom{n+1}{2}\\
	& =\frac{1}{4} n^{2}(n+1)^{2}=\binom{n+1}{2}^{2}
\end{align*}
\section{补充习题}


\begin{enumerate}
	\item  证明等式$$\sum_{k=0}^{n} \frac{1}{k+1} \binom{n}{k}   =\frac{2^{n+1}-1}{n+1}.$$

		\textbf{ 证明 } \quad 对
	$$
(1+x)^{n}=\sum_{k=0}^{n}\binom{n}{k} x^{k} \text { , }
	$$
	两边对 $x$ 求从 0 到 1 的定积分,
	\begin{align*}
		\int_{0}^{1}(1+x)^{n} \mathrm{~d} x & =\sum_{k=0}^{n}\binom{n}{k} \int_{0}^{1} x^{k} \mathrm{~d} x \\
		\left.\frac{1}{n+1}(1+x)^{n+1}\right|_{0} ^{1} & =\left.\sum_{k=0}^{n}\binom{n}{k} \cdot \frac{1}{k+1} x^{k+1}\right|_{0} ^{1} \\
		\frac{1}{n+1}\left(2^{n+1}-1\right) & =\sum_{k=0}^{n} \frac{1}{k+1}\binom{n}{k} .
	\end{align*}
	此即所证等式。
	使用这种方法证明不等式时一定要取定积分,否则易出现常数确定上的错误.
%	使用第二个方法时要注意不要漏项.


	\item
	计算 $$\sum_{k=1}^{n} k^{2}\binom{n}{k} .$$

	\textbf{解}  $\quad$ 对$$
	(1+x)^{n}=\sum_{k=0}^{n} \binom{n}{k}  x^{k}
	$$两边求导,得
	$$
	n(1+x)^{n-1}=\sum_{k=0}^{n} k\binom{n}{k}x^{k-1}
	$$
	两边乘 $x$, 得
	$$
	n x(1+x)^{n-1}=\sum_{k=0}^{n} k  \binom{n}{k}  x^{k}
	$$
	两边再求导,得
	$$
	n\left[(1+x)^{n-1}+(n-1) x(1+x)^{n-2}\right]=\sum_{k=0}^{n} k^{2}\binom{n}{k} x^{k-1}
	$$
	取 $x=1$, 得
	$$
	\sum_{k=1}^{n} k^{2} \binom{n}{k} =\sum_{k=0}^{n} k^{2} \binom{n}{k}
	=n\left(2^{n-1}+(n-1) \cdot 2^{n-2}\right)=n(n+1) 2^{n-2}
	$$



\item
证明 $$\binom{n}{k}=\frac{n}{k}\binom{n-1}{k-1}.$$


\textbf{证法一}$\quad$
$$\binom{n}{k}
=\frac{n !}{k !(n-k) !}=\frac{n \cdot(n-1) !}{k \cdot(k-1) !\left(n-1-(k-1) \right) !}=\frac{n}{k}  \binom{n-1}{k-1}.$$


\textbf{证法二} $\quad$ 用组合证明. 考虑从 $n$ 个人中选出带一个小组长的 $k$ 人小组. 若先选出 $k$ 个
人,再从 $k$ 个人中选小组长,则有 $k \binom{n}{k}$ 种选法. 若先选出组长,再从剩下的 $n-1$ 个人中
选 $k$ 一1 个组员,则有 $n \binom{n-1}{k-1}$ 种选法. 从而
$$
k\binom{n}{k}=n \binom{n-1}{k-1}, \quad
\text { 即 }\binom{n}{k}=\frac{n}{k}\binom{n-1}{k-1}.
$$


\item 用 $\binom{n}{k}=\frac{n}{k}\binom{n-1}{k-1}$ 证明下列恒等式。
\begin{enumerate}
 \item  $\sum_{k=0}^{n} k \binom{n}{k}  =n 2^{n-1}$.

 \item  $\sum_{k=0}^{n} k(k-1)  \binom{n}{k}  =n(n-1) 2^{n-2}$.

 \item  $\sum_{k=0}^{n} k^{2}  \binom{n}{k}    =n(n+1) 2^{n-2}$.

  \item $\sum_{k=0}^{n} \frac{1}{k+1} \binom{n}{k}   =\frac{2^{n+1}-1}{n+1}.$
\end{enumerate}

 \textbf{解}\quad
\begin{enumerate}
	\item  $ \sum_{k=1}^{n} k \binom{n}{k}
	=\sum_{k=1}^{n} k \cdot \frac{n}{k}\binom{n-1}{k-1}
	=n \sum_{k=1}^{n} \binom{n-1}{k-1}
	=n \sum_{i=0}^{n-1} \binom{n-1}{i}=n \cdot 2^{n-1}$

 \item
 $\sum_{k=2}^{n} k(k-1) \binom{n}{k}
 =n \sum_{k=2}^{n}(k-1)  \binom{n-1}{k-1}
 =n \sum_{k=1}^{n-1} k  \binom{n-1}{k}
 =n \cdot(n-1) \cdot 2^{n-2}$

  \item  利用前两个式子证明。
   \item  最后一个式子
可用 $\binom{n+1}{k+1}=\frac{n+1}{k+1}\binom{n}{k}$ 这个公式证明。
$$
\begin{aligned}
	\sum_{k=0}^{n} \frac{1}{k+1}\binom{n}{k}&=\sum_{k=0}^{n} \frac{1}{n+1}\binom{n+1}{k+1}=\frac{1}{n+1} \sum_{k=1}^{n+1}\binom{n+1}{k}\\
	&=\frac{1}{n+1}\left(\sum_{k=0}^{n+1}\binom{n+1}{k} -\binom{n+1}{0}\right)=\frac{2^{n+1}-1}{n+1}
\end{aligned}
$$
\end{enumerate}



\item
证明下列组合恒等式:
\begin{enumerate}
	\item $\binom{n}{k}\binom{k}{j}=\binom{n}{j} \binom{n-j}{k-j}$;
	\item  $\sum_{k=0}^{m}(-1)^{k}\binom{n-k}{m-k} \binom{n}{k}=0$
	\item  $\sum_{k=0}^{m}\binom{n-k}{n-m}  \binom{n}{k}=2^{m}\binom{n}{m}$
\end{enumerate}


\textbf{ 证明 } \quad
\begin{enumerate}
	\item
	$\binom{n}{k}\binom{k}{j}=\frac{n !}{k! (n-k) !} \cdot \frac{k}{j!(k-j) !}=\frac{n}{j!(n-j) !}\frac{(n-j) !}{ (k-j) !(n-k) !}=\binom{n}{j} \binom{n-j}{k-j}$
	\item
	$
	\sum_{k=0}^{m}(-1)^{k}\binom{n-k}{m-k} \binom{n}{k}
	= \sum_{k=0}^{m}(-1)^{k}\binom{n}{m}\binom{m}{k}
	= \binom{n}{m} \sum_{k}(-1)^{k}\binom{m}{k}
	= \binom{n}{m} 0=0
	$
	\item
	$\sum_{k=0}^{m}\binom{n-k}{n-m} \binom{n}{k}
	=\sum_{k=0}^{m}\binom{n-k}{m-k}\binom{n}{k}
	=\sum_{k=0}^{m}\binom{n} {m}\binom{m}{k}
	=\binom{n}{ m}\sum_{k=0}^{m}\binom{m}{k}=2^{m}\binom{n}{m}$
\end{enumerate}
\end{enumerate}

\section{思考题}

\begin{enumerate}
\item 设 $n$ 和 $k$ 均为正整数,给出下面式子的一个组合证明:
\begin{align*}
	\sum_{k=0}^{n} k \binom{n}{k}^{2}=n\binom{2n-1}{n-1}.
\end{align*}



\item
设 $n$ 是正整数,证明
$$
\sum_{k=0}^{n}(-1)^{k}\binom{n}{k}^{2}=\left\{\begin{array}{ll}
	0, & n \text { 为奇数 }, \\
	(-1)^{m}\binom{2m}{m}, & n \text { 为偶数 } 2 m .
\end{array}\right.
$$


提示: 考虑$(1-x^2)^n=(1-x)^n(1+x)^n$中$x^n$的系数。

\item
 设 $n$是正整数,证明
\begin{align*}
	\sum_{k=0}^{n-1}\frac{1}{k+1}\binom{n-1}{k}  \binom{n}{k}
=\binom{2n}{n}- \binom{2n}{n-1}.
\end{align*}



\item 证明下列恒等式
\begin{equation*}
	\binom{n+k}{k}^{2}=\sum_{j=0}^{k}\binom{k}{j}^{2}\binom{n+2k-j}{2k}.
\end{equation*}
\textbf{李善兰恒等式}为组合数学中的一个恒等式,由中国清代数学家李善兰于1859年在《垛积比类》一书中首次提出,因此得名。


\end{enumerate}




\section{高斯系数}
\subsection{高斯系数的定义}

我们下面要介绍的高斯系数是由德国数学家高斯最早提出的.高斯被认为是最重要的数学家,并有“数学王子”的美誉.
1792年,15岁的高斯进入Braunschweig学院.从此,高斯开始对高等数学作研究.独立发现了二项式定理的一般形式、数论上的“二次互反律”、素数定理及算术-几何平均数.
18岁时,高斯转入哥廷根大学学习.在他19 岁时,第一个成功的用尺规构造出了规则的17 边形.
1811年,高斯已是哥廷根大学的教授.一次偶然的机会,他在研究“二次高斯和”的估值的时候提出了这样一个多项式
$\left[{{n}\atop {k}}\right]$.

\begin{defi}
一般地,称
\begin{equation}\label{1.1}
{n\brack
k}=\frac{(q^n-1)(q^n-q)\cdots(q^n-q^{k-1})}{(q^k-1)(q^k-q)\cdots(q^k-q^{k-1})},\quad
1\leq k\leq n.
\end{equation}
为{\bf Gauss系数}.
\end{defi}

我们总是假定$|q|<1$. 若记$[n]!=[1]![2]!\cdots[n]!,$
其中$[n]=1+q+q^2+\ldots+q^{n-1},$ 则\eqref{1.1}式也可写为
$${n\brack k}=\frac{[n]!}{[k]![n-k]!}.$$


组合恒等式中最基本的就是二项系数$n \choose k$,它的组合意义大家都已十分清楚了.高斯系数是二项式系数的$q$-模拟.由定义可知
$$\displaystyle\lim_{q\rightarrow 1}{n\brack k}=\frac{n!}{k!(n-k)!}={n\choose
k},$$
这也是我们将$q$-Gauss系数,也称为$q$-二项式系数名称的由来.
\\[3pt]

\subsection{高斯系数的性质}
\begin{theo}
$q$-Gauss系数具有以下性质:
\begin{enumerate}
\item
$\displaystyle{n\brack 0}={n\brack n}=1; $
\item
$\displaystyle{n\brack k}={n\brack {n-k}};$
\item
$\displaystyle{n\brack k}={n-1\brack k}+q^{n-k}{n-1\brack{k-1}};$
\item
$\displaystyle{n\brack k}={n-1\brack{k-1}}+q^k{n-1\brack k}.$
\end{enumerate}
\end{theo}

\begin{pf}
(1)、(2)可直接由定义可得,我们只证(3),
\begin{align*}
{n\brack k}-{n-1\brack
k}&={\frac{(1-q^{n-1})\cdots{(1-q^{n-k+1})}}{(1-q^k)\cdots(1-q)}}[(1-q^n)-(1-q^{n-k})]\\&={\frac{(1-q^{n-1})\cdots{(1-q^{n-k+1}})q^{n-k}(1-q^k)}{(1-q^k)\cdots(1-q)}}\\
&=\frac{q^{n-k}(1-q^(n-1))\cdots(1-q^{n-k+1})}{(1-q^{k-1})\cdots(1-q)}\\
&=q^{n-k}{n-1\brack{k-1}}.
\end{align*}
类似可证明(4).
\end{pf}\\[15pt]
下面我们主要阐述Gauss系数的组合意义.\\[5pt]
\subsection{与线性代数的关系}
\noindent{\bf 组合意义1:}
为此,先给出有限域上的线性空间的一些概念. 设 $\mathbb{F}_q$ 为有限域, 其中 $q=p^{r}$, $p$ 为素数. 对自然数 $n$, 我们定义$V_{n}(q)$ 为$\mathbb{F}_q$ 上的有序 $n$ 元组
\begin{align*}
	\left(x_{1}, x_{2}, \ldots, x_{n}\right), \quad x_{i} \in \mathbb{F}_q, \quad i=1,2, \cdots, n
\end{align*}
组成的集合,并满足线性运算
\begin{align*}
	\begin{array}{c}
		\left(x_{1}, x_{2}, \cdots, x_{n}\right)+\left(y_{1}, y_{2}, \cdots, y_{n}\right)=\left(x_{1}+y_{1}, x_{2}+y_{2}, \cdots, x_{n}+y_{n}\right) \\
		\alpha\left(x_{1}, x_{2}, \cdots, x_{n}\right)=\left(\alpha x_{1}, \alpha x_{2}, \cdots, \alpha x_{n}\right), \quad \alpha \in \mathbb{F}_q
	\end{array}
\end{align*}
则 $V_{n}$ 构成 $\mathbb{F}_q$ 上的 $n$ 维线性空间, 其中的元素称为向量.若向量 $X_{1}, X_{2}, \cdots, X_{m}$ 满足
\begin{align*}
	\sum_{i=1}^{m} \alpha_{i} X_{i}=0 \Rightarrow \alpha_{i}=0, \quad i=1,2, \cdots, m
\end{align*}
则称向量 $X_{1}, X_{2}, \cdots, X_{m}$ 是线性无关的. $\quad V_{n}$ 中含 $n$ 个向量的线性无关组称为 $V_{n}$ 的一组基。 $V_{n}$ 中的任意向量都可以由 $V_{n}$ 的一组基线性表示, 即 $\forall X \in V_{n}$, $\exists\left\{\alpha_{1}, \alpha_{2}, \cdots, \alpha_{n}\right\} \in \mathbb{F}_q$, 使得
\begin{align*}
	X=\sum_{i=1}^{n} \alpha_{i} X_{i}
\end{align*}
用坐标表示为
\begin{align*}
	X=\left(\alpha_{1}, \alpha_{2}, \ldots, \alpha_{n}\right) .
\end{align*}

${n\brack
k}$的组合含义由下面定理给出.
\begin{theo}
有限域 $\mathbb{F}_q$ 上的$n$维线性空间$V_n(q)$的所有$k$维子空间的个数是${n\brack k}.$
\end{theo}

\begin{pf}
首先,从$V_n(q)$中选取一个由$k$个向量组成的元组构成一个$k$维子空间的(有序)基.为此,我们需要从空间$V_n(q)$中选取$k$个线性无关的向量.对于第一个向量$v_1$,可以选取任意非零向量,因此由$q^n-1$中选择.对于第二个向量$v_2$,我们不能选取$v_1$的倍数,因此有$q^n-q$种选择.对于第三个向量$v_3$,有$q^2$ 个不能选取的向量,它们是$v_1$和$v_2$的线性组合.以此类推,从$V_n(q)$中选取$k$ 个线性无关的向量的方法数为
\begin{equation}\label{v_n(q)}
(q^n-1)(q^n-q)\cdots(q^n-q^{k-1}),
\end{equation}
值得注意的是,一个子空间可以有很多组(有序)基.类似上面的讨论,选定一个$k$ 维子空间,在其中选一组基的方法数为
$$(q^k-1)(q^k-q)\cdots(q^k-q^{k-1}).$$
这就是\eqref{v_n(q)}中每个子空间重复计数的数目.因此,$V_n(q)$的$k$维子空间的个数是
$$\frac{(q^n-1)(q^n-q)\cdots(q^n-q^{k-1})}{(q^k-1)(q^k-q)\cdots(q^k-q^{k-1})}={n\brack k}.$$
\end{pf}

\subsection{与排列的关系}
\noindent{\bf  组合意义2:}

首先给出排列中逆序数的概念.一对元素$(i,j)$ 称为是一个{逆序}(inversion),如果满足$i<j$ 且$\pi_i>\pi_j$.$\pi$的
逆序的个数为$\pi$的逆序数,记作$inv(\pi)$.

\begin{prop}
$${n\brack k}=\sum_{\pi\in S(1^k2^{n-k})}q^{\mathrm{inv} \pi},$$
其中$S(1^k2^{n-k})$是由如下置换$\pi$构成的集合:$\pi=a_1a_2\cdots
a_n,$ 其中有$k$个$a_i$为$1$,$n-k$个$a_i$ 为$2$.
\end{prop}
\begin{pf}
对$n$用归纳法.

当$n=1$时,性质显然成立. 现在假设对$n-1$ 成立,考虑$n$的情形.
$\forall\ \pi=a_1a_2\cdots a_n\in S(1^k2^{n-k})$,分两种情况考虑:

若$a_n=2$,
则将$a_n$去掉后,$\pi$的逆序数不发生变化,且此时$a_1a_2\cdots
a_{n-1}\in S(1^k2^{n-k-1});$

若$a_n=1$,
则因为$\pi$中的每个$2$皆对$a_n$产生一个逆序数,故去掉$a_n$后,逆序数减少$n-k$ 个,且$a_1a_2\cdots
a_{n-1}\in S(1^{k-1}2^{n-k}).$

所以
$$ \sum_{\pi\in
S(1^k2^{n-k})}q^{\mathrm{inv} \pi}={n-1\brack
k}+q^{n-k}{n-1\brack{k-1}}={n\brack k}.$$
\end{pf}\\[15pt]
\subsection{与分拆的关系}
\subsubsection{分拆简介}

一个关于整数$n$的分拆$\lambda$是一个有限非递增的整数序列$(\lambda_1,\lambda_2,\ldots,\lambda_r)$,
使得$\sum_{i=1}^{r}\lambda_{r}=n$,则$\lambda_i$称为$\lambda$
的部分,$\lambda_1$为最大部分,$\lambda$ 的部分数称为$\lambda$ 的长度,记为$l(\lambda)$.
$\lambda$的权重是$\lambda$的各部分相加的和,记为$|\lambda|$.\par
例如,$5$的分拆共有$7$个,分别是:
\begin{center}
$(5),(4,1),(3,2),(3,1,1),(2,2,1),(2,1,1,1),(1,1,1,1,1).$
\end{center}
有时我们需要用到$\lambda$中相同部分出现的次数.若
$\lambda=(\lambda_1,\lambda_2,\ldots,\lambda_r)$ 中有$f_1$个$1$,$f_2$个$2$,……我们可以将其表示为
\begin{equation*}
\lambda=\langle1^{f_1},2^{f_2},3^{f_3},...\rangle
\end{equation*}
其中$f_j$表示有$j$出现的次数,注意$\sum_{i\geq1}if_i=n$.\par
所以上面的例子还可以写为:
\begin{center}
$\langle5^{1}\rangle,\langle1^{1},4^{1}\rangle,\langle2^{1},3^{1}\rangle,
\langle1^{2},3^{1}\rangle,\langle1^{1},2^{2}\rangle,\langle1^{3},2^{1}\rangle,
\langle1^{5}\rangle$
\end{center}

分拆还可以用它的Young图表示,如图所示:
\begin{center}\setlength{\unitlength}{1mm}
\begin{picture}(100,60)
\multiput(10,55)(0,-5){2}{\line(1,0){35}}\multiput(10,45)(0,-5){2}{\line(1,0){25}}
\put(10,35){\line(1,0){10}}\multiput(10,35)(5,0){3}{\line(0,1){20}}
\multiput(25,40)(5,0){3}{\line(0,1){15}}\multiput(40,50)(5,0){2}{\line(0,1){5}}
\put(40,35){$\lambda$} \multiput(65,55)(0,-5){3}{\line(1,0){20}}
\multiput(65,40)(0,-5){3}{\line(1,0){15}}\multiput(65,25)(0,-5){2}{\line(1,0){5}}
\multiput(65,20)(5,0){2}{\line(0,1){35}}\multiput(75,30)(5,0){2}{\line(0,1){25}}
\put(85,45){\line(0,1){10}}\put(85,30){$\lambda^{'}$} \put(0,10){图
1.1:
分拆$\lambda=(7,5,5,2)$及其共轭分拆$\lambda^{'}=(4,4,3,3,3,1,1)$}
\end{picture}
\end{center}

给定分拆$\lambda=(\lambda_1,\lambda_2,\ldots,\lambda_r)$ 我们定义$\lambda$的共轭分拆
$\lambda^{'}=(\lambda_1^{'},\lambda_2^{'},\ldots,\lambda_t^{'})$, 其中$\lambda_i^{'}$
表示$\lambda$中大于或者等于$i$的部分数.实际上共轭$\lambda^{'}$ 可以由分拆$\lambda$ 通过作
关于主对角线的翻转而得到,如图1.1所示.


\subsubsection{高斯系数与分拆的关系}
\noindent{\bf  组合意义3:}

令$Par$表示所有分拆的集合.
设
$$L(m,n)=\{\lambda\in Par: \ell(\lambda)\leq n, \ell(\lambda')\leq m\}.$$ 高斯系数和分拆的联系由下面的定理给出.

\begin{theo}
对任意的$m,n\in\mathbb{N}$,我们有
$$\left[{{m+n}\atop {n}}\right]=\sum_{\lambda\in L(m,n)}q^{|\lambda|}$$
\end{theo}

\begin{exa}
当$m=2,n=3$时,
$$\left[{{2+3}\atop {3}}\right]=1+q+2q^2+2q^3+2q^4+q^5+q^6$$
\end{exa}


假设$n$的分拆为
\begin{equation}
n=1\cdot a_1+2\cdot a_2+3\cdot a_3+\ldots+n\cdot a_n
\end{equation}
令$p(n,k,N)$表示$n$的所有分拆中满足
$$a_1+a_2+\ldots+a_n\leq k,\quad
a_{N+1}=\ldots=a_n=0$$的分拆个数,即分拆部分数$\leq k,$
且最大部分$\leq N$的分拆数.

显然
\begin{align*}
p(kN,k,N)&=1\text{,唯一的分拆是}<N^k>;\\
p(n,k,N)&=0,\quad \mbox{当}n>kN\mbox{时}.\\
\end{align*}
另外,我们令
\[
p(n,0,N)=p(n,k,0)=\begin{cases}1,\quad n=k=N=0,\\ 0,\quad
\mbox{其他}.
\end{cases}
\]
则有
\begin{lem}
\begin{equation}\label{1.3}
p(n,k,N)=p(n,k-1,N)+p(n-k,k,N-1).
\end{equation}
\end{lem}
\begin{pf}
有如下的组合解释:$p(n,k,N)-p(n,k-1,N)$计数的是$n$ 的分拆中分拆部分数恰为$k,$
且最大部分$\leq
N$的$n$的分拆个数;任给$n$的一个满足条件的分拆,从每一部分里减去1,得到$n-k$ 的一个分拆,且满足部分数$\leq
k,$ 最大部分$\leq N-1.$ 这样的分拆个数为$p(n-k,k,N-1).$

容易验证,这两种类型的分拆之间有一个一一对应.因此\eqref{1.3} 式成立.
考虑$p(n,k,N)$的生成函数
$$F(q;k,N)=\sum_{n=0}^{kN}p(n,k,N)q^n.$$
由\eqref{*}式,
$$F(q;k,N)=F(q;k-1,N)+q^kF(q;k;N-1),$$
其中$F(q;0,N)=F(q;k,0)=1.$

根据定理1.1.1中的(1),(3),我们看到$F(q;k,N)$ 和${N+k\brack
k}$有相同的递推关系式和初始值,因此
$$F(q;k,N)={N+k\brack k},$$ 也即
$${N+k\brack k}=\sum_{n=0}^{kN}p(n,k,N)q^n.$$
\end{pf}












\subsection{Cauchy二项式定理}

众所周知,二项式定理,又称牛顿二项式定理
\begin{equation}
	(1+z)^n=\sum_{j=0}^{n} {n \choose j} z^{j}
\end{equation}
其中
$$
{n \choose j}= \frac{n!}{j!(n-j)!}
$$

上述定理由艾萨克·牛顿(Newton)于1665年初提出。1643年1月4日牛顿生于英国林肯郡的沃尔索普村,父亲是一个农民,
在牛顿出生前就死了。虽然母亲也希望他务农,但幼年的牛顿却在做机械模型和实验上显示了他的爱好和才能。
例如,他做了一个玩具式的以老鼠为动力的磨和一架靠水推动的木钟。14 岁时,由于生活所迫,牛顿停学务农,
以后在舅父的帮助下又入学读书。1661年,不满19 岁的牛顿考入剑桥大学的三一学院。1665 年,鼠疫在英国流行,
剑桥大学关闭,牛顿只好回农村居住。在沃尔索普村的18个月里,牛顿给出了二项式定理的证明,发明了微积分,
提出了万有引力定律,还研究了光的性质。牛顿一生的重大成就大都发韧于这一期间。后来,他在追忆这段峥嵘的
青春岁月时说:“当年我正值发明创造能力最强的年华,比以后任何时期更专心致志于数学和哲学(科学)。”


下面是Cauchy二项式定理:
\begin{theo}
	\label{coro-cauchy binomial}
	\begin{equation*}
		\prod_{k=1}^{n}(1+yq^k)=\sum_{k=0}^{n}y^k q^{\frac{k(k+1)}{2}} {n \brack k}_q
	\end{equation*}
\end{theo}
~\\
由Gauss系数的性质可知,Cauchy二项式定理为牛顿二项式定理的$q$模拟。
下面给出推论 \ref{coro-cauchy binomial} 的四种证明。\\
方法一(数学归纳法)

假设 $n=m$时,推论\ref{coro-cauchy binomial} 显然成立,下面证明$n=m+1$ 时有
\begin{equation*}
	\prod_{k=1}^{m+1}(1+yq^k)=\sum_{k=0}^{m+1}y^k q^{\frac{k(k+1)}{2}} {{m+1}\brack k}_q
\end{equation*}
由
$$
{{m+1}\brack k}_q= {{m}\brack k}_q+ q^{m+1-k}{{m}\brack {k-1}}_q
$$
可知
\begin{align*}
	&\sum_{k=0}^{m+1}y^k q^{\frac{k(k+1)}{2}} {{m+1}\brack k}_q\\
	=&\sum_{k=1}^{m}y^k q^{\frac{k(k+1)}{2}} {{m}\brack k}_q +\sum_{k=1}^{m}y^k q^{\frac{k(k+1)}{2}+m+1-k} {{m}\brack {k-1}}_q
	+1+y^{m+1} q^{\frac{(m+1)(m+2)}{2}} {{m+1}\brack {m+1}}_q\\
	=&\sum_{k=0}^{m}y^k q^{\frac{k(k+1)}{2}} {{m}\brack k}_q +\sum_{k=0}^{m}y^{k+1} q^{\frac{k(k+1)}{2}+m+1} {{m}\brack k}_q \\
	=&(1+yq^{m+1})\prod_{k=1}^{m}(1+yq^k)\\
	=&\prod_{k=1}^{m+1}(1+yq^k)
\end{align*}
得证。\\
~\\
方法二(生成函数方法)

令
\begin{equation*}
	F(y)=\prod_{k=1}^n (1+yq^k)=\sum_{k=0}^n A_k  y^k
\end{equation*}
观察到
\begin{equation*}
	(1+yq)\prod_{k=2}^n (1+yq^k)(1+yq^{n+1})=(1+yq)F(yq)=(1+yq^{n+1})F(y)
\end{equation*}
则
\begin{equation*}
	(1+yq)\sum_{k=0}^n A_k  y^k q^k=(1+yq^{n+1})\sum_{k=0}^n A_k  y^k
\end{equation*}
比较$y^k$系数可知
\begin{equation*}
	A_k +q^{n+1}A_{k-1}=q^k A_k+q^k A_{k-1}
\end{equation*}
则
\begin{equation*}
	\frac{A_k}{A_{k-1}} =\frac{q^k(1-q^{n+1-k})}{1-q^k}
\end{equation*}
则
\begin{align*}
	&A_k =\frac{A_k}{A_{k-1}}\cdot\frac{A_{k-1}}{A_{k-2}}\cdots\frac{A_{1}}{A_{0}}\cdot A_{0}\\
	=&\frac{q^k(1-q^{n+1-k})}{1-q^k}\cdot\frac{q^{k-1}(1-q^{n+2-k})}{1-q^{k-1}}\cdots \frac{q(1-q^{n})}{1-q}\cdot A_{0}
\end{align*}
由$A_0=1$可知
\begin{equation*}
	A_k =q^{\frac{k(k+1)}{2}}{n \brack k}_q
\end{equation*}
得证。\\
~\\
方法三(应用排列相关性质)

由Gauss系数性质可知,
\begin{equation}
	\label{eq-2}
	{{i+j} \brack i}_q=\sum_{\sigma \epsilon S(i,j)} q^{inv(\sigma)}
\end{equation}
其中 $S(i,j)$为重集 $\{1^j 2^i\}$上的所有排列构成的集合。

\begin{lem}
	令$E=\{(a_1,a_2,\cdots,a_i)|0 \leq a_1 \leq a_2 \leq\cdots \leq a_i\leq j, a_1,a_2,\cdots,a_i \epsilon \mathbb{N} \}$,集合$E$与$S(i,j)$间存在
	一一映射。
\end{lem}

\pf
对于任意的$(a_1,a_2,\cdots,a_i)\epsilon E$,构造重集$\{1^j 2^i\}$上的排列$\sigma$,使得$\sigma$中第一个$2$贡献$a_i$个逆序,
第二个$2$贡献$a_{i-1}$个逆序,$\cdots\cdots$,最后一个$2$贡献$a_1$个逆序。例如:
\begin{equation*}
	(0,0,1,1,2,2,3)\longleftrightarrow 2122122122
\end{equation*}
易验证该映射为一一映射。
~\\
~\\
如果$(a_1,a_2,\cdots,a_i)\longleftrightarrow \sigma$,那么
\begin{equation*}
	a_1+a_2+\cdots+a_i=inv(\sigma)
\end{equation*}
由(\ref{eq-2})得
\begin{equation}
	\label{eq-3}
	{{i+j} \brack i}_q=\sum_{0 \leq a_1 \leq a_2 \leq\cdots \leq a_i\leq j} q^{a_1+a_2+\cdots+a_i}
\end{equation}

\begin{lem}
	\label{lem-1}
	令$F=\{(b_1,b_2,\cdots,b_i)|1 \leq  b_1 < b_2 <\cdots < b_i \leq j+i, b_1,b_2,\cdots,b_i \in p\mathbb{N} \}$,集合$E$ 与$F$间存在
	一一映射。
\end{lem}

\pf
该一一映射为
\begin{equation*}
	(a_1,a_2,\cdots,a_i)\longleftrightarrow (a_1+1,a_2+2,\cdots,a_i+i)
\end{equation*}
例如
\begin{equation*}
	(0,0,1,1,2,3)\longleftrightarrow (1,2,4,5,7,8,10).
\end{equation*}
~\\
由(\ref{eq-3})及引理\ref{lem-1}可知
\begin{align*}
	{{i+j} \brack i}_q=&\sum_{0 \leq a_1 \leq a_2 \leq\cdots \leq a_i\leq j} q^{a_1+a_2+\cdots+a_i}\\
	=&\sum_{0 \leq a_1 \leq a_2 \leq\cdots \leq a_i\leq j} q^{(a_1+1)+(a_2+2)+\cdots+(a_i+i)-\frac{i(i+1)}{2}}\\
	=&\sum_{1 \leq  b_1 < b_2 <\cdots < b_i \leq j+i} q^{b_1+b_2+\cdots+b_i-\frac{i(i+1)}{2}}
\end{align*}
令 $j=n-i$,则
\begin{equation}
	\label{eq-4}
	{n \brack i}_q=\sum_{1 \leq  b_1 < b_2 <\cdots < b_i \leq n} q^{b_1+b_2+\cdots+b_i-\frac{i(i+1)}{2}}
\end{equation}
注意到$\prod_{k=1}^{n}(1+yq^k)$中$y^i$ 的系数为
\begin{equation*}
	\sum_{1 \leq e_1 < e_2 <\cdots <e_i\leq n} q^{e_1+e_2+\cdots+e_i}=q^{\frac{i(i+1)}{2}} {n \brack i}_q
\end{equation*}
故推论\ref{coro-cauchy binomial}得证。\\
~\\
方法四(应用分拆相关性质)\\
由生成函数知识可知,$\prod_{k=1}^{n}(1+yq^k)$ 为各部分不同、最大部分$\leq n$ 且部分数$\leq n$ 的分拆的生成函数。给定各部分不同且最大部分$\leq n$ 的分拆$\lambda$,
%\begin{figure}[h,t,b,p]
%\begin{center}
%\scalebox{0.5}[0.5]{\includegraphics{pic-1.eps}}
%\end{center}
%\end{figure}
设$\lambda$的部分数为$k(0 \leq k \leq n)$。 将其如图所示做分解:图中,(1)对应$q^{\frac{k(k+1)}{2}}$,
由Guass系数的性质${n \brack k}_q$为满足部分数$\leq k$ 最大部分$\leq n-k$的分拆的生成函数可知(2)对应 ${n \brack k}_q$,
得证。\\


\end{document}










